% Template for Doctoral Theses at Uppsala
% University. The template is based on
% the layout and typography used for
% dissertations in the Acta Universitatis
% Upsaliensis series
% Ver 5.2 - 2012-08-08
% Latest version available at:
%   http://ub.uu.se/thesistemplate
%
% Support: Wolmar Nyberg Akerstrom
% Thesis Production
% Uppsala University Library
% avhandling@ub.uu.se
%
%%%%%%%%%%%%%%%%%%%%%%%%%%%%%%%%%%%%%%%%%%%


%%%%%%%%%%%%%%%%%%%%%%%%%%%%%%%%%%%%%%
% Radial Basis Function generated Finite Difference Methods for Pricing Financial Derivatives
%%%%%%%%%%%%%%%%%%%%%%%%%%%%%%%%%%%%%%

%%%
% Chosen: Berling as the main font and Pazo as the math font with Gill Sans for Sammanfattning and Преглед.
% Alternative: Palatino as the main font and Pazo as the math font or Computer Modern for all.
%%%

\documentclass{UUThesisTemplate}

% Package to determine wether XeTeX is used
\usepackage{ifxetex}

\ifxetex
	% XeTeX specific packages and settings
	% Language, diacritics and hyphenation
	\usepackage{polyglossia}
	\usepackage{fontspec,xltxtra,xunicode}
	\defaultfontfeatures{Mapping=tex-text}

	% Palatino with Pazo
	\setmainfont[
		Ligatures=TeX,
  		Extension=.ttf,
  		BoldFont=BerlingBold,
  		ItalicFont=BerlingItalic,
  		BoldItalicFont=BerlingBold_Italic,
	]{Berling}
	\renewcommand{\baselinestretch}{1.2} % Line width to support inline math!
	\usepackage{mathpazo} % I like this one the most for math!
%	\usepackage{mathpple}


%	% Computer Modern
%	\setmainfont[
%  		Ligatures=TeX,
%  		Extension=.otf,
%  		BoldFont=cmunbx,
%  		ItalicFont=cmunti,
%  		BoldItalicFont=cmunbi,
%	]{cmunrm}


	\setmainlanguage{english}
	\setotherlanguages{serbian, swedish}
	\setkeys{serbian}{script=Cyrillic}
	
	% Summary fonts
	\newfontfamily\cyrillicfont[Script=Cyrillic]{Gill Sans}
	\newfontfamily\swedishfont{Gill Sans}
	
%	\newfontfamily\cyrillicfont[Script=Cyrillic]{Palatino}
%	\newfontfamily\swedishfont{Palatino}
	
%	\newfontfamily\myfont{Palatino}
	
	% Font settings
%	\setmainfont{Times New Roman}
%	\setromanfont{Times New Roman}
%	\setsansfont{Arial}
%	\setmonofont{Courier New}


\else
	% Plain LaTeX specific packages and settings
	% Language, diacritics and hyphenation
    % Use English and Swedish languages.
	\usepackage[swedish,english]{babel}

	% Font settings
	\usepackage{type1cm}
	\usepackage[latin1]{inputenc}
	\usepackage[T1]{fontenc}
	\usepackage{mathptmx}

	% Enable scaling of images on import
	\usepackage{graphicx}
\fi


% Tables
\usepackage{booktabs}
\usepackage{tabularx}

% Document links and bookmarks
\usepackage{hyperref}

% Numbering of headings down to the subsection level
\numberingdepth{subsection}

% Including headings down to the subsection level in contents
\contentsdepth{subsection}


% Uncomment to use a custom abstract dummy text
%\abstractdummy{
%	\begin{abstract}
%		Please use no more than 300 words and avoid mathematics or complex script.
%	\end{abstract}
%}


\begin{document}
\frontmatter
    % Creates the front matter (title page(s), abstract, list of papers)
    % for either a Comprehensive Summary or a Monograph.
    % Authors of Comprehensive Summaries use this front matter
    \frontmatterCS
    % Monograph authors use this front matter
    %\frontmatterMonograph

   % Optional dedication
   \dedication{``These violent delights have violent ends''\\(Romeo and Juliet: Act 2, Scene 6, Line 9)}
   
\begin{swedish}
\chapter*{Sammanfattning}
Jag skriver på svenska här.
\end{swedish}

\begin{serbian}
\chapter*{Преглед}
\noindent Ова теза злата вреди!
\end{serbian}

% Environment used to create a list of papers   
    \begin{listofpapers}
    	\item Radial Basis Function generated Finite Differences for Option Pricing Problems \label{paper1}
	\item BENCHOP --- The BENCHmarking Project in Option Pricing  \label{paper2}
	\item BENCHOP-SLV: The BENCHmarking project in Option Pricing --- Stochastic and Local Volatility problems \label{paper4}
	\item Pricing Derivatives under Multiple Stochastic Factors by Localized Radial Basis Function Methods \label{paper3}
	\item Pricing Financial Derivatives using Radial Basis Function generated Finite Differences with Polyharmonic Splines on Smoothly Varying Node Layouts \label{paper5}
	\item SMOOTHING PAPER TITLE \label{paper6}
    \end{listofpapers}


    \begingroup
        % To adjust the indentation in your table of contents, uncomment and enter the widest numbers for each level
        %  E.g.  \settocnumwidth{widest chapter number}{widest section number}{widest subsection number}...{...}
       %  \settocnumwidth{5}{4}{5}{3}{3}{3}
        \tableofcontents
    \endgroup

    % Optional tables
    %\listoftables
    %\listoffigures

\mainmatter
    % This includes the "Instruction", "Problem and Solutions" and "Example" files. After reading it, remove it from Thesis.tex.
%    \input{Example/Instruction.tex}
%    \input{Example/ProblemsAndSolutions}
%    \input{Example/Example.tex}

    % Include your chapters here.
%    \input{Introduction.tex}
%
%
%%%
%\par
%\noindent abcde­fghijklmnopqrstu­vwxyz­abcde­fghijklmnopqrstu­vwxyz­abcdabcde­fghijklmnopqrstu­vwxya\\
%bcde­fghijklmnopqrstu­vwxyz­abcde­fghijklmnopqrstu­vwxyz­abcdabcde­fghijklmnopqrstu­vwxyab\\%max=90
%cde­fghijklmnopqrstu­vwxyz­abcde­fghijklmnopq\\%min=45
%rstu­vwxyz­abcdabcde­fghijklmnopqrstu­vwxfdasdadadasdadaadadadaa%64 current
\chapter{Introduction}
\label{ch:introduction}
\par
The purpose of this thesis is to report on state of the art in Radial Basis Function generated Finite Difference (RBF-FD) methods for pricing of financial derivatives. Based on the six appended papers, this work provides a detailed overview of RBF-FD properties and challenges that arise when the RBF-FD method is used in financial applications. Furthermore, the manuscript aims to motivate further development of RBF-FD for finance.
\par
Across the financial markets of the world, financial derivatives are traded in substantial volumes. Therefore, knowing the prices of those financial instruments is of utmost importance at any given time. In order to make that possible in practice, it is very often required to employ a set of skills incorporating knowledge in financial theory, engineering methods, mathematical tools, and the programming practice --- which altogether constitute the field known as \emph{financial engineering}. 
\par
Many of theoretical pricing models for financial derivatives can be represented using partial differential equations (PDEs). In great deal of cases, those equations are time-dependent, of high spatial dimensions, and with challenging boundary conditions --- which most often makes them analytically unsolvable. In those cases, numerical approximation as a mean of estimating their solution needs to be utilized. The field of \emph{numerical analysis} is concerned with obtaining approximate solutions while maintaining reasonable bounds on errors. Unfortunately, there is no universal numerical method which could be used to efficiently solve all problems of this type. In fact, there are tremendously many numerical methods for solving different types of PDEs, and all those methods are featured with individual limitations in performance, stability, and accuracy --- mostly depending on the problem details. Therefore, carefully selecting and developing numerical methods for particular applications has been the only way to build the most efficient PDE solvers in the ongoing practice. 
\par
RBF-FD is a numerical method that is developed to efficiently solve partial differential equations. RBF-FD, as its name suggests, is a finite difference type of a method from the radial basis function family. Such methods are mainly developed for numerical approximation of differential operators which is a crucial building block for many solvers partial differential equations (PDEs).
\par
This manuscript is organized in the following way. 
%%%
\chapter{Financial Derivatives}
\label{ch:finder}
\par
Across the financial markets of the world, financial derivatives are traded in substantial volumes. According to the reports of BIS, the estimated total notional value of these financial instruments has been above half a quadrillion of USD during the current decade. That value is about the order of magnitude larger than the total world gross domestic product (GDP). The main reason for this astronomical market size is that there are numerous financial derivatives in existence, available for almost every type of investment asset.

\section{Futures}
\label{sec:futures}
\section{Options}
\label{sec:options}
%%%
\chapter{Option Pricing}
\label{ch:optionpricing}
Talk about stochastic and deterministic formulation.
%%
\section{Market Models}
\label{sec:models}
State, motivate and explain the models used in BENCHOP.
%%
\section{Pricing Methods}
\label{sec:methods}
Talk about all the method groups from BENCHOP.
%
%
%%%
\chapter{Radial Basis Function generated Finite Difference Methods}
\label{ch:rbffd}
Similar to explanation in our papers, a bit of history and the method evolution throughout the papers.
\section{Choosing Shape Parameters}
\section{Constructing Node Layouts}
\section{Smoothing Payoff Functions}
%
%
%%%
\chapter{Outlook and Further Development}
Sumarize.

\backmatter
    % References
    % No restriction is set to the reference styles
    % Save your references in References.bib
    \nocite{*} % Remove this for your own citations
    \bibliographystyle{plain}
    \bibliography{References}

\chapter{Acknowledgment}
Wohoo!

\end{document}
