% Template for Doctoral Theses at Uppsala
% University. The template is based on
% the layout and typography used for
% dissertations in the Acta Universitatis
% Upsaliensis series
% Ver 5.2 - 2012-08-08
% Latest version available at:
%   http://ub.uu.se/thesistemplate
%
% Support: Wolmar Nyberg Akerstrom
% Thesis Production
% Uppsala University Library
% avhandling@ub.uu.se
%
%%%%%%%%%%%%%%%%%%%%%%%%%%%%%%%%%%%%%%%%%%%


%%%%%%%%%%%%%%%%%%%%%%%%%%%%%%%%%%%%%%
% Radial Basis Function generated Finite Difference Methods for Pricing of Financial Derivatives
%%%%%%%%%%%%%%%%%%%%%%%%%%%%%%%%%%%%%%

%%%
% Chosen: Berling as the main font and Pazo as the math font with Gill Sans for Sammanfattning and Преглед.
% Alternative: Palatino as the main font and Pazo as the math font or Computer Modern for all.
%%%

\documentclass{UUThesisTemplate}
\raggedbottom

% Package to determine wether XeTeX is used
\usepackage{ifxetex}

\ifxetex
    % XeTeX specific packages and settings
    % Language, diacritics and hyphenation
    \usepackage{polyglossia}
    \usepackage{fontspec,xltxtra,xunicode}
    \defaultfontfeatures{Mapping=tex-text}

    % Berling with Pazo
    \setsansfont[Ligatures=TeX]{Gill Sans}
    \setmainfont[
    	Ligatures=TeX,
        Extension=.ttf,
        BoldFont=BerlingBold,
        ItalicFont=BerlingItalic,
        BoldItalicFont=BerlingBold_Italic,
    ]{Berling}
    \renewcommand{\baselinestretch}{1.2} % Line width to support inline math!
    \usepackage{mathpazo} % I like this one the most for math!
%%    \usepackage{mathpple}

%    % Palatino with Pazo
%    \setmainfont{Palatino}
%    \renewcommand{\baselinestretch}{1.2} % Line width to support inline math!
%    \usepackage{mathpazo} % I like this one the most for math!
%%    \usepackage{mathpple}

    
%    % Computer Modern
%    \setmainfont[
%          Ligatures=TeX,
%          Extension=.otf,
%          BoldFont=cmunbx,
%          ItalicFont=cmunti,
%          BoldItalicFont=cmunbi,
%    ]{cmunrm}


    \setmainlanguage{english}
    \setotherlanguages{serbian, swedish}
    \setkeys{serbian}{script=Cyrillic}
    
    % Summary fonts
%    \newfontfamily\cyrillicfont[Script=Cyrillic]{Gill Sans Light}
%    \newfontfamily\swedishfont{Gill Sans Light}
    
    \newfontfamily\cyrillicfont[Script=Cyrillic]{Gill Sans}
    \newfontfamily\swedishfont{Gill Sans}
    
    \newfontfamily\myfont{Gill Sans Italic}
    \DeclareTextFontCommand{\gillsansitalicfont}{\myfont}
    
    \newfontfamily\noteunic{Gill Sans Light}
    \DeclareTextFontCommand{\bibnamefont}{\noteunic}
    
    
%    \newfontfamily\cyrillicfont[Script=Cyrillic]{Palatino}
%    \newfontfamily\swedishfont{Palatino}
    
    
    % Font settings
%    \setmainfont{Times New Roman}
%    \setromanfont{Times New Roman}
%    \setsansfont{Arial}
%    \setmonofont{Courier New}

\usepackage[dvipsnames]{xcolor}
\usepackage{commath}
\usepackage{mathtools}
\newtagform{brackets}{\rmfamily{(}}{\rmfamily{)}}
\usetagform{brackets}
%\renewcommand{\theequation}{{\rmfamily\arabic{equation}}}
%\renewcommand{\theequation}{\rmfamily{\arabic{chapter}.\arabic{equation}}}
%\renewcommand{\thetable}{\rmfamily{\arabic{table}}}
%\renewcommand{\thefigure}{\rmfamily{\arabic{table}}}

\else
    % Plain LaTeX specific packages and settings
    % Language, diacritics and hyphenation
    % Use English and Swedish languages.
    \usepackage[swedish,english]{babel}

    % Font settings
    \usepackage{type1cm}
    \usepackage[latin1]{inputenc}
    \usepackage[T1]{fontenc}
    \usepackage{mathptmx}

    % Enable scaling of images on import
    \usepackage{graphicx}
\fi
\usepackage{amsmath}

% Letter đ
\newcommand*\strike[1]{\bibnamefont{#1}\kern-0.3em{\raisebox{0.75ex}{-}}}

% Color grey
\definecolor{darkgrey}{rgb}{0.5,0.5,0.5}%

% Tables
\usepackage{booktabs}
\usepackage{tabularx}
\makeatletter
\def\hlinewd#1{\noalign{\ifnum0=`}\fi\hrule \@height #1\futurelet\reserved@a\@xhline}
\usepackage{arydshln}
\setlength{\arrayrulewidth}{1pt}

\renewcommand{\thefootnote}{\fnsymbol{footnote}}
\makeatother

\usepackage{float}
\usepackage{pgfplots}
\usepackage{tikz}
\usepackage{tikzsymbols}
\pgfplotsset{compat=1.15}
\graphicspath{{figures/}}

% Document links and bookmarks
\usepackage{hyperref}
%\usepackage{caption}


%\tikzset{
%    ultra thin/.style= {line width=0.1pt},
%    very thin/.style=  {line width=0.2pt},
%    thin/.style=       {line width=0.4pt},% thin is the default
%    semithick/.style=  {line width=0.6pt},
%    thick/.style=      {line width=0.8pt},
%    very thick/.style= {line width=1.2pt},
%    ultra thick/.style={line width=1.6pt}
%}

\tikzset{
    ultra thin/.style= {line width=0.15pt},
    very thin/.style=  {line width=0.3pt},
    thin/.style=       {line width=0.6pt},% thin is the default
    semithick/.style=  {line width=0.9pt},
    thick/.style=      {line width=1.2pt},
    very thick/.style= {line width=1.8pt},
    ultra thick/.style={line width=2.4pt}
}

\newcommand{\mybasiclinewidth}{thin}
%source is page 126 in the manual
\tikzset{
every picture/.style={
\mybasiclinewidth} %or use: "`line width=1 pt,"<-- note:if you write line width, you must use a value with unit
}
\pgfplotsset{
every axis/.append style={
thin,
grid style={
    thin,
},
tick style={
    very thin,
},
},
}

% Numbering of headings down to the subsection level
\numberingdepth{subsection}

% Including headings down to the subsection level in contents
\contentsdepth{subsection}

% Uncomment to use a custom abstract dummy text
\abstractdummy{
    \begin{abstract}
\par
The purpose of this thesis is to present state of the art in radial basis function generated finite difference (RBF-FD) methods for pricing of financial derivatives. This work provides a detailed overview of RBF-FD properties and challenges that arise when the RBF-FD methods are used in financial applications. 

\par
Across the financial markets of the world, financial derivatives such as futures, options, and others, are traded in substantial volumes. Knowing the prices of those financial instruments at any given time is of utmost importance. Many of the theoretical pricing models for financial derivatives can be represented using multi-dimensional PDEs, which are in most cases analytically unsolvable.

\par
We present RBF-FD as a recent numerical method with potential to efficiently approximate solutions of PDEs in finance. As its name suggests, the RBF-FD method is of a finite difference (FD) type, from the radial basis function (RBF) group of methods. When used to approximate differential operators, the method is featured with a sparse differentiation matrix, and it is relatively simple to implement --- like the standard FD methods. Moreover, the method is mesh-free, meaning that it does not require a structured discretization of the computational domain, and it is of a customizable order of accuracy --- which are the features it inherits from the global RBF approximations.

\par
The results in this thesis demonstrate how to successfully apply RBF-FD to different pricing problems by studying the effects of RBF shape parameters for Gaussian RBF-FD approximations, improving the approximation of differential operators in multiple dimensions by using polyharmonic splines augmented with polynomials, constructing suitable node layouts, and smoothing of the initial data to enable high order convergence of the method. Finally, we compare RBF-FD with other available methods on a plethora of pricing problems to form an objective image of the method's performance. 

\par
Future development of RBF-FD is expected to result in a solid mesh-free high order method for multi-dimensional PDEs, that can be used together with dimension reduction techniques to efficiently solve problems of high dimensionality that we often encounter in finance.
    \end{abstract}
}


\begin{document}
\frontmatter
    % Creates the front matter (title page(s), abstract, list of papers)
    % for either a Comprehensive Summary or a Monograph.
    % Authors of Comprehensive Summaries use this front matter
    \frontmatterCS
    % Monograph authors use this front matter
    %\frontmatterMonograph

   % Optional dedication
   \dedication{``These violent delights have violent ends''\\(Romeo and Juliet: Act 2, Scene 6, Line 9)}

% Environment used to create a list of papers   
    \begin{listofpapers}
%    \small{
    \item
    \bibnamefont{S. Milovanović and L. von Sydow}.\ %
    \emph{Radial Basis Function generated Finite Differences for Option Pricing Problems.}\ %
    Comp. Math. Appl., 75(4):1462--1481, 2017. \label{paper1} %
    \item
    \bibnamefont{Slobodan Milovanović.}\ %
    \emph{Pricing Financial Derivatives using Radial Basis Function generated Finite Differences with Polyharmonic Splines on Smoothly Varying Node Layouts.}\ %
    arXiv preprint, arXiv:1808.02365[q-fin.CP], 2018. \label{paper2} %
    \item
    \bibnamefont{S. Milovanović and L. von Sydow.}\ %
    \emph{A High Order Method for Pricing of Financial Derivatives using Radial Basis Function generated Finite Differences.}\ %
    arXiv preprint, arXiv:xxxx.yyyyy[q-fin.CP], 2018. \label{paper3} %
    \item
    \bibnamefont{S. Milovanović and V. Shcherbakov.}\ % 
    \emph{Pricing Derivatives under Multiple Stochastic Factors by Localized Radial Basis Function Methods.}\ %  	  
    Journ. Comp. Fin., (in review), 2018. \label{paper4} %arXiv:1711.09852[q-fin.CP] 
    \item
    \bibnamefont{L. von Sydow, L. J. Höök, E. Larsson, E. Lindström, S. Milovanović, J. Persson, V. Shcherbakov, Y. Shpolyanskiy, S. Sirén, J. Toivanen, J. Waldén, M. Wiktorsson, J. Levesley, J. Li, C. W. Oosterlee, M. J. Ruijter, A. Toropov, and Y. Zhao.}\ % 
    \emph{BENCHOP --- The BENCHmarking Project in Option Pricing.}\ %  	  
    Int. Journ. Comp. Math., 92(12): 2361--2379, 2015. \label{paper5}
    \item
    \bibnamefont{L. von Sydow, S. Milovanović, E. Larsson, K. in 't Hout, M. Wiktorsson, C. W. Oosterlee, V. Shcherbakov, M. Wyns, A. Leitao, S. Jain, T. Haentjens, and J. Waldén.}\ % 
    \emph{BENCHOP: The BENCHmarking Project in Option Pricing --- Stochastic and local volatility problems.}\ %  	  
    Int. Journ. Comp. Math., (in review), 2018. \label{paper6}
    \end{listofpapers}
	
    




%
\chapter*{Related Work}
    \noindent The following ongoing project, although not included, is related to the contents of this thesis.\\
    
	\noindent \bibnamefont{E. Larsson, S. Milovanović, V. Shcherbakov, L. von Sydow, et al.\footnote{{\textrm{The authors listed here, ordered alphabetically, represent the management group.}}}} 
    	\emph{BENCHOP: The BENCHmarking Project in Option Pricing --- Basket Options.}  	  
    	manuscript in preparation, 2018. \label{paper7}	

\section*{Supervised Theses}
The following bachelor and master theses were supervised by the author.\\
	
	\noindent \bibnamefont{T. Sundvall and D. Trång.} 
    	\emph{Examination of Impact from Different Boundary Conditions on the 2D Black--Scholes Model.}  	  
    	Bachelor's thesis, Department of Engineering Sciences, Uppsala University, 2014,
	supervised by: \bibnamefont{L. von Sydow and S. Milovanović.}\\
	
	\noindent \bibnamefont{A. Abrahamsson and R. Pettersson.} 
    	\emph{Smoothing of Initial Conditions for High Order Approximations in Option Pricing.}  	  
    	Bachelor's thesis, Department of Engineering Sciences, Uppsala University, 2016,
	supervised by: \bibnamefont{L. von Sydow and S. Milovanović.}\\
	
	\noindent \bibnamefont{Robin Eriksson.} 
    	\emph{Stencil Study for RBF-FD in Option Pricing.}  	  
    	Bachelor's thesis, Department of Engineering Sciences, Uppsala University, 2016,
	supervised by: \bibnamefont{S. Milovanović and L. von Sydow.}\\

	\noindent \bibnamefont{Stephane Dumanois.} 
    	\emph{Least Squares Radial Basis Function generated Finite Differences for Option Pricing.}  	  
    	Master's thesis, Department of Mathematics, Uppsala University, 2016,
	supervised by: \bibnamefont{S. Milovanović and L. von Sydow.}\\	




%%%%%%%%%%%%%%%%%%%%%%%%%%%%%%%%%%%%%%%%%%%%%%%%%%%%%%%%%%%%%%%%%%
    \begingroup
        % To adjust the indentation in your table of contents, uncomment and enter the widest numbers for each level
        %  E.g.  \settocnumwidth{widest chapter number}{widest section number}{widest subsection number}...{...}
       %  \settocnumwidth{5}{4}{5}{3}{3}{3}
        \tableofcontents
    \endgroup

    % Optional tables
    %\listoftables
    %\listoffigures
\mainmatter
    % This includes the "Instruction", "Problem and Solutions" and "Example" files. After reading it, remove it from Thesis.tex.
%    \input{Example/Instruction.tex}
%    \input{Example/ProblemsAndSolutions}
%    \input{Example/Example.tex}

    % Include your chapters here.
%    \input{Introduction.tex}
%
%
%%%
%\par
%\noindent abcde­fghijklmnopqrstu­vwxyz­abcde­fghijklmnopqrstu­vwxyz­abcdabcde­fghijklmnopqrstu­vwxya\\
%bcde­fghijklmnopqrstu­vwxyz­abcde­fghijklmnopqrstu­vwxyz­abcdabcde­fghijklmnopqrstu­vwxyab\\%max=90
%cde­fghijklmnopqrstu­vwxyz­abcde­fghijklmnopq\\%min=45
%rstu­vwxyz­abcdabcde­fghijklmnopqrstu­vwxfdasdadadasdadaadadadaa%64 current
%\addtolength{\jot}{0.3em}





%
\chapter{Introduction}
\label{ch:introduction}

\par
The purpose of this thesis is to present state of the art in radial basis function generated finite difference (RBF-FD) methods for pricing of financial derivatives. Based on the six appended papers which are referred to by their Roman numerals, this doctoral work provides a detailed overview of RBF-FD properties and challenges that arise when the RBF-FD methods are used in financial applications. Moreover, with this dissertation, we aim to motivate further development of RBF-FD for solving multi-dimensional partial differential equations (PDEs) in finance.

\par
Across the financial markets of the world, financial derivatives such as futures, options, and others, are traded in substantial volumes. The value of all assets that underly outstanding derivatives transactions is several times larger than the gross world product (GWP). Financial derivatives are the most commonly used instruments when it comes to hedging risks, speculation based investing, and performing arbitrage. Therefore, knowing the prices of those financial instruments is of utmost importance at any given time. In order to make that possible in practice, it is often required to employ a set of skills incorporating knowledge in financial theory, engineering methods, mathematical tools, and programming practice --- which altogether constitute the field known as \emph{financial engineering}. 

\par
Many of the theoretical pricing models for financial derivatives can be represented using PDEs. In many cases, those equations are time-dependent, of high spatial dimension, and with challenging boundary conditions --- which most often makes them analytically unsolvable. In those cases, we need to utilize numerical approximation as a mean of estimating their solution. The fields of \emph{numerical analysis} and \emph{scientific computing} are concerned with obtaining approximate solutions while maintaining reasonable bounds on errors. Unfortunately, there is no universal numerical method which can be used to solve all problems of this type efficiently. In fact, there are tremendously many numerical methods for solving different types of differential equations, and all those methods are featured with their own limitations in performance, stability, and accuracy --- mostly dependent on details of the problems they aim to solve. Therefore, carefully selecting and developing numerical methods for particular applications has been the only way to build efficient PDE solvers in ongoing practice. 

\par
RBF-FD is a recent numerical method with potential to efficiently approximate solutions of PDEs in finance. Over the past years, besides the purely academic development and research of its numerical properties, the method has been mainly applied for simulations of atmospheric phenomena. As its name suggests, the RBF-FD method is of a finite difference type, from the radial basis function family. As a finite difference method, RBF-FD approximates differential equations by linear systems of algebraic equations, known as difference equations. Radial basis functions (RBFs) are used as interpolants that enable local approximations of differential operators that are necessary for constructing the difference equations. Constructed like that, the method is featured with a sparse matrix of the linear system of difference equations, and it is relatively simple to implement --- like the standard finite difference methods. Moreover, the method is mesh-free, meaning that it does not require a structured discretization of the computational domain which makes it equally easy to use in spaces of different dimensions, and it is of a customizable order of accuracy --- which are the features it inherits from the global radial basis function approximations. It is those properties that led us to recognize RBF-FD as a method with high potential for efficiently approximating the solutions of some analytically unsolvable and computationally challenging pricing problems in finance.

\par
Nevertheless, being a young method, RBF-FD is still under intense development, and we face many challenges when moving from simple theoretical cases toward more complex real-world applications. The core of this thesis deals with finding solutions for overcoming obstacles when financial derivatives are priced using RBF-FD to solve PDEs of multiple spatial dimensions. Thus, it represents a contribution to making the RBF-FD methods more reliable and efficient for use in financial applications. 

\par
The rest of this manuscript is organized as follows. We introduce and define financial derivatives in Chapter \ref{ch:finder}. An overview of some popular financial models and techniques for the pricing of options are presented in Chapter \ref{ch:optionpricing}. We present the features and properties of RBF-FD methods for solving PDEs in finance in Chapter \ref{ch:rbffd}. Finally, we conclude with some unsolved challenges and suggestions for further development of the RBF-FD method for financial applications in Chapter \ref{ch:outlook}.
%
%
%%%





%
\chapter{Financial Derivatives}
\label{ch:finder}

\par
A \emph{financial derivative} is a market instrument whose value depends on the values of some other underlying variables. Most often, those underlying variables are the prices of another traded asset (e.g., a stock underlying stock options), but they may as well be almost any variables of stochastic nature (e.g., air temperatures underlying weather derivatives). There are numerous financial derivatives in existence, available for almost every type of investment asset, ranging from agricultural grains to cryptocurrencies. Futures and options are best known as \emph{exchange-traded} derivatives, standardized to be bought and sold on derivatives exchanges (e.g., Chicago Mercantile Exchange for futures and Chicago Board Options Exchange for options). On the other hand, much larger volumes of financial derivatives are traded bilaterally \emph{over-the-counter} in a highly customizable fashion. That gave birth to many contracts with tailored properties such as forward contracts, swaps, exotic options, and other custom financial instruments.

\par
When it comes to traders, three categories can be readily identified: \emph{hedgers}, \emph{speculators}, and \emph{arbitrageurs}~\cite{hull2017options}. Hedgers use derivatives to reduce risks from potential future movements in a market variable, speculators use them to bet on the future outcome of a market variable, and arbitrageurs aim at making riskless profit by exploiting discrepancies in values of the same underlying variable traded under different derivatives or across different markets. Thanks to the traders, derivatives markets have been highly liquid over the past decades as many of the traders find trading derivatives more attractive compared to trading their underlying assets.

\par
Financial derivatives are traded in extremely large volumes across the planet. %According to statistics maintained by the Bank for International Settlements (BIS),% 
The estimated total notional value of these financial instruments has been above half a quadrillion of USD during the current decade~\cite{bank2018annual}. That is about an order of magnitude larger than GWP~\cite{worldgdp2018annual}. Moreover, derivatives markets have received significant criticism due to their role in the most recent global financial crisis.  As a result of the crisis, strict regulations in trading of derivatives have been introduced in order to increase transparency on the markets, improve market efficiency, and reduce systemic risk. Now, in the post-crisis period, methods for valuation of financial derivatives are still under the spotlight of financial institutions, as they look for the most efficient ways to solve mathematical problems stemming from the regulations. Those problems involve estimation of value adjustments (known as xVA) that keep different sources of counterparty risk under control. 

\par
In order to bring financial derivatives closer to the mathematical framework, it is useful for us to define several of their features. We assume that the contract representing a particular financial derivative is signed at time $t=t_0\equiv0$ and expires at $t=T$, where $T$ is also known as the time of \emph{maturity} of the contract. The contract is issued on the underlying stochastic variable $S(t)$. At the expiration of the contract, the holder receives payoff $g(S(T))$, which is equivalent to the value of the financial derivative at the time of maturity $T$, i.e., $u(S(T),T) = g(S(T))$. The value of the contract is represented by a function $u(t,S(t))$.

\par
When it comes to hierarchy of financial derivatives, we can see most of them either as a type of a forward/futures contract, or as a type of an option. Therefore, it is common to study forwards and futures as binding contracts ($-\infty < g(S(T)) < \infty\ $), and options as non-obligatory contracts towards their holders ($0\leq g(S(T))<\infty$\ ). In the following sections, we consider them in more detail.
%
%%





%
\section{Forwards and Futures}
\label{sec:futures}

\par
A \emph{forward} contract is an agreement between two parties signed at $t=t_0$ to buy or sell an underlying $S(t)$ at a certain future time $T$ for a certain price $K(t_0)=K_0$. The price $K(t)$ is called the \emph{forward price} of the contract, and it is determined at time $t_0$ in such a way that the value of the forward contract at the time of signing is equal to zero, i.e., $u(t_0,S(t_0))=u_0=0$. One of the parties in the contract takes a \emph{long} position and agrees to the payoff 
$$g_l(S(T))=S(T)-K_0.$$
The other party assumes a \emph{short} position and agrees to sell $S(t)$ at the same time $T$ for the stipulated forward price $K_0$, effectively obliging to the payoff 
$$g_s(S(T))=K_0-S(T).$$
Forward contracts are traded in over-the-counter markets and may be further customized according to the preferences of the signing parties.

\par
A \emph{futures} contract is an exchange-traded, and thus standardized financial derivative, that is very similar to a forward contract. It is in agreement signed at no cost between two parties at $t=t_0$ to buy or sell an underlying $S(t)$ at a certain time $T$. The principal difference from the forward contract lies in the way in which the payments are realized. Namely, at every point in time $t_0 \leq t \leq T$, there exists a price $K(t)$, now called the \emph{futures price} of the contract, that is quoted on the exchange. At time $T$, the long position holder of the contract is entitled to the payoff 
$$g_l(S(T))=S(T)-K(T),$$
while the short position holder gets 
$$g_s(S(T))=K(T)-S(T).$$
Moreover, during an arbitrary time interval $(t_i,t_j]$, where $t_0 \leq t_i < t_j < T$, the long holder of the contract receives the amount $K(t_j)-K(t_i)$, and the short holder receives $K(t_i)-K(t_j)$. The futures price $K(t)$ evolves in such way that obtaining the futures contract at any time $t_0 \leq t \leq T$ incurs a zero cost, i.e., $u(t,S(t))=0$. 

\par
As far as the pricing of forwards and futures is concerned, it is clear that these contracts are designed in such a way that their prices are equal to zero at the signing. Thus, computational problems of interest here are related to fairly determining the defined forward and futures prices. For more details on forwards and futures, it is wise to turn to~\cite{duffie1989futures,hull2017options}.  
%
%%





%
\section{Options}
\label{sec:options}

\par
An \emph{option} is a contract that gives its holder the right, but not the obligation to buy or sell an underlying $S(t)$ by a certain time of maturity $T$ for a certain price $K$. The price $K$ in the contract is known as the strike price. If the contract gives the buying right to its owner, then it is called a \emph{call} option, and if it gives the selling right, it is called a \emph{put} option. Call options are characterized with 
\begin{equation}
\label{eq:callop}
g_c(S(T))=\max(S(T)-K,\ 0),
\end{equation}
and put options with 
\begin{equation}
\label{eq:putop}
g_p(S(T))=\max(K-S(T),\ 0),
\end{equation}
as their respective payoff functions. Options that can be exercised at any time $t_0 < t \leq T$ are called \emph{American} options, and options that can be exercised only at time $t=T$ are known as \emph{European} options. Since options are traded both on exchanges and in over-the-counter markets, there are many more types of them (e.g., binary options, barrier options, Asian options, Bermudan options, and other exotic options) --- as the ways of customizing them are limitless. For instance, \emph{rainbow} options are defined in such a way that their payoffs may depend on more than one underlying asset, consequently requiring a multi-dimensional pricing model in order to estimate their value. An example of such a multi-asset derivative is an arithmetic European call basket option issued on $D$ underlying assets $S_1,\ldots,S_D$, whose payoff function is 
\begin{equation}
\label{eq:basketop}
g_{cb}(S_1(T),\ldots,S_D(T)) = \max\left(\frac{1}{D}\sum_{d=1}^D S_d(T) - K,\ 0\right).
\end{equation}
Moreover, for a given underlying $S$, there may be a large number of options with different dates of expiration $T$, and different strike prices $K$.

\par
Due to their versatility, options have been among the most popular financial derivatives on the financial markets, and many examples of their applications can be found in~\cite{hull2017options}.
%
%
%%%





%
\chapter{Option Pricing}
\label{ch:optionpricing}

\par
We emphasize that an option gives the right to the holder to do something, and that the holder does not need to use that right. That is the main difference between options and other financial derivatives. Whereas it costs nothing to buy a forward or futures contract, there is always a non-negative price for acquiring an option. That very detail is the cornerstone of one of the most involving fundamental problems in financial markets, known as \emph{option pricing}. Depending on the option characteristics, the pricing problem can be as trivial as deriving an analytical pricing formula --- such is the case for the standard European call option with certain market assumptions. Nevertheless, in many other cases of option valuation, we are faced against an eternal struggle of balancing between reasonable market assumptions for deriving delicate mathematical models and developing efficient numerical solvers that can estimate the solutions of the equations posed by those models.

\par
As the option gives stipulated rights, but not the obligations to their holder, it is natural to assume that this contract must have some objective non-negative value at any time. The central task of option pricing is to objectively determine the fair value of an option at any given time $t \leq T$. The fundamental mathematical framework for approaching this problem is the \emph{arbitrage theory}. In order to model option prices, the theory heavily relies on carefully argued assumptions about the market and mathematical ingredients such as martingale measures, stochastic differential equations (SDEs), It\^o calculus, Feynman--Kac representations, and PDEs. We refer to these topics throughout the manuscript in limited capacity, as the detailed definitions and proofs can be found elegantly presented in~\cite{bjork2009arbitrage}.
%
%%





%
\section{Market Models}
\label{sec:models}
\par
In order to be able to price an option, we need a set of assumptions that can be used to build a financial market model. The models range from the simple ones capturing a rough approximate picture of reality to extremely advanced ones aimed at capturing very fine details of the market. Once the model is defined, we can set up an option pricing problem that needs to be solved in order to estimate the option value. Difficulty of such pricing problems strongly depends on the complexity of the chosen market model as well as on the complexity of the specifics in the option contract that we want to price.  
%



%
\subsection{Black--Scholes--Merton Model}
\label{sub:bs}

\par
We start by considering a plain European option on a stock that does not pay dividends, under the famous Black--Scholes--Merton model~\cite{black73,merton73}. Creation of that model in 1973 is considered as one of the most successful quantitative breakthroughs in social sciences, initiating a pricing framework that still keeps occupied thousands of researchers across financial institutions and universities of the world. That was recognized by the Royal Swedish Academy of Sciences, when the \emph{Bank of Sweden Prize in Economic Sciences in Memory of Alfred Nobel} was awarded to Robert C. Merton and Myron S. Scholes in 1997, while Fischer S. Black was credited with equal contribution since he had passed away two years before the prize was awarded. 

\par
The main feature of the Black--Scholes--Merton model is that it allows the prices of European call and put options to be calculated analytically using parameters that are either directly observable on the market or can be easily estimated. The model is still widely used as a benchmark, although more advanced models have been developed over the years to take into account more realistic features of asset price dynamics, such as jumps and stochastic volatility. Being able to calculate prices of some options analytically makes estimation of the model parameters simple --- which is useful for calibration of more advanced models and pricing of more complex financial instruments.
\par
The model consists of two assets, a riskless bond $B(t)$ and a risky stock $S(t)$, with dynamics given by the following SDEs
\begin{align}
\dif B(t) &= r B(t) \dif t, \label{eq:bond} \\
\dif S(t) &= \mu S(t) + \sigma S(t) \dif W(t), \label{eq:stock}
\end{align}
where $r$ is the risk-neutral interest rate, $\mu$ is the drift coefficient, and $\sigma$ is the volatility of the stock --- all three being constant in the model. Moreover, $W(t)$ is the Wiener process.

\par
The Black--Scholes--Merton model stands on several important assumptions. The main assumption is that the considered financial market is arbitrage free, meaning that it is not possible to make positive earnings on the market without being exposed to risk. The next assumption states that the market is complete and efficient, which means that every contract on the market can be hedged and that the market prices fully reflect all available information. Those assumptions allow us to determine a unique price of the option whose payoff function is $g(S(T))$, using the following valuation under the risk-neutral measure $\mathbb{Q}$,
\begin{equation}
\label{eq:mc}
u(S(t), t)=\exp\left(-r(T-t)\right)\mathbb{E}^{\mathbb{Q}}\left[g(S(T))\right].
\end{equation}
That effectively means that the expected value is calculated on an adapted dynamics by using $r$ instead of $\mu$ as the drift constant of the stochastic process $S(t)$ defined in \eqref{eq:stock}. 

\par
Moreover, using the It\^o's lemma and the Feynman--Kac theorem, we can equivalently express the option price as the solution of the following PDE, known as the Black--Scholes--Merton equation
\begin{align}
%{\displaystyle{\frac{\partial}{\partial t} u(s,t) + r s \frac{\partial} {\partial s} u(s,t) + \frac{1}{2} s^2 \sigma^2 \frac{\partial^2}{\partial s^2} u(s,t) - r u(s,t)}} &=& 0,\\
\frac{\partial u}{\partial t} + r s \frac{\partial u} {\partial s} + \frac{1}{2} s^2 \sigma^2 \frac{\partial^2 u}{\partial s^2} - r u &= 0, \nonumber \\
u(s,T) &= g(s), \label{eq:bs}
\end{align}
where $s$ is the deterministic representation of the stochastic asset price $S$. Equation \eqref{eq:bs} is a parabolic PDE that has an analytical solution $u=u(s,t)$ in case of European call and put options. %In order to have a fully defined PDE problem, formulation in \eqref{eq:bs} requires appropriate boundary conditions. We omit stating the boundary conditions on purpose in this chapter for readability, and discuss them with particular problems examples in the subsequent chapters as they may vary from case to case. %Otherwise, the option price $u(s,t)$ can be numerically approximated by integration backward in time.

\par
In order to make better trading decisions, investors often look at the hedging parameters, which are also known as the \emph{greeks}. The most commonly used ones are \emph{delta} $\Delta = \frac{\partial u}{\partial s}$, \emph{gamma} $\Gamma = \frac{\partial^2 u}{\partial s^2}$, and \emph{vega} $\nu = \frac{\partial u}{\partial \sigma}$. As these hedging parameters represent risk sensitivities, being able to compute them is of great importance.

\par
We can use this basic framework to price financial derivatives with different payoffs or extend it in order to be able to valuate options with different underlying assets (e.g., stocks that pay discrete dividends). Also, we can further adapt the model to capture different market features more accurately. (e.g., introduce local volatility instead of the constant one). Moreover, it is sometimes beneficial to use the Merton model~\cite{merton1976option} to describe underlying assets with jumps. On the other hand, stochastic volatility models, such as the Heston model presented in Section \ref{sub:multifactor}, are useful when there are prominent volatility smiles in the underlying asset. To push things even further, it is not uncommon to have a stochastic volatility model with jumps --- the most known representative is the Bates model~\cite{bates1996jumps}. An overview of such extensions of the Black--Scholes--Merton framework can be seen in \textbf{Paper \ref{paper5}}.

\begin{figure}[H]
\centering
% This file was created by matlab2tikz.
%
%The latest updates can be retrieved from
%  http://www.mathworks.com/matlabcentral/fileexchange/22022-matlab2tikz-matlab2tikz
%where you can also make suggestions and rate matlab2tikz.
%
\definecolor{mycolor1}{rgb}{0.00000,0.44700,0.74100}%
\definecolor{mycolor2}{rgb}{0.85000,0.32500,0.09800}%
%
\begin{tikzpicture}[trim axis left, trim axis right, baseline]

\begin{axis}[%
width=6.028*0.13\textwidth,
height=4.651*0.13\textwidth,
at={(0\textwidth,0\textwidth)},
scale only axis,
xmin=0,
xmax=400,
ymin=0,
ymax=308.606881527401,
axis background/.style={fill=white},
xmajorgrids,
ymajorgrids,
xtick={0,100,400},
xticklabels={$0$,$K$,$4K$},
ytick={0,100,200,300,400},
yticklabels={$0$,$K$,$2K$,$3K$,$4K$},
xlabel={$s$},
ylabel={$u(s,t)$},
legend style={legend cell align=left, align=left, draw=white!15!black},
title={$r=0.03$, $\sigma=0.15$, $T=3$},
legend pos = north west,
]
\addplot [color=mycolor1,thick]
  table[row sep=crcr]{%
0	0\\
4.04040404040404	0\\
8.08080808080808	0\\
12.1212121212121	0\\
16.1616161616162	0\\
20.2020202020202	0\\
24.2424242424242	0\\
28.2828282828283	0\\
32.3232323232323	0\\
36.3636363636364	0\\
40.4040404040404	0\\
44.4444444444444	0\\
48.4848484848485	0\\
52.5252525252525	0\\
56.5656565656566	0\\
60.6060606060606	0\\
64.6464646464647	0\\
68.6868686868687	0\\
72.7272727272727	0\\
76.7676767676768	0\\
80.8080808080808	0\\
84.8484848484848	0\\
88.8888888888889	0\\
92.9292929292929	0\\
96.969696969697	0\\
100 0\\
101.010101010101	1.01010101010101\\
105.050505050505	5.05050505050505\\
109.090909090909	9.09090909090909\\
113.131313131313	13.1313131313131\\
117.171717171717	17.1717171717172\\
121.212121212121	21.2121212121212\\
125.252525252525	25.2525252525252\\
129.292929292929	29.2929292929293\\
133.333333333333	33.3333333333333\\
137.373737373737	37.3737373737374\\
141.414141414141	41.4141414141414\\
145.454545454545	45.4545454545455\\
149.49494949495	49.4949494949495\\
153.535353535354	53.5353535353535\\
157.575757575758	57.5757575757576\\
161.616161616162	61.6161616161616\\
165.656565656566	65.6565656565656\\
169.69696969697	69.6969696969697\\
173.737373737374	73.7373737373737\\
177.777777777778	77.7777777777778\\
181.818181818182	81.8181818181818\\
185.858585858586	85.8585858585859\\
189.89898989899	89.8989898989899\\
193.939393939394	93.9393939393939\\
197.979797979798	97.979797979798\\
202.020202020202	102.020202020202\\
206.060606060606	106.060606060606\\
210.10101010101	110.10101010101\\
214.141414141414	114.141414141414\\
218.181818181818	118.181818181818\\
222.222222222222	122.222222222222\\
226.262626262626	126.262626262626\\
230.30303030303	130.30303030303\\
234.343434343434	134.343434343434\\
238.383838383838	138.383838383838\\
242.424242424242	142.424242424242\\
246.464646464646	146.464646464646\\
250.50505050505	150.50505050505\\
254.545454545455	154.545454545455\\
258.585858585859	158.585858585859\\
262.626262626263	162.626262626263\\
266.666666666667	166.666666666667\\
270.707070707071	170.707070707071\\
274.747474747475	174.747474747475\\
278.787878787879	178.787878787879\\
282.828282828283	182.828282828283\\
286.868686868687	186.868686868687\\
290.909090909091	190.909090909091\\
294.949494949495	194.949494949495\\
298.989898989899	198.989898989899\\
303.030303030303	203.030303030303\\
307.070707070707	207.070707070707\\
311.111111111111	211.111111111111\\
315.151515151515	215.151515151515\\
319.191919191919	219.191919191919\\
323.232323232323	223.232323232323\\
327.272727272727	227.272727272727\\
331.313131313131	231.313131313131\\
335.353535353535	235.353535353535\\
339.393939393939	239.393939393939\\
343.434343434343	243.434343434343\\
347.474747474747	247.474747474747\\
351.515151515152	251.515151515152\\
355.555555555556	255.555555555556\\
359.59595959596	259.59595959596\\
363.636363636364	263.636363636364\\
367.676767676768	267.676767676768\\
371.717171717172	271.717171717172\\
375.757575757576	275.757575757576\\
379.79797979798	279.79797979798\\
383.838383838384	283.838383838384\\
387.878787878788	287.878787878788\\
391.919191919192	291.919191919192\\
395.959595959596	295.959595959596\\
400	300\\
};
\addlegendentry{$u(s,T)=g_c(s)$}

\addplot [color=mycolor2,thick]
  table[row sep=crcr]{%
0	0\\
4.04040404040404	0\\
8.08080808080808	0\\
12.1212121212121	4.06518274685513e-15\\
16.1616161616162	1.84110226260438e-11\\
20.2020202020202	5.6570893425188e-09\\
24.2424242424242	3.61754452061055e-07\\
28.2828282828283	8.47218861555263e-06\\
32.3232323232323	9.98647852595565e-05\\
36.3636363636364	0.00071966703651701\\
40.4040404040404	0.00359845105956065\\
44.4444444444444	0.0136101617332664\\
48.4848484848485	0.0413941343712883\\
52.5252525252525	0.105862922529326\\
56.5656565656566	0.235403119330865\\
60.6060606060606	0.466944235830593\\
64.6464646464647	0.842883472743563\\
68.6868686868687	1.40662006030599\\
72.7272727272727	2.19781847486442\\
76.7676767676768	3.2484302386748\\
80.8080808080808	4.58012154573924\\
84.8484848484848	6.203297849965\\
88.8888888888889	8.1175485478212\\
92.9292929292929	10.3131237730385\\
96.969696969697	12.7729952067817\\
101.010101010101	15.4750993178966\\
105.050505050505	18.3944625991648\\
109.090909090909	21.5050221822407\\
113.131313131313	24.7810555196482\\
117.171717171717	28.1982078353119\\
121.212121212121	31.7341540722855\\
125.252525252525	35.3689570104479\\
129.292929292929	39.0851911389302\\
133.333333333333	42.8678988009122\\
137.373737373737	46.7044360787997\\
141.414141414141	50.58425447158\\
145.454545454545	54.498652955167\\
149.49494949495	58.440524802846\\
153.535353535354	62.4041151394003\\
157.575757575758	66.3847987080162\\
161.616161616162	70.3788825821819\\
165.656565656566	74.3834352748955\\
169.69696969697	78.3961415734878\\
173.737373737374	82.4151811686068\\
177.777777777778	86.4391285016292\\
181.818181818182	90.4668710271427\\
185.858585858586	94.4975431252303\\
189.89898989899	98.5304730924492\\
193.939393939394	102.565140914428\\
197.979797979798	106.601144826313\\
202.020202020202	110.638174968188\\
206.060606060606	114.67599272272\\
210.10101010101	118.714414572371\\
214.141414141414	122.753299530226\\
218.181818181818	126.792539382121\\
222.222222222222	130.832051130643\\
226.262626262626	134.871771157106\\
230.30303030303	138.911650719532\\
234.343434343434	142.951652486606\\
238.383838383838	146.991747872971\\
242.424242424242	151.031914993026\\
246.464646464646	155.072137091239\\
250.50505050505	159.112401339001\\
254.545454545455	163.152697913036\\
258.585858585859	167.193019289856\\
262.626262626263	171.233359705835\\
266.666666666667	175.273714744121\\
270.707070707071	179.314081018674\\
274.747474747475	183.354455932582\\
278.787878787879	187.394837493224\\
282.828282828283	191.435224170864\\
286.868686868687	195.475614790484\\
290.909090909091	199.516008449002\\
294.949494949495	203.556404451913\\
298.989898989899	207.596802264768\\
303.030303030303	211.637201476017\\
307.070707070707	215.677601768527\\
311.111111111111	219.718002897743\\
315.151515151515	223.758404674944\\
319.191919191919	227.798806954384\\
323.232323232323	231.839209623421\\
327.272727272727	235.879612594926\\
331.313131313131	239.920015801456\\
335.353535353535	243.960419190761\\
339.393939393939	248.000822722328\\
343.434343434343	252.04122636472\\
347.474747474747	256.081630093523\\
351.515151515152	260.122033889758\\
355.555555555556	264.162437738662\\
359.59595959596	268.202841628739\\
363.636363636364	272.243245551032\\
367.676767676768	276.283649498554\\
371.717171717172	280.324053465851\\
375.757575757576	284.36445744866\\
379.79797979798	288.404861443651\\
383.838383838384	292.445265448215\\
387.878787878788	296.485669460309\\
391.919191919192	300.526073478331\\
395.959595959596	304.566477501024\\
400	308.606881527401\\
};
\addlegendentry{$u(s,t_0)$}

\end{axis}
\end{tikzpicture}%

\caption{\emph{An example of a European call option price $u(s,t)$ calculated using the Black--Scholes--Merton model, and its payoff function $g_c(s)$.}}
\label{fig:eucallpayoff}
\end{figure}

%
%
%\subsection{Multi-Asset Options}

\par
To price multi-asset financial derivatives, such as rainbow options issued on $D$ underlying assets $S_1,S_2,\ldots,S_D$, we consider a multi-dimensional analogue to \eqref{eq:bond} and \eqref{eq:stock}
\begin{align}
\dif B(t)&=rB(t)\dif t, \nonumber \\
\dif S_1(t)&=\mu_1 S_1(t)\dif t+\sigma_1 S_1(t)\dif W_1(t), \nonumber \\
%\dif S_2(t)&=&\mu_2 S_2(t)\dif t+\sigma_2 S_2(t)\dif W_2(t),\\
\vdots \nonumber \\ 
\dif S_D(t)&=\mu_D S_D(t)\dif t+\sigma_D S_D(t)\dif W_D(t), \label{eq:multi}
\end{align}
where the Wiener processes are correlated such that $\dif W_i(t)\dif W_j(t)=\rho_{ij}\dif t$. In this high-dimensional setting, an option with the payoff function $g(S_1(T),\ldots,S_D(T))$, can be priced by solving the corresponding high-dimensional Black--Scholes--Merton equation
%\begin{equation}
%\label{eq:mcD}
%u(S_1(t),\ldots,S_D(t),t)=\exp\left(-r(T-t)\right)\mathbb{E}^{{{\mathbb{Q}}}}_{t}[g(S_1(T),\ldots,S_D(T))],
%\end{equation}
%and the corresponding high-dimensional Black--Scholes--Merton equation reads as
%\begin{equation}\label{eq:bsD}
%\begin{array}{rcl}
%{\displaystyle{\frac{\partial u}{\partial t}+r\sum\limits_{i}^{D}s_i\frac{\partial u}{\partial{s_i}}+\frac{1}{2}\sum\limits_{i,j}^{D}\rho_{ij}\sigma_i\sigma_j s_is_j\frac{\partial^2u}{\partial s_i \partial s_j}-ru}}&=&0,\\
%{\displaystyle{u(s_1,\ldots,s_D,T)}}&=&g(s_1,\ldots,s_D).
%\end{array}
%\end{equation}
\begin{align}
\frac{\partial u}{\partial t}+\mathcal{L}_b u&=0, \nonumber \\
u(s_1,\ldots,s_D,T)&=g(s_1,\ldots,s_D), \label{eq:bsD}
\end{align}
where
\begin{equation}
\label{eqBSop}
\mathcal{L}_b u \equiv r\sum\limits_{i}^{D}s_i\frac{\partial u}{\partial{s_i}}+\frac{1}{2}\sum\limits_{i,j}^{D}\rho_{i,j}\sigma_i\sigma_j s_is_j\frac{\partial^2u}{\partial s_i \partial s_j}-ru.
\end{equation}
We observe \eqref{eq:bsD} as a time-dependent PDE with $D$ spatial dimensions.

\par
When it comes to American options, since these financial derivatives can be exercised at any $t \leq T$, as opposed to the European options (that can only be exercised at $t=T$), instead of using a PDE as a model, we formulate the pricing task as a linear complementarity problem (LCP)
\begin{align}
\frac{\partial u}{\partial t}+\mathcal{L}_b u&\geq 0,\nonumber \\
u(s_1,\ldots, s_D,t)&\geq g(s_1,\ldots, s_D), \nonumber \\
\left( \frac{\partial u}{\partial t}+\mathcal{L}_b u\right) \big(u(s_1,\ldots, s_D,t) &-g(s_1,\ldots, s_D)\big)=0, \nonumber \\ 
u(s_1,\ldots,s_D,T)&=g(s_1,\ldots,s_D). \label{eqlcp}
\end{align}
This formulation also applies to pricing of a single-asset American option by choosing $D=1$. 
%



%
\subsection{Multi-Factor Models}
\label{sub:multifactor}

\par
Another direction in development of pricing models is to include more stochastic factors. Models with multiple stochastic factors allow for better simulation of market features compared to the standard Black--Scholes--Merton formulation, which is known to fall short in capturing heavy tails of return distributions and volatility smiles. Therefore, various models with local volatilities, local stochastic volatilities, stochastic interest rates, and their combinations have been getting popular. In this section, we present two models with multiple stochastic factors that are used for pricing options.

\par
The attention to local volatility models started with~\cite{dupire1994pricing}. The first multi-factor model that we introduce is the Heston model~\cite{heston1993closed}, featured with a stochastic volatility. 

\par
The adapted dynamics for this model is as follows
\begin{align}
\dif S(t) & =  rS(t)\dif t + \sqrt{V(t)} S(t) \dif W_s(t), \label{qlsvSDE1} \\
\dif V(t) & =  \kappa\big(\eta-V(t)\big)\dif t + \sigma \sqrt{V(t)}\dif W_v(t), \label{qlsvSDE2}
\end{align}
where $V(t)$ is the stochastic volatility, $\sigma$ is the constant volatility of volatility, $\kappa$ is the speed of mean reversion of the volatility process, $\eta$ is the mean reversion level, $r$ is the risk-free interest rate, $W_s(t)$ and $W_v(t)$ are correlated Wiener processes with constant correlation $\rho$, i.e., $\dif W_s(t) \dif W_v(t) = \rho \dif t$. After using the It\^{o}'s lemma and the Feynman--Kac theorem, the PDE for the Heston model reads as
\begin{align}
\frac{\partial u}{\partial t}+\mathcal{L}_h u&=0, \nonumber \\
u(s,v,T) &= g(s), \label{hstPDE}
\end{align}
where
\begin{align}
\mathcal{L}_{h} u \equiv \frac{1}{2}vs^2\frac{\partial^2 u}{\partial s^2} &+ \rho\sigma v s \frac{\partial^2 u}{\partial s\partial v} + \frac{1}{2}\sigma^2v\frac{\partial^2 u}{\partial v^2} \nonumber \\ 
               &+ rs\frac{\partial u}{\partial s} + \kappa(\eta-v)\frac{\partial u}{\partial v} - ru, \label{eqHSTop}
\end{align}
$s$ and $v$ are deterministic representations of the stochastic asset price and volatility processes, respectively.
\par
When it comes to path dependent options, the Heston model has a great advantage over the Black--Scholes--Merton model and models with deterministic local volatility. There is clear evidence that in practice the volatility of asset prices is in itself random, and cannot be simply described as a function of time and underlying strike price~\cite{cont2001empirical}. 

\par
The Heston--Hull--White model~\cite{grzelak2011heston, grzelak2012extension}, is an enhancement of the Heston stochastic volatility model. The improvement consists of adding a stochastic interest rate that follows the Hull--White process~\cite{hull1990pricing}, as the interest rates on the market are not constant. The model is useful when pricing long-term derivatives in which we observe an implied volatility smile in the underlying asset. Another notable property of the Hull--White model is that the interest rates can be negative, as nowadays happens in some economies. 

\par
The adapted dynamics for this model is as follows
\begin{align}
\dif S(t) & =  R(t)S(t)\dif t + \sqrt{V(t)}S(t)\dif W_s(t), \label{hhwSDE1} \\
\dif V(t) & =  \kappa\big(\eta-V(t)\big)\dif t + \sigma_v \sqrt{V(t)}\dif W_v(t), \label{hhwSDE2} \\
\dif R(t) & = a\big(b-R(t)\big)\dif t + \sigma_r\dif W_r(t), {\color{white} \sqrt{V(t)}} \label{hhwSDE3}
\end{align}
where $R(t)$ is the stochastic interest rate, $a$ is the speed of mean reversion of the interest rate process, $b$ is its mean reversion level, $\sigma_r$ is its volatility,  $W_s(t)$, $W_v(t)$,  and $W_r(t)$ are correlated Wiener processes.

\par
We can apply the It\^{o}'s lemma and the Feynman--Kac theorem to derive the pricing PDE
\begin{align}
\frac{\partial u}{\partial t}+\mathcal{L}_w u&=0, \nonumber \\
u(s,v,r,T) &= g(s), \label{hhwPDE}
\end{align}
where
\begin{align}
\mathcal{L}_{w} u &\equiv  \frac{1}{2}vs^2\frac{\partial^2 u}{\partial s^2} + \frac{1}{2}\sigma_v^2v\frac{\partial^2 u}{\partial v^2}  + \frac{1}{2}\sigma_r^2\frac{\partial^2 u}{\partial r^2} + \nonumber \\
                             & \rho_{sv}\sigma_v vs\frac{\partial^2 u}{\partial s\partial v} + \rho_{sr}\sigma_r \sqrt{v} s\frac{\partial^2 u}{\partial s\partial r} + \rho_{vr}\sigma_v\sigma_r \sqrt{v}\frac{\partial^2 u}{\partial v\partial r} + \nonumber \\
                             & rs\frac{\partial u}{\partial s} + \kappa(\eta-v)\frac{\partial u}{\partial v} + a(b-r)\frac{\partial u}{\partial r} - ru, \label{eqHHWop}
\end{align}

\par
Here, it becomes clear how advanced models easily grow in complexity, which in turn makes it difficult to calibrate and valuate them in practice. Several other multi-factor models are discussed in more detail in \textbf{Papers \ref{paper4}} and \textbf{\ref{paper6}}. 
%
%%





%
\section{Pricing Methods}
\label{sec:methods}

\par
For a small number of cases, such as plain European call or put options under the Black--Scholes--Merton model, calculating the option price can be done by closed form solutions, derived using analytical methods. In some other cases, it is possible to approximate the solutions using semi-analytical schemes. Commonly used methods for pricing of financial derivatives in the absence of analytical or semi-analytical solutions can be split into three main groups: stochastic methods, methods based on the Fourier transform, and deterministic methods. Performance of these methods when pricing several option types across different market models is presented in \textbf{Papers \ref{paper5}} and \textbf{\ref{paper6}}.%



%



%
\subsection{Stochastic Methods}

\par
Stochastic methods, such as Monte Carlo (MC), aim at approximating option prices using the form showed in \eqref{eq:mc}. The idea of estimating expectations by repeated random sampling was used in different forms for centuries, but it was officially defined in~\cite{metropolis1949monte}. The first application of an MC method in option pricing was reported in 1977 for European options~\cite{boyle1977options}. A least square MC method for pricing American options was introduced in 2001~\cite{longstaff2001valuing}, and more recently, a new regression based MC method, named stochastic grid bundling method (SGBM), has been developed for efficient pricing of early-exercise options and their hedging parameters~\cite{jain2015stochastic}. Furthermore, quasi-MC~\cite{paskov1995faster} --- methods that use deterministic sequences of numbers to boost convergence --- became successful at efficiently tackling problems in hundreds of dimensions~\cite{dick2013high}. More recently, many advanced versions of MC methods have been developed, of which some of the most notable are multilevel MC methods~\cite{giles2008multilevel}, which are inspired by the multigrid ideas for the iterative solution of PDEs. % 
%These methods can greatly reduce the computational cost of classical MC by drawing many samples with a low accuracy at a low computational cost, and few samples at a high accuracy with a high cost. Those samples are then used to construct the final approximation of the solution.% 
Interestingly, in the time of publishing of this thesis, some pioneering approaches in development of quantum computing MC algorithms for pricing of financial derivatives have been made~\cite{rebentrost2018quantum}.

\par
Discrete models like binomial trees that appeared in 1979~\cite{cox1979option, rendleman1979two}, also fall in the group of stochastic methods. These models work by simulating stochastic trajectories of the underlying dynamics on predefined discrete lattices and are among the simplest nontrivial models of financial markets. 

\par
Stochastic methods are most suitable for multi-asset derivatives and multi-factor models --- both of which, as well as their combinations, result in problems of high dimensionality. The classical versions of these methods are arguably easy to implement and use. MC methods are significantly less efficient than other methods when used for problems in smaller dimensions, as their convergence rate is much slower in comparison.
%



%
\subsection{Fourier Methods}

\par
This group consists of methods based on the Fourier transform such as Carr--Madan fast Fourier transform method~\cite{carr1999option}. More recently, Fourier-cosine series expansions (COS) for European options~\cite{fang2008novel} and early-exercise options~\cite{fang2009pricing}, showed to be extremely efficient in pricing. In 2012, the COS method was extended to higher dimensions~\cite{ruijter2012two}. The methods from this category are very fast and accurate, but they typically require existence of the characteristic function for the price process of the underlying asset in closed form, or at least its approximation --- which is available for a fairly large class of the models, but not all. 
%



%
\subsection{Deterministic Methods}

\par
Deterministic methods are used to solve pricing problems in PDE form such as \eqref{eq:bs}, by discretizing its differential operators. 

\par
The main methods in this category are the finite differences (FD). The first time an FD method was used for pricing of a contract was in 1976~\cite{brennan1976pricing}, to solve a one-dimensional Black--Scholes--Merton equation. A few years later, FD schemes, together with MC methods, were established as a standard numerical approach for pricing financial derivatives when analytical solutions are not available~\cite{brennan1978finite}. Moreover, a notable operator splitting scheme was introduced in~\cite{ikonen2004operator}, enabling FD methods to price American options efficiently. Over the years, FD methods have been used to solve mostly one-dimensional and two-dimensional pricing problems. More recently, hierarchical approximation using sparse grids and asymptotic expansions~\cite{reisinger2007efficient, reisinger2015numerical} of high-dimensional option pricing problems have been developed --- enabling state of the art FD~\cite{lotstedt2007space, persson2010pricing, foulon2010adi, haentjens2012adi} to be used for pricing high-dimensional options by solving a sequence of lower dimensional problems~\cite{vonsydow2016pricing}. Apparently, many high-dimensional pricing problems have such a configuration of volatilities and correlations that their effective dimensionality is low, and as such can be represented by a small number of lower dimensional components~\cite{wang2005high}.

\par
Although they are used less often, the finite element methods can excel in certain cases~\cite{zvan1998general, forsyth1999finite, heinecke2012highly}, and the same applies for finite volumes~\cite{zvan2001finite}, which finds its use in convection dominated or degenerate cases.

\par
Finally, RBF methods are a more recent group of deterministic methods to be used for option pricing --- first time applied in 1999 for one-dimensional European options~\cite{hon1999radial}. Ever since, these methods have been becoming popular, as they possess potential to cope with PDEs of moderately high dimensions.
%



%
\subsection{Method Selection}

\par
Based on the presented details and the results reported in \textbf{Papers \ref{paper5}} and \textbf{\ref{paper6}}, a basic guide for selecting an appropriate option pricing method is first to check if it is possible to calculate the solution to the pricing problem analytically. In case that is not possible, the next best option is a Fourier transform based method. Deterministic methods come into play as robust numerical schemes when Fourier methods are not applicable. Nevertheless, they often suffer from the curse of dimensionality as the degrees of freedom in the resulting approximations grow exponentially with the dimensionality of the problem. Therefore, if the pricing problem is of a higher dimensionality that cannot be reduced, Monte Carlo methods are the most common alternative.

\par
Typically, deterministic methods are used to solve pricing problems of up to no more than three dimensions. In the following chapter, we present a localized RBF method that might become an alternative to Monte Carlo methods for moderately high-dimensional problems, i.e., of dimensionality three to five. 
%
%
%%%





%
\chapter{Radial Basis Function generated Finite Differences}
\label{ch:rbffd}

\par
Using the RBF methods for approximating solutions of PDEs dates back to the beginning of the nineties in the previous century~\cite{kansa1990multiquadrics1,kansa1990multiquadrics2}. Ever since, these methods have been used in different fields, including financial engineering~\cite{hon1999radial, fasshauer2004using,pettersson2008improved}. 

\par
In order to apply an RBF method, we observe option pricing problems on the truncated computational domain $\Omega\subset \mathbb{R}^{d}$ in the following PDE form
\begin{alignat}{2}
\frac{\partial}{\partial t}u(\underline{x},t) + \mathcal{L}u(\underline{x},t) &= 0, \quad &&\underline{x} \in \Omega \label{eqPDE} \\
{\color{white} \frac{\partial}{\partial t}} \mathcal{B}u(\underline{x},t) &= f(\underline{x},t), \quad &&\underline{x} \in \partial \Omega, \label{eqBC} \\
{\color{white} \frac{\partial}{\partial t}} u(\underline{x},T) &= g(\underline{x}), \quad &&\underline{x} \in \Omega, \label{eqTC}
\end{alignat}
where $u(\underline{x},t)$ is the option price, $\underline{x}$ is the spatial variable representing underlying assets and/or stochastic factors, with $\mathcal{L}$ as the differential operator of the pricing model; $\mathcal{B}$ is the boundary operator which together with the function $f(\underline{x},t)$ models the boundary conditions; initial data are defined by the terminal condition $g(\underline{x})$. 

\par
To construct a global RBF approximation in space, we scatter $N$ nodes $\underline{x}_j$, where $j=1,\ldots,N$, across the computational domain $\Omega$. Then, we consider an interpolant
\begin{equation}
\label{eq:RBFint}
	\tilde{u}(\underline{x},t) = \sum_{j=1}^N \lambda_j(t) \phi(\|\underline{x}-\underline{x}_j\|),
\end{equation}
where $\phi$ is the RBF, and $\lambda_j(t)$ are the time-dependent interpolation coefficients. At any time $t$, the value of the interpolant in every point $\underline{x}$ only depends on the distance to the nodes and this expression is valid for any number of dimensions. 

\par
Some examples of commonly used RBFs are listed in \textbf{Table \ref{tabrbf}}, split into two groups. The first group in the table consists of infinitely smooth RBFs that can provide spectral accuracy for interpolation and are featured with a shape parameter $\varepsilon$. The second kind is a piecewise smooth RBF that can give algebraic convergence for interpolation.%
\begin{table}[H]
%\label{tabrbf}
\begin{center}
\caption{\emph{Commonly used RBFs, where $\varepsilon\in \mathbb{R}^+$ is the shape parameter for the infinitely smooth RBFs, and $q\in\{2m-1,\ m \in \mathbb{N}\}$ is the degree of the polyharmonic spline as a piecewise smooth RBF.}}
\label{tabrbf}
\begin{tabular}{ l  c  c  c  r  }
%\hline\hline 
RBF & & &  & $\phi(r)$   \\ 
\hline
Gaussian (GA) &  & &  &  $\exp{(-\varepsilon^2r^2)}$ \\
Multiquadric (MQ) &  & &  & $\sqrt{1+\varepsilon^2r^2}$ \\
Inverse Multiquadric (IMQ) & & &  & $1/\sqrt{1+\varepsilon^2r^2}$ \\
Inverse Quadratic (IQ) & & &  & $1/(1+\varepsilon^2r^2)$ \\
\hlinewd{0.5pt}
Polyharmonic Spline (PHS) & & &  & $r^q$\\
\hline
\end{tabular}
\end{center}
\end{table}
\noindent In this thesis, we consider GA and PHS basis functions for approximating solutions of the pricing equations. Those two RBFs are shown plotted on a unit domain in \textbf{Figure \ref{fig:RBF}}.
\begin{figure}[H]
\centering
% This file was created by matlab2tikz.
%
%The latest updates can be retrieved from
%  http://www.mathworks.com/matlabcentral/fileexchange/22022-matlab2tikz-matlab2tikz
%where you can also make suggestions and rate matlab2tikz.
%
\rmfamily
\definecolor{mycolor1}{rgb}{0.00000,0.44700,0.74100}%
\definecolor{mycolor2}{rgb}{0.85000,0.32500,0.09800}%
\definecolor{mycolor3}{rgb}{0.92900,0.69400,0.12500}%
\definecolor{mycolor4}{rgb}{0.49400,0.18400,0.55600}%
\definecolor{mycolor5}{rgb}{0.46600,0.67400,0.18800}%
\definecolor{mycolor6}{rgb}{0.30100,0.74500,0.93300}%
%
\begin{tikzpicture}[trim axis left, trim axis right, baseline]

  \begin{axis}[
  grid=major,
  %tick label style = {font=\sansmath\sffamily},
  width=0.4\textwidth,
  height=0.4\textwidth,
  at={(0\textwidth,0\textwidth)},
  scale only axis,
  unbounded coords=jump,
  xmin=0,
  xmax=1,
  ymin=0,
  ymax=1,
  xlabel={$r$},
  ylabel={$\phi(r)$},
  axis background/.style={fill=white},
  %title style={font=\bfseries},
  title={GA},
  legend pos=north east,
  legend style={legend cell align=left,align=left,draw=white!15!black}
  ]
\addplot [color=mycolor1, style=dashed,semithick]
  table[row sep=crcr]{%
  0	1\\
  0.002002002002002	0.999995991996016\\
  0.004004004004004	0.999983968080449\\
  0.00600600600600601	0.999963928542447\\
  0.00800800800800801	0.999935873863912\\
  0.01001001001001	0.999899804719482\\
  0.012012012012012	0.999855721976499\\
  0.014014014014014	0.999803626694977\\
  0.016016016016016	0.999743520127562\\
  0.018018018018018	0.999675403719478\\
  0.02002002002002	0.99959927910847\\
  0.022022022022022	0.999515148124739\\
  0.024024024024024	0.99942301279087\\
  0.026026026026026	0.999322875321747\\
  0.028028028028028	0.999214738124469\\
  0.03003003003003	0.99909860379825\\
  0.032032032032032	0.998974475134316\\
  0.034034034034034	0.998842355115795\\
  0.036036036036036	0.998702246917593\\
  0.038038038038038	0.998554153906274\\
  0.04004004004004	0.998398079639916\\
  0.042042042042042	0.998234027867976\\
  0.044044044044044	0.998062002531139\\
  0.046046046046046	0.997882007761154\\
  0.048048048048048	0.997694047880678\\
  0.0500500500500501	0.997498127403093\\
  0.0520520520520521	0.997294251032336\\
  0.0540540540540541	0.997082423662702\\
  0.0560560560560561	0.996862650378651\\
  0.0580580580580581	0.996634936454606\\
  0.0600600600600601	0.996399287354742\\
  0.0620620620620621	0.996155708732764\\
  0.0640640640640641	0.995904206431688\\
  0.0660660660660661	0.995644786483597\\
  0.0680680680680681	0.995377455109409\\
  0.0700700700700701	0.995102218718627\\
  0.0720720720720721	0.994819083909076\\
  0.0740740740740741	0.994528057466649\\
  0.0760760760760761	0.994229146365028\\
  0.0780780780780781	0.993922357765412\\
  0.0800800800800801	0.993607699016224\\
  0.0820820820820821	0.993285177652825\\
  0.0840840840840841	0.992954801397209\\
  0.0860860860860861	0.992616578157693\\
  0.0880880880880881	0.992270516028608\\
  0.0900900900900901	0.99191662328997\\
  0.0920920920920921	0.991554908407152\\
  0.0940940940940941	0.991185380030548\\
  0.0960960960960961	0.990808046995225\\
  0.0980980980980981	0.990422918320575\\
  0.1001001001001	0.99003000320995\\
  0.102102102102102	0.989629311050304\\
  0.104104104104104	0.989220851411808\\
  0.106106106106106	0.988804634047481\\
  0.108108108108108	0.988380668892793\\
  0.11011011011011	0.987948966065272\\
  0.112112112112112	0.987509535864105\\
  0.114114114114114	0.987062388769724\\
  0.116116116116116	0.986607535443393\\
  0.118118118118118	0.98614498672678\\
  0.12012012012012	0.985674753641531\\
  0.122122122122122	0.985196847388831\\
  0.124124124124124	0.984711279348956\\
  0.126126126126126	0.984218061080826\\
  0.128128128128128	0.983717204321545\\
  0.13013013013013	0.983208720985933\\
  0.132132132132132	0.982692623166056\\
  0.134134134134134	0.982168923130746\\
  0.136136136136136	0.981637633325118\\
  0.138138138138138	0.981098766370071\\
  0.14014014014014	0.980552335061792\\
  0.142142142142142	0.979998352371252\\
  0.144144144144144	0.979436831443688\\
  0.146146146146146	0.978867785598085\\
  0.148148148148148	0.97829122832665\\
  0.15015015015015	0.977707173294281\\
  0.152152152152152	0.977115634338022\\
  0.154154154154154	0.976516625466521\\
  0.156156156156156	0.975910160859477\\
  0.158158158158158	0.975296254867076\\
  0.16016016016016	0.974674922009433\\
  0.162162162162162	0.974046176976012\\
  0.164164164164164	0.973410034625051\\
  0.166166166166166	0.972766509982977\\
  0.168168168168168	0.972115618243815\\
  0.17017017017017	0.971457374768587\\
  0.172172172172172	0.970791795084711\\
  0.174174174174174	0.970118894885389\\
  0.176176176176176	0.969438690028991\\
  0.178178178178178	0.968751196538434\\
  0.18018018018018	0.96805643060055\\
  0.182182182182182	0.96735440856545\\
  0.184184184184184	0.966645146945889\\
  0.186186186186186	0.965928662416612\\
  0.188188188188188	0.965204971813704\\
  0.19019019019019	0.964474092133932\\
  0.192192192192192	0.963736040534077\\
  0.194194194194194	0.962990834330263\\
  0.196196196196196	0.962238490997283\\
  0.198198198198198	0.961479028167913\\
  0.2002002002002	0.960712463632226\\
  0.202202202202202	0.959938815336897\\
  0.204204204204204	0.9591581013845\\
  0.206206206206206	0.958370340032806\\
  0.208208208208208	0.957575549694071\\
  0.21021021021021	0.956773748934317\\
  0.212212212212212	0.955964956472611\\
  0.214214214214214	0.955149191180338\\
  0.216216216216216	0.954326472080464\\
  0.218218218218218	0.953496818346799\\
  0.22022022022022	0.952660249303254\\
  0.222222222222222	0.951816784423089\\
  0.224224224224224	0.950966443328158\\
  0.226226226226226	0.95010924578815\\
  0.228228228228228	0.949245211719822\\
  0.23023023023023	0.948374361186227\\
  0.232232232232232	0.947496714395942\\
  0.234234234234234	0.946612291702283\\
  0.236236236236236	0.945721113602521\\
  0.238238238238238	0.944823200737087\\
  0.24024024024024	0.943918573888782\\
  0.242242242242242	0.943007253981969\\
  0.244244244244244	0.942089262081774\\
  0.246246246246246	0.941164619393267\\
  0.248248248248248	0.940233347260654\\
  0.25025025025025	0.939295467166448\\
  0.252252252252252	0.938351000730654\\
  0.254254254254254	0.93739996970993\\
  0.256256256256256	0.936442395996754\\
  0.258258258258258	0.935478301618592\\
  0.26026026026026	0.934507708737042\\
  0.262262262262262	0.933530639646998\\
  0.264264264264264	0.932547116775787\\
  0.266266266266266	0.93155716268232\\
  0.268268268268268	0.930560800056224\\
  0.27027027027027	0.92955805171698\\
  0.272272272272272	0.928548940613052\\
  0.274274274274274	0.927533489821011\\
  0.276276276276276	0.926511722544659\\
  0.278278278278278	0.925483662114144\\
  0.28028028028028	0.924449331985075\\
  0.282282282282282	0.923408755737629\\
  0.284284284284284	0.922361957075658\\
  0.286286286286286	0.92130895982579\\
  0.288288288288288	0.920249787936524\\
  0.29029029029029	0.919184465477328\\
  0.292292292292292	0.918113016637725\\
  0.294294294294294	0.917035465726377\\
  0.296296296296296	0.915951837170173\\
  0.298298298298298	0.914862155513303\\
  0.3003003003003	0.913766445416335\\
  0.302302302302302	0.912664731655285\\
  0.304304304304304	0.911557039120684\\
  0.306306306306306	0.910443392816647\\
  0.308308308308308	0.909323817859926\\
  0.31031031031031	0.908198339478977\\
  0.312312312312312	0.907066983013005\\
  0.314314314314314	0.905929773911021\\
  0.316316316316316	0.904786737730889\\
  0.318318318318318	0.903637900138368\\
  0.32032032032032	0.902483286906156\\
  0.322322322322322	0.901322923912926\\
  0.324324324324324	0.900156837142364\\
  0.326326326326326	0.898985052682199\\
  0.328328328328328	0.897807596723234\\
  0.33033033033033	0.896624495558373\\
  0.332332332332332	0.895435775581642\\
  0.334334334334334	0.894241463287214\\
  0.336336336336336	0.893041585268424\\
  0.338338338338338	0.891836168216785\\
  0.34034034034034	0.890625238921003\\
  0.342342342342342	0.889408824265987\\
  0.344344344344344	0.888186951231853\\
  0.346346346346346	0.886959646892937\\
  0.348348348348348	0.88572693841679\\
  0.35035035035035	0.884488853063185\\
  0.352352352352352	0.88324541818311\\
  0.354354354354354	0.88199666121777\\
  0.356356356356356	0.880742609697576\\
  0.358358358358358	0.879483291241139\\
  0.36036036036036	0.878218733554258\\
  0.362362362362362	0.876948964428911\\
  0.364364364364364	0.875674011742238\\
  0.366366366366366	0.874393903455524\\
  0.368368368368368	0.873108667613182\\
  0.37037037037037	0.871818332341735\\
  0.372372372372372	0.870522925848789\\
  0.374374374374374	0.869222476422014\\
  0.376376376376376	0.867917012428118\\
  0.378378378378378	0.866606562311816\\
  0.38038038038038	0.865291154594808\\
  0.382382382382382	0.863970817874743\\
  0.384384384384384	0.862645580824192\\
  0.386386386386386	0.861315472189611\\
  0.388388388388388	0.85998052079031\\
  0.39039039039039	0.858640755517417\\
  0.392392392392392	0.857296205332837\\
  0.394394394394394	0.855946899268219\\
  0.396396396396396	0.854592866423914\\
  0.398398398398398	0.853234135967937\\
  0.4004004004004	0.851870737134919\\
  0.402402402402402	0.850502699225073\\
  0.404404404404404	0.849130051603145\\
  0.406406406406406	0.847752823697371\\
  0.408408408408408	0.846371044998432\\
  0.41041041041041	0.844984745058407\\
  0.412412412412412	0.843593953489727\\
  0.414414414414414	0.842198699964128\\
  0.416416416416416	0.840799014211601\\
  0.418418418418418	0.839394926019345\\
  0.42042042042042	0.837986465230716\\
  0.422422422422422	0.836573661744177\\
  0.424424424424424	0.835156545512252\\
  0.426426426426426	0.833735146540469\\
  0.428428428428428	0.832309494886311\\
  0.43043043043043	0.830879620658169\\
  0.432432432432432	0.829445554014283\\
  0.434434434434434	0.828007325161697\\
  0.436436436436436	0.826564964355201\\
  0.438438438438438	0.825118501896286\\
  0.44044044044044	0.823667968132084\\
  0.442442442442442	0.822213393454322\\
  0.444444444444444	0.820754808298268\\
  0.446446446446446	0.819292243141678\\
  0.448448448448448	0.817825728503745\\
  0.45045045045045	0.81635529494405\\
  0.452452452452452	0.814880973061504\\
  0.454454454454454	0.813402793493306\\
  0.456456456456456	0.811920786913886\\
  0.458458458458458	0.810434984033856\\
  0.46046046046046	0.808945415598962\\
  0.462462462462462	0.807452112389032\\
  0.464464464464464	0.805955105216931\\
  0.466466466466466	0.804454424927511\\
  0.468468468468468	0.802950102396562\\
  0.47047047047047	0.801442168529768\\
  0.472472472472472	0.79993065426166\\
  0.474474474474474	0.79841559055457\\
  0.476476476476476	0.796897008397589\\
  0.478478478478478	0.795374938805521\\
  0.48048048048048	0.793849412817842\\
  0.482482482482482	0.792320461497659\\
  0.484484484484485	0.790788115930669\\
  0.486486486486487	0.789252407224118\\
  0.488488488488488	0.787713366505768\\
  0.49049049049049	0.786171024922853\\
  0.492492492492492	0.78462541364105\\
  0.494494494494495	0.783076563843438\\
  0.496496496496497	0.781524506729473\\
  0.498498498498498	0.779969273513948\\
  0.500500500500501	0.778410895425968\\
  0.502502502502503	0.776849403707919\\
  0.504504504504504	0.775284829614443\\
  0.506506506506507	0.773717204411409\\
  0.508508508508508	0.77214655937489\\
  0.510510510510511	0.770572925790143\\
  0.512512512512513	0.768996334950585\\
  0.514514514514514	0.767416818156776\\
  0.516516516516517	0.7658344067154\\
  0.518518518518518	0.764249131938253\\
  0.520520520520521	0.762661025141225\\
  0.522522522522523	0.761070117643292\\
  0.524524524524524	0.759476440765504\\
  0.526526526526527	0.757880025829981\\
  0.528528528528528	0.756280904158901\\
  0.530530530530531	0.754679107073505\\
  0.532532532532533	0.75307466589309\\
  0.534534534534535	0.75146761193401\\
  0.536536536536537	0.749857976508682\\
  0.538538538538539	0.74824579092459\\
  0.540540540540541	0.746631086483293\\
  0.542542542542543	0.745013894479435\\
  0.544544544544545	0.743394246199757\\
  0.546546546546547	0.741772172922114\\
  0.548548548548549	0.74014770591449\\
  0.550550550550551	0.73852087643402\\
  0.552552552552553	0.736891715726011\\
  0.554554554554555	0.735260255022971\\
  0.556556556556557	0.733626525543629\\
  0.558558558558559	0.731990558491975\\
  0.560560560560561	0.730352385056288\\
  0.562562562562563	0.728712036408173\\
  0.564564564564565	0.727069543701598\\
  0.566566566566567	0.725424938071941\\
  0.568568568568569	0.723778250635031\\
  0.570570570570571	0.722129512486194\\
  0.572572572572573	0.720478754699309\\
  0.574574574574575	0.718826008325857\\
  0.576576576576577	0.717171304393979\\
  0.578578578578579	0.715514673907537\\
  0.580580580580581	0.713856147845175\\
  0.582582582582583	0.712195757159386\\
  0.584584584584585	0.710533532775583\\
  0.586586586586587	0.708869505591169\\
  0.588588588588589	0.707203706474613\\
  0.590590590590591	0.705536166264531\\
  0.592592592592593	0.703866915768767\\
  0.594594594594595	0.702195985763479\\
  0.596596596596597	0.70052340699223\\
  0.598598598598599	0.698849210165079\\
  0.600600600600601	0.697173425957677\\
  0.602602602602603	0.695496085010367\\
  0.604604604604605	0.69381721792729\\
  0.606606606606607	0.692136855275489\\
  0.608608608608609	0.690455027584021\\
  0.610610610610611	0.68877176534307\\
  0.612612612612613	0.687087099003067\\
  0.614614614614615	0.685401058973809\\
  0.616616616616617	0.683713675623589\\
  0.618618618618619	0.68202497927832\\
  0.620620620620621	0.680335000220672\\
  0.622622622622623	0.678643768689208\\
  0.624624624624625	0.676951314877522\\
  0.626626626626627	0.67525766893339\\
  0.628628628628629	0.673562860957914\\
  0.630630630630631	0.671866921004675\\
  0.632632632632633	0.670169879078892\\
  0.634634634634635	0.668471765136582\\
  0.636636636636637	0.666772609083725\\
  0.638638638638639	0.665072440775433\\
  0.640640640640641	0.663371290015122\\
  0.642642642642643	0.661669186553692\\
  0.644644644644645	0.659966160088706\\
  0.646646646646647	0.658262240263578\\
  0.648648648648649	0.656557456666763\\
  0.650650650650651	0.654851838830949\\
  0.652652652652653	0.653145416232258\\
  0.654654654654655	0.651438218289449\\
  0.656656656656657	0.649730274363123\\
  0.658658658658659	0.648021613754938\\
  0.660660660660661	0.646312265706825\\
  0.662662662662663	0.644602259400205\\
  0.664664664664665	0.642891623955219\\
  0.666666666666667	0.641180388429955\\
  0.668668668668669	0.63946858181968\\
  0.670670670670671	0.637756233056085\\
  0.672672672672673	0.636043371006523\\
  0.674674674674675	0.634330024473259\\
  0.676676676676677	0.632616222192721\\
  0.678678678678679	0.63090199283476\\
  0.680680680680681	0.62918736500191\\
  0.682682682682683	0.627472367228652\\
  0.684684684684685	0.625757027980692\\
  0.686686686686687	0.624041375654231\\
  0.688688688688689	0.622325438575247\\
  0.690690690690691	0.620609244998783\\
  0.692692692692693	0.618892823108233\\
  0.694694694694695	0.617176201014643\\
  0.696696696696697	0.615459406756007\\
  0.698698698698699	0.613742468296569\\
  0.700700700700701	0.612025413526142\\
  0.702702702702703	0.610308270259412\\
  0.704704704704705	0.608591066235266\\
  0.706706706706707	0.606873829116112\\
  0.708708708708709	0.605156586487208\\
  0.710710710710711	0.603439365856001\\
  0.712712712712713	0.601722194651462\\
  0.714714714714715	0.60000510022343\\
  0.716716716716717	0.598288109841966\\
  0.718718718718719	0.596571250696703\\
  0.720720720720721	0.59485454989621\\
  0.722722722722723	0.593138034467352\\
  0.724724724724725	0.591421731354662\\
  0.726726726726727	0.589705667419717\\
  0.728728728728729	0.587989869440515\\
  0.730730730730731	0.586274364110865\\
  0.732732732732733	0.58455917803977\\
  0.734734734734735	0.58284433775083\\
  0.736736736736737	0.581129869681639\\
  0.738738738738739	0.57941580018319\\
  0.740740740740741	0.577702155519287\\
  0.742742742742743	0.575988961865964\\
  0.744744744744745	0.574276245310902\\
  0.746746746746747	0.572564031852858\\
  0.748748748748749	0.570852347401098\\
  0.750750750750751	0.569141217774833\\
  0.752752752752753	0.56743066870266\\
  0.754754754754755	0.565720725822013\\
  0.756756756756757	0.564011414678617\\
  0.758758758758759	0.562302760725941\\
  0.760760760760761	0.560594789324668\\
  0.762762762762763	0.558887525742158\\
  0.764764764764765	0.557180995151929\\
  0.766766766766767	0.555475222633133\\
  0.768768768768769	0.553770233170041\\
  0.770770770770771	0.552066051651537\\
  0.772772772772773	0.550362702870608\\
  0.774774774774775	0.548660211523853\\
  0.776776776776777	0.546958602210982\\
  0.778778778778779	0.545257899434335\\
  0.780780780780781	0.543558127598394\\
  0.782782782782783	0.54185931100931\\
  0.784784784784785	0.54016147387443\\
  0.786786786786787	0.538464640301829\\
  0.788788788788789	0.536768834299852\\
  0.790790790790791	0.535074079776658\\
  0.792792792792793	0.53338040053977\\
  0.794794794794795	0.53168782029563\\
  0.796796796796797	0.529996362649158\\
  0.798798798798799	0.528306051103326\\
  0.800800800800801	0.526616909058719\\
  0.802802802802803	0.524928959813123\\
  0.804804804804805	0.523242226561101\\
  0.806806806806807	0.521556732393583\\
  0.808808808808809	0.519872500297461\\
  0.810810810810811	0.518189553155188\\
  0.812812812812813	0.516507913744381\\
  0.814814814814815	0.514827604737433\\
  0.816816816816817	0.513148648701129\\
  0.818818818818819	0.511471068096266\\
  0.820820820820821	0.509794885277278\\
  0.822822822822823	0.508120122491872\\
  0.824824824824825	0.506446801880665\\
  0.826826826826827	0.504774945476822\\
  0.828828828828829	0.503104575205712\\
  0.830830830830831	0.501435712884555\\
  0.832832832832833	0.499768380222087\\
  0.834834834834835	0.498102598818223\\
  0.836836836836837	0.496438390163725\\
  0.838838838838839	0.494775775639882\\
  0.840840840840841	0.493114776518188\\
  0.842842842842843	0.491455413960032\\
  0.844844844844845	0.489797709016386\\
  0.846846846846847	0.488141682627506\\
  0.848848848848849	0.486487355622636\\
  0.850850850850851	0.484834748719712\\
  0.852852852852853	0.483183882525081\\
  0.854854854854855	0.481534777533219\\
  0.856856856856857	0.479887454126453\\
  0.858858858858859	0.478241932574695\\
  0.860860860860861	0.476598233035176\\
  0.862862862862863	0.474956375552188\\
  0.864864864864865	0.47331638005683\\
  0.866866866866867	0.471678266366761\\
  0.868868868868869	0.470042054185954\\
  0.870870870870871	0.468407763104465\\
  0.872872872872873	0.466775412598193\\
  0.874874874874875	0.465145022028661\\
  0.876876876876877	0.463516610642792\\
  0.878878878878879	0.461890197572691\\
  0.880880880880881	0.460265801835437\\
  0.882882882882883	0.458643442332879\\
  0.884884884884885	0.457023137851433\\
  0.886886886886887	0.455404907061891\\
  0.888888888888889	0.453788768519229\\
  0.890890890890891	0.452174740662425\\
  0.892892892892893	0.450562841814279\\
  0.894894894894895	0.448953090181242\\
  0.896896896896897	0.447345503853245\\
  0.898898898898899	0.445740100803539\\
  0.900900900900901	0.444136898888537\\
  0.902902902902903	0.442535915847661\\
  0.904904904904905	0.440937169303194\\
  0.906906906906907	0.43934067676014\\
  0.908908908908909	0.437746455606089\\
  0.910910910910911	0.436154523111081\\
  0.912912912912913	0.434564896427484\\
  0.914914914914915	0.43297759258987\\
  0.916916916916917	0.431392628514901\\
  0.918918918918919	0.429810021001218\\
  0.920920920920921	0.428229786729336\\
  0.922922922922923	0.426651942261541\\
  0.924924924924925	0.425076504041797\\
  0.926926926926927	0.423503488395654\\
  0.928928928928929	0.421932911530164\\
  0.930930930930931	0.420364789533802\\
  0.932932932932933	0.418799138376387\\
  0.934934934934935	0.417235973909016\\
  0.936936936936937	0.415675311863995\\
  0.938938938938939	0.414117167854784\\
  0.940940940940941	0.412561557375935\\
  0.942942942942943	0.411008495803049\\
  0.944944944944945	0.409457998392726\\
  0.946946946946947	0.407910080282524\\
  0.948948948948949	0.406364756490931\\
  0.950950950950951	0.404822041917323\\
  0.952952952952953	0.403281951341949\\
  0.954954954954955	0.401744499425904\\
  0.956956956956957	0.400209700711116\\
  0.958958958958959	0.398677569620332\\
  0.960960960960961	0.397148120457116\\
  0.962962962962963	0.395621367405844\\
  0.964964964964965	0.394097324531708\\
  0.966966966966967	0.392576005780725\\
  0.968968968968969	0.391057424979749\\
  0.970970970970971	0.38954159583649\\
  0.972972972972973	0.388028531939533\\
  0.974974974974975	0.386518246758369\\
  0.976976976976977	0.385010753643423\\
  0.978978978978979	0.383506065826092\\
  0.980980980980981	0.382004196418786\\
  0.982982982982983	0.380505158414973\\
  0.984984984984985	0.379008964689227\\
  0.986986986986987	0.377515627997286\\
  0.988988988988989	0.37602516097611\\
  0.990990990990991	0.374537576143943\\
  0.992992992992993	0.373052885900383\\
  0.994994994994995	0.371571102526453\\
  0.996996996996997	0.370092238184678\\
  0.998998998998999	0.36861630491917\\
  1.001001001001	0.367143314655709\\
};
\addlegendentry{$\varepsilon=1$}

\addplot [color=mycolor2, style=semithick]
  table[row sep=crcr]{%
  0	1\\
  0.002002002002002	0.999983968080449\\
  0.004004004004004	0.999935873863912\\
  0.00600600600600601	0.999855721976499\\
  0.00800800800800801	0.999743520127562\\
  0.01001001001001	0.99959927910847\\
  0.012012012012012	0.99942301279087\\
  0.014014014014014	0.999214738124469\\
  0.016016016016016	0.998974475134316\\
  0.018018018018018	0.998702246917593\\
  0.02002002002002	0.998398079639916\\
  0.022022022022022	0.998062002531139\\
  0.024024024024024	0.997694047880678\\
  0.026026026026026	0.997294251032336\\
  0.028028028028028	0.996862650378651\\
  0.03003003003003	0.996399287354742\\
  0.032032032032032	0.995904206431688\\
  0.034034034034034	0.995377455109409\\
  0.036036036036036	0.994819083909076\\
  0.038038038038038	0.994229146365028\\
  0.04004004004004	0.993607699016224\\
  0.042042042042042	0.992954801397209\\
  0.044044044044044	0.992270516028608\\
  0.046046046046046	0.991554908407152\\
  0.048048048048048	0.990808046995225\\
  0.0500500500500501	0.99003000320995\\
  0.0520520520520521	0.989220851411808\\
  0.0540540540540541	0.988380668892793\\
  0.0560560560560561	0.987509535864105\\
  0.0580580580580581	0.986607535443393\\
  0.0600600600600601	0.985674753641531\\
  0.0620620620620621	0.984711279348956\\
  0.0640640640640641	0.983717204321545\\
  0.0660660660660661	0.982692623166056\\
  0.0680680680680681	0.981637633325118\\
  0.0700700700700701	0.980552335061792\\
  0.0720720720720721	0.979436831443688\\
  0.0740740740740741	0.97829122832665\\
  0.0760760760760761	0.977115634338022\\
  0.0780780780780781	0.975910160859477\\
  0.0800800800800801	0.974674922009433\\
  0.0820820820820821	0.973410034625051\\
  0.0840840840840841	0.972115618243815\\
  0.0860860860860861	0.970791795084711\\
  0.0880880880880881	0.969438690028991\\
  0.0900900900900901	0.96805643060055\\
  0.0920920920920921	0.966645146945889\\
  0.0940940940940941	0.965204971813704\\
  0.0960960960960961	0.963736040534077\\
  0.0980980980980981	0.962238490997283\\
  0.1001001001001	0.960712463632226\\
  0.102102102102102	0.9591581013845\\
  0.104104104104104	0.957575549694071\\
  0.106106106106106	0.955964956472611\\
  0.108108108108108	0.954326472080464\\
  0.11011011011011	0.952660249303254\\
  0.112112112112112	0.950966443328158\\
  0.114114114114114	0.949245211719822\\
  0.116116116116116	0.947496714395942\\
  0.118118118118118	0.945721113602521\\
  0.12012012012012	0.943918573888782\\
  0.122122122122122	0.942089262081774\\
  0.124124124124124	0.940233347260654\\
  0.126126126126126	0.938351000730654\\
  0.128128128128128	0.936442395996754\\
  0.13013013013013	0.934507708737042\\
  0.132132132132132	0.932547116775787\\
  0.134134134134134	0.930560800056224\\
  0.136136136136136	0.928548940613052\\
  0.138138138138138	0.926511722544659\\
  0.14014014014014	0.924449331985075\\
  0.142142142142142	0.922361957075658\\
  0.144144144144144	0.920249787936524\\
  0.146146146146146	0.918113016637725\\
  0.148148148148148	0.915951837170173\\
  0.15015015015015	0.913766445416335\\
  0.152152152152152	0.911557039120684\\
  0.154154154154154	0.909323817859926\\
  0.156156156156156	0.907066983013005\\
  0.158158158158158	0.904786737730889\\
  0.16016016016016	0.902483286906156\\
  0.162162162162162	0.900156837142364\\
  0.164164164164164	0.897807596723234\\
  0.166166166166166	0.895435775581642\\
  0.168168168168168	0.893041585268424\\
  0.17017017017017	0.890625238921003\\
  0.172172172172172	0.888186951231853\\
  0.174174174174174	0.88572693841679\\
  0.176176176176176	0.88324541818311\\
  0.178178178178178	0.880742609697576\\
  0.18018018018018	0.878218733554258\\
  0.182182182182182	0.875674011742238\\
  0.184184184184184	0.873108667613182\\
  0.186186186186186	0.870522925848789\\
  0.188188188188188	0.867917012428118\\
  0.19019019019019	0.865291154594808\\
  0.192192192192192	0.862645580824192\\
  0.194194194194194	0.85998052079031\\
  0.196196196196196	0.857296205332837\\
  0.198198198198198	0.854592866423914\\
  0.2002002002002	0.851870737134919\\
  0.202202202202202	0.849130051603145\\
  0.204204204204204	0.846371044998432\\
  0.206206206206206	0.843593953489727\\
  0.208208208208208	0.840799014211601\\
  0.21021021021021	0.837986465230716\\
  0.212212212212212	0.835156545512252\\
  0.214214214214214	0.832309494886311\\
  0.216216216216216	0.829445554014283\\
  0.218218218218218	0.826564964355201\\
  0.22022022022022	0.823667968132084\\
  0.222222222222222	0.820754808298268\\
  0.224224224224224	0.817825728503745\\
  0.226226226226226	0.814880973061504\\
  0.228228228228228	0.811920786913886\\
  0.23023023023023	0.808945415598962\\
  0.232232232232232	0.805955105216931\\
  0.234234234234234	0.802950102396562\\
  0.236236236236236	0.79993065426166\\
  0.238238238238238	0.796897008397589\\
  0.24024024024024	0.793849412817842\\
  0.242242242242242	0.790788115930669\\
  0.244244244244244	0.787713366505768\\
  0.246246246246246	0.78462541364105\\
  0.248248248248248	0.781524506729473\\
  0.25025025025025	0.778410895425968\\
  0.252252252252252	0.775284829614443\\
  0.254254254254254	0.77214655937489\\
  0.256256256256256	0.768996334950585\\
  0.258258258258258	0.7658344067154\\
  0.26026026026026	0.762661025141225\\
  0.262262262262262	0.759476440765504\\
  0.264264264264264	0.756280904158901\\
  0.266266266266266	0.75307466589309\\
  0.268268268268268	0.749857976508682\\
  0.27027027027027	0.746631086483293\\
  0.272272272272272	0.743394246199757\\
  0.274274274274274	0.74014770591449\\
  0.276276276276276	0.736891715726011\\
  0.278278278278278	0.733626525543629\\
  0.28028028028028	0.730352385056288\\
  0.282282282282282	0.727069543701598\\
  0.284284284284284	0.723778250635031\\
  0.286286286286286	0.720478754699309\\
  0.288288288288288	0.717171304393979\\
  0.29029029029029	0.713856147845175\\
  0.292292292292292	0.710533532775583\\
  0.294294294294294	0.707203706474613\\
  0.296296296296296	0.703866915768767\\
  0.298298298298298	0.70052340699223\\
  0.3003003003003	0.697173425957677\\
  0.302302302302302	0.69381721792729\\
  0.304304304304304	0.690455027584021\\
  0.306306306306306	0.687087099003067\\
  0.308308308308308	0.683713675623589\\
  0.31031031031031	0.680335000220672\\
  0.312312312312312	0.676951314877522\\
  0.314314314314314	0.673562860957914\\
  0.316316316316316	0.670169879078892\\
  0.318318318318318	0.666772609083725\\
  0.32032032032032	0.663371290015122\\
  0.322322322322322	0.659966160088706\\
  0.324324324324324	0.656557456666763\\
  0.326326326326326	0.653145416232258\\
  0.328328328328328	0.649730274363123\\
  0.33033033033033	0.646312265706825\\
  0.332332332332332	0.642891623955219\\
  0.334334334334334	0.63946858181968\\
  0.336336336336336	0.636043371006523\\
  0.338338338338338	0.632616222192721\\
  0.34034034034034	0.62918736500191\\
  0.342342342342342	0.625757027980692\\
  0.344344344344344	0.622325438575247\\
  0.346346346346346	0.618892823108233\\
  0.348348348348348	0.615459406756007\\
  0.35035035035035	0.612025413526142\\
  0.352352352352352	0.608591066235266\\
  0.354354354354354	0.605156586487208\\
  0.356356356356356	0.601722194651462\\
  0.358358358358358	0.598288109841966\\
  0.36036036036036	0.59485454989621\\
  0.362362362362362	0.591421731354662\\
  0.364364364364364	0.587989869440515\\
  0.366366366366366	0.58455917803977\\
  0.368368368368368	0.581129869681639\\
  0.37037037037037	0.577702155519287\\
  0.372372372372372	0.574276245310902\\
  0.374374374374374	0.570852347401098\\
  0.376376376376376	0.56743066870266\\
  0.378378378378378	0.564011414678617\\
  0.38038038038038	0.560594789324668\\
  0.382382382382382	0.557180995151929\\
  0.384384384384384	0.553770233170041\\
  0.386386386386386	0.550362702870608\\
  0.388388388388388	0.546958602210982\\
  0.39039039039039	0.543558127598394\\
  0.392392392392392	0.54016147387443\\
  0.394394394394394	0.536768834299852\\
  0.396396396396396	0.53338040053977\\
  0.398398398398398	0.529996362649158\\
  0.4004004004004	0.526616909058719\\
  0.402402402402402	0.523242226561101\\
  0.404404404404404	0.519872500297461\\
  0.406406406406406	0.516507913744381\\
  0.408408408408408	0.513148648701129\\
  0.41041041041041	0.509794885277278\\
  0.412412412412412	0.506446801880665\\
  0.414414414414414	0.503104575205712\\
  0.416416416416416	0.499768380222087\\
  0.418418418418418	0.496438390163725\\
  0.42042042042042	0.493114776518188\\
  0.422422422422422	0.489797709016386\\
  0.424424424424424	0.486487355622636\\
  0.426426426426426	0.483183882525081\\
  0.428428428428428	0.479887454126453\\
  0.43043043043043	0.476598233035176\\
  0.432432432432432	0.47331638005683\\
  0.434434434434434	0.470042054185954\\
  0.436436436436436	0.466775412598193\\
  0.438438438438438	0.463516610642792\\
  0.44044044044044	0.460265801835437\\
  0.442442442442442	0.457023137851433\\
  0.444444444444444	0.453788768519229\\
  0.446446446446446	0.450562841814279\\
  0.448448448448448	0.447345503853245\\
  0.45045045045045	0.444136898888537\\
  0.452452452452452	0.440937169303194\\
  0.454454454454454	0.437746455606089\\
  0.456456456456456	0.434564896427484\\
  0.458458458458458	0.431392628514901\\
  0.46046046046046	0.428229786729336\\
  0.462462462462462	0.425076504041797\\
  0.464464464464464	0.421932911530164\\
  0.466466466466466	0.418799138376387\\
  0.468468468468468	0.415675311863995\\
  0.47047047047047	0.412561557375935\\
  0.472472472472472	0.409457998392726\\
  0.474474474474474	0.406364756490931\\
  0.476476476476476	0.403281951341949\\
  0.478478478478478	0.400209700711116\\
  0.48048048048048	0.397148120457116\\
  0.482482482482482	0.394097324531708\\
  0.484484484484485	0.391057424979749\\
  0.486486486486487	0.388028531939533\\
  0.488488488488488	0.385010753643423\\
  0.49049049049049	0.382004196418786\\
  0.492492492492492	0.379008964689227\\
  0.494494494494495	0.37602516097611\\
  0.496496496496497	0.373052885900383\\
  0.498498498498498	0.370092238184678\\
  0.500500500500501	0.367143314655709\\
  0.502502502502503	0.364206210246949\\
  0.504504504504504	0.361281018001586\\
  0.506506506506507	0.358367829075761\\
  0.508508508508508	0.355466732742081\\
  0.510510510510511	0.352577816393403\\
  0.512512512512513	0.349701165546886\\
  0.514514514514514	0.346836863848313\\
  0.516516516516517	0.343984993076677\\
  0.518518518518518	0.341145633149023\\
  0.520520520520521	0.338318862125551\\
  0.522522522522523	0.335504756214974\\
  0.524524524524524	0.332703389780123\\
  0.526526526526527	0.329914835343807\\
  0.528528528528528	0.327139163594914\\
  0.530530530530531	0.324376443394747\\
  0.532532532532533	0.321626741783615\\
  0.534534534534535	0.318890123987644\\
  0.536536536536537	0.316166653425825\\
  0.538538538538539	0.313456391717294\\
  0.540540540540541	0.310759398688832\\
  0.542542542542543	0.308075732382591\\
  0.544544544544545	0.305405449064036\\
  0.546546546546547	0.302748603230103\\
  0.548548548548549	0.300105247617569\\
  0.550550550550551	0.29747543321163\\
  0.552552552552553	0.294859209254682\\
  0.554554554554555	0.292256623255308\\
  0.556556556556557	0.28966772099746\\
  0.558558558558559	0.287092546549832\\
  0.560560560560561	0.284531142275429\\
  0.562562562562563	0.281983548841325\\
  0.564564564564565	0.279449805228594\\
  0.566566566566567	0.276929948742437\\
  0.568568568568569	0.274424015022471\\
  0.570570570570571	0.2719320380532\\
  0.572572572572573	0.269454050174652\\
  0.574574574574575	0.266990082093185\\
  0.576576576576577	0.264540162892454\\
  0.578578578578579	0.26210432004453\\
  0.580580580580581	0.259682579421189\\
  0.582582582582583	0.25727496530534\\
  0.584584584584585	0.254881500402606\\
  0.586586586586587	0.252502205853047\\
  0.588588588588589	0.250137101243029\\
  0.590590590590591	0.247786204617227\\
  0.592592592592593	0.245449532490759\\
  0.594594594594595	0.243127099861455\\
  0.596596596596597	0.240818920222253\\
  0.598598598598599	0.23852500557371\\
  0.600600600600601	0.236245366436644\\
  0.602602602602603	0.233980011864883\\
  0.604604604604605	0.231728949458131\\
  0.606606606606607	0.22949218537494\\
  0.608608608608609	0.227269724345791\\
  0.610610610610611	0.225061569686272\\
  0.612612612612613	0.222867723310358\\
  0.614614614614615	0.220688185743786\\
  0.616616616616617	0.218522956137517\\
  0.618618618618619	0.216372032281293\\
  0.620620620620621	0.214235410617268\\
  0.622622622622623	0.212113086253734\\
  0.624624624624625	0.210005052978911\\
  0.626626626626627	0.207911303274817\\
  0.628628628628629	0.205831828331212\\
  0.630630630630631	0.203766618059605\\
  0.632632632632633	0.201715661107324\\
  0.634634634634635	0.199678944871649\\
  0.636636636636637	0.197656455514005\\
  0.638638638638639	0.195648177974202\\
  0.640640640640641	0.193654095984734\\
  0.642642642642643	0.191674192085119\\
  0.644644644644645	0.189708447636285\\
  0.646646646646647	0.187756842835001\\
  0.648648648648649	0.185819356728341\\
  0.650650650650651	0.183895967228184\\
  0.652652652652653	0.181986651125748\\
  0.654654654654655	0.180091384106149\\
  0.656656656656657	0.178210140762991\\
  0.658658658658659	0.176342894612968\\
  0.660660660660661	0.1744896181105\\
  0.662662662662663	0.172650282662372\\
  0.664664664664665	0.170824858642393\\
  0.666666666666667	0.169013315406066\\
  0.668668668668669	0.167215621305262\\
  0.670670670670671	0.1654317437029\\
  0.672672672672673	0.163661648987634\\
  0.674674674674675	0.161905302588527\\
  0.676676676676677	0.160162668989733\\
  0.678678678678679	0.158433711745169\\
  0.680680680680681	0.156718393493172\\
  0.682682682682683	0.155016675971151\\
  0.684684684684685	0.153328520030222\\
  0.686686686686687	0.151653885649826\\
  0.688688688688689	0.149992731952322\\
  0.690690690690691	0.148345017217569\\
  0.692692692692693	0.146710698897467\\
  0.694694694694695	0.145089733630489\\
  0.696696696696697	0.143482077256163\\
  0.698698698698699	0.14188768482954\\
  0.700700700700701	0.140306510635613\\
  0.702702702702703	0.138738508203708\\
  0.704704704704705	0.137183630321828\\
  0.706706706706707	0.135641829050962\\
  0.708708708708709	0.134113055739347\\
  0.710710710710711	0.132597261036679\\
  0.712712712712713	0.131094394908284\\
  0.714714714714715	0.12960440664923\\
  0.716716716716717	0.128127244898394\\
  0.718718718718719	0.126662857652466\\
  0.720720720720721	0.1252111922799\\
  0.722722722722723	0.12377219553481\\
  0.724724724724725	0.122345813570795\\
  0.726726726726727	0.120931991954709\\
  0.728728728728729	0.119530675680364\\
  0.730730730730731	0.118141809182165\\
  0.732732732732733	0.11676533634868\\
  0.734734734734735	0.115401200536131\\
  0.736736736736737	0.114049344581825\\
  0.738738738738739	0.112709710817499\\
  0.740740740740741	0.111382241082602\\
  0.742742742742743	0.110066876737484\\
  0.744744744744745	0.108763558676524\\
  0.746746746746747	0.10747222734116\\
  0.748748748748749	0.106192822732854\\
  0.750750750750751	0.104925284425957\\
  0.752752752752753	0.103669551580503\\
  0.754754754754755	0.102425562954908\\
  0.756756756756757	0.101193256918584\\
  0.758758758758759	0.0999725714644625\\
  0.760760760760761	0.0987634442214314\\
  0.762762762762763	0.0975658124666755\\
  0.764764764764765	0.0963796131379261\\
  0.766766766766767	0.0952047828456159\\
  0.768768768768769	0.0940412578849393\\
  0.770770770770771	0.0928889742478148\\
  0.772772772772773	0.0917478676347503\\
  0.774774774774775	0.0906178734666105\\
  0.776776776776777	0.0894989268962829\\
  0.778778778778779	0.0883909628202439\\
  0.780780780780781	0.0872939158900236\\
  0.782782782782783	0.0862077205235662\\
  0.784784784784785	0.0851323109164892\\
  0.786786786786787	0.0840676210532357\\
  0.788788788788789	0.0830135847181231\\
  0.790790790790791	0.0819701355062847\\
  0.792792792792793	0.0809372068345051\\
  0.794794794794795	0.0799147319519477\\
  0.796796796796797	0.0789026439507732\\
  0.798798798798799	0.0779008757766512\\
  0.800800800800801	0.0769093602391591\\
  0.802802802802803	0.075928030022074\\
  0.804804804804805	0.0749568176935514\\
  0.806806806806807	0.0739956557161938\\
  0.808808808808809	0.0730444764570076\\
  0.810810810810811	0.0721032121972465\\
  0.812812812812813	0.0711717951421438\\
  0.814814814814815	0.0702501574305303\\
  0.816816816816817	0.0693382311443393\\
  0.818818818818819	0.0684359483179977\\
  0.820820820820821	0.0675432409477025\\
  0.822822822822823	0.0666600410005835\\
  0.824824824824825	0.0657862804237503\\
  0.826826826826827	0.0649218911532256\\
  0.828828828828829	0.064066805122762\\
  0.830830830830831	0.0632209542725446\\
  0.832832832832833	0.062384270557777\\
  0.834834834834835	0.0615566859571527\\
  0.836836836836837	0.0607381324812103\\
  0.838838838838839	0.0599285421805725\\
  0.840840840840841	0.0591278471540705\\
  0.842842842842843	0.0583359795567512\\
  0.844844844844845	0.0575528716077691\\
  0.846846846846847	0.0567784555981634\\
  0.848848848848849	0.0560126638985176\\
  0.850850850850851	0.0552554289665045\\
  0.852852852852853	0.0545066833543163\\
  0.854854854854855	0.0537663597159776\\
  0.856856856856857	0.0530343908145451\\
  0.858858858858859	0.052310709529191\\
  0.860860860860861	0.0515952488621715\\
  0.862862862862863	0.0508879419456822\\
  0.864864864864865	0.0501887220485974\\
  0.866866866866867	0.0494975225830964\\
  0.868868868868869	0.0488142771111754\\
  0.870870870870871	0.0481389193510471\\
  0.872872872872873	0.0474713831834255\\
  0.874874874874875	0.0468116026576997\\
  0.876876876876877	0.0461595119979938\\
  0.878878878878879	0.0455150456091162\\
  0.880880880880881	0.0448781380823965\\
  0.882882882882883	0.0442487242014113\\
  0.884884884884885	0.0436267389476001\\
  0.886886886886887	0.0430121175057703\\
  0.888888888888889	0.0424047952694926\\
  0.890890890890891	0.0418047078463877\\
  0.892892892892893	0.0412117910633039\\
  0.894894894894895	0.0406259809713871\\
  0.896896896896897	0.0400472138510424\\
  0.898898898898899	0.03947542621679\\
  0.900900900900901	0.0389105548220135\\
  0.902902902902903	0.0383525366636035\\
  0.904904904904905	0.037801308986495\\
  0.906906906906907	0.0372568092881017\\
  0.908908908908909	0.0367189753226449\\
  0.910910910910911	0.0361877451053805\\
  0.912912912912913	0.0356630569167229\\
  0.914914914914915	0.035144849306267\\
  0.916916916916917	0.0346330610967097\\
  0.918918918918919	0.0341276313876702\\
  0.920920920920921	0.0336284995594116\\
  0.922922922922923	0.0331356052764625\\
  0.924924924924925	0.0326488884911416\\
  0.926926926926927	0.032168289446984\\
  0.928928928928929	0.0316937486820711\\
  0.930930930930931	0.0312252070322653\\
  0.932932932932933	0.0307626056343484\\
  0.934934934934935	0.0303058859290672\\
  0.936936936936937	0.0298549896640844\\
  0.938938938938939	0.0294098588968379\\
  0.940940940940941	0.0289704359973074\\
  0.942942942942943	0.028536663650691\\
  0.944944944944945	0.0281084848599909\\
  0.946946946946947	0.0276858429485101\\
  0.948948948948949	0.0272686815622613\\
  0.950950950950951	0.0268569446722873\\
  0.952952952952953	0.0264505765768954\\
  0.954954954954955	0.0260495219038055\\
  0.956956956956957	0.0256537256122146\\
  0.958958958958959	0.0252631329947754\\
  0.960960960960961	0.0248776896794931\\
  0.962962962962963	0.02449734163154\\
  0.964964964964965	0.024122035154988\\
  0.966966966966967	0.0237517168944617\\
  0.968968968968969	0.023386333836711\\
  0.970970970970971	0.0230258333121064\\
  0.972972972972973	0.0226701629960551\\
  0.974974974974975	0.0223192709103419\\
  0.976976976976977	0.0219731054243935\\
  0.978978978978979	0.0216316152564681\\
  0.980980980980981	0.0212947494747709\\
  0.982982982982983	0.0209624574984971\\
  0.984984984984985	0.0206346890988026\\
  0.986986986986987	0.0203113943997031\\
  0.988988988988989	0.0199925238789039\\
  0.990990990990991	0.01967802836856\\
  0.992992992992993	0.0193678590559676\\
  0.994994994994995	0.019061967484189\\
  0.996996996996997	0.0187603055526104\\
  0.998998998998999	0.0184628255174343\\
  1.001001001001	0.018169479992108\\
};
\addlegendentry{$\varepsilon=2$}

\addplot [color=mycolor3, style=semithick]
  table[row sep=crcr]{%
  0	1\\
  0.002002002002002	0.999935873863912\\
  0.004004004004004	0.999743520127562\\
  0.00600600600600601	0.99942301279087\\
  0.00800800800800801	0.998974475134316\\
  0.01001001001001	0.998398079639916\\
  0.012012012012012	0.997694047880678\\
  0.014014014014014	0.996862650378651\\
  0.016016016016016	0.995904206431688\\
  0.018018018018018	0.994819083909076\\
  0.02002002002002	0.993607699016224\\
  0.022022022022022	0.992270516028608\\
  0.024024024024024	0.990808046995225\\
  0.026026026026026	0.989220851411808\\
  0.028028028028028	0.987509535864105\\
  0.03003003003003	0.985674753641531\\
  0.032032032032032	0.983717204321545\\
  0.034034034034034	0.981637633325118\\
  0.036036036036036	0.979436831443688\\
  0.038038038038038	0.977115634338022\\
  0.04004004004004	0.974674922009433\\
  0.042042042042042	0.972115618243815\\
  0.044044044044044	0.969438690028991\\
  0.046046046046046	0.966645146945889\\
  0.048048048048048	0.963736040534077\\
  0.0500500500500501	0.960712463632226\\
  0.0520520520520521	0.957575549694071\\
  0.0540540540540541	0.954326472080464\\
  0.0560560560560561	0.950966443328158\\
  0.0580580580580581	0.947496714395942\\
  0.0600600600600601	0.943918573888782\\
  0.0620620620620621	0.940233347260654\\
  0.0640640640640641	0.936442395996754\\
  0.0660660660660661	0.932547116775787\\
  0.0680680680680681	0.928548940613052\\
  0.0700700700700701	0.924449331985075\\
  0.0720720720720721	0.920249787936524\\
  0.0740740740740741	0.915951837170173\\
  0.0760760760760761	0.911557039120684\\
  0.0780780780780781	0.907066983013005\\
  0.0800800800800801	0.902483286906156\\
  0.0820820820820821	0.897807596723234\\
  0.0840840840840841	0.893041585268424\\
  0.0860860860860861	0.888186951231853\\
  0.0880880880880881	0.88324541818311\\
  0.0900900900900901	0.878218733554258\\
  0.0920920920920921	0.873108667613182\\
  0.0940940940940941	0.867917012428118\\
  0.0960960960960961	0.862645580824192\\
  0.0980980980980981	0.857296205332837\\
  0.1001001001001	0.851870737134919\\
  0.102102102102102	0.846371044998432\\
  0.104104104104104	0.840799014211601\\
  0.106106106106106	0.835156545512252\\
  0.108108108108108	0.829445554014283\\
  0.11011011011011	0.823667968132084\\
  0.112112112112112	0.817825728503745\\
  0.114114114114114	0.811920786913886\\
  0.116116116116116	0.805955105216931\\
  0.118118118118118	0.79993065426166\\
  0.12012012012012	0.793849412817842\\
  0.122122122122122	0.787713366505768\\
  0.124124124124124	0.781524506729473\\
  0.126126126126126	0.775284829614443\\
  0.128128128128128	0.768996334950585\\
  0.13013013013013	0.762661025141225\\
  0.132132132132132	0.756280904158901\\
  0.134134134134134	0.749857976508682\\
  0.136136136136136	0.743394246199757\\
  0.138138138138138	0.736891715726011\\
  0.14014014014014	0.730352385056288\\
  0.142142142142142	0.723778250635031\\
  0.144144144144144	0.717171304393979\\
  0.146146146146146	0.710533532775583\\
  0.148148148148148	0.703866915768767\\
  0.15015015015015	0.697173425957677\\
  0.152152152152152	0.690455027584021\\
  0.154154154154154	0.683713675623589\\
  0.156156156156156	0.676951314877522\\
  0.158158158158158	0.670169879078892\\
  0.16016016016016	0.663371290015122\\
  0.162162162162162	0.656557456666763\\
  0.164164164164164	0.649730274363123\\
  0.166166166166166	0.642891623955219\\
  0.168168168168168	0.636043371006523\\
  0.17017017017017	0.62918736500191\\
  0.172172172172172	0.622325438575247\\
  0.174174174174174	0.615459406756007\\
  0.176176176176176	0.608591066235266\\
  0.178178178178178	0.601722194651462\\
  0.18018018018018	0.59485454989621\\
  0.182182182182182	0.587989869440515\\
  0.184184184184184	0.581129869681639\\
  0.186186186186186	0.574276245310902\\
  0.188188188188188	0.56743066870266\\
  0.19019019019019	0.560594789324668\\
  0.192192192192192	0.553770233170041\\
  0.194194194194194	0.546958602210982\\
  0.196196196196196	0.54016147387443\\
  0.198198198198198	0.53338040053977\\
  0.2002002002002	0.526616909058719\\
  0.202202202202202	0.519872500297461\\
  0.204204204204204	0.513148648701129\\
  0.206206206206206	0.506446801880665\\
  0.208208208208208	0.499768380222087\\
  0.21021021021021	0.493114776518188\\
  0.212212212212212	0.486487355622636\\
  0.214214214214214	0.479887454126453\\
  0.216216216216216	0.47331638005683\\
  0.218218218218218	0.466775412598193\\
  0.22022022022022	0.460265801835437\\
  0.222222222222222	0.453788768519229\\
  0.224224224224224	0.447345503853245\\
  0.226226226226226	0.440937169303194\\
  0.228228228228228	0.434564896427484\\
  0.23023023023023	0.428229786729336\\
  0.232232232232232	0.421932911530164\\
  0.234234234234234	0.415675311863995\\
  0.236236236236236	0.409457998392726\\
  0.238238238238238	0.403281951341949\\
  0.24024024024024	0.397148120457116\\
  0.242242242242242	0.391057424979749\\
  0.244244244244244	0.385010753643423\\
  0.246246246246246	0.379008964689227\\
  0.248248248248248	0.373052885900383\\
  0.25025025025025	0.367143314655709\\
  0.252252252252252	0.361281018001586\\
  0.254254254254254	0.355466732742081\\
  0.256256256256256	0.349701165546886\\
  0.258258258258258	0.343984993076677\\
  0.26026026026026	0.338318862125551\\
  0.262262262262262	0.332703389780123\\
  0.264264264264264	0.327139163594914\\
  0.266266266266266	0.321626741783615\\
  0.268268268268268	0.316166653425825\\
  0.27027027027027	0.310759398688832\\
  0.272272272272272	0.305405449064036\\
  0.274274274274274	0.300105247617569\\
  0.276276276276276	0.294859209254682\\
  0.278278278278278	0.28966772099746\\
  0.28028028028028	0.284531142275429\\
  0.282282282282282	0.279449805228594\\
  0.284284284284284	0.274424015022471\\
  0.286286286286286	0.269454050174652\\
  0.288288288288288	0.264540162892454\\
  0.29029029029029	0.259682579421189\\
  0.292292292292292	0.254881500402606\\
  0.294294294294294	0.250137101243029\\
  0.296296296296296	0.245449532490759\\
  0.298298298298298	0.240818920222253\\
  0.3003003003003	0.236245366436644\\
  0.302302302302302	0.231728949458131\\
  0.304304304304304	0.227269724345791\\
  0.306306306306306	0.222867723310358\\
  0.308308308308308	0.218522956137517\\
  0.31031031031031	0.214235410617268\\
  0.312312312312312	0.210005052978911\\
  0.314314314314314	0.205831828331212\\
  0.316316316316316	0.201715661107324\\
  0.318318318318318	0.197656455514005\\
  0.32032032032032	0.193654095984734\\
  0.322322322322322	0.189708447636285\\
  0.324324324324324	0.185819356728341\\
  0.326326326326326	0.181986651125748\\
  0.328328328328328	0.178210140762991\\
  0.33033033033033	0.1744896181105\\
  0.332332332332332	0.170824858642393\\
  0.334334334334334	0.167215621305262\\
  0.336336336336336	0.163661648987634\\
  0.338338338338338	0.160162668989733\\
  0.34034034034034	0.156718393493172\\
  0.342342342342342	0.153328520030222\\
  0.344344344344344	0.149992731952322\\
  0.346346346346346	0.146710698897467\\
  0.348348348348348	0.143482077256163\\
  0.35035035035035	0.140306510635613\\
  0.352352352352352	0.137183630321828\\
  0.354354354354354	0.134113055739347\\
  0.356356356356356	0.131094394908284\\
  0.358358358358358	0.128127244898394\\
  0.36036036036036	0.1252111922799\\
  0.362362362362362	0.122345813570795\\
  0.364364364364364	0.119530675680364\\
  0.366366366366366	0.11676533634868\\
  0.368368368368368	0.114049344581825\\
  0.37037037037037	0.111382241082602\\
  0.372372372372372	0.108763558676524\\
  0.374374374374374	0.106192822732854\\
  0.376376376376376	0.103669551580503\\
  0.378378378378378	0.101193256918584\\
  0.38038038038038	0.0987634442214314\\
  0.382382382382382	0.0963796131379261\\
  0.384384384384384	0.0940412578849393\\
  0.386386386386386	0.0917478676347503\\
  0.388388388388388	0.0894989268962829\\
  0.39039039039039	0.0872939158900236\\
  0.392392392392392	0.0851323109164892\\
  0.394394394394394	0.0830135847181231\\
  0.396396396396396	0.0809372068345051\\
  0.398398398398398	0.0789026439507732\\
  0.4004004004004	0.0769093602391591\\
  0.402402402402402	0.0749568176935514\\
  0.404404404404404	0.0730444764570076\\
  0.406406406406406	0.0711717951421438\\
  0.408408408408408	0.0693382311443393\\
  0.41041041041041	0.0675432409477025\\
  0.412412412412412	0.0657862804237503\\
  0.414414414414414	0.064066805122762\\
  0.416416416416416	0.062384270557777\\
  0.418418418418418	0.0607381324812103\\
  0.42042042042042	0.0591278471540705\\
  0.422422422422422	0.0575528716077691\\
  0.424424424424424	0.0560126638985176\\
  0.426426426426426	0.0545066833543163\\
  0.428428428428428	0.0530343908145451\\
  0.43043043043043	0.0515952488621715\\
  0.432432432432432	0.0501887220485974\\
  0.434434434434434	0.0488142771111754\\
  0.436436436436436	0.0474713831834255\\
  0.438438438438438	0.0461595119979938\\
  0.44044044044044	0.0448781380823965\\
  0.442442442442442	0.0436267389476001\\
  0.444444444444444	0.0424047952694926\\
  0.446446446446446	0.0412117910633039\\
  0.448448448448448	0.0400472138510424\\
  0.45045045045045	0.0389105548220135\\
  0.452452452452452	0.037801308986495\\
  0.454454454454454	0.0367189753226449\\
  0.456456456456456	0.0356630569167229\\
  0.458458458458458	0.0346330610967097\\
  0.46046046046046	0.0336284995594116\\
  0.462462462462462	0.0326488884911416\\
  0.464464464464464	0.0316937486820711\\
  0.466466466466466	0.0307626056343484\\
  0.468468468468468	0.0298549896640844\\
  0.47047047047047	0.0289704359973074\\
  0.472472472472472	0.0281084848599909\\
  0.474474474474474	0.0272686815622613\\
  0.476476476476476	0.0264505765768954\\
  0.478478478478478	0.0256537256122146\\
  0.48048048048048	0.0248776896794931\\
  0.482482482482482	0.024122035154988\\
  0.484484484484485	0.023386333836711\\
  0.486486486486487	0.0226701629960551\\
  0.488488488488488	0.0219731054243935\\
  0.49049049049049	0.0212947494747709\\
  0.492492492492492	0.0206346890988026\\
  0.494494494494495	0.0199925238789039\\
  0.496496496496497	0.0193678590559676\\
  0.498498498498498	0.0187603055526104\\
  0.500500500500501	0.018169479992108\\
  0.502502502502503	0.0175950047131407\\
  0.504504504504504	0.0170365077804675\\
  0.506506506506507	0.0164936229916508\\
  0.508508508508508	0.0159659898799503\\
  0.510510510510511	0.0154532537135062\\
  0.512512512512513	0.01495506549093\\
  0.514514514514514	0.0144710819334215\\
  0.516516516516517	0.0140009654735295\\
  0.518518518518518	0.0135443842406727\\
  0.520520520520521	0.0131010120435366\\
  0.522522522522523	0.0126705283494622\\
  0.524524524524524	0.0122526182609376\\
  0.526526526526527	0.0118469724893087\\
  0.528528528528528	0.0114532873258174\\
  0.530530530530531	0.0110712646100782\\
  0.532532532532533	0.010700611696103\\
  0.534534534534535	0.010341041415979\\
  0.536536536536537	0.00999227204130794\\
  0.538538538538539	0.00965402724250886\\
  0.540540540540541	0.00932603604608787\\
  0.542542542542543	0.00900803278997578\\
  0.544544544544545	0.00869975707703245\\
  0.546546546546547	0.00840095372681561\\
  0.548548548548549	0.00811137272570954\\
  0.550550550550551	0.00783076917550767\\
  0.552552552552553	0.00755890324054099\\
  0.554554554554555	0.00729554009344228\\
  0.556556556556557	0.00704044985963456\\
  0.558558558558559	0.00679340756062973\\
  0.560560560560561	0.00655419305622201\\
  0.562562562562563	0.00632259098565841\\
  0.564564564564565	0.00609839070786638\\
  0.566566566566567	0.00588138624081742\\
  0.568568568568569	0.00567137620010261\\
  0.570570570570571	0.0054681637367947\\
  0.572572572572573	0.00527155647466898\\
  0.574574574574575	0.00508136644685325\\
  0.576576576576577	0.00489741003197525\\
  0.578578578578579	0.00471950788987386\\
  0.580580580580581	0.00454748489693835\\
  0.582582582582583	0.00438117008113804\\
  0.584584584584585	0.00422039655680256\\
  0.586586586586587	0.00406500145921134\\
  0.588588588588589	0.00391482587904829\\
  0.590590590590591	0.00376971479677654\\
  0.592592592592593	0.00362951701698553\\
  0.594594594594595	0.00349408510276098\\
  0.596596596596597	0.00336327531012665\\
  0.598598598598599	0.00323694752260455\\
  0.600600600600601	0.0031149651859387\\
  0.602602602602603	0.00299719524302548\\
  0.604604604604605	0.00288350806909204\\
  0.606606606606607	0.00277377740716237\\
  0.608608608608609	0.00266788030384865\\
  0.610610610610611	0.00256569704550431\\
  0.612612612612613	0.00246711109477298\\
  0.614614614614615	0.00237200902756624\\
  0.616616616616617	0.00228028047050124\\
  0.618618618618619	0.00219181803882769\\
  0.620620620620621	0.00210651727487238\\
  0.622622622622623	0.00202427658702731\\
  0.624624624624625	0.0019449971893068\\
  0.626626626626627	0.00186858304149669\\
  0.628628628628629	0.0017949407899178\\
  0.630630630630631	0.0017239797088243\\
  0.632632632632633	0.00165561164245617\\
  0.634634634634635	0.00158975094776373\\
  0.636636636636637	0.00152631443782086\\
  0.638638638638639	0.00146522132594222\\
  0.640640640640641	0.00140639317051871\\
  0.642642642642643	0.00134975382058396\\
  0.644644644644645	0.00129522936212375\\
  0.646646646646647	0.00124274806513898\\
  0.648648648648649	0.00119224033147167\\
  0.650650650650651	0.00114363864340267\\
  0.652652652652653	0.00109687751302834\\
  0.654654654654655	0.00105189343242286\\
  0.656656656656657	0.00100862482459169\\
  0.658658658658659	0.00096701199522068\\
  0.660660660660661	0.000926997085224698\\
  0.662662662662663	0.000888524024098541\\
  0.664664664664665	0.000851538484072273\\
  0.666666666666667	0.000815987835072148\\
  0.668668668668669	0.000781821100487695\\
  0.670670670670671	0.00074898891374469\\
  0.672672672672673	0.000717443475683079\\
  0.674674674674675	0.000687138512738294\\
  0.676676676676677	0.000658029235923672\\
  0.678678678678679	0.000630072300611159\\
  0.680680680680681	0.000603225767106878\\
  0.682682682682683	0.000577449062017537\\
  0.684684684684685	0.000552702940403214\\
  0.686686686686687	0.000528949448711488\\
  0.688688688688689	0.00050615188848743\\
  0.690690690690691	0.000484274780853554\\
  0.692692692692693	0.000463283831753343\\
  0.694694694694695	0.000443145897951638\\
  0.696696696696697	0.000423828953784741\\
  0.698698698698699	0.000405302058652786\\
  0.700700700700701	0.000387535325246566\\
  0.702702702702703	0.000370499888500711\\
  0.704704704704705	0.000354167875264849\\
  0.706706706706707	0.000338512374684072\\
  0.708708708708709	0.000323507409279842\\
  0.710710710710711	0.000309127906722198\\
  0.712712712712713	0.000295349672283959\\
  0.714714714714715	0.000282149361967407\\
  0.716716716716717	0.000269504456293773\\
  0.718718718718719	0.000257393234745716\\
  0.720720720720721	0.000245794750852819\\
  0.722722722722723	0.000234688807910016\\
  0.724724724724725	0.000224055935318801\\
  0.726726726726727	0.000213877365540911\\
  0.728728728728729	0.000204135011654161\\
  0.730730730730731	0.00019481144550002\\
  0.732732732732733	0.000185889876412481\\
  0.734734734734735	0.000177354130517736\\
  0.736736736736737	0.000169188630594141\\
  0.738738738738739	0.000161378376481967\\
  0.740740740740741	0.000153908926032411\\
  0.742742742742743	0.000146766376585349\\
  0.744744744744745	0.000139937346965366\\
  0.746746746746747	0.000133408959985581\\
  0.748748748748749	0.000127168825448877\\
  0.750750750750751	0.000121205023636131\\
  0.752752752752753	0.000115506089271177\\
  0.754754754754755	0.000110060995952205\\
  0.756756756756757	0.000104859141039452\\
  0.758758758758759	9.989033098907e-05\\
  0.760760760760761	9.51447671231586e-05\\
  0.762762762762763	9.06130318260363e-05\\
  0.764764764764765	8.62860751569233e-05\\
  0.766766766766767	8.21552018693039e-05\\
  0.768768768768769	7.82120588273552e-05\\
  0.770770770770771	7.4448622809923e-05\\
  0.772772772772773	7.08571886926557e-05\\
  0.774774774774775	6.74303579990209e-05\\
  0.776776776776777	6.41610278110449e-05\\
  0.778778778778779	6.10423800307562e-05\\
  0.780780780780781	5.80678709834278e-05\\
  0.782782782782783	5.52312213538545e-05\\
  0.784784784784785	5.25264064470334e-05\\
  0.786786786786787	4.99476467647448e-05\\
  0.788788788788789	4.74893988896832e-05\\
  0.790790790790791	4.51463466689098e-05\\
  0.792792792792793	4.29133926885556e-05\\
  0.794794794794795	4.07856500318333e-05\\
  0.796796796796797	3.8758434312568e-05\\
  0.798798798798799	3.68272559765955e-05\\
  0.800800800800801	3.49878128635228e-05\\
  0.802802802802803	3.32359830214923e-05\\
  0.804804804804805	3.15678177677309e-05\\
  0.806806806806807	2.99795349878163e-05\\
  0.808808808808809	2.84675126667326e-05\\
  0.810810810810811	2.70282826449336e-05\\
  0.812812812812813	2.56585245927765e-05\\
  0.814814814814815	2.43550601968314e-05\\
  0.816816816816817	2.31148475517143e-05\\
  0.818818818818819	2.19349757512343e-05\\
  0.820820820820821	2.08126596727837e-05\\
  0.822822822822823	1.97452349490437e-05\\
  0.824824824824825	1.87301531212122e-05\\
  0.826826826826827	1.77649769681005e-05\\
  0.828828828828829	1.68473760055814e-05\\
  0.830830830830831	1.59751221510024e-05\\
  0.832832832832833	1.51460855473163e-05\\
  0.834834834834835	1.43582305418056e-05\\
  0.836836836836837	1.36096118144141e-05\\
  0.838838838838839	1.28983706508208e-05\\
  0.840840840840841	1.22227313555204e-05\\
  0.842842842842843	1.15809978002999e-05\\
  0.844844844844845	1.09715501036188e-05\\
  0.846846846846847	1.03928414365268e-05\\
  0.848848848848849	9.84339495086653e-06\\
  0.850850850850851	9.32180082562701e-06\\
  0.852852852852853	8.82671342743044e-06\\
  0.854854854854855	8.35684858124256e-06\\
  0.856856856856857	7.91098094751246e-06\\
  0.858858858858859	7.48794150205201e-06\\
  0.860860860860861	7.08661511507427e-06\\
  0.862862862862863	6.70593822591287e-06\\
  0.864864864864865	6.34489661004694e-06\\
  0.866866866866867	6.00252323515674e-06\\
  0.868868868868869	5.67789620303201e-06\\
  0.870870870870871	5.37013677425244e-06\\
  0.872872872872873	5.07840747265192e-06\\
  0.874874874874875	4.80191026667098e-06\\
  0.876876876876877	4.53988482479125e-06\\
  0.878878878878879	4.29160684233294e-06\\
  0.880880880880881	4.05638643698256e-06\\
  0.882882882882883	3.83356661050106e-06\\
  0.884884884884885	3.62252177414423e-06\\
  0.886886886886887	3.42265633540698e-06\\
  0.888888888888889	3.23340334378008e-06\\
  0.890890890890891	3.05422319328424e-06\\
  0.892892892892893	2.88460237961938e-06\\
  0.894894894894895	2.72405230983897e-06\\
  0.896896896896897	2.5721081625295e-06\\
  0.898898898898899	2.42832779654239e-06\\
  0.900900900900901	2.29229070639257e-06\\
  0.902902902902903	2.16359702250185e-06\\
  0.904904904904905	2.04186655452802e-06\\
  0.906906906906907	1.92673787608142e-06\\
  0.908908908908909	1.81786744918974e-06\\
  0.910910910910911	1.71492878692957e-06\\
  0.912912912912913	1.61761165269851e-06\\
  0.914914914914915	1.52562129465641e-06\\
  0.916916916916917	1.43867771391636e-06\\
  0.918918918918919	1.3565149651175e-06\\
  0.920920920920921	1.27888048806074e-06\\
  0.922922922922923	1.20553446913696e-06\\
  0.924924924924925	1.13624923132336e-06\\
  0.926926926926927	1.07080865156898e-06\\
  0.928928928928929	1.00900760443389e-06\\
  0.930930930930931	9.5065143088898e-07\\
  0.932932932932933	8.95555431224198e-07\\
  0.934934934934935	8.43544381052677e-07\\
  0.936936936936937	7.94452069436678e-07\\
  0.938938938938939	7.48120858198213e-07\\
  0.940940940940941	7.04401261513335e-07\\
  0.942942942942943	6.63151544923574e-07\\
  0.944944944944945	6.24237342931731e-07\\
  0.946946946946947	5.87531294381554e-07\\
  0.948948948948949	5.52912694852197e-07\\
  0.950950950950951	5.20267165328632e-07\\
  0.952952952952953	4.89486336438369e-07\\
  0.954954954954955	4.60467547573087e-07\\
  0.956956956956957	4.33113560240972e-07\\
  0.958958958958959	4.07332285021849e-07\\
  0.960960960960961	3.83036521522528e-07\\
  0.962962962962963	3.60143710754205e-07\\
  0.964964964964965	3.38575699377358e-07\\
  0.966966966966967	3.1825851528226e-07\\
  0.968968968968969	2.99122153995094e-07\\
  0.970970970970971	2.81100375420815e-07\\
  0.972972972972973	2.6413051045413e-07\\
  0.974974974974975	2.48153277009569e-07\\
  0.976976976976977	2.33112605040443e-07\\
  0.978978978978979	2.18955470134588e-07\\
  0.980980980980981	2.05631735292255e-07\\
  0.982982982982983	1.9309400050825e-07\\
  0.984984984984985	1.81297459796603e-07\\
  0.986986986986987	1.70199765311505e-07\\
  0.988988988988989	1.59760898233227e-07\\
  0.990990990990991	1.49943046102014e-07\\
  0.992992992992993	1.40710486296757e-07\\
  0.994994994994995	1.32029475368464e-07\\
  0.996996996996997	1.23868143951254e-07\\
  0.998998998998999	1.16196396985821e-07\\
  1.001001001001	1.08985819002012e-07\\
};
\addlegendentry{$\varepsilon=4$}

\addplot [color=mycolor4, style=semithick]
  table[row sep=crcr]{%
  0	1\\
  0.002002002002002	0.999743520127562\\
  0.004004004004004	0.998974475134316\\
  0.00600600600600601	0.997694047880678\\
  0.00800800800800801	0.995904206431688\\
  0.01001001001001	0.993607699016224\\
  0.012012012012012	0.990808046995225\\
  0.014014014014014	0.987509535864105\\
  0.016016016016016	0.983717204321545\\
  0.018018018018018	0.979436831443688\\
  0.02002002002002	0.974674922009433\\
  0.022022022022022	0.969438690028991\\
  0.024024024024024	0.963736040534077\\
  0.026026026026026	0.957575549694071\\
  0.028028028028028	0.950966443328158\\
  0.03003003003003	0.943918573888782\\
  0.032032032032032	0.936442395996754\\
  0.034034034034034	0.928548940613052\\
  0.036036036036036	0.920249787936524\\
  0.038038038038038	0.911557039120684\\
  0.04004004004004	0.902483286906156\\
  0.042042042042042	0.893041585268424\\
  0.044044044044044	0.88324541818311\\
  0.046046046046046	0.873108667613182\\
  0.048048048048048	0.862645580824192\\
  0.0500500500500501	0.851870737134919\\
  0.0520520520520521	0.840799014211601\\
  0.0540540540540541	0.829445554014283\\
  0.0560560560560561	0.817825728503745\\
  0.0580580580580581	0.805955105216931\\
  0.0600600600600601	0.793849412817842\\
  0.0620620620620621	0.781524506729473\\
  0.0640640640640641	0.768996334950585\\
  0.0660660660660661	0.756280904158901\\
  0.0680680680680681	0.743394246199757\\
  0.0700700700700701	0.730352385056288\\
  0.0720720720720721	0.717171304393979\\
  0.0740740740740741	0.703866915768767\\
  0.0760760760760761	0.690455027584021\\
  0.0780780780780781	0.676951314877522\\
  0.0800800800800801	0.663371290015122\\
  0.0820820820820821	0.649730274363123\\
  0.0840840840840841	0.636043371006523\\
  0.0860860860860861	0.622325438575247\\
  0.0880880880880881	0.608591066235266\\
  0.0900900900900901	0.59485454989621\\
  0.0920920920920921	0.581129869681639\\
  0.0940940940940941	0.56743066870266\\
  0.0960960960960961	0.553770233170041\\
  0.0980980980980981	0.54016147387443\\
  0.1001001001001	0.526616909058719\\
  0.102102102102102	0.513148648701129\\
  0.104104104104104	0.499768380222087\\
  0.106106106106106	0.486487355622636\\
  0.108108108108108	0.47331638005683\\
  0.11011011011011	0.460265801835437\\
  0.112112112112112	0.447345503853245\\
  0.114114114114114	0.434564896427484\\
  0.116116116116116	0.421932911530164\\
  0.118118118118118	0.409457998392726\\
  0.12012012012012	0.397148120457116\\
  0.122122122122122	0.385010753643423\\
  0.124124124124124	0.373052885900383\\
  0.126126126126126	0.361281018001586\\
  0.128128128128128	0.349701165546886\\
  0.13013013013013	0.338318862125551\\
  0.132132132132132	0.327139163594914\\
  0.134134134134134	0.316166653425825\\
  0.136136136136136	0.305405449064036\\
  0.138138138138138	0.294859209254682\\
  0.14014014014014	0.284531142275429\\
  0.142142142142142	0.274424015022471\\
  0.144144144144144	0.264540162892454\\
  0.146146146146146	0.254881500402606\\
  0.148148148148148	0.245449532490759\\
  0.15015015015015	0.236245366436644\\
  0.152152152152152	0.227269724345791\\
  0.154154154154154	0.218522956137517\\
  0.156156156156156	0.210005052978911\\
  0.158158158158158	0.201715661107324\\
  0.16016016016016	0.193654095984734\\
  0.162162162162162	0.185819356728341\\
  0.164164164164164	0.178210140762991\\
  0.166166166166166	0.170824858642393\\
  0.168168168168168	0.163661648987634\\
  0.17017017017017	0.156718393493172\\
  0.172172172172172	0.149992731952322\\
  0.174174174174174	0.143482077256163\\
  0.176176176176176	0.137183630321828\\
  0.178178178178178	0.131094394908284\\
  0.18018018018018	0.1252111922799\\
  0.182182182182182	0.119530675680364\\
  0.184184184184184	0.114049344581825\\
  0.186186186186186	0.108763558676524\\
  0.188188188188188	0.103669551580503\\
  0.19019019019019	0.0987634442214314\\
  0.192192192192192	0.0940412578849393\\
  0.194194194194194	0.0894989268962829\\
  0.196196196196196	0.0851323109164892\\
  0.198198198198198	0.0809372068345051\\
  0.2002002002002	0.0769093602391591\\
  0.202202202202202	0.0730444764570076\\
  0.204204204204204	0.0693382311443393\\
  0.206206206206206	0.0657862804237503\\
  0.208208208208208	0.062384270557777\\
  0.21021021021021	0.0591278471540705\\
  0.212212212212212	0.0560126638985176\\
  0.214214214214214	0.0530343908145451\\
  0.216216216216216	0.0501887220485974\\
  0.218218218218218	0.0474713831834255\\
  0.22022022022022	0.0448781380823965\\
  0.222222222222222	0.0424047952694926\\
  0.224224224224224	0.0400472138510424\\
  0.226226226226226	0.037801308986495\\
  0.228228228228228	0.0356630569167229\\
  0.23023023023023	0.0336284995594116\\
  0.232232232232232	0.0316937486820711\\
  0.234234234234234	0.0298549896640844\\
  0.236236236236236	0.0281084848599909\\
  0.238238238238238	0.0264505765768954\\
  0.24024024024024	0.0248776896794931\\
  0.242242242242242	0.023386333836711\\
  0.244244244244244	0.0219731054243935\\
  0.246246246246246	0.0206346890988026\\
  0.248248248248248	0.0193678590559676\\
  0.25025025025025	0.018169479992108\\
  0.252252252252252	0.0170365077804675\\
  0.254254254254254	0.0159659898799503\\
  0.256256256256256	0.01495506549093\\
  0.258258258258258	0.0140009654735295\\
  0.26026026026026	0.0131010120435366\\
  0.262262262262262	0.0122526182609376\\
  0.264264264264264	0.0114532873258174\\
  0.266266266266266	0.010700611696103\\
  0.268268268268268	0.00999227204130794\\
  0.27027027027027	0.00932603604608787\\
  0.272272272272272	0.00869975707703245\\
  0.274274274274274	0.00811137272570954\\
  0.276276276276276	0.00755890324054099\\
  0.278278278278278	0.00704044985963456\\
  0.28028028028028	0.00655419305622201\\
  0.282282282282282	0.00609839070786638\\
  0.284284284284284	0.00567137620010261\\
  0.286286286286286	0.00527155647466898\\
  0.288288288288288	0.00489741003197525\\
  0.29029029029029	0.00454748489693835\\
  0.292292292292292	0.00422039655680256\\
  0.294294294294294	0.00391482587904829\\
  0.296296296296296	0.00362951701698553\\
  0.298298298298298	0.00336327531012665\\
  0.3003003003003	0.0031149651859387\\
  0.302302302302302	0.00288350806909204\\
  0.304304304304304	0.00266788030384865\\
  0.306306306306306	0.00246711109477298\\
  0.308308308308308	0.00228028047050124\\
  0.31031031031031	0.00210651727487238\\
  0.312312312312312	0.0019449971893068\\
  0.314314314314314	0.0017949407899178\\
  0.316316316316316	0.00165561164245617\\
  0.318318318318318	0.00152631443782086\\
  0.32032032032032	0.00140639317051871\\
  0.322322322322322	0.00129522936212375\\
  0.324324324324324	0.00119224033147167\\
  0.326326326326326	0.00109687751302834\\
  0.328328328328328	0.00100862482459169\\
  0.33033033033033	0.000926997085224698\\
  0.332332332332332	0.000851538484072273\\
  0.334334334334334	0.000781821100487695\\
  0.336336336336336	0.000717443475683079\\
  0.338338338338338	0.000658029235923672\\
  0.34034034034034	0.000603225767106878\\
  0.342342342342342	0.000552702940403214\\
  0.344344344344344	0.00050615188848743\\
  0.346346346346346	0.000463283831753343\\
  0.348348348348348	0.000423828953784741\\
  0.35035035035035	0.000387535325246566\\
  0.352352352352352	0.000354167875264849\\
  0.354354354354354	0.000323507409279842\\
  0.356356356356356	0.000295349672283959\\
  0.358358358358358	0.000269504456293773\\
  0.36036036036036	0.000245794750852819\\
  0.362362362362362	0.000224055935318801\\
  0.364364364364364	0.000204135011654161\\
  0.366366366366366	0.000185889876412481\\
  0.368368368368368	0.000169188630594141\\
  0.37037037037037	0.000153908926032411\\
  0.372372372372372	0.000139937346965366\\
  0.374374374374374	0.000127168825448877\\
  0.376376376376376	0.000115506089271177\\
  0.378378378378378	0.000104859141039452\\
  0.38038038038038	9.51447671231586e-05\\
  0.382382382382382	8.62860751569233e-05\\
  0.384384384384384	7.82120588273552e-05\\
  0.386386386386386	7.08571886926557e-05\\
  0.388388388388388	6.41610278110449e-05\\
  0.39039039039039	5.80678709834278e-05\\
  0.392392392392392	5.25264064470334e-05\\
  0.394394394394394	4.74893988896832e-05\\
  0.396396396396396	4.29133926885556e-05\\
  0.398398398398398	3.8758434312568e-05\\
  0.4004004004004	3.49878128635228e-05\\
  0.402402402402402	3.15678177677309e-05\\
  0.404404404404404	2.84675126667326e-05\\
  0.406406406406406	2.56585245927765e-05\\
  0.408408408408408	2.31148475517143e-05\\
  0.41041041041041	2.08126596727837e-05\\
  0.412412412412412	1.87301531212122e-05\\
  0.414414414414414	1.68473760055814e-05\\
  0.416416416416416	1.51460855473163e-05\\
  0.418418418418418	1.36096118144141e-05\\
  0.42042042042042	1.22227313555204e-05\\
  0.422422422422422	1.09715501036188e-05\\
  0.424424424424424	9.84339495086653e-06\\
  0.426426426426426	8.82671342743044e-06\\
  0.428428428428428	7.91098094751246e-06\\
  0.43043043043043	7.08661511507427e-06\\
  0.432432432432432	6.34489661004694e-06\\
  0.434434434434434	5.67789620303201e-06\\
  0.436436436436436	5.07840747265192e-06\\
  0.438438438438438	4.53988482479125e-06\\
  0.44044044044044	4.05638643698256e-06\\
  0.442442442442442	3.62252177414423e-06\\
  0.444444444444444	3.23340334378008e-06\\
  0.446446446446446	2.88460237961938e-06\\
  0.448448448448448	2.5721081625295e-06\\
  0.45045045045045	2.29229070639257e-06\\
  0.452452452452452	2.04186655452802e-06\\
  0.454454454454454	1.81786744918974e-06\\
  0.456456456456456	1.61761165269851e-06\\
  0.458458458458458	1.43867771391636e-06\\
  0.46046046046046	1.27888048806074e-06\\
  0.462462462462462	1.13624923132336e-06\\
  0.464464464464464	1.00900760443389e-06\\
  0.466466466466466	8.95555431224198e-07\\
  0.468468468468468	7.94452069436678e-07\\
  0.47047047047047	7.04401261513335e-07\\
  0.472472472472472	6.24237342931731e-07\\
  0.474474474474474	5.52912694852197e-07\\
  0.476476476476476	4.89486336438369e-07\\
  0.478478478478478	4.33113560240972e-07\\
  0.48048048048048	3.83036521522528e-07\\
  0.482482482482482	3.38575699377358e-07\\
  0.484484484484485	2.99122153995094e-07\\
  0.486486486486487	2.6413051045413e-07\\
  0.488488488488488	2.33112605040443e-07\\
  0.49049049049049	2.05631735292255e-07\\
  0.492492492492492	1.81297459796603e-07\\
  0.494494494494495	1.59760898233227e-07\\
  0.496496496496497	1.40710486296757e-07\\
  0.498498498498498	1.23868143951254e-07\\
  0.500500500500501	1.08985819002012e-07\\
  0.502502502502503	9.58423712276194e-08\\
  0.504504504504504	8.4240765318756e-08\\
  0.506506506506507	7.40055436366592e-08\\
  0.508508508508508	6.49805523499179e-08\\
  0.510510510510511	5.70268968488203e-08\\
  0.512512512512513	5.00211044866114e-08\\
  0.514514514514514	4.38534746703481e-08\\
  0.516516516516517	3.84265981337059e-08\\
  0.518518518518518	3.36540288821191e-08\\
  0.520520520520521	2.94590938185305e-08\\
  0.522522522522523	2.57738264464327e-08\\
  0.524524524524524	2.2538012315777e-08\\
  0.526526526526527	1.96983350360695e-08\\
  0.528528528528528	1.7207612738231e-08\\
  0.530530530530531	1.50241158306804e-08\\
  0.532532532532533	1.31109577731599e-08\\
  0.534534534534535	1.14355513910219e-08\\
  0.536536536536537	9.9691239795817e-09\\
  0.538538538538539	8.6862851086914e-09\\
  0.540540540540541	7.56464163751588e-09\\
  0.542542542542543	6.58445499373051e-09\\
  0.544544544544545	5.72833626478742e-09\\
  0.546546546546547	4.98097509590931e-09\\
  0.548548548548549	4.32889879411773e-09\\
  0.550550550550551	3.76025840362782e-09\\
  0.552552552552553	3.26463884877125e-09\\
  0.554554554554555	2.83289053940626e-09\\
  0.556556556556557	2.45698010343587e-09\\
  0.558558558558559	2.12985815423821e-09\\
  0.560560560560561	1.84534221995798e-09\\
  0.562562562562563	1.5980131589403e-09\\
  0.564564564564565	1.38312356315066e-09\\
  0.566566566566567	1.19651681107991e-09\\
  0.568568568568569	1.03455557508086e-09\\
  0.570570570570571	8.94058716876972e-10\\
  0.572572572572573	7.72245620534734e-10\\
  0.574574574574575	6.66687115787724e-10\\
  0.576576576576577	5.75262237411977e-10\\
  0.578578578578579	4.96120149441727e-10\\
  0.580580580580581	4.27646637348062e-10\\
  0.582582582582583	3.68434637755395e-10\\
  0.584584584584585	3.17258334635764e-10\\
  0.586586586586587	2.73050403917427e-10\\
  0.588588588588589	2.34882035721572e-10\\
  0.590590590590591	2.01945405587576e-10\\
  0.592592592592593	1.73538303593035e-10\\
  0.594594594594595	1.49050663699278e-10\\
  0.596596596596597	1.27952765387855e-10\\
  0.598598598598599	1.09784906087541e-10\\
  0.600600600600601	9.41483663751601e-11\\
  0.602602602602603	8.06975107817756e-11\\
  0.604604604604605	6.9132885531875e-11\\
  0.606606606606607	5.91951909410541e-11\\
  0.608608608608609	5.06600207257497e-11\\
  0.610610610610611	4.33332733411594e-11\\
  0.612612612612613	3.70471518438681e-11\\
  0.614614614614615	3.1656678837962e-11\\
  0.616616616616617	2.70366619541787e-11\\
  0.618618618618619	2.3079053162093e-11\\
  0.620620620620621	1.96906521425747e-11\\
  0.622622622622623	1.67911100562476e-11\\
  0.624624624624625	1.4311195426666e-11\\
  0.626626626626627	1.21912885973959e-11\\
  0.628628628628629	1.03800753940974e-11\\
  0.630630630630631	8.8334142918308e-12\\
  0.632632632632633	7.51335461259997e-12\\
  0.634634634634635	6.3872861105886e-12\\
  0.636636636636637	5.42720278871636e-12\\
  0.638638638638639	4.60906597097191e-12\\
  0.640640640640641	3.91225356671613e-12\\
  0.642642642642643	3.31908413790456e-12\\
  0.644644644644645	2.81440584638762e-12\\
  0.646646646646647	2.38524164124027e-12\\
  0.648648648648649	2.02048316767699e-12\\
  0.650650650650651	1.71062685914312e-12\\
  0.652652652652653	1.44754652995895e-12\\
  0.654654654654655	1.22429753270312e-12\\
  0.656656656656657	1.03494819581567e-12\\
  0.658658658658659	8.74434824532864e-13\\
  0.660660660660661	7.38437042668298e-13\\
  0.662662662662663	6.23270683098951e-13\\
  0.664664664664665	5.25795809165802e-13\\
  0.666666666666667	4.43337774632805e-13\\
  0.668668668668669	3.73619512580881e-13\\
  0.670670670670671	3.14703489091351e-13\\
  0.672672672672673	2.64941970571803e-13\\
  0.674674674674675	2.22934438277518e-13\\
  0.676676676676677	1.87491143642159e-13\\
  0.678678678678679	1.57601936653105e-13\\
  0.680680680680681	1.32409619485519e-13\\
  0.682682682682683	1.11187181387768e-13\\
  0.684684684684685	9.33183605205282e-14\\
  0.686686686686687	7.8281055951676e-14\\
  0.688688688688689	6.56331799194024e-14\\
  0.690690690690691	5.50005982076777e-14\\
  0.692692692692693	4.60668562597036e-14\\
  0.694694694694695	3.85644315545314e-14\\
  0.696696696696697	3.2267289718092e-14\\
  0.698698698698699	2.69845536391873e-14\\
  0.700700700700701	2.25551222135462e-14\\
  0.702702702702703	1.88430988522694e-14\\
  0.704704704704705	1.57339100913844e-14\\
  0.706706706706707	1.31310119828638e-14\\
  0.708708708708709	1.09530968292952e-14\\
  0.710710710710711	9.13172558598924e-15\\
  0.712712712712713	7.60932219111835e-15\\
  0.714714714714715	6.33747545167967e-15\\
  0.716716716716717	5.2755021310624e-15\\
  0.718718718718719	4.38923174293588e-15\\
  0.720720720720721	3.6499794200433e-15\\
  0.722722722722723	3.03367823660757e-15\\
  0.724724724724725	2.52014664114709e-15\\
  0.726726726726727	2.09247030729431e-15\\
  0.728728728728729	1.73648082382349e-15\\
  0.730730730730731	1.44031629895146e-15\\
  0.732732732732733	1.19405121461161e-15\\
  0.734734734734735	9.89384790914729e-16\\
  0.736736736736737	8.19378758395815e-16\\
  0.738738738738739	6.78236827904125e-16\\
  0.740740740740741	5.61119331111645e-16\\
  0.742742742742743	4.63987509407738e-16\\
  0.744744744744745	3.83472781781509e-16\\
  0.746746746746747	3.1676704571789e-16\\
  0.748748748748749	2.61530678418688e-16\\
  0.750750750750751	2.15815425263823e-16\\
  0.752752752752753	1.779998023992e-16\\
  0.754754754754755	1.46735012658156e-16\\
  0.756756756756757	1.20899688908227e-16\\
  0.758758758758759	9.95620450663337e-17\\
  0.760760760760761	8.19482398519215e-17\\
  0.762762762762763	6.74159481533321e-17\\
  0.764764764764765	5.54322950277696e-17\\
  0.766766766766767	4.55554423948379e-17\\
  0.768768768768769	3.7419232286244e-17\\
  0.770770770770771	3.07203863629852e-17\\
  0.772772772772773	2.5207842091205e-17\\
  0.774774774774775	2.06738738393496e-17\\
  0.776776776776777	1.69467042221772e-17\\
  0.778778778778779	1.38843589646121e-17\\
  0.780780780780781	1.13695588228276e-17\\
  0.782782782782783	9.3054758926155e-18\\
  0.784784784784785	7.6122099795969e-18\\
  0.786786786786787	6.22386446659997e-18\\
  0.788788788788789	5.0861210201869e-18\\
  0.790790790790791	4.1542291462328e-18\\
  0.792792792792793	3.39134055244451e-18\\
  0.794794794794795	2.76712994076899e-18\\
  0.796796796796797	2.25665360276196e-18\\
  0.798798798798799	1.83940533576228e-18\\
  0.800800800800801	1.49853599750904e-18\\
  0.802802802802803	1.22020869044532e-18\\
  0.804804804804805	9.93066298138431e-19\\
  0.806806806806807	8.07792039266109e-19\\
  0.808808808808809	6.56746988860011e-19\\
  0.810810810810811	5.33671250442501e-19\\
  0.812812812812813	4.33437737226693e-19\\
  0.814814814814815	3.5184941177539e-19\\
  0.816816816816817	2.8547240511906e-19\\
  0.818818818818819	2.31498741576493e-19\\
  0.820820820820821	1.87633478922457e-19\\
  0.822822822822823	1.52001972290179e-19\\
  0.824824824824825	1.23073715325648e-19\\
  0.826826826826827	9.95998295258974e-20\\
  0.828828828828829	8.05617839893331e-20\\
  0.830830830830831	6.51293509787151e-20\\
  0.832832832832833	5.2626152733427e-20\\
  0.834834834834835	4.25014443406071e-20\\
  0.836836836836837	3.43070165606427e-20\\
  0.838838838838839	2.76782999266265e-20\\
  0.840840840840841	2.23189143660279e-20\\
  0.842842842842843	1.79880429761999e-20\\
  0.844844844844845	1.4490119361057e-20\\
  0.846846846846847	1.16664091993466e-20\\
  0.848848848848849	9.38814185935807e-21\\
  0.850850850850851	7.55090971219219e-21\\
  0.852852852852853	6.07010365324639e-21\\
  0.854854854854855	4.8771951450445e-21\\
  0.856856856856857	3.9167094362176e-21\\
  0.858858858858859	3.14376280712038e-21\\
  0.860860860860861	2.5220598290001e-21\\
  0.862862862862863	2.02226559778303e-21\\
  0.864864864864865	1.62068345536115e-21\\
  0.866866866866867	1.2981814507716e-21\\
  0.868868868868869	1.03932122348845e-21\\
  0.870870870870871	8.31651525090858e-22\\
  0.872872872872873	6.65135574447926e-22\\
  0.874874874874875	5.31687145434738e-22\\
  0.876876876876877	4.24794945353247e-22\\
  0.878878878878879	3.39218645867713e-22\\
  0.880880880880881	2.70743031710686e-22\\
  0.882882882882883	2.15979263130781e-22\\
  0.884884884884885	1.7220431056579e-22\\
  0.886886886886887	1.3723130000421e-22\\
  0.888888888888889	1.09304875118982e-22\\
  0.890890890890891	8.70167949673772e-23\\
  0.892892892892893	6.92378908512896e-23\\
  0.894894894894895	5.50632412557156e-23\\
  0.896896896896897	4.37680210368958e-23\\
  0.898898898898899	3.47719658669348e-23\\
  0.900900900900901	2.76107862969354e-23\\
  0.902902902902903	2.1913184756569e-23\\
  0.904904904904905	1.73823872909306e-23\\
  0.906906906906907	1.37813111930338e-23\\
  0.908908908908909	1.09206591633831e-23\\
  0.910910910910911	8.64936773798891e-24\\
  0.912912912912913	6.84694859070699e-24\\
  0.914914914914915	5.4173509084843e-24\\
  0.916916916916917	4.2840453997268e-24\\
  0.918918918918919	3.38608890781214e-24\\
  0.920920920920921	2.67497572651693e-24\\
  0.922922922922923	2.11211971153407e-24\\
  0.924924924924925	1.66684190297653e-24\\
  0.926926926926927	1.31476303605963e-24\\
  0.928928928928929	1.03652016934234e-24\\
  0.930930930930931	8.1674263120575e-25\\
  0.932932932932933	6.43235327135271e-25\\
  0.934934934934935	5.06327770257759e-25\\
  0.936936936936937	3.98355486121784e-25\\
  0.938938938938939	3.13247096027174e-25\\
  0.940940940940941	2.46195720934498e-25\\
  0.942942942942943	1.93397627863476e-25\\
  0.944944944944945	1.51844470455469e-25\\
  0.946946946946947	1.19158222353034e-25\\
  0.948948948948949	9.34601009408543e-26\\
  0.950950950950951	7.3266538172335e-26\\
  0.952952952952953	5.74066618196579e-26\\
  0.954954954954955	4.49568745658938e-26\\
  0.956956956956957	3.51890161553109e-26\\
  0.958958958958959	2.75293104855126e-26\\
  0.960960960960961	2.15258712872233e-26\\
  0.962962962962963	1.68229959396695e-26\\
  0.964964964964965	1.31408402872043e-26\\
  0.966966966966967	1.02593566972145e-26\\
  0.968968968968969	8.00560795163635e-27\\
  0.970970970970971	6.24375283124566e-27\\
  0.972972972972973	4.86714497691188e-27\\
  0.974974974974975	3.79210242136721e-27\\
  0.976976976976977	2.95299716902169e-27\\
  0.978978978978979	2.29838722107388e-27\\
  0.980980980980981	1.78797137634777e-27\\
  0.982982982982983	1.390193076503e-27\\
  0.984984984984985	1.08035597088834e-27\\
  0.986986986986987	8.39142713055722e-28\\
  0.988988988988989	6.5145132916705e-28\\
  0.990990990990991	5.05481560172131e-28\\
  0.992992992992993	3.92017862085783e-28\\
  0.994994994994995	3.03867036458095e-28\\
  0.996996996996997	2.35417379199348e-28\\
  0.998998998998999	1.82293274725694e-28\\
  1.001001001001	1.41084716119847e-28\\
};
\addlegendentry{$\varepsilon=8$}

\addplot [color=mycolor5, style=dotted,semithick]
  table[row sep=crcr]{%
  0	1\\
  0.002002002002002	0.998974475134316\\
  0.004004004004004	0.995904206431688\\
  0.00600600600600601	0.990808046995225\\
  0.00800800800800801	0.983717204321545\\
  0.01001001001001	0.974674922009433\\
  0.012012012012012	0.963736040534077\\
  0.014014014014014	0.950966443328158\\
  0.016016016016016	0.936442395996754\\
  0.018018018018018	0.920249787936524\\
  0.02002002002002	0.902483286906156\\
  0.022022022022022	0.88324541818311\\
  0.024024024024024	0.862645580824192\\
  0.026026026026026	0.840799014211601\\
  0.028028028028028	0.817825728503745\\
  0.03003003003003	0.793849412817842\\
  0.032032032032032	0.768996334950585\\
  0.034034034034034	0.743394246199757\\
  0.036036036036036	0.717171304393979\\
  0.038038038038038	0.690455027584021\\
  0.04004004004004	0.663371290015122\\
  0.042042042042042	0.636043371006523\\
  0.044044044044044	0.608591066235266\\
  0.046046046046046	0.581129869681639\\
  0.048048048048048	0.553770233170041\\
  0.0500500500500501	0.526616909058719\\
  0.0520520520520521	0.499768380222087\\
  0.0540540540540541	0.47331638005683\\
  0.0560560560560561	0.447345503853245\\
  0.0580580580580581	0.421932911530164\\
  0.0600600600600601	0.397148120457116\\
  0.0620620620620621	0.373052885900383\\
  0.0640640640640641	0.349701165546886\\
  0.0660660660660661	0.327139163594914\\
  0.0680680680680681	0.305405449064036\\
  0.0700700700700701	0.284531142275429\\
  0.0720720720720721	0.264540162892454\\
  0.0740740740740741	0.245449532490759\\
  0.0760760760760761	0.227269724345791\\
  0.0780780780780781	0.210005052978911\\
  0.0800800800800801	0.193654095984734\\
  0.0820820820820821	0.178210140762991\\
  0.0840840840840841	0.163661648987634\\
  0.0860860860860861	0.149992731952322\\
  0.0880880880880881	0.137183630321828\\
  0.0900900900900901	0.1252111922799\\
  0.0920920920920921	0.114049344581825\\
  0.0940940940940941	0.103669551580503\\
  0.0960960960960961	0.0940412578849393\\
  0.0980980980980981	0.0851323109164892\\
  0.1001001001001	0.0769093602391591\\
  0.102102102102102	0.0693382311443393\\
  0.104104104104104	0.062384270557777\\
  0.106106106106106	0.0560126638985176\\
  0.108108108108108	0.0501887220485974\\
  0.11011011011011	0.0448781380823965\\
  0.112112112112112	0.0400472138510424\\
  0.114114114114114	0.0356630569167229\\
  0.116116116116116	0.0316937486820711\\
  0.118118118118118	0.0281084848599909\\
  0.12012012012012	0.0248776896794931\\
  0.122122122122122	0.0219731054243935\\
  0.124124124124124	0.0193678590559676\\
  0.126126126126126	0.0170365077804675\\
  0.128128128128128	0.01495506549093\\
  0.13013013013013	0.0131010120435366\\
  0.132132132132132	0.0114532873258174\\
  0.134134134134134	0.00999227204130794\\
  0.136136136136136	0.00869975707703245\\
  0.138138138138138	0.00755890324054099\\
  0.14014014014014	0.00655419305622201\\
  0.142142142142142	0.00567137620010261\\
  0.144144144144144	0.00489741003197525\\
  0.146146146146146	0.00422039655680256\\
  0.148148148148148	0.00362951701698553\\
  0.15015015015015	0.0031149651859387\\
  0.152152152152152	0.00266788030384865\\
  0.154154154154154	0.00228028047050124\\
  0.156156156156156	0.0019449971893068\\
  0.158158158158158	0.00165561164245617\\
  0.16016016016016	0.00140639317051871\\
  0.162162162162162	0.00119224033147167\\
  0.164164164164164	0.00100862482459169\\
  0.166166166166166	0.000851538484072273\\
  0.168168168168168	0.000717443475683079\\
  0.17017017017017	0.000603225767106878\\
  0.172172172172172	0.00050615188848743\\
  0.174174174174174	0.000423828953784741\\
  0.176176176176176	0.000354167875264849\\
  0.178178178178178	0.000295349672283959\\
  0.18018018018018	0.000245794750852819\\
  0.182182182182182	0.000204135011654161\\
  0.184184184184184	0.000169188630594141\\
  0.186186186186186	0.000139937346965366\\
  0.188188188188188	0.000115506089271177\\
  0.19019019019019	9.51447671231586e-05\\
  0.192192192192192	7.82120588273552e-05\\
  0.194194194194194	6.41610278110449e-05\\
  0.196196196196196	5.25264064470334e-05\\
  0.198198198198198	4.29133926885556e-05\\
  0.2002002002002	3.49878128635228e-05\\
  0.202202202202202	2.84675126667326e-05\\
  0.204204204204204	2.31148475517143e-05\\
  0.206206206206206	1.87301531212122e-05\\
  0.208208208208208	1.51460855473163e-05\\
  0.21021021021021	1.22227313555204e-05\\
  0.212212212212212	9.84339495086653e-06\\
  0.214214214214214	7.91098094751246e-06\\
  0.216216216216216	6.34489661004694e-06\\
  0.218218218218218	5.07840747265192e-06\\
  0.22022022022022	4.05638643698256e-06\\
  0.222222222222222	3.23340334378008e-06\\
  0.224224224224224	2.5721081625295e-06\\
  0.226226226226226	2.04186655452802e-06\\
  0.228228228228228	1.61761165269851e-06\\
  0.23023023023023	1.27888048806074e-06\\
  0.232232232232232	1.00900760443389e-06\\
  0.234234234234234	7.94452069436678e-07\\
  0.236236236236236	6.24237342931731e-07\\
  0.238238238238238	4.89486336438369e-07\\
  0.24024024024024	3.83036521522528e-07\\
  0.242242242242242	2.99122153995094e-07\\
  0.244244244244244	2.33112605040443e-07\\
  0.246246246246246	1.81297459796603e-07\\
  0.248248248248248	1.40710486296757e-07\\
  0.25025025025025	1.08985819002012e-07\\
  0.252252252252252	8.4240765318756e-08\\
  0.254254254254254	6.49805523499179e-08\\
  0.256256256256256	5.00211044866114e-08\\
  0.258258258258258	3.84265981337059e-08\\
  0.26026026026026	2.94590938185305e-08\\
  0.262262262262262	2.2538012315777e-08\\
  0.264264264264264	1.7207612738231e-08\\
  0.266266266266266	1.31109577731599e-08\\
  0.268268268268268	9.9691239795817e-09\\
  0.27027027027027	7.56464163751588e-09\\
  0.272272272272272	5.72833626478742e-09\\
  0.274274274274274	4.32889879411773e-09\\
  0.276276276276276	3.26463884877125e-09\\
  0.278278278278278	2.45698010343587e-09\\
  0.28028028028028	1.84534221995798e-09\\
  0.282282282282282	1.38312356315066e-09\\
  0.284284284284284	1.03455557508086e-09\\
  0.286286286286286	7.72245620534734e-10\\
  0.288288288288288	5.75262237411977e-10\\
  0.29029029029029	4.27646637348062e-10\\
  0.292292292292292	3.17258334635764e-10\\
  0.294294294294294	2.34882035721572e-10\\
  0.296296296296296	1.73538303593035e-10\\
  0.298298298298298	1.27952765387855e-10\\
  0.3003003003003	9.41483663751601e-11\\
  0.302302302302302	6.9132885531875e-11\\
  0.304304304304304	5.06600207257497e-11\\
  0.306306306306306	3.70471518438681e-11\\
  0.308308308308308	2.70366619541787e-11\\
  0.31031031031031	1.96906521425747e-11\\
  0.312312312312312	1.4311195426666e-11\\
  0.314314314314314	1.03800753940974e-11\\
  0.316316316316316	7.51335461259997e-12\\
  0.318318318318318	5.42720278871636e-12\\
  0.32032032032032	3.91225356671613e-12\\
  0.322322322322322	2.81440584638762e-12\\
  0.324324324324324	2.02048316767699e-12\\
  0.326326326326326	1.44754652995895e-12\\
  0.328328328328328	1.03494819581567e-12\\
  0.33033033033033	7.38437042668298e-13\\
  0.332332332332332	5.25795809165802e-13\\
  0.334334334334334	3.73619512580881e-13\\
  0.336336336336336	2.64941970571803e-13\\
  0.338338338338338	1.87491143642159e-13\\
  0.34034034034034	1.32409619485519e-13\\
  0.342342342342342	9.33183605205282e-14\\
  0.344344344344344	6.56331799194024e-14\\
  0.346346346346346	4.60668562597036e-14\\
  0.348348348348348	3.2267289718092e-14\\
  0.35035035035035	2.25551222135462e-14\\
  0.352352352352352	1.57339100913844e-14\\
  0.354354354354354	1.09530968292952e-14\\
  0.356356356356356	7.60932219111835e-15\\
  0.358358358358358	5.2755021310624e-15\\
  0.36036036036036	3.6499794200433e-15\\
  0.362362362362362	2.52014664114709e-15\\
  0.364364364364364	1.73648082382349e-15\\
  0.366366366366366	1.19405121461161e-15\\
  0.368368368368368	8.19378758395815e-16\\
  0.37037037037037	5.61119331111645e-16\\
  0.372372372372372	3.83472781781509e-16\\
  0.374374374374374	2.61530678418688e-16\\
  0.376376376376376	1.779998023992e-16\\
  0.378378378378378	1.20899688908227e-16\\
  0.38038038038038	8.19482398519215e-17\\
  0.382382382382382	5.54322950277696e-17\\
  0.384384384384384	3.7419232286244e-17\\
  0.386386386386386	2.5207842091205e-17\\
  0.388388388388388	1.69467042221772e-17\\
  0.39039039039039	1.13695588228276e-17\\
  0.392392392392392	7.6122099795969e-18\\
  0.394394394394394	5.0861210201869e-18\\
  0.396396396396396	3.39134055244451e-18\\
  0.398398398398398	2.25665360276196e-18\\
  0.4004004004004	1.49853599750904e-18\\
  0.402402402402402	9.93066298138431e-19\\
  0.404404404404404	6.56746988860011e-19\\
  0.406406406406406	4.33437737226693e-19\\
  0.408408408408408	2.8547240511906e-19\\
  0.41041041041041	1.87633478922457e-19\\
  0.412412412412412	1.23073715325648e-19\\
  0.414414414414414	8.05617839893331e-20\\
  0.416416416416416	5.2626152733427e-20\\
  0.418418418418418	3.43070165606427e-20\\
  0.42042042042042	2.23189143660279e-20\\
  0.422422422422422	1.4490119361057e-20\\
  0.424424424424424	9.38814185935807e-21\\
  0.426426426426426	6.07010365324639e-21\\
  0.428428428428428	3.9167094362176e-21\\
  0.43043043043043	2.5220598290001e-21\\
  0.432432432432432	1.62068345536115e-21\\
  0.434434434434434	1.03932122348845e-21\\
  0.436436436436436	6.65135574447926e-22\\
  0.438438438438438	4.24794945353247e-22\\
  0.44044044044044	2.70743031710686e-22\\
  0.442442442442442	1.7220431056579e-22\\
  0.444444444444444	1.09304875118982e-22\\
  0.446446446446446	6.92378908512896e-23\\
  0.448448448448448	4.37680210368958e-23\\
  0.45045045045045	2.76107862969354e-23\\
  0.452452452452452	1.73823872909306e-23\\
  0.454454454454454	1.09206591633831e-23\\
  0.456456456456456	6.84694859070699e-24\\
  0.458458458458458	4.2840453997268e-24\\
  0.46046046046046	2.67497572651693e-24\\
  0.462462462462462	1.66684190297653e-24\\
  0.464464464464464	1.03652016934234e-24\\
  0.466466466466466	6.43235327135271e-25\\
  0.468468468468468	3.98355486121784e-25\\
  0.47047047047047	2.46195720934498e-25\\
  0.472472472472472	1.51844470455469e-25\\
  0.474474474474474	9.34601009408543e-26\\
  0.476476476476476	5.74066618196579e-26\\
  0.478478478478478	3.51890161553109e-26\\
  0.48048048048048	2.15258712872233e-26\\
  0.482482482482482	1.31408402872043e-26\\
  0.484484484484485	8.00560795163635e-27\\
  0.486486486486487	4.86714497691188e-27\\
  0.488488488488488	2.95299716902169e-27\\
  0.49049049049049	1.78797137634777e-27\\
  0.492492492492492	1.08035597088834e-27\\
  0.494494494494495	6.5145132916705e-28\\
  0.496496496496497	3.92017862085783e-28\\
  0.498498498498498	2.35417379199348e-28\\
  0.500500500500501	1.41084716119847e-28\\
  0.502502502502503	8.43781890287139e-29\\
  0.504504504504504	5.03604050934065e-29\\
  0.506506506506507	2.99955626682643e-29\\
  0.508508508508508	1.78292713421238e-29\\
  0.510510510510511	1.057593953968e-29\\
  0.512512512512513	6.2605589261762e-30\\
  0.514514514514514	3.69841838377954e-30\\
  0.516516516516517	2.1803577312146e-30\\
  0.518518518518518	1.28276839866882e-30\\
  0.520520520520521	7.53143156321642e-31\\
  0.522522522522523	4.41281376404664e-31\\
  0.524524524524524	2.58025392576517e-31\\
  0.526526526526527	1.50562937473473e-31\\
  0.528528528528528	8.76763565911447e-32\\
  0.530530530530531	5.09513496799717e-32\\
  0.532532532532533	2.95486520879938e-32\\
  0.534534534534535	1.71012729905619e-32\\
  0.536536536536537	9.87706673901798e-33\\
  0.538538538538539	5.692936244911e-33\\
  0.540540540540541	3.27456364168981e-33\\
  0.542542542542543	1.87966014931682e-33\\
  0.544544544544545	1.07674785682353e-33\\
  0.546546546546547	6.15541702411878e-34\\
  0.548548548548549	3.51163791972396e-34\\
  0.550550550550551	1.99926683659488e-34\\
  0.552552552552553	1.13590125001654e-34\\
  0.554554554554555	6.44049394447904e-35\\
  0.556556556556557	3.64423653969662e-35\\
  0.558558558558559	2.05779791965638e-35\\
  0.560560560560561	1.15959857415433e-35\\
  0.562562562562563	6.52110818007191e-36\\
  0.564564564564565	3.65968680709442e-36\\
  0.566566566566567	2.04962882119048e-36\\
  0.568568568568569	1.14555330234229e-36\\
  0.570570570570571	6.38946017053118e-37\\
  0.572572572572573	3.55649183720364e-37\\
  0.574574574574575	1.97555101307992e-37\\
  0.576576576576577	1.09512442247886e-37\\
  0.578578578578579	6.05825379993492e-38\\
  0.580580580580581	3.34456966028148e-38\\
  0.582582582582583	1.84264558789125e-38\\
  0.584584584584585	1.0130996393464e-38\\
  0.586586586586587	5.55867513369959e-39\\
  0.588588588588589	3.04368153174199e-39\\
  0.590590590590591	1.66316718789179e-39\\
  0.592592592592593	9.06945918978766e-40\\
  0.594594594594595	4.9355511472551e-40\\
  0.596596596596597	2.68039442627628e-40\\
  0.598598598598599	1.45268194500991e-40\\
  0.600600600600601	7.85689871968659e-41\\
  0.602602602602603	4.24072933285726e-41\\
  0.604604604604605	2.28422424552832e-41\\
  0.606606606606607	1.22785109838681e-41\\
  0.608608608608609	6.5866024676394e-42\\
  0.610610610610611	3.52602985641496e-42\\
  0.612612612612613	1.88373280706643e-42\\
  0.614614614614615	1.0042952324764e-42\\
  0.616616616616617	5.3433335338867e-43\\
  0.618618618618619	2.83708240386353e-43\\
  0.620620620620621	1.50328180082075e-43\\
  0.622622622622623	7.94909400201196e-44\\
  0.624624624624625	4.19472649420668e-44\\
  0.626626626626627	2.20901390072586e-44\\
  0.628628628628629	1.16091930141098e-44\\
  0.630630630630631	6.08855730908506e-45\\
  0.632632632632633	3.1866586719126e-45\\
  0.634634634634635	1.66442979349158e-45\\
  0.636636636636637	8.675693439921e-46\\
  0.638638638638639	4.51285830194273e-46\\
  0.640640640640641	2.34265308700306e-46\\
  0.642642642642643	1.21359295645326e-46\\
  0.644644644644645	6.27403442228563e-47\\
  0.646646646646647	3.23690184665464e-47\\
  0.648648648648649	1.66655997368571e-47\\
  0.650650650650651	8.56290541780898e-48\\
  0.652652652652653	4.39066326014695e-48\\
  0.654654654654655	2.24671454598176e-48\\
  0.656656656656657	1.14729327297266e-48\\
  0.658658658658659	5.84668613709479e-49\\
  0.660660660660661	2.97340383598134e-49\\
  0.662662662662663	1.50906100739761e-49\\
  0.664664664664665	7.64308133166702e-50\\
  0.666666666666667	3.86312666304906e-50\\
  0.668668668668669	1.94857980901526e-50\\
  0.670670670670671	9.80858210298282e-51\\
  0.672672672672673	4.92723242006215e-51\\
  0.674674674674675	2.47006651880497e-51\\
  0.676676676676677	1.23572841335913e-51\\
  0.678678678678679	6.16944645956164e-52\\
  0.680680680680681	3.07381800394214e-52\\
  0.682682682682683	1.52833614323024e-52\\
  0.684684684684685	7.58347747008427e-53\\
  0.686686686686687	3.75514491292845e-53\\
  0.688688688688689	1.8556402544983e-53\\
  0.690690690690691	9.15102311370457e-54\\
  0.692692692692693	4.5035428866491e-54\\
  0.694694694694695	2.21180958985222e-54\\
  0.696696696696697	1.0840515980131e-54\\
  0.698698698698699	5.30225920315469e-55\\
  0.700700700700701	2.58809812755191e-55\\
  0.702702702702703	1.26069289683603e-55\\
  0.704704704704705	6.12839368758679e-56\\
  0.706706706706707	2.97298549704386e-56\\
  0.708708708708709	1.43928801167599e-56\\
  0.710710710710711	6.95362728553642e-57\\
  0.712712712712713	3.35261661449831e-57\\
  0.714714714714715	1.61311437140834e-57\\
  0.716716716716717	7.7456026027285e-58\\
  0.718718718718719	3.71153914573074e-58\\
  0.720720720720721	1.77485003307347e-58\\
  0.722722722722723	8.46989572994323e-59\\
  0.724724724724725	4.03369677771717e-59\\
  0.726726726726727	1.91706666639821e-59\\
  0.728728728728729	9.09243001228646e-60\\
  0.730730730730731	4.30359605933562e-60\\
  0.732732732732733	2.03278673890265e-60\\
  0.734734734734735	9.58210487739691e-61\\
  0.736736736736737	4.50753185291361e-61\\
  0.738738738738739	2.11604779174399e-61\\
  0.740740740740741	9.91336104136461e-62\\
  0.742742742742743	4.63473767095646e-62\\
  0.744744744744745	2.16241067032967e-62\\
  0.746746746746747	1.00683887780728e-62\\
  0.748748748748749	4.67832686207072e-63\\
  0.750750750750751	2.16935151505935e-63\\
  0.752752752752753	1.0038713983294e-63\\
  0.754754754754755	4.63591020602085e-64\\
  0.756756756756757	2.13648935573497e-64\\
  0.758758758758759	9.82596549728641e-65\\
  0.760760760760761	4.50981284833521e-65\\
  0.762762762762763	2.06562075026903e-65\\
  0.764764764764765	9.44172700270056e-66\\
  0.766766766766767	4.30686316592785e-66\\
  0.768768768768769	1.96055708527636e-66\\
  0.770770770770771	8.90649223586235e-67\\
  0.772772772772773	4.0377802416542e-67\\
  0.774774774774775	1.82678504164337e-67\\
  0.776776776776777	8.24785464110643e-68\\
  0.778778778778779	3.71623640437107e-68\\
  0.780780780780781	1.67099231174768e-68\\
  0.782782782782783	7.49815395695855e-69\\
  0.784784784784785	3.35770887378673e-69\\
  0.786786786786787	1.5005155721918e-69\\
  0.788788788788789	6.69185864520053e-70\\
  0.790790790790791	2.97825441140992e-70\\
  0.792792792792793	1.32277388498924e-70\\
  0.794794794794795	5.86297731828261e-71\\
  0.796796796796797	2.59334083931249e-71\\
  0.798798798798799	1.14474766888704e-71\\
  0.800800800800801	5.0427648821443e-72\\
  0.802802802802803	2.21685074948949e-72\\
  0.804804804804805	9.72552318809333e-73\\
  0.806806806806807	4.25792762988525e-73\\
  0.808808808808809	1.86034015798939e-73\\
  0.810810810810811	8.11138900464679e-74\\
  0.812812812812813	3.5294487643878e-74\\
  0.814814814814815	1.53259469250778e-74\\
  0.816816816816817	6.64135256608224e-75\\
  0.818818818818819	2.87206680951382e-75\\
  0.820820820820821	1.23948513781599e-75\\
  0.822822822822823	5.33822521756962e-76\\
  0.824824824824825	2.29435832125983e-76\\
  0.826826826826827	9.8408900680989e-77\\
  0.828828828828829	4.21227095336975e-77\\
  0.830830830830831	1.79931417610846e-77\\
  0.832832832832833	7.67019644962314e-78\\
  0.834834834834835	3.26298258794649e-78\\
  0.836836836836837	1.38526164179668e-78\\
  0.838838838838839	5.86891263215508e-79\\
  0.840840840840841	2.48137420663692e-79\\
  0.842842842842843	1.0469734436022e-79\\
  0.844844844844845	4.40846961489642e-80\\
  0.846846846846847	1.85245992277633e-80\\
  0.848848848848849	7.76816735848956e-81\\
  0.850850850850851	3.25085133656714e-81\\
  0.852852852852853	1.35763938597532e-81\\
  0.854854854854855	5.65822914339953e-82\\
  0.856856856856857	2.35334401317488e-82\\
  0.858858858858859	9.7678523575032e-83\\
  0.860860860860861	4.04595957524844e-83\\
  0.862862862862863	1.67244858456831e-83\\
  0.864864864864865	6.89910563580803e-84\\
  0.866866866866867	2.84015209232805e-84\\
  0.868868868868869	1.16680742365409e-84\\
  0.870870870870871	4.78371781264882e-85\\
  0.872872872872873	1.95722478132568e-85\\
  0.874874874874875	7.9914326215931e-86\\
  0.876876876876877	3.25624715867544e-86\\
  0.878878878878879	1.32409414330579e-86\\
  0.880880880880881	5.37315230283772e-87\\
  0.882882882882883	2.17594653691381e-87\\
  0.884884884884885	8.79378966144112e-88\\
  0.886886886886887	3.54660408437283e-88\\
  0.888888888888889	1.42744087796636e-88\\
  0.890890890890891	5.73340119572291e-89\\
  0.892892892892893	2.2981338470286e-89\\
  0.894894894894895	9.19278470147358e-90\\
  0.896896896896897	3.6696753279839e-90\\
  0.898898898898899	1.4618976856176e-90\\
  0.900900900900901	5.81185938775419e-91\\
  0.902902902902903	2.30580194747098e-91\\
  0.904904904904905	9.12930440340745e-92\\
  0.906906906906907	3.60713302101938e-92\\
  0.908908908908909	1.42231375967896e-92\\
  0.910910910910911	5.59676984882374e-93\\
  0.912912912912913	2.19780050165186e-93\\
  0.914914914914915	8.61286959146087e-94\\
  0.916916916916917	3.36834260291921e-94\\
  0.918918918918919	1.31459939598294e-94\\
  0.920920920920921	5.12011106621393e-95\\
  0.922922922922923	1.99009642104083e-95\\
  0.924924924924925	7.71929501139737e-96\\
  0.926926926926927	2.98806432466968e-96\\
  0.928928928928929	1.15427962392401e-96\\
  0.930930930930931	4.44980417884703e-97\\
  0.932932932932933	1.71190457729755e-97\\
  0.934934934934935	6.57244544831907e-98\\
  0.936936936936937	2.51815935874049e-98\\
  0.938938938938939	9.62826897358568e-99\\
  0.940940940940941	3.67385491248564e-99\\
  0.942942942942943	1.39895766323142e-99\\
  0.944944944944945	5.31613407355124e-100\\
  0.946946946946947	2.01602569240361e-100\\
  0.948948948948949	7.62965645176742e-101\\
  0.950950950950951	2.88152695709872e-101\\
  0.952952952952953	1.08604838476499e-101\\
  0.954954954954955	4.08492836144216e-102\\
  0.956956956956957	1.53330481156851e-102\\
  0.958958958958959	5.74356229471148e-103\\
  0.960960960960961	2.1470539457501e-103\\
  0.962962962962963	8.00964670645918e-104\\
  0.964964964964965	2.98189638004426e-104\\
  0.966966966966967	1.10784886858218e-104\\
  0.968968968968969	4.1074971670586e-105\\
  0.970970970970971	1.51978729640029e-105\\
  0.972972972972973	5.61173469530672e-106\\
  0.974974974974975	2.06785572665902e-106\\
  0.976976976976977	7.6041753404534e-107\\
  0.978978978978979	2.79056917941503e-107\\
  0.980980980980981	1.02197964881104e-107\\
  0.982982982982983	3.73508496189483e-108\\
  0.984984984984985	1.3622835301978e-108\\
  0.986986986986987	4.95841999725284e-109\\
  0.988988988988989	1.80105882656034e-109\\
  0.990990990990991	6.52861816561783e-110\\
  0.992992992992993	2.36169289731457e-110\\
  0.994994994994995	8.5257846984795e-111\\
  0.996996996996997	3.0715251966424e-111\\
  0.998998998998999	1.1042885948613e-111\\
  1.001001001001	3.96204929462e-112\\
};
\addlegendentry{$\varepsilon=16$}
\end{axis}
\end{tikzpicture}%
\hspace{1em}
% This file was created by matlab2tikz.
%
%The latest updates can be retrieved from
%  http://www.mathworks.com/matlabcentral/fileexchange/22022-matlab2tikz-matlab2tikz
%where you can also make suggestions and rate matlab2tikz.
%
\rmfamily
\definecolor{mycolor1}{rgb}{0.00000,0.44700,0.74100}%
\definecolor{mycolor2}{rgb}{0.85000,0.32500,0.09800}%
\definecolor{mycolor3}{rgb}{0.92900,0.69400,0.12500}%
\definecolor{mycolor4}{rgb}{0.49400,0.18400,0.55600}%
\definecolor{mycolor5}{rgb}{0.46600,0.67400,0.18800}%
\definecolor{mycolor6}{rgb}{0.30100,0.74500,0.93300}%
%
\begin{tikzpicture}[trim axis left, trim axis right, baseline]

  \begin{axis}[
  grid=major,
  %tick label style = {font=\sansmath\sffamily},
  width=0.4\textwidth,
  height=0.4\textwidth,
  at={(0\textwidth,0\textwidth)},
  scale only axis,
  unbounded coords=jump,
  xmin=0,
  xmax=1,
  ymin=0,
  ymax=1,
  xlabel={$r$},
  ytick=\empty,
  % ylabel={$\phi(r)$},
  axis background/.style={fill=white},
  %title style={font=\bfseries},
  title={PHS},
  legend pos=north west,
  legend style={legend cell align=left,align=left,draw=white!15!black}
  ]
\addplot [color=mycolor1, style=dashed,semithick]
  table[row sep=crcr]{%
0 0\\
0.0050251256281407  0.0050251256281407\\
0.0100502512562814  0.0100502512562814\\
0.0150753768844221  0.0150753768844221\\
0.0201005025125628  0.0201005025125628\\
0.0251256281407035  0.0251256281407035\\
0.0301507537688442  0.0301507537688442\\
0.0351758793969849  0.0351758793969849\\
0.0402010050251256  0.0402010050251256\\
0.0452261306532663  0.0452261306532663\\
0.050251256281407 0.050251256281407\\
0.0552763819095477  0.0552763819095477\\
0.0603015075376884  0.0603015075376884\\
0.0653266331658292  0.0653266331658292\\
0.0703517587939698  0.0703517587939698\\
0.0753768844221105  0.0753768844221105\\
0.0804020100502513  0.0804020100502513\\
0.085427135678392 0.085427135678392\\
0.0904522613065327  0.0904522613065327\\
0.0954773869346734  0.0954773869346734\\
0.100502512562814 0.100502512562814\\
0.105527638190955 0.105527638190955\\
0.110552763819095 0.110552763819095\\
0.115577889447236 0.115577889447236\\
0.120603015075377 0.120603015075377\\
0.125628140703518 0.125628140703518\\
0.130653266331658 0.130653266331658\\
0.135678391959799 0.135678391959799\\
0.14070351758794  0.14070351758794\\
0.14572864321608  0.14572864321608\\
0.150753768844221 0.150753768844221\\
0.155778894472362 0.155778894472362\\
0.160804020100503 0.160804020100503\\
0.165829145728643 0.165829145728643\\
0.170854271356784 0.170854271356784\\
0.175879396984925 0.175879396984925\\
0.180904522613065 0.180904522613065\\
0.185929648241206 0.185929648241206\\
0.190954773869347 0.190954773869347\\
0.195979899497487 0.195979899497487\\
0.201005025125628 0.201005025125628\\
0.206030150753769 0.206030150753769\\
0.21105527638191  0.21105527638191\\
0.21608040201005  0.21608040201005\\
0.221105527638191 0.221105527638191\\
0.226130653266332 0.226130653266332\\
0.231155778894472 0.231155778894472\\
0.236180904522613 0.236180904522613\\
0.241206030150754 0.241206030150754\\
0.246231155778894 0.246231155778894\\
0.251256281407035 0.251256281407035\\
0.256281407035176 0.256281407035176\\
0.261306532663317 0.261306532663317\\
0.266331658291457 0.266331658291457\\
0.271356783919598 0.271356783919598\\
0.276381909547739 0.276381909547739\\
0.281407035175879 0.281407035175879\\
0.28643216080402  0.28643216080402\\
0.291457286432161 0.291457286432161\\
0.296482412060302 0.296482412060302\\
0.301507537688442 0.301507537688442\\
0.306532663316583 0.306532663316583\\
0.311557788944724 0.311557788944724\\
0.316582914572864 0.316582914572864\\
0.321608040201005 0.321608040201005\\
0.326633165829146 0.326633165829146\\
0.331658291457286 0.331658291457286\\
0.336683417085427 0.336683417085427\\
0.341708542713568 0.341708542713568\\
0.346733668341709 0.346733668341709\\
0.351758793969849 0.351758793969849\\
0.35678391959799  0.35678391959799\\
0.361809045226131 0.361809045226131\\
0.366834170854271 0.366834170854271\\
0.371859296482412 0.371859296482412\\
0.376884422110553 0.376884422110553\\
0.381909547738693 0.381909547738693\\
0.386934673366834 0.386934673366834\\
0.391959798994975 0.391959798994975\\
0.396984924623116 0.396984924623116\\
0.402010050251256 0.402010050251256\\
0.407035175879397 0.407035175879397\\
0.412060301507538 0.412060301507538\\
0.417085427135678 0.417085427135678\\
0.422110552763819 0.422110552763819\\
0.42713567839196  0.42713567839196\\
0.4321608040201 0.4321608040201\\
0.437185929648241 0.437185929648241\\
0.442211055276382 0.442211055276382\\
0.447236180904523 0.447236180904523\\
0.452261306532663 0.452261306532663\\
0.457286432160804 0.457286432160804\\
0.462311557788945 0.462311557788945\\
0.467336683417085 0.467336683417085\\
0.472361809045226 0.472361809045226\\
0.477386934673367 0.477386934673367\\
0.482412060301508 0.482412060301508\\
0.487437185929648 0.487437185929648\\
0.492462311557789 0.492462311557789\\
0.49748743718593  0.49748743718593\\
0.50251256281407  0.50251256281407\\
0.507537688442211 0.507537688442211\\
0.512562814070352 0.512562814070352\\
0.517587939698492 0.517587939698492\\
0.522613065326633 0.522613065326633\\
0.527638190954774 0.527638190954774\\
0.532663316582915 0.532663316582915\\
0.537688442211055 0.537688442211055\\
0.542713567839196 0.542713567839196\\
0.547738693467337 0.547738693467337\\
0.552763819095477 0.552763819095477\\
0.557788944723618 0.557788944723618\\
0.562814070351759 0.562814070351759\\
0.5678391959799 0.5678391959799\\
0.57286432160804  0.57286432160804\\
0.577889447236181 0.577889447236181\\
0.582914572864322 0.582914572864322\\
0.587939698492462 0.587939698492462\\
0.592964824120603 0.592964824120603\\
0.597989949748744 0.597989949748744\\
0.603015075376884 0.603015075376884\\
0.608040201005025 0.608040201005025\\
0.613065326633166 0.613065326633166\\
0.618090452261307 0.618090452261307\\
0.623115577889447 0.623115577889447\\
0.628140703517588 0.628140703517588\\
0.633165829145729 0.633165829145729\\
0.638190954773869 0.638190954773869\\
0.64321608040201  0.64321608040201\\
0.648241206030151 0.648241206030151\\
0.653266331658292 0.653266331658292\\
0.658291457286432 0.658291457286432\\
0.663316582914573 0.663316582914573\\
0.668341708542714 0.668341708542714\\
0.673366834170854 0.673366834170854\\
0.678391959798995 0.678391959798995\\
0.683417085427136 0.683417085427136\\
0.688442211055276 0.688442211055276\\
0.693467336683417 0.693467336683417\\
0.698492462311558 0.698492462311558\\
0.703517587939699 0.703517587939699\\
0.708542713567839 0.708542713567839\\
0.71356783919598  0.71356783919598\\
0.718592964824121 0.718592964824121\\
0.723618090452261 0.723618090452261\\
0.728643216080402 0.728643216080402\\
0.733668341708543 0.733668341708543\\
0.738693467336683 0.738693467336683\\
0.743718592964824 0.743718592964824\\
0.748743718592965 0.748743718592965\\
0.753768844221106 0.753768844221106\\
0.758793969849246 0.758793969849246\\
0.763819095477387 0.763819095477387\\
0.768844221105528 0.768844221105528\\
0.773869346733668 0.773869346733668\\
0.778894472361809 0.778894472361809\\
0.78391959798995  0.78391959798995\\
0.78894472361809  0.78894472361809\\
0.793969849246231 0.793969849246231\\
0.798994974874372 0.798994974874372\\
0.804020100502513 0.804020100502513\\
0.809045226130653 0.809045226130653\\
0.814070351758794 0.814070351758794\\
0.819095477386935 0.819095477386935\\
0.824120603015075 0.824120603015075\\
0.829145728643216 0.829145728643216\\
0.834170854271357 0.834170854271357\\
0.839195979899497 0.839195979899497\\
0.844221105527638 0.844221105527638\\
0.849246231155779 0.849246231155779\\
0.85427135678392  0.85427135678392\\
0.85929648241206  0.85929648241206\\
0.864321608040201 0.864321608040201\\
0.869346733668342 0.869346733668342\\
0.874371859296482 0.874371859296482\\
0.879396984924623 0.879396984924623\\
0.884422110552764 0.884422110552764\\
0.889447236180904 0.889447236180904\\
0.894472361809045 0.894472361809045\\
0.899497487437186 0.899497487437186\\
0.904522613065327 0.904522613065327\\
0.909547738693467 0.909547738693467\\
0.914572864321608 0.914572864321608\\
0.919597989949749 0.919597989949749\\
0.924623115577889 0.924623115577889\\
0.92964824120603  0.92964824120603\\
0.934673366834171 0.934673366834171\\
0.939698492462312 0.939698492462312\\
0.944723618090452 0.944723618090452\\
0.949748743718593 0.949748743718593\\
0.954773869346734 0.954773869346734\\
0.959798994974874 0.959798994974874\\
0.964824120603015 0.964824120603015\\
0.969849246231156 0.969849246231156\\
0.974874371859296 0.974874371859296\\
0.979899497487437 0.979899497487437\\
0.984924623115578 0.984924623115578\\
0.989949748743719 0.989949748743719\\
0.994974874371859 0.994974874371859\\
1 1\\
};
\addlegendentry{$q=1$}

\addplot [color=mycolor2, style=semithick]
  table[row sep=crcr]{%
0 0\\
0.0050251256281407  1.26893907430133e-07\\
0.0100502512562814  1.01515125944107e-06\\
0.0150753768844221  3.4261355006136e-06\\
0.0201005025125628  8.12121007552852e-06\\
0.0251256281407035  1.58617384287666e-05\\
0.0301507537688442  2.74090840049088e-05\\
0.0351758793969849  4.35246102485357e-05\\
0.0402010050251256  6.49696806042282e-05\\
0.0452261306532663  9.25056585165671e-05\\
0.050251256281407 0.000126893907430133\\
0.0552763819095477  0.000168895790789507\\
0.0603015075376884  0.00021927267203927\\
0.0653266331658292  0.000278785914624003\\
0.0703517587939698  0.000348196881988285\\
0.0753768844221105  0.000428266937576699\\
0.0804020100502513  0.000519757444833826\\
0.085427135678392 0.000623429767204244\\
0.0904522613065327  0.000740045268132537\\
0.0954773869346734  0.000870365311063283\\
0.100502512562814 0.00101515125944107\\
0.105527638190955 0.00117516447671046\\
0.110552763819095 0.00135116632631606\\
0.115577889447236 0.00154391817170243\\
0.120603015075377 0.00175418137631416\\
0.125628140703518 0.00198271730359583\\
0.130653266331658 0.00223028731699202\\
0.135678391959799 0.00249765277994731\\
0.14070351758794  0.00278557505590628\\
0.14572864321608  0.00309481550831352\\
0.150753768844221 0.00342613550061359\\
0.155778894472362 0.0037802963962511\\
0.160804020100503 0.0041580595586706\\
0.165829145728643 0.0045601863513167\\
0.170854271356784 0.00498743813763395\\
0.175879396984925 0.00544057628106696\\
0.180904522613065 0.00592036214506029\\
0.185929648241206 0.00642755709305854\\
0.190954773869347 0.00696292248850627\\
0.195979899497487 0.00752721969484807\\
0.201005025125628 0.00812121007552852\\
0.206030150753769 0.00874565499399221\\
0.21105527638191  0.00940131581368371\\
0.21608040201005  0.0100889538980476\\
0.221105527638191 0.0108093306105285\\
0.226130653266332 0.0115632073145709\\
0.231155778894472 0.0123513453736194\\
0.236180904522613 0.0131745061511187\\
0.241206030150754 0.0140334510105133\\
0.246231155778894 0.0149289413152477\\
0.251256281407035 0.0158617384287666\\
0.256281407035176 0.0168326037145146\\
0.261306532663317 0.0178422985359362\\
0.266331658291457 0.0188915842564759\\
0.271356783919598 0.0199812222395785\\
0.276381909547739 0.0211119738486884\\
0.281407035175879 0.0222846004472503\\
0.28643216080402  0.0234998633987087\\
0.291457286432161 0.0247585240665081\\
0.296482412060302 0.0260613438140933\\
0.301507537688442 0.0274090840049088\\
0.306532663316583 0.0288025060023991\\
0.311557788944724 0.0302423711700088\\
0.316582914572864 0.0317294408711825\\
0.321608040201005 0.0332644764693648\\
0.326633165829146 0.0348482393280003\\
0.331658291457286 0.0364814908105336\\
0.336683417085427 0.0381649922804091\\
0.341708542713568 0.0398995051010716\\
0.346733668341709 0.0416857906359656\\
0.351758793969849 0.0435246102485357\\
0.35678391959799  0.0454167253022264\\
0.361809045226131 0.0473628971604824\\
0.366834170854271 0.0493638871867481\\
0.371859296482412 0.0514204567444683\\
0.376884422110553 0.0535333671970874\\
0.381909547738693 0.0557033799080501\\
0.386934673366834 0.057931256240801\\
0.391959798994975 0.0602177575587845\\
0.396984924623116 0.0625636452254454\\
0.402010050251256 0.0649696806042282\\
0.407035175879397 0.0674366250585774\\
0.412060301507538 0.0699652399519377\\
0.417085427135678 0.0725562866477535\\
0.422110552763819 0.0752105265094696\\
0.42713567839196  0.0779287209005305\\
0.4321608040201 0.0807116311843808\\
0.437185929648241 0.083560018724465\\
0.442211055276382 0.0864746448842277\\
0.447236180904523 0.0894562710271135\\
0.452261306532663 0.0925056585165671\\
0.457286432160804 0.0956235687160329\\
0.462311557788945 0.0988107629889555\\
0.467336683417085 0.10206800269878\\
0.472361809045226 0.10539604920895\\
0.477386934673367 0.10879566388291\\
0.482412060301508 0.112267608084106\\
0.487437185929648 0.115812643175982\\
0.492462311557789 0.119431530521982\\
0.49748743718593  0.123125031485551\\
0.50251256281407  0.126893907430133\\
0.507537688442211 0.130738919719174\\
0.512562814070352 0.134660829716117\\
0.517587939698492 0.138660398784407\\
0.522613065326633 0.142738388287489\\
0.527638190954774 0.146895559588808\\
0.532663316582915 0.151132674051807\\
0.537688442211055 0.155450493039933\\
0.542713567839196 0.159849777916628\\
0.547738693467337 0.164331290045338\\
0.552763819095477 0.168895790789507\\
0.557788944723618 0.17354404151258\\
0.562814070351759 0.178276803578002\\
0.5678391959799 0.183094838349217\\
0.57286432160804  0.187998907189669\\
0.577889447236181 0.192989771462804\\
0.582914572864322 0.198068192532065\\
0.587939698492462 0.203234931760898\\
0.592964824120603 0.208490750512747\\
0.597989949748744 0.213836410151056\\
0.603015075376884 0.21927267203927\\
0.608040201005025 0.224800297540834\\
0.613065326633166 0.230420048019192\\
0.618090452261307 0.23613268483779\\
0.623115577889447 0.24193896936007\\
0.628140703517588 0.247839662949479\\
0.633165829145729 0.25383552696946\\
0.638190954773869 0.259927322783458\\
0.64321608040201  0.266115811754919\\
0.648241206030151 0.272401755247285\\
0.653266331658292 0.278785914624003\\
0.658291457286432 0.285269051248515\\
0.663316582914573 0.291851926484268\\
0.668341708542714 0.298535301694706\\
0.673366834170854 0.305319938243273\\
0.678391959798995 0.312206597493414\\
0.683417085427136 0.319196040808573\\
0.688442211055276 0.326289029552195\\
0.693467336683417 0.333486325087725\\
0.698492462311558 0.340788688778607\\
0.703517587939699 0.348196881988285\\
0.708542713567839 0.355711666080205\\
0.71356783919598  0.363333802417811\\
0.718592964824121 0.371064052364547\\
0.723618090452261 0.378903177283859\\
0.728643216080402 0.38685193853919\\
0.733668341708543 0.394911097493985\\
0.738693467336683 0.403081415511689\\
0.743718592964824 0.411363653955746\\
0.748743718592965 0.419758574189602\\
0.753768844221106 0.4282669375767\\
0.758793969849246 0.436889505480484\\
0.763819095477387 0.445627039264401\\
0.768844221105528 0.454480300291894\\
0.773869346733668 0.463450049926408\\
0.778894472361809 0.472537049531387\\
0.78391959798995  0.481742060470276\\
0.78894472361809  0.49106584410652\\
0.793969849246231 0.500509161803563\\
0.798994974874372 0.51007277492485\\
0.804020100502513 0.519757444833826\\
0.809045226130653 0.529563932893933\\
0.814070351758794 0.539493000468619\\
0.819095477386935 0.549545408921327\\
0.824120603015075 0.559721919615501\\
0.829145728643216 0.570023293914587\\
0.834170854271357 0.580450293182028\\
0.839195979899497 0.59100367878127\\
0.844221105527638 0.601684212075757\\
0.849246231155779 0.612492654428934\\
0.85427135678392  0.623429767204244\\
0.85929648241206  0.634496311765133\\
0.864321608040201 0.645693049475046\\
0.869346733668342 0.657020741697427\\
0.874371859296482 0.66848014979572\\
0.879396984924623 0.68007203513337\\
0.884422110552764 0.691797159073822\\
0.889447236180904 0.70365628298052\\
0.894472361809045 0.715650168216908\\
0.899497487437186 0.727779576146433\\
0.904522613065327 0.740045268132537\\
0.909547738693467 0.752448005538665\\
0.914572864321608 0.764988549728263\\
0.919597989949749 0.777667662064774\\
0.924623115577889 0.790486103911644\\
0.92964824120603  0.803444636632317\\
0.934673366834171 0.816544021590237\\
0.939698492462312 0.829785020148849\\
0.944723618090452 0.843168393671598\\
0.949748743718593 0.856694903521928\\
0.954773869346734 0.870365311063283\\
0.959798994974874 0.88418037765911\\
0.964824120603015 0.89814086467285\\
0.969849246231156 0.912247533467951\\
0.974874371859296 0.926501145407855\\
0.979899497487437 0.940902461856009\\
0.984924623115578 0.955452244175855\\
0.989949748743719 0.970151253730839\\
0.994974874371859 0.985000251884406\\
1 1\\
};
\addlegendentry{$q=3$}

\addplot [color=mycolor3, style=semithick]
  table[row sep=crcr]{%
0 0\\
0.0050251256281407  3.20431068483455e-12\\
0.0100502512562814  1.02537941914706e-10\\
0.0150753768844221  7.78647496414796e-10\\
0.0201005025125628  3.28121414127058e-09\\
0.0251256281407035  1.0013470890108e-08\\
0.0301507537688442  2.49167198852735e-08\\
0.0351758793969849  5.38548496800143e-08\\
0.0402010050251256  1.04998852520659e-07\\
0.0452261306532663  1.89211341628796e-07\\
0.050251256281407 3.20431068483455e-07\\
0.0552763819095477  5.1605744010329e-07\\
0.0603015075376884  7.97335036328751e-07\\
0.0653266331658292  1.18973812710428e-06\\
0.0703517587939698  1.72335518976046e-06\\
0.0753768844221105  2.43327342629624e-06\\
0.0804020100502513  3.35996328066108e-06\\
0.085427135678392 4.54966295603713e-06\\
0.0904522613065327  6.05476293212146e-06\\
0.0954773869346734  7.93419048240815e-06\\
0.100502512562814 1.02537941914706e-05\\
0.105527638190955 1.30867284722435e-05\\
0.110552763819095 1.65138380833053e-05\\
0.115577889447236 2.06240426461601e-05\\
0.120603015075377 2.551472116252e-05\\
0.125628140703518 3.12920965315874e-05\\
0.130653266331658 3.80716200673368e-05\\
0.135678391959799 4.59783560157973e-05\\
0.14070351758794  5.51473660723347e-05\\
0.14572864321608  6.57240938989336e-05\\
0.150753768844221 7.78647496414796e-05\\
0.155778894472362 9.17366944470419e-05\\
0.160804020100503 0.000107518824981154\\
0.165829145728643 0.000125401957945099\\
0.170854271356784 0.000145589214593188\\
0.175879396984925 0.000168296405250045\\
0.180904522613065 0.000193752413827887\\
0.185929648241206 0.000222199582343808\\
0.190954773869347 0.000253894095437061\\
0.195979899497487 0.000289106364886339\\
0.201005025125628 0.000328121414127058\\
0.206030150753769 0.00037123926276864\\
0.21105527638191  0.000418775311111792\\
0.21608040201005  0.000471060724665791\\
0.221105527638191 0.000528442818665769\\
0.226130653266332 0.000591285442589986\\
0.231155778894472 0.000659969364677123\\
0.236180904522613 0.000734892656443555\\
0.241206030150754 0.000816471077200642\\
0.246231155778894 0.000905138458572001\\
0.251256281407035 0.0010013470890108\\
0.256281407035176 0.00110556809831702\\
0.261306532663317 0.00121829184215478\\
0.266331658291457 0.00134002828656955\\
0.271356783919598 0.00147130739250551\\
0.276381909547739 0.00161267950032278\\
0.281407035175879 0.00176471571431471\\
0.28643216080402  0.00192800828722518\\
0.291457286432161 0.00210317100476587\\
0.296482412060302 0.00229083957013355\\
0.301507537688442 0.00249167198852735\\
0.306532663316583 0.00270634895166604\\
0.311557788944724 0.00293557422230534\\
0.316582914572864 0.00318007501875517\\
0.321608040201005 0.00344060239939694\\
0.326633165829146 0.00371793164720086\\
0.331658291457286 0.00401286265424318\\
0.336683417085427 0.0043262203062235\\
0.341708542713568 0.00465885486698202\\
0.346733668341709 0.0050116423630169\\
0.351758793969849 0.00538548496800144\\
0.35678391959799  0.00578131138730141\\
0.361809045226131 0.00620007724249237\\
0.366834170854271 0.00664276545587689\\
0.371859296482412 0.00711038663500185\\
0.376884422110553 0.00760397945717575\\
0.381909547738693 0.00812461105398595\\
0.386934673366834 0.00867337739581599\\
0.391959798994975 0.00925140367636285\\
0.396984924623116 0.00985984469715424\\
0.402010050251256 0.0104998852520659\\
0.407035175879397 0.0111727405118387\\
0.412060301507538 0.0118796564085965\\
0.417085427135678 0.0126219100203625\\
0.422110552763819 0.0134008099555773\\
0.42713567839196  0.014217696737616\\
0.4321608040201 0.0150739431893053\\
0.437185929648241 0.0159709548174409\\
0.442211055276382 0.0169101701973046\\
0.447236180904523 0.017893061357182\\
0.452261306532663 0.0189211341628796\\
0.457286432160804 0.0199959287022416\\
0.462311557788945 0.0211190196696679\\
0.467336683417085 0.0222920167506312\\
0.472361809045226 0.0235165650061938\\
0.477386934673367 0.0247943452575255\\
0.482412060301508 0.0261270744704205\\
0.487437185929648 0.027516506139815\\
0.492462311557789 0.028964430674304\\
0.49748743718593  0.0304726757806592\\
0.50251256281407  0.0320431068483455\\
0.507537688442211 0.0336776273340393\\
0.512562814070352 0.0353781791461448\\
0.517587939698492 0.0371467430293118\\
0.522613065326633 0.0389853389489529\\
0.527638190954774 0.0408960264757609\\
0.532663316582915 0.0428809051702257\\
0.537688442211055 0.0449421149671521\\
0.542713567839196 0.0470818365601765\\
0.547738693467337 0.0493022917862847\\
0.552763819095477 0.0516057440103289\\
0.557788944723618 0.0539944985095453\\
0.562814070351759 0.0564709028580707\\
0.5678391959799 0.0590373473114606\\
0.57286432160804  0.0616962651912058\\
0.577889447236181 0.0644501332692502\\
0.582914572864322 0.067301472152508\\
0.587939698492462 0.0702528466673804\\
0.592964824120603 0.0733068662442737\\
0.597989949748744 0.0764661853021162\\
0.603015075376884 0.0797335036328751\\
0.608040201005025 0.0831115667860749\\
0.613065326633166 0.0866031664533133\\
0.618090452261307 0.0902111408527795\\
0.623115577889447 0.0939383751137709\\
0.628140703517588 0.0977878016612108\\
0.633165829145729 0.101762400600165\\
0.638190954773869 0.105865200100361\\
0.64321608040201  0.110099276780702\\
0.648241206030151 0.114467756093787\\
0.653266331658292 0.118973812710428\\
0.658291457286432 0.123620670904163\\
0.663316582914573 0.128411604935782\\
0.668341708542714 0.133349939437834\\
0.673366834170854 0.138439049799152\\
0.678391959798995 0.143682362549367\\
0.683417085427136 0.149083355743425\\
0.688442211055276 0.154645559346106\\
0.693467336683417 0.160372555616541\\
0.698492462311558 0.166267979492727\\
0.703517587939699 0.172335518976046\\
0.708542713567839 0.178578915515784\\
0.71356783919598  0.185001964393645\\
0.718592964824121 0.191608515108271\\
0.723618090452261 0.198402471759756\\
0.728643216080402 0.205387793434167\\
0.733668341708543 0.21256849458806\\
0.738693467336683 0.219948645432996\\
0.743718592964824 0.227532372320059\\
0.748743718592965 0.235323858124374\\
0.753768844221106 0.243327342629624\\
0.758793969849246 0.251547122912566\\
0.763819095477387 0.25998755372755\\
0.768844221105528 0.268653047891037\\
0.773869346733668 0.277548076666112\\
0.778894472361809 0.286677170147006\\
0.78391959798995  0.296044917643611\\
0.78894472361809  0.305655968065999\\
0.793969849246231 0.315515030308936\\
0.798994974874372 0.325626873636402\\
0.804020100502513 0.335996328066108\\
0.809045226130653 0.346628284754012\\
0.814070351758794 0.35752769637884\\
0.819095477386935 0.368699577526596\\
0.824120603015075 0.380149005075087\\
0.829145728643216 0.391881118578436\\
0.834170854271357 0.403901120651599\\
0.839195979899497 0.416214277354886\\
0.844221105527638 0.428825918578475\\
0.849246231155779 0.441741438426928\\
0.85427135678392  0.454966295603714\\
0.85929648241206  0.468506013795719\\
0.864321608040201 0.48236618205777\\
0.869346733668342 0.496552455197149\\
0.874371859296482 0.511070554158108\\
0.879396984924623 0.52592626640639\\
0.884422110552764 0.541125446313747\\
0.889447236180904 0.556674015542453\\
0.894472361809045 0.572577963429826\\
0.899497487437186 0.588843347372739\\
0.904522613065327 0.605476293212146\\
0.909547738693467 0.622482995617591\\
0.914572864321608 0.63986971847173\\
0.919597989949749 0.657642795254848\\
0.924623115577889 0.675808629429374\\
0.92964824120603  0.6943736948244\\
0.934673366834171 0.713344536020198\\
0.939698492462312 0.732727768732736\\
0.944723618090452 0.752530080198201\\
0.949748743718593 0.772758229557506\\
0.954773869346734 0.793419048240815\\
0.959798994974874 0.814519440352061\\
0.964824120603015 0.836066383053457\\
0.969849246231156 0.858066926950019\\
0.974874371859296 0.88052819647408\\
0.979899497487437 0.90345739026981\\
0.984924623115578 0.926861781577729\\
0.989949748743719 0.95074871861923\\
0.994974874371859 0.975125624981093\\
1 1\\
};
\addlegendentry{$q=5$}

\addplot [color=mycolor4, style=semithick]
  table[row sep=crcr]{%
0 0\\
0.0050251256281407  8.09148931803377e-17\\
0.0100502512562814  1.03571063270832e-14\\
0.0150753768844221  1.76960871385399e-13\\
0.0201005025125628  1.32570960986665e-12\\
0.0251256281407035  6.32147602971389e-12\\
0.0301507537688442  2.2650991537331e-11\\
0.0351758793969849  6.66368938744149e-11\\
0.0402010050251256  1.69690830062932e-10\\
0.0452261306532663  3.87013425719867e-10\\
0.050251256281407 8.09148931803377e-10\\
0.0552763819095477  1.57680235985198e-09\\
0.0603015075376884  2.89932691677837e-09\\
0.0653266331658292  5.07728955027961e-09\\
0.0703517587939698  8.5295224159251e-09\\
0.0753768844221105  1.38250680769843e-08\\
0.0804020100502513  2.17204262480552e-08\\
0.085427135678392 3.32025098935565e-08\\
0.0904522613065327  4.9537718492143e-08\\
0.0954773869346734  7.23275362781077e-08\\
0.100502512562814 1.03571063270832e-07\\
0.105527638190955 1.45734886903345e-07\\
0.110552763819095 2.01830702061053e-07\\
0.115577889447236 2.75501087341701e-07\\
0.120603015075377 3.71113845347631e-07\\
0.125628140703518 4.93865314821397e-07\\
0.130653266331658 6.4989306243579e-07\\
0.135678391959799 8.46398362049349e-07\\
0.14070351758794  1.09177886923841e-06\\
0.14572864321608  1.39577189891677e-06\\
0.150753768844221 1.76960871385398e-06\\
0.155778894472362 2.22618023190342e-06\\
0.160804020100503 2.78021455975107e-06\\
0.165829145728643 3.44846676099627e-06\\
0.170854271356784 4.24992126637523e-06\\
0.175879396984925 5.20600733393866e-06\\
0.180904522613065 6.3408279669943e-06\\
0.185929648241206 7.68140269762564e-06\\
0.190954773869347 9.25792464359779e-06\\
0.195979899497487 1.11040322464615e-05\\
0.201005025125628 1.32570960986665e-05\\
0.206030150753769 1.57585212674954e-05\\
0.21105527638191  1.86540655236282e-05\\
0.21608040201005  2.19941738821507e-05\\
0.221105527638191 2.58343298638148e-05\\
0.226130653266332 3.02354238843646e-05\\
0.231155778894472 3.52641391797377e-05\\
0.236180904522613 4.0993355674953e-05\\
0.241206030150754 4.75025722044968e-05\\
0.246231155778894 5.48783474920173e-05\\
0.251256281407035 6.32147602971388e-05\\
0.256281407035176 7.26138891372081e-05\\
0.261306532663317 8.31863119917811e-05\\
0.266331658291457 9.50516263976636e-05\\
0.271356783919598 0.000108338990342317\\
0.276381909547739 0.000123187684363435\\
0.281407035175879 0.000139747695262517\\
0.28643216080402  0.000158180321840222\\
0.291457286432161 0.000178658803061347\\
0.296482412060302 0.000201368969057218\\
0.301507537688442 0.00022650991537331\\
0.306532663316583 0.000254294700869911\\
0.311557788944724 0.000284951069683637\\
0.316582914572864 0.000318722197657616\\
0.321608040201005 0.000355867463648137\\
0.326633165829146 0.000396663246115594\\
0.331658291457286 0.000441403745407522\\
0.336683417085427 0.000490401832141544\\
0.341708542713568 0.00054398992209603\\
0.346733668341709 0.000602520878016299\\
0.351758793969849 0.000666368938744149\\
0.35678391959799  0.000735930676078544\\
0.361809045226131 0.00081162597977527\\
0.366834170854271 0.000893899071093354\\
0.371859296482412 0.000983219545296082\\
0.376884422110553 0.0010800834435144\\
0.381909547738693 0.00118501435438052\\
0.386934673366834 0.00129856454583958\\
0.391959798994975 0.00142131612754707\\
0.396984924623116 0.00155388224425998\\
0.402010050251256 0.00169690830062932\\
0.407035175879397 0.00185107321780192\\
0.412060301507538 0.00201709072223941\\
0.417085427135678 0.00219571066716187\\
0.422110552763819 0.00238772038702441\\
0.42713567839196  0.0025939460854341\\
0.4321608040201 0.00281525425691528\\
0.437185929648241 0.00305255314293098\\
0.442211055276382 0.00330679422256829\\
0.447236180904523 0.00357897373829547\\
0.452261306532663 0.00387013425719867\\
0.457286432160804 0.00418136626810592\\
0.462311557788945 0.00451380981500642\\
0.467336683417085 0.00486865616717277\\
0.472361809045226 0.00524714952639399\\
0.477386934673367 0.00565058877172717\\
0.482412060301508 0.00608032924217559\\
0.487437185929648 0.00653778455770105\\
0.492462311557789 0.00702442847897821\\
0.49748743718593  0.00754179680629884\\
0.50251256281407  0.00809148931803377\\
0.507537688442211 0.00867517174906025\\
0.512562814070352 0.00929457780956264\\
0.517587939698492 0.00995151124461424\\
0.522613065326633 0.010647847934948\\
0.527638190954774 0.0113855380393238\\
0.532663316582915 0.0121666081789009\\
0.537688442211055 0.0129931636640217\\
0.542713567839196 0.0138673907638165\\
0.547738693467337 0.0147915590190361\\
0.552763819095477 0.0157680235985197\\
0.557788944723618 0.0167992276997073\\
0.562814070351759 0.0178877049936022\\
0.5678391959799 0.0190360821145941\\
0.57286432160804  0.0202470811955484\\
0.577889447236181 0.0215235224485703\\
0.582914572864322 0.0228683267918524\\
0.587939698492462 0.0242845185230113\\
0.592964824120603 0.0257752280393239\\
0.597989949748744 0.0273436946052692\\
0.603015075376884 0.0289932691677837\\
0.608040201005025 0.030727417219639\\
0.613065326633166 0.0325497217113486\\
0.618090452261307 0.0344638860120123\\
0.623115577889447 0.0364737369195056\\
0.628140703517588 0.0385832277204217\\
0.633165829145729 0.0407964413001749\\
0.638190954773869 0.0431175933036722\\
0.64321608040201  0.0455510353469616\\
0.648241206030151 0.0481012582802635\\
0.653266331658292 0.0507728955027961\\
0.658291457286432 0.0535707263297984\\
0.663316582914573 0.0564996794121629\\
0.668341708542714 0.0595648362090817\\
0.673366834170854 0.0627714345141176\\
0.678391959798995 0.0661248720351054\\
0.683417085427136 0.0696307100282918\\
0.688442211055276 0.0732946769871232\\
0.693467336683417 0.0771226723860863\\
0.698492462311558 0.0811207704800124\\
0.703517587939699 0.0852952241592511\\
0.708542713567839 0.0896524688611222\\
0.71356783919598  0.0941991265380536\\
0.718592964824121 0.0989420096828118\\
0.723618090452261 0.103888125411235\\
0.728643216080402 0.109044679602873\\
0.733668341708543 0.114419081099949\\
0.738693467336683 0.120018945965042\\
0.743718592964824 0.125852101797898\\
0.748743718592965 0.131926592111796\\
0.753768844221106 0.138250680769843\\
0.758793969849246 0.14483285648164\\
0.763819095477387 0.151681837360706\\
0.768844221105528 0.158806575543074\\
0.773869346733668 0.166216261867466\\
0.778894472361809 0.173920330617455\\
0.78391959798995  0.181928464326025\\
0.78894472361809  0.190250598642933\\
0.793969849246231 0.198896927265278\\
0.798994974874372 0.20787790693169\\
0.804020100502513 0.217204262480553\\
0.809045226130653 0.226886991972646\\
0.814070351758794 0.236937371878646\\
0.819095477386935 0.247366962331864\\
0.824120603015075 0.258187612446644\\
0.829145728643216 0.269411465702834\\
0.834170854271357 0.281050965396719\\
0.839195979899497 0.293118860158845\\
0.844221105527638 0.305628209539124\\
0.849246231155779 0.318592389659642\\
0.85427135678392  0.332025098935565\\
0.85929648241206  0.345940363864564\\
0.864321608040201 0.360352544885156\\
0.869346733668342 0.375276342304373\\
0.874371859296482 0.390726802295166\\
0.879396984924623 0.406719322963958\\
0.884422110552764 0.423269660488741\\
0.889447236180904 0.440393935328136\\
0.894472361809045 0.45810863850182\\
0.899497487437186 0.476430637942727\\
0.904522613065327 0.49537718492143\\
0.909547738693467 0.514965920543115\\
0.914572864321608 0.535214882317557\\
0.919597989949749 0.556142510802495\\
0.924623115577889 0.577767656320822\\
0.92964824120603  0.600109585752003\\
0.934673366834171 0.623187989398115\\
0.939698492462312 0.647022987924928\\
0.944723618090452 0.67163513937843\\
0.949748743718593 0.697045446277207\\
0.954773869346734 0.723275362781077\\
0.959798994974874 0.750346801936404\\
0.964824120603015 0.778282142998476\\
0.969849246231156 0.807104238831374\\
0.974874371859296 0.836836423385735\\
0.979899497487437 0.867502519254805\\
0.984924623115578 0.899126845309211\\
0.989949748743719 0.931734224410841\\
0.994974874371859 0.965349991206251\\
1 1\\
};
\addlegendentry{$q=7$}

\addplot [color=mycolor5, style=dotted,semithick]
  table[row sep=crcr]{%
0 0\\
0.0050251256281407  2.04325378602403e-21\\
0.0100502512562814  1.0461459384443e-18\\
0.0150753768844221  4.0217364270311e-17\\
0.0201005025125628  5.35626720483484e-16\\
0.0251256281407035  3.99073005082819e-15\\
0.0301507537688442  2.05912905063992e-14\\
0.0351758793969849  8.24526602824759e-14\\
0.0402010050251256  2.74240880887544e-13\\
0.0452261306532663  7.91598380932532e-13\\
0.050251256281407 2.04325378602403e-12\\
0.0552763819095477  4.81788554688238e-12\\
0.0603015075376884  1.05427407392764e-11\\
0.0653266331658292  2.16676834927717e-11\\
0.0703517587939698  4.22157620646277e-11\\
0.0753768844221105  7.85495395904512e-11\\
0.0804020100502513  1.40411331014422e-10\\
0.085427135678392 2.42305127629046e-10\\
0.0904522613065327  4.05298371037457e-10\\
0.0954773869346734  6.5933286019032e-10\\
0.100502512562814 1.04614593844431e-09\\
0.105527638190955 1.62291571233997e-09\\
0.110552763819095 2.46675740000378e-09\\
0.115577889447236 3.68021199474154e-09\\
0.120603015075377 5.39788325850952e-09\\
0.125628140703518 7.7943946305238e-09\\
0.130653266331658 1.10938539482991e-08\\
0.135678391959799 1.5581030931895e-08\\
0.14070351758794  2.16144701770894e-08\\
0.14572864321608  2.96417809395976e-08\\
0.150753768844221 4.0217364270311e-08\\
0.155778894472362 5.4022858080836e-08\\
0.160804020100503 7.18906014793843e-08\\
0.165829145728643 9.48304412192858e-08\\
0.170854271356784 1.24060225346071e-07\\
0.175879396984925 1.61040352114211e-07\\
0.180904522613065 2.07512765971178e-07\\
0.185929648241206 2.65544816874561e-07\\
0.190954773869347 3.37578424417444e-07\\
0.195979899497487 4.26485014188226e-07\\
0.201005025125628 5.35626720483484e-07\\
0.206030150753769 6.68924376926332e-07\\
0.21105527638191  8.30932844718066e-07\\
0.21608040201005  1.02692425716766e-06\\
0.221105527638191 1.26297978880193e-06\\
0.226130653266332 1.54609058775885e-06\\
0.231155778894472 1.88426854130767e-06\\
0.236180904522613 2.2866675762221e-06\\
0.241206030150754 2.76371622835688e-06\\
0.246231155778894 3.32726224914354e-06\\
0.251256281407035 3.99073005082819e-06\\
0.256281407035176 4.76929182712251e-06\\
0.261306532663317 5.68005322152916e-06\\
0.266331658291457 6.74225445193397e-06\\
0.271356783919598 7.97748783713026e-06\\
0.276381909547739 9.40993270875463e-06\\
0.281407035175879 1.10666087306698e-05\\
0.28643216080402  1.29776486871261e-05\\
0.291457286432161 1.5176591841074e-05\\
0.296482412060302 1.77006990047771e-05\\
0.301507537688442 2.05912905063992e-05\\
0.306532663316583 2.38941082785015e-05\\
0.311557788944724 2.7659703337388e-05\\
0.316582914572864 3.19438499659877e-05\\
0.321608040201005 3.68079879574448e-05\\
0.326633165829146 4.23196943218198e-05\\
0.331658291457286 4.85531859042743e-05\\
0.336683417085427 5.55898544098227e-05\\
0.341708542713568 6.35188353771885e-05\\
0.346733668341709 7.24376126924977e-05\\
0.351758793969849 8.2452660282476e-05\\
0.35678391959799  9.36801226764965e-05\\
0.361809045226131 0.000106246536177243\\
0.366834170854271 0.00012028959243091\\
0.371859296482412 0.000135958946239775\\
0.376884422110553 0.0001534170695126\\
0.381909547738693 0.000172840153301731\\
0.386934673366834 0.000194419059929872\\
0.391959798994975 0.000218360327264372\\
0.396984924623116 0.000244887227252508\\
0.402010050251256 0.000274240880887544\\
0.407035175879397 0.00030668143183249\\
0.412060301507538 0.000342489280986282\\
0.417085427135678 0.000381966384335701\\
0.422110552763819 0.00042543761649565\\
0.42713567839196  0.00047325220240048\\
0.4321608040201 0.000525785219669843\\
0.437185929648241 0.000583439174234101\\
0.442211055276382 0.00064664565186659\\
0.447236180904523 0.000715867048333083\\
0.452261306532663 0.000791598380932532\\
0.457286432160804 0.000874369184267697\\
0.462311557788945 0.000964745493149526\\
0.467336683417085 0.00106333191560509\\
0.472361809045226 0.00117077379902571\\
0.477386934673367 0.00128775949255922\\
0.482412060301508 0.00141502270891872\\
0.487437185929648 0.001553344988849\\
0.492462311557789 0.00170355827156149\\
0.49748743718593  0.0018665475745192\\
0.50251256281407  0.00204325378602403\\
0.507537688442211 0.00223467657413105\\
0.512562814070352 0.00244187741548672\\
0.517587939698492 0.00266598274776174\\
0.522613065326633 0.00290818724942293\\
0.527638190954774 0.00316975725066401\\
0.532663316582915 0.00345203427939019\\
0.537688442211055 0.00375643874622824\\
0.542713567839196 0.00408447377261069\\
0.547738693467337 0.00443772916606065\\
0.552763819095477 0.00481788554688237\\
0.557788944723618 0.00522671863054199\\
0.562814070351759 0.00566610367010291\\
0.5678391959799 0.00613802006316133\\
0.57286432160804  0.00664455612780856\\
0.577889447236181 0.00718791405222956\\
0.582914572864322 0.00777041502262988\\
0.587939698492462 0.00839450453426685\\
0.592964824120603 0.00906275789044586\\
0.597989949748744 0.00977788589442736\\
0.603015075376884 0.0105427407392764\\
0.608040201005025 0.0113603221007736\\
0.613065326633166 0.0122337834385928\\
0.618090452261307 0.013166438511041\\
0.623115577889447 0.0141617681087427\\
0.628140703517588 0.0152234270127418\\
0.633165829145729 0.0163552511825857\\
0.638190954773869 0.0175612651800442\\
0.64321608040201  0.0188456898342117\\
0.648241206030151 0.0202129501538311\\
0.653266331658292 0.0216676834927717\\
0.658291457286432 0.0232147479746893\\
0.663316582914573 0.0248592311829885\\
0.668341708542714 0.0266064591223061\\
0.673366834170854 0.0284620054578292\\
0.678391959798995 0.0304317010388575\\
0.683417085427136 0.0325216437131205\\
0.688442211055276 0.0347382084384565\\
0.693467336683417 0.0370880576985588\\
0.698492462311558 0.0395781522295982\\
0.703517587939699 0.0422157620646277\\
0.708542713567839 0.0450084779027795\\
0.71356783919598  0.0479642228103662\\
0.718592964824121 0.0510912642610999\\
0.723618090452261 0.0543982265227484\\
0.728643216080402 0.0578941033976517\\
0.733668341708543 0.0615882713246262\\
0.738693467336683 0.0654905028498923\\
0.743718592964824 0.069610980474765\\
0.748743718592965 0.073960310887957\\
0.753768844221106 0.0785495395904513\\
0.758793969849246 0.0833901659210087\\
0.763819095477387 0.0884941584904864\\
0.768844221105528 0.0938739710332523\\
0.773869346733668 0.0995425586840942\\
0.778894472361809 0.105513394689133\\
0.78391959798995  0.111800487559358\\
0.78894472361809  0.11841839867553\\
0.793969849246231 0.125382260353284\\
0.798994974874372 0.132707794377416\\
0.804020100502513 0.140411331014422\\
0.809045226130653 0.148509828512486\\
0.814070351758794 0.157020893098235\\
0.819095477386935 0.165962799479692\\
0.824120603015075 0.175354511864976\\
0.829145728643216 0.185215705506418\\
0.834170854271357 0.195566788779879\\
0.839195979899497 0.206428925809197\\
0.844221105527638 0.217824059645773\\
0.849246231155779 0.22977493601346\\
0.85427135678392  0.242305127629046\\
0.85929648241206  0.255439059108702\\
0.864321608040201 0.269202032470959\\
0.869346733668342 0.283620253246826\\
0.874371859296482 0.298720857207859\\
0.879396984924623 0.314531937723068\\
0.884422110552764 0.331082573755694\\
0.889447236180904 0.348402858511027\\
0.894472361809045 0.366523928746539\\
0.899497487437186 0.385477994755762\\
0.904522613065327 0.405298371037457\\
0.909547738693467 0.426019507661751\\
0.914572864321608 0.447677022345061\\
0.919597989949749 0.470307733245745\\
0.924623115577889 0.493949692492557\\
0.92964824120603  0.518642220458128\\
0.934673366834171 0.544425940789808\\
0.939698492462312 0.571342816210368\\
0.944723618090452 0.599436185101165\\
0.949748743718593 0.628750798880536\\
0.954773869346734 0.659332860190321\\
0.959798994974874 0.691230061903537\\
0.964824120603015 0.724491626966385\\
0.969849246231156 0.759168349087898\\
0.974874371859296 0.795312634290687\\
0.979899497487437 0.832978543336379\\
0.984924623115578 0.872221835039485\\
0.989949748743719 0.913100010483582\\
0.994974874371859 0.955672358153831\\
1 1\\
};
\addlegendentry{$q=9$}
\end{axis}
\end{tikzpicture}%

\caption{\emph{Examples of GAs with different shape parameter values and PHSs of different degrees.}}
\label{fig:RBF}
\end{figure}
%

\par
We can apply the global RBF method by collocating at the same $\underline{x}_j$ points through substituting \eqref{eq:RBFint} into \eqref{eqPDE}. Thus, we obtain a dense linear system of ordinary differential equations (ODEs) of size $N$, where $\lambda_j(t)$ are the unknowns. 
%\begin{align}
%\frac{\partial}{\partial t} \mathbf{l}(t) + P\mathbf{l}(t)=0
%\end{align}
Starting from the terminal condition \eqref{eqTC}, we can use a backward time integration method of our choice to compute the coefficients $\lambda_j(t)$, and therefore evaluate the interpolant $\tilde u$ which approximates the option price.

\par
Even though the global RBF methods possess desirable properties such as spectral convergence and mesh-free domain discretization, they are featured with dense system matrices which makes the method very computationally demanding. To overcome that weakness, several localized RBF approaches were introduced, among which radial basis function partition of unity (RBF-PU) methods~\cite{wendland2002fast} and RBF-FD~\cite{tolstykh2000using, wright2006scattered}, are the most popular, and still actively developed. These localized RBF methods are featured with sparser system matrices while still maintaining great properties from the global RBF methods, such as being mesh-free and of high order.

\par
The RBF-PU method has been used in finance for pricing multi-asset derivatives~\cite{safdari2015radial, shcherbakov2016radialvanilla, shcherbakov2016radial}, and its performance when pricing one-dimensional options and their hedging parameters is also documented in \textbf{Paper \ref{paper5}}. Moreover, RBF-PU is extensively compared with the RBF-FD method at solving multiple stochastic factors problems, which is reported in \textbf{Paper \ref{paper4}}. While in that paper both methods performed similarly, on a more comprehensive study with stochastic and local volatility problems, presented in \textbf{Paper \ref{paper6}} --- RBF-FD showed as a robust method that performs more efficiently in most of the considered cases. As RBF-FD and RBF-PU are still in development, it is hard to say which method has a better future potential. Therefore, it is important for the field of computational finance that both of these methods continue developing.
%
%%





%
\section{Method Definition}

\par
In this thesis, we focus on development of the RBF-FD methods. RBF-FD can be seen as a kind of an FD method that belongs to the family of RBF methods. To construct an RBF-FD approximation, as firstly introduced by Andrei I. Tolstykh in 2000~\cite{tolstykh2000using}, we can reuse the same $N$ scattered nodes across the computational domain $\Omega$. For each node $\underline{x}_j$, we define an array of nodes $\mathbf{x}_j$ consisting of $n_j-1$ neighboring nodes and $\underline{x}_j$ itself, and consider it as a stencil of size $n_j$ centered at $\underline{x}_j$. The differential operator $\mathcal{L}$ defined in (\ref{eqPDE})  is approximated in every node  $\underline{x}_j$ as

\begin{equation}
\mathcal{L}u(\underline{x}_j, t)\approx\sum_{i=1}^{n_j}{w}_{j}^{i}u_j^{i}\equiv \mathbf{w}_j u(\mathbf{x}_j, t),\quad j=1,\ldots,N,
\label{eqRBFFD}
\end{equation}
where $u_j^{i}=u(\underline{x}_j^i,t)$ and $\underline{x}_j^i$ is a locally indexed node in $\mathbf{x}_j$, while $\mathbf{w}_j$ is the array of differentiation weights for the stencil centered at $\underline{x}_j$. In the standard RBF-FD methods, the weights ${w}_j^i$ are calculated by enforcing (\ref{eqRBFFD}) to be exact for RBFs centered at each of the nodes in $\mathbf{x}_j$, yielding
\begin{equation}
\label{eqRBFFDmat}
{\footnotesize{
\underbrace{\left[\begin{array}{cccc}
\phi(\|\underline{x}_j^{1}-\underline{x}_j^{1}\|) & \ldots & \phi(\|\underline{x}_j^{1}-\underline{x}_{j}^{n_j}\|)\\
\vdots & \ddots & \vdots\\
\phi(\|\underline{x}_{j}^{n_j}-\underline{x}_j^{1}\|) & \ldots & \phi(\|\underline{x}_{j}^{n_j}-\underline{x}_{j}^{n_j}\|)
\end{array}\right]}_{\mathbf{A}_j} \underbrace{\left[\begin{array}{c}
{w}_j^{1}\\
\vdots\\
{w}_{j}^{n_j}
\end{array}\right]}_{\mathbf{w}_j}=
\underbrace{\left[\begin{array}{c}
\mathcal{L}\phi(\|\underline{x}_{j}-\underline{x}_j^{1}\|)\\
\vdots \\
\mathcal{L}\phi(\|\underline{x}_{j}-\underline{x}_{{j}}^{n_j}\|)
\end{array}\right]}_{\mathbf{l}_j}.}}
\end{equation}
In theory on RBF interpolation, it is known that (\ref{eqRBFFDmat}) forms a nonsingular system of equations. Therefore, a unique set of weights can be computed for each node $\underline{x}_j$ by solving $N$ linear systems of size $n_j\times n_j$. We arrange those weights in a differentiation matrix $\mathbf{L}$, which now represents a discrete approximation of the spatial operator $\mathcal{L}$ on the chosen set of nodes $\{\underline{x}_j\}_{j=1}^N$. Since $n_j \ll N$, the resulting differentiation matrix is sparse, having $n_j$ non-zero elements per row. It is important to note that the linear systems \eqref{eqRBFFDmat} can be solved in parallel, which significantly reduces the weights computation overhead that RBF-FD has compared to FD. Moreover, when it comes to the boundary nodes and the nodes that are close to the boundary, the nearest neighbor based stencils automatically form according to the shape of the boundary and require no special treatment for computing the differentiation weights --- which can be seen in \textbf{Figure \ref{fig:gridsten}}. The only data that is required for approximation of differential operators are Euclidian distances between the nodes. This means that (\ref{eqRBFFDmat}) represents a way to approximate a differential operator in any number of dimensions.

\par
After the weights are computed and stored in the differentiation matrix $\mathbf{L}$, an approximation of (\ref{eqPDE}) can be presented in the form of the following semi-discrete equation
\begin{align}
\label{eqdRBFFD}
\frac{\mathrm{d}}{\mathrm{d} t}\mathbf{u}(t)&=\mathbf{L}\mathbf{u}(t),\\
\mathbf{u}(T)&=\mathbf{g},
\end{align}
where $\mathbf{u}(t)\equiv u(\mathbf{x},t)$ is the discrete numerical solution of the pricing equation, $\mathbf{g}\equiv g(\mathbf{x})$, while $\mathbf{x}$ is the array of all nodes in the computational domain. To compute the option price $\mathbf{u}$, we need to integrate \eqref{eqdRBFFD} in time. 

\begin{figure}[H]
\centering
% This file was created by matlab2tikz.
%
%The latest updates can be retrieved from
%  http://www.mathworks.com/matlabcentral/fileexchange/22022-matlab2tikz-matlab2tikz
%where you can also make suggestions and rate matlab2tikz.
%
\definecolor{mycolor1}{rgb}{0.00000,0.44700,0.74100}%
\definecolor{mycolor2}{rgb}{0.85000,0.32500,0.09800}%
\definecolor{mycolor3}{rgb}{0.92900,0.69400,0.12500}%
\definecolor{mycolor4}{rgb}{0.49400,0.18400,0.55600}%
\definecolor{mycolor5}{rgb}{0.46600,0.67400,0.18800}%
\definecolor{mycolor6}{rgb}{0.30100,0.74500,0.93300}%
\definecolor{mycolor7}{rgb}{0.63500,0.07800,0.18400}%
%
\begin{tikzpicture}[trim axis left, trim axis right]

\begin{axis}[%
axis x line*=bottom,
axis y line*=left,
width=0.75\textwidth,
height=0.75\textwidth,
xmin=0,
xmax=8,
ymin=0,
ymax=8,
xtick={0,1,2,8},
xticklabels={$0$,$K$,$2K$,$8K$},
% xticklabels={},
ytick={0,1,2,8},
yticklabels={$0$,$K$,$2K$,$8K$},
% yticklabels={},
xlabel={$s_1$},
ylabel={$s_2$},
]
\addplot [color=black,mark size=0.5pt,only marks,mark=*,mark options={solid},forget plot]
  table[row sep=crcr]{%
  0	0\\
  0	0.180451956615514\\
  0.180451956615514	0\\
  0	0.347061520741313\\
  0.173530760370657	0.173530760370657\\
  0.347061520741313	0\\
  0	0.501096191187092\\
  0.167032063729031	0.334064127458061\\
  0.334064127458061	0.167032063729031\\
  0.501096191187092	0\\
  0	0.643727802009712\\
  0.160931950502428	0.482795851507284\\
  0.321863901004856	0.321863901004856\\
  0.482795851507284	0.160931950502428\\
  0.643727802009712	0\\
  0	0.776041437357198\\
  0.15520828747144	0.620833149885759\\
  0.310416574942879	0.465624862414319\\
  0.465624862414319	0.310416574942879\\
  0.620833149885759	0.15520828747144\\
  0.776041437357198	0\\
  0	0.899043686354185\\
  0.149840614392364	0.749203071961821\\
  0.299681228784728	0.599362457569457\\
  0.449521843177093	0.449521843177093\\
  0.599362457569457	0.299681228784728\\
  0.749203071961821	0.149840614392364\\
  0.899043686354185	0\\
  0	1.01367030082868\\
  0.144810042975525	0.86886025785315\\
  0.28962008595105	0.724050214877625\\
  0.434430128926575	0.5792401719021\\
  0.5792401719021	0.434430128926575\\
  0.724050214877625	0.28962008595105\\
  0.86886025785315	0.144810042975525\\
  1.01367030082868	0\\
  0	1.12079331413702\\
  0.140099164267128	0.980694149869895\\
  0.280198328534256	0.840594985602767\\
  0.420297492801383	0.700495821335639\\
  0.560396657068511	0.560396657068511\\
  0.700495821335639	0.420297492801383\\
  0.840594985602767	0.280198328534256\\
  0.980694149869895	0.140099164267128\\
  1.12079331413702	0\\
  0	1.22122767524432\\
  0.135691963916036	1.08553571132829\\
  0.271383927832072	0.949843747412251\\
  0.407075891748107	0.814151783496215\\
  0.542767855664143	0.678459819580179\\
  0.678459819580179	0.542767855664143\\
  0.814151783496215	0.407075891748107\\
  0.949843747412251	0.271383927832072\\
  1.08553571132829	0.135691963916036\\
  1.22122767524432	0\\
  0	1.31573744852961\\
  0.131573744852961	1.18416370367665\\
  0.263147489705923	1.05258995882369\\
  0.394721234558884	0.921016213970729\\
  0.526294979411845	0.789442469117768\\
  0.657868724264807	0.657868724264807\\
  0.789442469117768	0.526294979411845\\
  0.921016213970729	0.394721234558884\\
  1.05258995882369	0.263147489705923\\
  1.18416370367665	0.131573744852961\\
  1.31573744852961	0\\
  0	1.40504162648154\\
  0.127731056952868	1.27731056952868\\
  0.255462113905735	1.14957951257581\\
  0.383193170858603	1.02184845562294\\
  0.510924227811471	0.894117398670074\\
  0.638655284764338	0.766386341717206\\
  0.766386341717206	0.638655284764338\\
  0.894117398670074	0.510924227811471\\
  1.02184845562294	0.383193170858603\\
  1.14957951257581	0.255462113905735\\
  1.27731056952868	0.127731056952868\\
  1.40504162648154	0\\
  0	1.48981959950515\\
  0.124151633292096	1.36566796621306\\
  0.248303266584192	1.24151633292096\\
  0.372454899876289	1.11736469962887\\
  0.496606533168385	0.99321306633677\\
  0.620758166460481	0.869061433044674\\
  0.744909799752577	0.744909799752577\\
  0.869061433044674	0.620758166460481\\
  0.99321306633677	0.496606533168385\\
  1.11736469962887	0.372454899876289\\
  1.24151633292096	0.248303266584192\\
  1.36566796621306	0.124151633292096\\
  1.48981959950515	0\\
  0	1.57071632445187\\
  0.120824332650144	1.44989199180172\\
  0.241648665300287	1.32906765915158\\
  0.362472997950431	1.20824332650144\\
  0.483297330600574	1.08741899385129\\
  0.604121663250718	0.966594661201149\\
  0.724945995900862	0.845770328551005\\
  0.845770328551005	0.724945995900862\\
  0.966594661201149	0.604121663250718\\
  1.08741899385129	0.483297330600574\\
  1.20824332650144	0.362472997950431\\
  1.32906765915158	0.241648665300287\\
  1.44989199180172	0.120824332650144\\
  1.57071632445187	0\\
  0	1.64834723119281\\
  0.117739087942343	1.53060814325046\\
  0.235478175884687	1.41286905530812\\
  0.35321726382703	1.29512996736578\\
  0.470956351769373	1.17739087942343\\
  0.588695439711716	1.05965179148109\\
  0.70643452765406	0.941912703538746\\
  0.824173615596403	0.824173615596403\\
  0.941912703538746	0.70643452765406\\
  1.05965179148109	0.588695439711716\\
  1.17739087942343	0.470956351769373\\
  1.29512996736578	0.35321726382703\\
  1.41286905530812	0.235478175884687\\
  1.53060814325046	0.117739087942343\\
  1.64834723119281	0\\
  0	1.72330290456268\\
  0.114886860304178	1.6084160442585\\
  0.229773720608357	1.49352918395432\\
  0.344660580912535	1.37864232365014\\
  0.459547441216714	1.26375546334596\\
  0.574434301520892	1.14886860304178\\
  0.689321161825071	1.03398174273761\\
  0.804208022129249	0.919094882433428\\
  0.919094882433428	0.804208022129249\\
  1.03398174273761	0.689321161825071\\
  1.14886860304178	0.574434301520892\\
  1.26375546334596	0.459547441216714\\
  1.37864232365014	0.344660580912535\\
  1.49352918395432	0.229773720608357\\
  1.6084160442585	0.114886860304178\\
  1.72330290456268	0\\
  0	1.79615357729256\\
  0.112259598580785	1.68389397871178\\
  0.22451919716157	1.57163438013099\\
  0.336778795742356	1.45937478155021\\
  0.449038394323141	1.34711518296942\\
  0.561297992903926	1.23485558438864\\
  0.673557591484711	1.12259598580785\\
  0.785817190065496	1.01033638722707\\
  0.898076788646281	0.898076788646281\\
  1.01033638722707	0.785817190065496\\
  1.12259598580785	0.673557591484711\\
  1.23485558438864	0.561297992903926\\
  1.34711518296942	0.449038394323141\\
  1.45937478155021	0.336778795742356\\
  1.57163438013099	0.22451919716157\\
  1.68389397871178	0.112259598580785\\
  1.79615357729256	0\\
  0	1.86745346811207\\
  0.109850204006593	1.75760326410548\\
  0.219700408013185	1.64775306009889\\
  0.329550612019778	1.5379028560923\\
  0.43940081602637	1.4280526520857\\
  0.549251020032963	1.31820244807911\\
  0.659101224039555	1.20835224407252\\
  0.768951428046148	1.09850204006593\\
  0.87880163205274	0.988651836059333\\
  0.988651836059333	0.87880163205274\\
  1.09850204006593	0.768951428046148\\
  1.20835224407252	0.659101224039555\\
  1.31820244807911	0.549251020032963\\
  1.4280526520857	0.43940081602637\\
  1.5379028560923	0.329550612019778\\
  1.64775306009889	0.219700408013185\\
  1.75760326410548	0.109850204006593\\
  1.86745346811207	0\\
  0	1.93774499802338\\
  0.107652499890188	1.83009249813319\\
  0.215304999780376	1.72243999824301\\
  0.322957499670564	1.61478749835282\\
  0.430609999560752	1.50713499846263\\
  0.538262499450939	1.39948249857244\\
  0.645914999341127	1.29182999868225\\
  0.753567499231315	1.18417749879207\\
  0.861219999121503	1.07652499890188\\
  0.968872499011691	0.968872499011691\\
  1.07652499890188	0.861219999121503\\
  1.18417749879207	0.753567499231315\\
  1.29182999868225	0.645914999341127\\
  1.39948249857244	0.53826249945094\\
  1.50713499846263	0.430609999560752\\
  1.61478749835282	0.322957499670564\\
  1.72243999824301	0.215304999780376\\
  1.83009249813319	0.107652499890188\\
  1.93774499802338	0\\
  0	2.00756291682289\\
  0.105661206148573	1.90190171067431\\
  0.211322412297146	1.79624050452574\\
  0.316983618445719	1.69057929837717\\
  0.422644824594292	1.5849180922286\\
  0.528306030742865	1.47925688608002\\
  0.633967236891438	1.37359567993145\\
  0.739628443040011	1.26793447378288\\
  0.845289649188584	1.1622732676343\\
  0.950950855337157	1.05661206148573\\
  1.05661206148573	0.950950855337157\\
  1.1622732676343	0.845289649188584\\
  1.26793447378288	0.739628443040011\\
  1.37359567993145	0.633967236891438\\
  1.47925688608002	0.528306030742865\\
  1.5849180922286	0.422644824594292\\
  1.69057929837717	0.316983618445719\\
  1.79624050452574	0.211322412297146\\
  1.90190171067431	0.105661206148573\\
  2.00756291682289	0\\
  0	2.0774383712634\\
  0.10387191856317	1.97356645270023\\
  0.20774383712634	1.86969453413706\\
  0.311615755689511	1.76582261557389\\
  0.415487674252681	1.66195069701072\\
  0.519359592815851	1.55807877844755\\
  0.623231511379021	1.45420685988438\\
  0.727103429942191	1.35033494132121\\
  0.830975348505362	1.24646302275804\\
  0.934847267068532	1.14259110419487\\
  1.0387191856317	1.0387191856317\\
  1.14259110419487	0.934847267068532\\
  1.24646302275804	0.830975348505362\\
  1.35033494132121	0.727103429942192\\
  1.45420685988438	0.623231511379021\\
  1.55807877844755	0.519359592815851\\
  1.66195069701072	0.415487674252681\\
  1.76582261557389	0.311615755689511\\
  1.86969453413706	0.20774383712634\\
  1.97356645270023	0.10387191856317\\
  2.0774383712634	0\\
  0	2.14790294580586\\
  0.102281092657422	2.04562185314843\\
  0.204562185314843	1.94334076049101\\
  0.306843277972265	1.84105966783359\\
  0.409124370629687	1.73877857517617\\
  0.511405463287108	1.63649748251875\\
  0.61368655594453	1.53421638986133\\
  0.715967648601952	1.4319352972039\\
  0.818248741259374	1.32965420454648\\
  0.920529833916795	1.22737311188906\\
  1.02281092657422	1.12509201923164\\
  1.12509201923164	1.02281092657422\\
  1.22737311188906	0.920529833916795\\
  1.32965420454648	0.818248741259374\\
  1.4319352972039	0.715967648601952\\
  1.53421638986133	0.61368655594453\\
  1.63649748251875	0.511405463287109\\
  1.73877857517617	0.409124370629687\\
  1.84105966783359	0.306843277972265\\
  1.94334076049101	0.204562185314843\\
  2.04562185314843	0.102281092657422\\
  2.14790294580586	0\\
  0	2.21949270670086\\
  0.100886032122766	2.11860667457809\\
  0.201772064245533	2.01772064245533\\
  0.302658096368299	1.91683461033256\\
  0.403544128491065	1.81594857820979\\
  0.504430160613832	1.71506254608703\\
  0.605316192736598	1.61417651396426\\
  0.706202224859364	1.5132904818415\\
  0.807088256982131	1.41240444971873\\
  0.907974289104897	1.31151841759596\\
  1.00886032122766	1.2106323854732\\
  1.10974635335043	1.10974635335043\\
  1.2106323854732	1.00886032122766\\
  1.31151841759596	0.907974289104897\\
  1.41240444971873	0.807088256982131\\
  1.51329048184149	0.706202224859364\\
  1.61417651396426	0.605316192736598\\
  1.71506254608703	0.504430160613832\\
  1.81594857820979	0.403544128491065\\
  1.91683461033256	0.302658096368299\\
  2.01772064245533	0.201772064245533\\
  2.11860667457809	0.100886032122766\\
  2.21949270670086	0\\
  0	2.29275228016598\\
  0.0996848817463471	2.19306739841964\\
  0.199369763492694	2.09338251667329\\
  0.299054645239041	1.99369763492694\\
  0.398739526985389	1.8940127531806\\
  0.498424408731736	1.79432787143425\\
  0.598109290478083	1.6946429896879\\
  0.69779417222443	1.59495810794155\\
  0.797479053970777	1.49527322619521\\
  0.897163935717124	1.39558834444886\\
  0.996848817463471	1.29590346270251\\
  1.09653369920982	1.19621858095617\\
  1.19621858095617	1.09653369920982\\
  1.29590346270251	0.996848817463472\\
  1.39558834444886	0.897163935717124\\
  1.49527322619521	0.797479053970777\\
  1.59495810794155	0.69779417222443\\
  1.6946429896879	0.598109290478083\\
  1.79432787143425	0.498424408731736\\
  1.8940127531806	0.398739526985389\\
  1.99369763492694	0.299054645239041\\
  2.09338251667329	0.199369763492694\\
  2.19306739841964	0.0996848817463469\\
  2.29275228016598	0\\
  0	2.3682389956838\\
  0.0986766248201584	2.26956237086364\\
  0.197353249640317	2.17088574604349\\
  0.296029874460475	2.07220912122333\\
  0.394706499280634	1.97353249640317\\
  0.493383124100792	1.87485587158301\\
  0.592059748920951	1.77617924676285\\
  0.690736373741109	1.67750262194269\\
  0.789412998561267	1.57882599712253\\
  0.888089623381426	1.48014937230238\\
  0.986766248201584	1.38147274748222\\
  1.08544287302174	1.28279612266206\\
  1.1841194978419	1.1841194978419\\
  1.28279612266206	1.08544287302174\\
  1.38147274748222	0.986766248201584\\
  1.48014937230238	0.888089623381426\\
  1.57882599712253	0.789412998561267\\
  1.67750262194269	0.690736373741109\\
  1.77617924676285	0.59205974892095\\
  1.87485587158301	0.493383124100792\\
  1.97353249640317	0.394706499280634\\
  2.07220912122333	0.296029874460475\\
  2.17088574604349	0.197353249640317\\
  2.26956237086364	0.0986766248201585\\
  2.3682389956838	0\\
  0	2.44652712594129\\
  0.0978610850376515	2.34866604090364\\
  0.195722170075303	2.25080495586599\\
  0.293583255112955	2.15294387082833\\
  0.391444340150606	2.05508278579068\\
  0.489305425188258	1.95722170075303\\
  0.587166510225909	1.85936061571538\\
  0.685027595263561	1.76149953067773\\
  0.782888680301212	1.66363844564008\\
  0.880749765338864	1.56577736060242\\
  0.978610850376516	1.46791627556477\\
  1.07647193541417	1.37005519052712\\
  1.17433302045182	1.27219410548947\\
  1.27219410548947	1.17433302045182\\
  1.37005519052712	1.07647193541417\\
  1.46791627556477	0.978610850376515\\
  1.56577736060242	0.880749765338864\\
  1.66363844564008	0.782888680301213\\
  1.76149953067773	0.685027595263561\\
  1.85936061571538	0.587166510225909\\
  1.95722170075303	0.489305425188258\\
  2.05508278579068	0.391444340150606\\
  2.15294387082833	0.293583255112955\\
  2.25080495586599	0.195722170075303\\
  2.34866604090364	0.0978610850376516\\
  2.44652712594129	0\\
  0	2.52821225566634\\
  0.0972389329102438	2.4309733227561\\
  0.194477865820488	2.33373438984585\\
  0.291716798730731	2.23649545693561\\
  0.388955731640975	2.13925652402536\\
  0.486194664551219	2.04201759111512\\
  0.583433597461463	1.94477865820488\\
  0.680672530371707	1.84753972529463\\
  0.77791146328195	1.75030079238439\\
  0.875150396192194	1.65306185947414\\
  0.972389329102438	1.5558229265639\\
  1.06962826201268	1.45858399365366\\
  1.16686719492293	1.36134506074341\\
  1.26410612783317	1.26410612783317\\
  1.36134506074341	1.16686719492293\\
  1.45858399365366	1.06962826201268\\
  1.5558229265639	0.972389329102438\\
  1.65306185947414	0.875150396192194\\
  1.75030079238439	0.77791146328195\\
  1.84753972529463	0.680672530371707\\
  1.94477865820488	0.583433597461463\\
  2.04201759111512	0.486194664551219\\
  2.13925652402536	0.388955731640975\\
  2.23649545693561	0.291716798730731\\
  2.33373438984585	0.194477865820488\\
  2.4309733227561	0.0972389329102441\\
  2.52821225566634	0\\
  0	2.61391581259775\\
  0.0968116967628795	2.51710411583487\\
  0.193623393525759	2.42029241907199\\
  0.290435090288639	2.32348072230911\\
  0.387246787051518	2.22666902554623\\
  0.484058483814397	2.12985732878335\\
  0.580870180577277	2.03304563202047\\
  0.677681877340157	1.93623393525759\\
  0.774493574103036	1.83942223849471\\
  0.871305270865915	1.74261054173183\\
  0.968116967628795	1.64579884496895\\
  1.06492866439167	1.54898714820607\\
  1.16174036115455	1.45217545144319\\
  1.25855205791743	1.35536375468031\\
  1.35536375468031	1.25855205791743\\
  1.45217545144319	1.16174036115455\\
  1.54898714820607	1.06492866439167\\
  1.64579884496895	0.968116967628795\\
  1.74261054173183	0.871305270865915\\
  1.83942223849471	0.774493574103036\\
  1.93623393525759	0.677681877340157\\
  2.03304563202047	0.580870180577277\\
  2.12985732878335	0.484058483814398\\
  2.22666902554623	0.387246787051518\\
  2.32348072230911	0.290435090288638\\
  2.42029241907199	0.193623393525759\\
  2.51710411583487	0.0968116967628796\\
  2.61391581259775	0\\
  0	2.70428979505846\\
  0.096581778394945	2.60770801666351\\
  0.19316355678989	2.51112623826857\\
  0.289745335184835	2.41454445987362\\
  0.38632711357978	2.31796268147868\\
  0.482908891974725	2.22138090308373\\
  0.57949067036967	2.12479912468879\\
  0.676072448764615	2.02821734629384\\
  0.77265422715956	1.9316355678989\\
  0.869236005554505	1.83505378950395\\
  0.96581778394945	1.73847201110901\\
  1.06239956234439	1.64189023271406\\
  1.15898134073934	1.54530845431912\\
  1.25556311913428	1.44872667592417\\
  1.35214489752923	1.35214489752923\\
  1.44872667592417	1.25556311913428\\
  1.54530845431912	1.15898134073934\\
  1.64189023271406	1.06239956234439\\
  1.73847201110901	0.96581778394945\\
  1.83505378950395	0.869236005554505\\
  1.9316355678989	0.77265422715956\\
  2.02821734629384	0.676072448764615\\
  2.12479912468879	0.57949067036967\\
  2.22138090308373	0.482908891974725\\
  2.31796268147868	0.38632711357978\\
  2.41454445987362	0.289745335184835\\
  2.51112623826857	0.19316355678989\\
  2.60770801666351	0.096581778394945\\
  2.70428979505846	0\\
  0	2.80002173209759\\
  0.0965524735206065	2.70346925857698\\
  0.193104947041213	2.60691678505637\\
  0.289657420561819	2.51036431153577\\
  0.386209894082426	2.41381183801516\\
  0.482762367603032	2.31725936449456\\
  0.579314841123639	2.22070689097395\\
  0.675867314644245	2.12415441745334\\
  0.772419788164852	2.02760194393274\\
  0.868972261685458	1.93104947041213\\
  0.965524735206065	1.83449699689152\\
  1.06207720872667	1.73794452337092\\
  1.15862968224728	1.64139204985031\\
  1.25518215576788	1.5448395763297\\
  1.35173462928849	1.4482871028091\\
  1.4482871028091	1.35173462928849\\
  1.5448395763297	1.25518215576788\\
  1.64139204985031	1.15862968224728\\
  1.73794452337092	1.06207720872667\\
  1.83449699689152	0.965524735206065\\
  1.93104947041213	0.868972261685458\\
  2.02760194393274	0.772419788164852\\
  2.12415441745334	0.675867314644246\\
  2.22070689097395	0.579314841123639\\
  2.31725936449456	0.482762367603032\\
  2.41381183801516	0.386209894082425\\
  2.51036431153577	0.28965742056182\\
  2.60691678505637	0.193104947041213\\
  2.70346925857698	0.0965524735206063\\
  2.80002173209759	0\\
  0	2.90183991393594\\
  0.0967279971311982	2.80511191680475\\
  0.193455994262396	2.70838391967355\\
  0.290183991393595	2.61165592254235\\
  0.386911988524793	2.51492792541115\\
  0.483639985655991	2.41819992827995\\
  0.580367982787189	2.32147193114876\\
  0.677095979918387	2.22474393401756\\
  0.773823977049585	2.12801593688636\\
  0.870551974180783	2.03128793975516\\
  0.967279971311982	1.93455994262396\\
  1.06400796844318	1.83783194549277\\
  1.16073596557438	1.74110394836157\\
  1.25746396270558	1.64437595123037\\
  1.35419195983677	1.54764795409917\\
  1.45091995696797	1.45091995696797\\
  1.54764795409917	1.35419195983677\\
  1.64437595123037	1.25746396270558\\
  1.74110394836157	1.16073596557438\\
  1.83783194549277	1.06400796844318\\
  1.93455994262396	0.967279971311982\\
  2.03128793975516	0.870551974180783\\
  2.12801593688636	0.773823977049585\\
  2.22474393401756	0.677095979918387\\
  2.32147193114876	0.580367982787189\\
  2.41819992827995	0.483639985655991\\
  2.51492792541115	0.386911988524793\\
  2.61165592254235	0.290183991393595\\
  2.70838391967355	0.193455994262396\\
  2.80511191680475	0.0967279971311981\\
  2.90183991393594	0\\
  0	3.01051893250625\\
  0.0971135139518144	2.91340541855443\\
  0.194227027903629	2.81629190460262\\
  0.291340541855443	2.7191783906508\\
  0.388454055807258	2.62206487669899\\
  0.485567569759072	2.52495136274717\\
  0.582681083710886	2.42783784879536\\
  0.679794597662701	2.33072433484355\\
  0.776908111614515	2.23361082089173\\
  0.87402162556633	2.13649730693992\\
  0.971135139518144	2.0393837929881\\
  1.06824865346996	1.94227027903629\\
  1.16536216742177	1.84515676508447\\
  1.26247568137359	1.74804325113266\\
  1.3595891953254	1.65092973718085\\
  1.45670270927722	1.55381622322903\\
  1.55381622322903	1.45670270927722\\
  1.65092973718084	1.3595891953254\\
  1.74804325113266	1.26247568137359\\
  1.84515676508447	1.16536216742177\\
  1.94227027903629	1.06824865346996\\
  2.0393837929881	0.971135139518144\\
  2.13649730693992	0.87402162556633\\
  2.23361082089173	0.776908111614516\\
  2.33072433484355	0.679794597662701\\
  2.42783784879536	0.582681083710886\\
  2.52495136274717	0.485567569759072\\
  2.62206487669899	0.388454055807258\\
  2.7191783906508	0.291340541855444\\
  2.81629190460262	0.194227027903629\\
  2.91340541855443	0.0971135139518142\\
  3.01051893250625	0\\
  0	3.12688557423819\\
  0.0977151741949434	3.02917040004325\\
  0.195430348389887	2.9314552258483\\
  0.29314552258483	2.83374005165336\\
  0.390860696779774	2.73602487745842\\
  0.488575870974717	2.63830970326347\\
  0.58629104516966	2.54059452906853\\
  0.684006219364604	2.44287935487359\\
  0.781721393559547	2.34516418067864\\
  0.879436567754491	2.2474490064837\\
  0.977151741949434	2.14973383228876\\
  1.07486691614438	2.05201865809381\\
  1.17258209033932	1.95430348389887\\
  1.27029726453426	1.85658830970392\\
  1.36801243872921	1.75887313550898\\
  1.46572761292415	1.66115796131404\\
  1.56344278711909	1.56344278711909\\
  1.66115796131404	1.46572761292415\\
  1.75887313550898	1.36801243872921\\
  1.85658830970392	1.27029726453426\\
  1.95430348389887	1.17258209033932\\
  2.05201865809381	1.07486691614438\\
  2.14973383228876	0.977151741949434\\
  2.2474490064837	0.879436567754491\\
  2.34516418067864	0.781721393559547\\
  2.44287935487359	0.684006219364604\\
  2.54059452906853	0.586291045169661\\
  2.63830970326347	0.488575870974717\\
  2.73602487745842	0.390860696779773\\
  2.83374005165336	0.29314552258483\\
  2.9314552258483	0.195430348389887\\
  3.02917040004325	0.0977151741949434\\
  3.12688557423819	0\\
  0	3.25182510991834\\
  0.0985401548460102	3.15328495507233\\
  0.19708030969202	3.05474480022632\\
  0.295620464538031	2.95620464538031\\
  0.394160619384041	2.8576644905343\\
  0.492700774230051	2.75912433568829\\
  0.591240929076061	2.66058418084228\\
  0.689781083922072	2.56204402599627\\
  0.788321238768082	2.46350387115026\\
  0.886861393614092	2.36496371630425\\
  0.985401548460102	2.26642356145824\\
  1.08394170330611	2.16788340661222\\
  1.18248185815212	2.06934325176621\\
  1.28102201299813	1.9708030969202\\
  1.37956216784414	1.87226294207419\\
  1.47810232269015	1.77372278722818\\
  1.57664247753616	1.67518263238217\\
  1.67518263238217	1.57664247753616\\
  1.77372278722818	1.47810232269015\\
  1.87226294207419	1.37956216784414\\
  1.9708030969202	1.28102201299813\\
  2.06934325176621	1.18248185815212\\
  2.16788340661223	1.08394170330611\\
  2.26642356145824	0.985401548460102\\
  2.36496371630425	0.886861393614092\\
  2.46350387115026	0.788321238768082\\
  2.56204402599627	0.689781083922072\\
  2.66058418084228	0.591240929076061\\
  2.75912433568829	0.492700774230051\\
  2.8576644905343	0.394160619384041\\
  2.95620464538031	0.295620464538031\\
  3.05474480022632	0.197080309692021\\
  3.15328495507233	0.0985401548460101\\
  3.25182510991834	0\\
  0	3.38628802947558\\
  0.0995967067492818	3.2866913227263\\
  0.199193413498564	3.18709461597702\\
  0.298790120247845	3.08749790922774\\
  0.398386826997127	2.98790120247845\\
  0.497983533746409	2.88830449572917\\
  0.597580240495691	2.78870778897989\\
  0.697176947244972	2.68911108223061\\
  0.796773653994254	2.58951437548133\\
  0.896370360743536	2.48991766873204\\
  0.995967067492818	2.39032096198276\\
  1.0955637742421	2.29072425523348\\
  1.19516048099138	2.1911275484842\\
  1.29475718774066	2.09153084173492\\
  1.39435389448994	1.99193413498564\\
  1.49395060123923	1.89233742823635\\
  1.59354730798851	1.79274072148707\\
  1.69314401473779	1.69314401473779\\
  1.79274072148707	1.59354730798851\\
  1.89233742823635	1.49395060123923\\
  1.99193413498564	1.39435389448994\\
  2.09153084173492	1.29475718774066\\
  2.1911275484842	1.19516048099138\\
  2.29072425523348	1.0955637742421\\
  2.39032096198276	0.995967067492818\\
  2.48991766873204	0.896370360743536\\
  2.58951437548133	0.796773653994254\\
  2.68911108223061	0.697176947244972\\
  2.78870778897989	0.597580240495691\\
  2.88830449572917	0.497983533746409\\
  2.98790120247845	0.398386826997127\\
  3.08749790922774	0.298790120247845\\
  3.18709461597702	0.199193413498564\\
  3.2866913227263	0.0995967067492818\\
  3.38628802947558	0\\
  0	3.53129727292769\\
  0.100894207797934	3.43040306512975\\
  0.201788415595868	3.32950885733182\\
  0.302682623393802	3.22861464953388\\
  0.403576831191736	3.12772044173595\\
  0.504471038989669	3.02682623393802\\
  0.605365246787603	2.92593202614008\\
  0.706259454585537	2.82503781834215\\
  0.807153662383471	2.72414361054421\\
  0.908047870181405	2.62324940274628\\
  1.00894207797934	2.52235519494835\\
  1.10983628577727	2.42146098715041\\
  1.21073049357521	2.32056677935248\\
  1.31162470137314	2.21967257155455\\
  1.41251890917107	2.11877836375661\\
  1.51341311696901	2.01788415595868\\
  1.61430732476694	1.91698994816074\\
  1.71520153256488	1.81609574036281\\
  1.81609574036281	1.71520153256488\\
  1.91698994816074	1.61430732476694\\
  2.01788415595868	1.51341311696901\\
  2.11877836375661	1.41251890917107\\
  2.21967257155455	1.31162470137314\\
  2.32056677935248	1.21073049357521\\
  2.42146098715041	1.10983628577727\\
  2.52235519494835	1.00894207797934\\
  2.62324940274628	0.908047870181405\\
  2.72414361054422	0.807153662383471\\
  2.82503781834215	0.706259454585537\\
  2.92593202614008	0.605365246787604\\
  3.02682623393802	0.504471038989669\\
  3.12772044173595	0.403576831191736\\
  3.22861464953388	0.302682623393802\\
  3.32950885733182	0.201788415595868\\
  3.43040306512975	0.100894207797934\\
  3.53129727292769	0\\
  0	3.68795601249909\\
  0.102443222569419	3.58551278992967\\
  0.204886445138839	3.48306956736026\\
  0.307329667708258	3.38062634479084\\
  0.409772890277677	3.27818312222142\\
  0.512216112847096	3.175739899652\\
  0.614659335416516	3.07329667708258\\
  0.717102557985935	2.97085345451316\\
  0.819545780555354	2.86841023194374\\
  0.921989003124773	2.76596700937432\\
  1.02443222569419	2.6635237868049\\
  1.12687544826361	2.56108056423548\\
  1.22931867083303	2.45863734166606\\
  1.33176189340245	2.35619411909664\\
  1.43420511597187	2.25375089652722\\
  1.53664833854129	2.1513076739578\\
  1.63909156111071	2.04886445138839\\
  1.74153478368013	1.94642122881897\\
  1.84397800624955	1.84397800624955\\
  1.94642122881897	1.74153478368013\\
  2.04886445138839	1.63909156111071\\
  2.1513076739578	1.53664833854129\\
  2.25375089652722	1.43420511597187\\
  2.35619411909664	1.33176189340245\\
  2.45863734166606	1.22931867083303\\
  2.56108056423548	1.12687544826361\\
  2.6635237868049	1.02443222569419\\
  2.76596700937432	0.921989003124774\\
  2.86841023194374	0.819545780555354\\
  2.97085345451316	0.717102557985935\\
  3.07329667708258	0.614659335416516\\
  3.175739899652	0.512216112847097\\
  3.27818312222142	0.409772890277677\\
  3.38062634479084	0.307329667708258\\
  3.48306956736026	0.204886445138839\\
  3.58551278992968	0.102443222569419\\
  3.68795601249909	0\\
  0	3.85745604511321\\
  0.104255568786844	3.75320047632637\\
  0.208511137573687	3.64894490753952\\
  0.312766706360531	3.54468933875268\\
  0.417022275147374	3.44043376996584\\
  0.521277843934218	3.33617820117899\\
  0.625533412721061	3.23192263239215\\
  0.729788981507905	3.12766706360531\\
  0.834044550294748	3.02341149481846\\
  0.938300119081592	2.91915592603162\\
  1.04255568786844	2.81490035724477\\
  1.14681125665528	2.71064478845793\\
  1.25106682544212	2.60638921967109\\
  1.35532239422897	2.50213365088424\\
  1.45957796301581	2.3978780820974\\
  1.56383353180265	2.29362251331056\\
  1.6680891005895	2.18936694452371\\
  1.77234466937634	2.08511137573687\\
  1.87660023816318	1.98085580695003\\
  1.98085580695003	1.87660023816318\\
  2.08511137573687	1.77234466937634\\
  2.18936694452371	1.6680891005895\\
  2.29362251331056	1.56383353180265\\
  2.3978780820974	1.45957796301581\\
  2.50213365088424	1.35532239422897\\
  2.60638921967109	1.25106682544212\\
  2.71064478845793	1.14681125665528\\
  2.81490035724477	1.04255568786844\\
  2.91915592603162	0.938300119081592\\
  3.02341149481846	0.834044550294748\\
  3.12766706360531	0.729788981507904\\
  3.23192263239215	0.625533412721061\\
  3.33617820117899	0.521277843934218\\
  3.44043376996584	0.417022275147374\\
  3.54468933875268	0.31276670636053\\
  3.64894490753952	0.208511137573687\\
  3.75320047632637	0.104255568786844\\
  3.85745604511321	0\\
  0	4.04108685910588\\
  0.106344391029102	3.93474246807678\\
  0.212688782058204	3.82839807704768\\
  0.319033173087307	3.72205368601858\\
  0.425377564116409	3.61570929498947\\
  0.531721955145511	3.50936490396037\\
  0.638066346174613	3.40302051293127\\
  0.744410737203715	3.29667612190217\\
  0.850755128232818	3.19033173087307\\
  0.95709951926192	3.08398733984396\\
  1.06344391029102	2.97764294881486\\
  1.16978830132012	2.87129855778576\\
  1.27613269234923	2.76495416675666\\
  1.38247708337833	2.65860977572755\\
  1.48882147440743	2.55226538469845\\
  1.59516586543653	2.44592099366935\\
  1.70151025646564	2.33957660264025\\
  1.80785464749474	2.23323221161115\\
  1.91419903852384	2.12688782058204\\
  2.02054342955294	2.02054342955294\\
  2.12688782058204	1.91419903852384\\
  2.23323221161115	1.80785464749474\\
  2.33957660264025	1.70151025646563\\
  2.44592099366935	1.59516586543653\\
  2.55226538469845	1.48882147440743\\
  2.65860977572755	1.38247708337833\\
  2.76495416675666	1.27613269234923\\
  2.87129855778576	1.16978830132012\\
  2.97764294881486	1.06344391029102\\
  3.08398733984396	0.95709951926192\\
  3.19033173087307	0.850755128232818\\
  3.29667612190217	0.744410737203716\\
  3.40302051293127	0.638066346174613\\
  3.50936490396037	0.531721955145511\\
  3.61570929498947	0.425377564116408\\
  3.72205368601858	0.319033173087307\\
  3.82839807704768	0.212688782058204\\
  3.93474246807678	0.106344391029102\\
  4.04108685910588	0\\
  0	4.24024544413602\\
  0.108724242157334	4.13152120197869\\
  0.217448484314668	4.02279695982135\\
  0.326172726472002	3.91407271766402\\
  0.434896968629336	3.80534847550669\\
  0.543621210786669	3.69662423334935\\
  0.652345452944003	3.58789999119202\\
  0.761069695101337	3.47917574903468\\
  0.869793937258671	3.37045150687735\\
  0.978518179416005	3.26172726472002\\
  1.08724242157334	3.15300302256268\\
  1.19596666373067	3.04427878040535\\
  1.30469090588801	2.93555453824802\\
  1.41341514804534	2.82683029609068\\
  1.52213939020267	2.71810605393335\\
  1.63086363236001	2.60938181177601\\
  1.73958787451734	2.50065756961868\\
  1.84831211667468	2.39193332746135\\
  1.95703635883201	2.28320908530401\\
  2.06576060098934	2.17448484314668\\
  2.17448484314668	2.06576060098934\\
  2.28320908530401	1.95703635883201\\
  2.39193332746135	1.84831211667468\\
  2.50065756961868	1.73958787451734\\
  2.60938181177601	1.63086363236001\\
  2.71810605393335	1.52213939020267\\
  2.82683029609068	1.41341514804534\\
  2.93555453824802	1.30469090588801\\
  3.04427878040535	1.19596666373067\\
  3.15300302256268	1.08724242157334\\
  3.26172726472002	0.978518179416005\\
  3.37045150687735	0.869793937258671\\
  3.47917574903468	0.761069695101337\\
  3.58789999119202	0.652345452944004\\
  3.69662423334935	0.543621210786669\\
  3.80534847550669	0.434896968629335\\
  3.91407271766402	0.326172726472002\\
  4.02279695982135	0.217448484314668\\
  4.13152120197869	0.108724242157334\\
  4.24024544413602	0\\
  0	4.45644691892319\\
  0.11141117297308	4.34503574595011\\
  0.22282234594616	4.23362457297703\\
  0.334233518919239	4.12221340000395\\
  0.445644691892319	4.01080222703087\\
  0.557055864865399	3.89939105405779\\
  0.668467037838479	3.78797988108471\\
  0.779878210811559	3.67656870811163\\
  0.891289383784638	3.56515753513855\\
  1.00270055675772	3.45374636216547\\
  1.1141117297308	3.34233518919239\\
  1.22552290270388	3.23092401621931\\
  1.33693407567696	3.11951284324623\\
  1.44834524865004	3.00810167027315\\
  1.55975642162312	2.89669049730007\\
  1.6711675945962	2.78527932432699\\
  1.78257876756928	2.67386815135391\\
  1.89398994054236	2.56245697838084\\
  2.00540111351544	2.45104580540776\\
  2.11681228648852	2.33963463243468\\
  2.2282234594616	2.2282234594616\\
  2.33963463243468	2.11681228648852\\
  2.45104580540776	2.00540111351544\\
  2.56245697838084	1.89398994054236\\
  2.67386815135391	1.78257876756928\\
  2.78527932432699	1.6711675945962\\
  2.89669049730007	1.55975642162312\\
  3.00810167027315	1.44834524865004\\
  3.11951284324623	1.33693407567696\\
  3.23092401621931	1.22552290270388\\
  3.34233518919239	1.1141117297308\\
  3.45374636216547	1.00270055675772\\
  3.56515753513855	0.891289383784638\\
  3.67656870811163	0.779878210811559\\
  3.78797988108471	0.668467037838479\\
  3.89939105405779	0.557055864865398\\
  4.01080222703087	0.445644691892319\\
  4.12221340000395	0.334233518919239\\
  4.23362457297703	0.22282234594616\\
  4.34503574595011	0.11141117297308\\
  4.45644691892319	0\\
  0	4.69133605766379\\
  0.114422830674727	4.57691322698906\\
  0.228845661349453	4.46249039631433\\
  0.34326849202418	4.34806756563961\\
  0.457691322698906	4.23364473496488\\
  0.572114153373633	4.11922190429016\\
  0.686536984048359	4.00479907361543\\
  0.800959814723086	3.8903762429407\\
  0.915382645397812	3.77595341226598\\
  1.02980547607254	3.66153058159125\\
  1.14422830674727	3.54710775091652\\
  1.25865113742199	3.4326849202418\\
  1.37307396809672	3.31826208956707\\
  1.48749679877145	3.20383925889234\\
  1.60191962944617	3.08941642821762\\
  1.7163424601209	2.97499359754289\\
  1.83076529079562	2.86057076686816\\
  1.94518812147035	2.74614793619344\\
  2.05961095214508	2.63172510551871\\
  2.1740337828198	2.51730227484398\\
  2.28845661349453	2.40287944416926\\
  2.40287944416926	2.28845661349453\\
  2.51730227484398	2.1740337828198\\
  2.63172510551871	2.05961095214508\\
  2.74614793619344	1.94518812147035\\
  2.86057076686816	1.83076529079562\\
  2.97499359754289	1.7163424601209\\
  3.08941642821762	1.60191962944617\\
  3.20383925889234	1.48749679877145\\
  3.31826208956707	1.37307396809672\\
  3.4326849202418	1.25865113742199\\
  3.54710775091652	1.14422830674727\\
  3.66153058159125	1.02980547607254\\
  3.77595341226598	0.915382645397812\\
  3.8903762429407	0.800959814723086\\
  4.00479907361543	0.686536984048359\\
  4.11922190429016	0.572114153373633\\
  4.23364473496488	0.457691322698906\\
  4.34806756563961	0.34326849202418\\
  4.46249039631433	0.228845661349453\\
  4.57691322698906	0.114422830674727\\
  4.69133605766379	0\\
  0	4.94669980281412\\
  0.11777856673367	4.82892123608045\\
  0.235557133467339	4.71114266934678\\
  0.353335700201009	4.59336410261311\\
  0.471114266934678	4.47558553587944\\
  0.588892833668348	4.35780696914577\\
  0.706671400402017	4.2400284024121\\
  0.824449967135687	4.12224983567843\\
  0.942228533869356	4.00447126894476\\
  1.06000710060303	3.88669270221109\\
  1.1777856673367	3.76891413547742\\
  1.29556423407036	3.65113556874376\\
  1.41334280080403	3.53335700201009\\
  1.5311213675377	3.41557843527642\\
  1.64889993427137	3.29779986854275\\
  1.76667850100504	3.18002130180908\\
  1.88445706773871	3.06224273507541\\
  2.00223563447238	2.94446416834174\\
  2.12001420120605	2.82668560160807\\
  2.23779276793972	2.7089070348744\\
  2.35557133467339	2.59112846814073\\
  2.47334990140706	2.47334990140706\\
  2.59112846814073	2.35557133467339\\
  2.7089070348744	2.23779276793972\\
  2.82668560160807	2.12001420120605\\
  2.94446416834174	2.00223563447238\\
  3.06224273507541	1.88445706773871\\
  3.18002130180908	1.76667850100504\\
  3.29779986854275	1.64889993427137\\
  3.41557843527642	1.5311213675377\\
  3.53335700201009	1.41334280080403\\
  3.65113556874376	1.29556423407036\\
  3.76891413547742	1.1777856673367\\
  3.88669270221109	1.06000710060303\\
  4.00447126894476	0.942228533869356\\
  4.12224983567843	0.824449967135687\\
  4.2400284024121	0.706671400402016\\
  4.35780696914577	0.588892833668348\\
  4.47558553587944	0.471114266934678\\
  4.59336410261311	0.353335700201009\\
  4.71114266934678	0.235557133467339\\
  4.82892123608045	0.117778566733669\\
  4.94669980281412	0\\
  0	5.22448085943266\\
  0.121499554870527	5.10298130456213\\
  0.242999109741054	4.9814817496916\\
  0.364498664611581	4.85998219482108\\
  0.485998219482108	4.73848263995055\\
  0.607497774352635	4.61698308508002\\
  0.728997329223162	4.4954835302095\\
  0.850496884093689	4.37398397533897\\
  0.971996438964216	4.25248442046844\\
  1.09349599383474	4.13098486559792\\
  1.21499554870527	4.00948531072739\\
  1.3364951035758	3.88798575585686\\
  1.45799465844632	3.76648620098633\\
  1.57949421331685	3.64498664611581\\
  1.70099376818738	3.52348709124528\\
  1.8224933230579	3.40198753637475\\
  1.94399287792843	3.28048798150423\\
  2.06549243279896	3.1589884266337\\
  2.18699198766948	3.03748887176317\\
  2.30849154254001	2.91598931689265\\
  2.42999109741054	2.79448976202212\\
  2.55149065228107	2.67299020715159\\
  2.67299020715159	2.55149065228107\\
  2.79448976202212	2.42999109741054\\
  2.91598931689265	2.30849154254001\\
  3.03748887176317	2.18699198766948\\
  3.1589884266337	2.06549243279896\\
  3.28048798150423	1.94399287792843\\
  3.40198753637475	1.8224933230579\\
  3.52348709124528	1.70099376818738\\
  3.64498664611581	1.57949421331685\\
  3.76648620098634	1.45799465844632\\
  3.88798575585686	1.3364951035758\\
  4.00948531072739	1.21499554870527\\
  4.13098486559792	1.09349599383474\\
  4.25248442046844	0.971996438964216\\
  4.37398397533897	0.850496884093689\\
  4.4954835302095	0.728997329223161\\
  4.61698308508002	0.607497774352635\\
  4.73848263995055	0.485998219482108\\
  4.85998219482108	0.36449866461158\\
  4.9814817496916	0.242999109741054\\
  5.10298130456213	0.121499554870527\\
  5.22448085943266	0\\
  0	5.52679247450175\\
  0.12560891987504	5.40118355462671\\
  0.25121783975008	5.27557463475167\\
  0.376826759625119	5.14996571487663\\
  0.502435679500159	5.02435679500159\\
  0.628044599375199	4.89874787512655\\
  0.753653519250239	4.77313895525151\\
  0.879262439125279	4.64753003537647\\
  1.00487135900032	4.52192111550143\\
  1.13048027887536	4.39631219562639\\
  1.2560891987504	4.27070327575135\\
  1.38169811862544	4.14509435587631\\
  1.50730703850048	4.01948543600127\\
  1.63291595837552	3.89387651612623\\
  1.75852487825056	3.76826759625119\\
  1.8841337981256	3.64265867637615\\
  2.00974271800064	3.51704975650111\\
  2.13535163787568	3.39144083662608\\
  2.26096055775072	3.26583191675104\\
  2.38656947762576	3.140222996876\\
  2.5121783975008	3.01461407700096\\
  2.63778731737584	2.88900515712592\\
  2.76339623725088	2.76339623725088\\
  2.88900515712592	2.63778731737584\\
  3.01461407700096	2.5121783975008\\
  3.140222996876	2.38656947762576\\
  3.26583191675104	2.26096055775072\\
  3.39144083662607	2.13535163787568\\
  3.51704975650111	2.00974271800064\\
  3.64265867637615	1.8841337981256\\
  3.76826759625119	1.75852487825056\\
  3.89387651612623	1.63291595837552\\
  4.01948543600127	1.50730703850048\\
  4.14509435587631	1.38169811862544\\
  4.27070327575135	1.2560891987504\\
  4.39631219562639	1.13048027887536\\
  4.52192111550143	1.00487135900032\\
  4.64753003537647	0.879262439125279\\
  4.77313895525151	0.753653519250239\\
  4.89874787512655	0.628044599375199\\
  5.02435679500159	0.502435679500159\\
  5.14996571487663	0.37682675962512\\
  5.27557463475167	0.25121783975008\\
  5.40118355462671	0.12560891987504\\
  5.52679247450175	0\\
  0	5.85593451366397\\
  0.130131878081422	5.72580263558255\\
  0.260263756162843	5.59567075750113\\
  0.390395634244265	5.46553887941971\\
  0.520527512325686	5.33540700133829\\
  0.650659390407108	5.20527512325686\\
  0.78079126848853	5.07514324517544\\
  0.910923146569951	4.94501136709402\\
  1.04105502465137	4.8148794890126\\
  1.17118690273279	4.68474761093118\\
  1.30131878081422	4.55461573284976\\
  1.43145065889564	4.42448385476833\\
  1.56158253697706	4.29435197668691\\
  1.69171441505848	4.16422009860549\\
  1.8218462931399	4.03408822052407\\
  1.95197817122132	3.90395634244265\\
  2.08211004930275	3.77382446436123\\
  2.21224192738417	3.6436925862798\\
  2.34237380546559	3.51356070819838\\
  2.47250568354701	3.38342883011696\\
  2.60263756162843	3.25329695203554\\
  2.73276943970985	3.12316507395412\\
  2.86290131779127	2.9930331958727\\
  2.9930331958727	2.86290131779128\\
  3.12316507395412	2.73276943970985\\
  3.25329695203554	2.60263756162843\\
  3.38342883011696	2.47250568354701\\
  3.51356070819838	2.34237380546559\\
  3.6436925862798	2.21224192738417\\
  3.77382446436123	2.08211004930275\\
  3.90395634244265	1.95197817122132\\
  4.03408822052407	1.8218462931399\\
  4.16422009860549	1.69171441505848\\
  4.29435197668691	1.56158253697706\\
  4.42448385476833	1.43145065889564\\
  4.55461573284976	1.30131878081422\\
  4.68474761093118	1.17118690273279\\
  4.8148794890126	1.04105502465137\\
  4.94501136709402	0.910923146569951\\
  5.07514324517544	0.780791268488529\\
  5.20527512325686	0.650659390407109\\
  5.33540700133829	0.520527512325686\\
  5.46553887941971	0.390395634244265\\
  5.59567075750113	0.260263756162844\\
  5.72580263558255	0.130131878081422\\
  5.85593451366397	0\\
  0	6.21441095767845\\
  0.135095890384314	6.07931506729413\\
  0.270191780768628	5.94421917690982\\
  0.405287671152942	5.80912328652551\\
  0.540383561537256	5.67402739614119\\
  0.67547945192157	5.53893150575688\\
  0.810575342305885	5.40383561537256\\
  0.945671232690199	5.26873972498825\\
  1.08076712307451	5.13364383460394\\
  1.21586301345883	4.99854794421962\\
  1.35095890384314	4.86345205383531\\
  1.48605479422745	4.72835616345099\\
  1.62115068461177	4.59326027306668\\
  1.75624657499608	4.45816438268236\\
  1.8913424653804	4.32306849229805\\
  2.02643835576471	4.18797260191374\\
  2.16153424614903	4.05287671152942\\
  2.29663013653334	3.91778082114511\\
  2.43172602691765	3.78268493076079\\
  2.56682191730197	3.64758904037648\\
  2.70191780768628	3.51249314999217\\
  2.8370136980706	3.37739725960785\\
  2.97210958845491	3.24230136922354\\
  3.10720547883922	3.10720547883922\\
  3.24230136922354	2.97210958845491\\
  3.37739725960785	2.8370136980706\\
  3.51249314999217	2.70191780768628\\
  3.64758904037648	2.56682191730197\\
  3.78268493076079	2.43172602691765\\
  3.91778082114511	2.29663013653334\\
  4.05287671152942	2.16153424614903\\
  4.18797260191374	2.02643835576471\\
  4.32306849229805	1.8913424653804\\
  4.45816438268236	1.75624657499608\\
  4.59326027306668	1.62115068461177\\
  4.72835616345099	1.48605479422745\\
  4.86345205383531	1.35095890384314\\
  4.99854794421962	1.21586301345883\\
  5.13364383460394	1.08076712307451\\
  5.26873972498825	0.945671232690199\\
  5.40383561537256	0.810575342305885\\
  5.53893150575688	0.675479451921571\\
  5.67402739614119	0.540383561537256\\
  5.80912328652551	0.405287671152943\\
  5.94421917690982	0.270191780768628\\
  6.07931506729413	0.135095890384314\\
  6.21441095767845	0\\
  0	6.60494895170324\\
  0.140530828759643	6.46441812294359\\
  0.281061657519287	6.32388729418395\\
  0.42159248627893	6.18335646542431\\
  0.562123315038573	6.04282563666466\\
  0.702654143798217	5.90229480790502\\
  0.84318497255786	5.76176397914538\\
  0.983715801317503	5.62123315038573\\
  1.12424663007715	5.48070232162609\\
  1.26477745883679	5.34017149286645\\
  1.40530828759643	5.1996406641068\\
  1.54583911635608	5.05910983534716\\
  1.68636994511572	4.91857900658752\\
  1.82690077387536	4.77804817782787\\
  1.96743160263501	4.63751734906823\\
  2.10796243139465	4.49698652030859\\
  2.24849326015429	4.35645569154894\\
  2.38902408891394	4.2159248627893\\
  2.52955491767358	4.07539403402966\\
  2.67008574643322	3.93486320527001\\
  2.81061657519287	3.79433237651037\\
  2.95114740395251	3.65380154775073\\
  3.09167823271215	3.51327071899108\\
  3.2322090614718	3.37273989023144\\
  3.37273989023144	3.2322090614718\\
  3.51327071899108	3.09167823271215\\
  3.65380154775073	2.95114740395251\\
  3.79433237651037	2.81061657519287\\
  3.93486320527001	2.67008574643322\\
  4.07539403402966	2.52955491767358\\
  4.2159248627893	2.38902408891394\\
  4.35645569154894	2.24849326015429\\
  4.49698652030859	2.10796243139465\\
  4.63751734906823	1.96743160263501\\
  4.77804817782787	1.82690077387536\\
  4.91857900658752	1.68636994511572\\
  5.05910983534716	1.54583911635608\\
  5.1996406641068	1.40530828759643\\
  5.34017149286645	1.26477745883679\\
  5.48070232162609	1.12424663007715\\
  5.62123315038573	0.983715801317503\\
  5.76176397914538	0.843184972557861\\
  5.90229480790502	0.702654143798217\\
  6.04282563666466	0.562123315038574\\
  6.18335646542431	0.42159248627893\\
  6.32388729418395	0.281061657519287\\
  6.46441812294359	0.140530828759643\\
  6.60494895170324	0\\
  0	7.03051955232302\\
  0.146469157340063	6.88405039498295\\
  0.292938314680126	6.73758123764289\\
  0.439407472020189	6.59111208030283\\
  0.585876629360251	6.44464292296276\\
  0.732345786700314	6.2981737656227\\
  0.878814944040377	6.15170460828264\\
  1.02528410138044	6.00523545094258\\
  1.1717532587205	5.85876629360251\\
  1.31822241606057	5.71229713626245\\
  1.46469157340063	5.56582797892239\\
  1.61116073074069	5.41935882158232\\
  1.75762988808075	5.27288966424226\\
  1.90409904542082	5.1264205069022\\
  2.05056820276088	4.97995134956214\\
  2.19703736010094	4.83348219222207\\
  2.34350651744101	4.68701303488201\\
  2.48997567478107	4.54054387754195\\
  2.63644483212113	4.39407472020189\\
  2.78291398946119	4.24760556286182\\
  2.92938314680126	4.10113640552176\\
  3.07585230414132	3.9546672481817\\
  3.22232146148138	3.80819809084163\\
  3.36879061882144	3.66172893350157\\
  3.51525977616151	3.51525977616151\\
  3.66172893350157	3.36879061882144\\
  3.80819809084163	3.22232146148138\\
  3.9546672481817	3.07585230414132\\
  4.10113640552176	2.92938314680126\\
  4.24760556286182	2.78291398946119\\
  4.39407472020188	2.63644483212113\\
  4.54054387754195	2.48997567478107\\
  4.68701303488201	2.34350651744101\\
  4.83348219222207	2.19703736010094\\
  4.97995134956214	2.05056820276088\\
  5.1264205069022	1.90409904542082\\
  5.27288966424226	1.75762988808075\\
  5.41935882158232	1.61116073074069\\
  5.56582797892239	1.46469157340063\\
  5.71229713626245	1.31822241606057\\
  5.85876629360251	1.1717532587205\\
  6.00523545094258	1.02528410138044\\
  6.15170460828264	0.878814944040377\\
  6.2981737656227	0.732345786700313\\
  6.44464292296276	0.585876629360252\\
  6.59111208030283	0.439407472020188\\
  6.73758123764289	0.292938314680126\\
  6.88405039498295	0.146469157340062\\
  7.03051955232302	0\\
  0	7.49436033015719\\
  0.152946129186881	7.34141420097031\\
  0.305892258373763	7.18846807178343\\
  0.458838387560644	7.03552194259655\\
  0.611784516747526	6.88257581340967\\
  0.764730645934407	6.72962968422279\\
  0.917676775121289	6.5766835550359\\
  1.07062290430817	6.42373742584902\\
  1.22356903349505	6.27079129666214\\
  1.37651516268193	6.11784516747526\\
  1.52946129186881	5.96489903828838\\
  1.6824074210557	5.8119529091015\\
  1.83535355024258	5.65900677991461\\
  1.98829967942946	5.50606065072773\\
  2.14124580861634	5.35311452154085\\
  2.29419193780322	5.20016839235397\\
  2.4471380669901	5.04722226316709\\
  2.60008419617699	4.89427613398021\\
  2.75303032536387	4.74133000479333\\
  2.90597645455075	4.58838387560644\\
  3.05892258373763	4.43543774641956\\
  3.21186871292451	4.28249161723268\\
  3.36481484211139	4.1295454880458\\
  3.51776097129827	3.97659935885892\\
  3.67070710048516	3.82365322967204\\
  3.82365322967204	3.67070710048516\\
  3.97659935885892	3.51776097129827\\
  4.1295454880458	3.36481484211139\\
  4.28249161723268	3.21186871292451\\
  4.43543774641956	3.05892258373763\\
  4.58838387560644	2.90597645455075\\
  4.74133000479333	2.75303032536387\\
  4.89427613398021	2.60008419617699\\
  5.04722226316709	2.4471380669901\\
  5.20016839235397	2.29419193780322\\
  5.35311452154085	2.14124580861634\\
  5.50606065072773	1.98829967942946\\
  5.65900677991461	1.83535355024258\\
  5.8119529091015	1.6824074210557\\
  5.96489903828838	1.52946129186881\\
  6.11784516747526	1.37651516268193\\
  6.27079129666214	1.22356903349505\\
  6.42373742584902	1.07062290430817\\
  6.5766835550359	0.917676775121288\\
  6.72962968422279	0.764730645934407\\
  6.88257581340967	0.611784516747527\\
  7.03552194259655	0.458838387560644\\
  7.18846807178343	0.305892258373762\\
  7.34141420097031	0.152946129186882\\
  7.49436033015719	0\\
  0	8\\
  0.16	7.84\\
  0.32	7.68\\
  0.48	7.52\\
  0.64	7.36\\
  0.8	7.2\\
  0.96	7.04\\
  1.12	6.88\\
  1.28	6.72\\
  1.44	6.56\\
  1.6	6.4\\
  1.76	6.24\\
  1.92	6.08\\
  2.08	5.92\\
  2.24	5.76\\
  2.4	5.6\\
  2.56	5.44\\
  2.72	5.28\\
  2.88	5.12\\
  3.04	4.96\\
  3.2	4.8\\
  3.36	4.64\\
  3.52	4.48\\
  3.68	4.32\\
  3.84	4.16\\
  4	4\\
  4.16	3.84\\
  4.32	3.68\\
  4.48	3.52\\
  4.64	3.36\\
  4.8	3.2\\
  4.96	3.04\\
  5.12	2.88\\
  5.28	2.72\\
  5.44	2.56\\
  5.6	2.4\\
  5.76	2.24\\
  5.92	2.08\\
  6.08	1.92\\
  6.24	1.76\\
  6.4	1.6\\
  6.56	1.44\\
  6.72	1.28\\
  6.88	1.12\\
  7.04	0.96\\
  7.2	0.8\\
  7.36	0.64\\
  7.52	0.48\\
  7.68	0.32\\
  7.84	0.16\\
  8	0\\
};
% \addplot [color=black,mark size=0.5pt,only marks,mark=*,mark options={solid},forget plot]
%   table[row sep=crcr]{%
% 0	8\\
% 0.16	7.84\\
% 0.32	7.68\\
% 0.48	7.52\\
% 0.64	7.36\\
% 0.8	7.2\\
% 0.96	7.04\\
% 1.12	6.88\\
% 1.28	6.72\\
% 1.44	6.56\\
% 1.6	6.4\\
% 1.76	6.24\\
% 1.92	6.08\\
% 2.08	5.92\\
% 2.24	5.76\\
% 2.4	5.6\\
% 2.56	5.44\\
% 2.72	5.28\\
% 2.88	5.12\\
% 3.04	4.96\\
% 3.2	4.8\\
% 3.36	4.64\\
% 3.52	4.48\\
% 3.68	4.32\\
% 3.84	4.16\\
% 4	4\\
% 4.16	3.84\\
% 4.32	3.68\\
% 4.48	3.52\\
% 4.64	3.36\\
% 4.8	3.2\\
% 4.96	3.04\\
% 5.12	2.88\\
% 5.28	2.72\\
% 5.44	2.56\\
% 5.6	2.4\\
% 5.76	2.24\\
% 5.92	2.08\\
% 6.08	1.92\\
% 6.24	1.76\\
% 6.4	1.6\\
% 6.56	1.44\\
% 6.72	1.28\\
% 6.88	1.12\\
% 7.04	0.96\\
% 7.2	0.8\\
% 7.36	0.640000000000001\\
% 7.52	0.48\\
% 7.68	0.32\\
% 7.84	0.16\\
% 8	0\\
% 0	0\\
% };
% \addplot [color=black,mark size=0.5pt,only marks,mark=*,mark options={solid},forget plot]
%   table[row sep=crcr]{%
% 0	0\\
% };
\addplot [color=mycolor1,line width=1.0pt,mark size=0.75pt,only marks,mark=*,mark options={solid},forget plot]
  table[row sep=crcr]{%
  3.1589884266337	2.06549243279896\\
  3.03748887176317	2.18699198766948\\
  3.28048798150423	1.94399287792843\\
  3.06224273507541	1.88445706773871\\
  3.26583191675104	2.26096055775072\\
  2.94446416834174	2.00223563447238\\
  3.39144083662607	2.13535163787568\\
  3.18002130180908	1.76667850100504\\
  3.140222996876	2.38656947762576\\
  2.82668560160807	2.12001420120605\\
  2.91598931689265	2.30849154254001\\
  3.40198753637475	1.8224933230579\\
  3.51704975650111	2.00974271800064\\
  2.86057076686816	1.83076529079562\\
  2.97499359754289	1.7163424601209\\
  2.74614793619344	1.94518812147035\\
  3.29779986854275	1.64889993427137\\
  3.51356070819838	2.34237380546559\\
  3.38342883011696	2.47250568354701\\
  3.08941642821762	1.60191962944617\\
  3.01461407700096	2.5121783975008\\
  2.7089070348744	2.23779276793972\\
  3.6436925862798	2.21224192738417\\
  2.79448976202212	2.42999109741054\\
  3.52348709124528	1.70099376818738\\
  3.64265867637615	1.8841337981256\\
  2.63172510551871	2.05961095214508\\
  2.78527932432699	1.6711675945962\\
  3.25329695203554	2.60263756162843\\
  2.67386815135391	1.78257876756928\\
  2.89669049730007	1.55975642162312\\
  3.20383925889234	1.48749679877145\\
  3.41557843527642	1.5311213675377\\
  3.77382446436123	2.08211004930275\\
  2.56245697838084	1.89398994054236\\
  2.88900515712592	2.63778731737584\\
  3.00810167027315	1.44834524865004\\
  2.59112846814073	2.35557133467339\\
  2.51730227484398	2.1740337828198\\
  3.12316507395412	2.73276943970985\\
  3.76826759625119	1.75852487825056\\
  2.67299020715159	2.55149065228107\\
  3.64498664611581	1.57949421331685\\
  2.71810605393335	1.52213939020267\\
  3.64758904037648	2.56682191730197\\
  2.60938181177601	1.63086363236001\\
  2.45104580540776	2.00540111351544\\
  3.31826208956707	1.37307396809672\\
  3.78268493076079	2.43172602691765\\
  3.51249314999217	2.70191780768628\\
  3.11951284324623	1.33693407567696\\
  2.82683029609068	1.41341514804534\\
  2.50065756961868	1.73958787451734\\
  3.53335700201009	1.41334280080403\\
  3.90395634244265	1.95197817122132\\
  2.40287944416926	2.28845661349453\\
  2.93555453824802	1.30469090588801\\
  3.91778082114511	2.29663013653334\\
  2.39193332746135	1.84831211667468\\
  2.47334990140706	2.47334990140706\\
  3.37739725960785	2.8370136980706\\
  2.76339623725088	2.76339623725088\\
  2.9930331958727	2.86290131779128\\
  2.33963463243468	2.11681228648852\\
  2.55226538469845	1.48882147440743\\
  3.23092401621931	1.22552290270388\\
  2.65860977572755	1.38247708337833\\
  3.4326849202418	1.25865113742199\\
  3.89387651612623	1.63291595837552\\
  2.44592099366935	1.59516586543653\\
  2.55149065228107	2.67299020715159\\
  3.76648620098634	1.45799465844632\\
  3.04427878040535	1.19596666373067\\
  2.76495416675666	1.27613269234923\\
  2.28320908530401	1.95703635883201\\
};

\addplot [color=mycolor2,line width=1.0pt,mark size=0.75pt,only marks,mark=*,mark options={solid},forget plot]
  table[row sep=crcr]{%
  0.774493574103036	1.83942223849471\\
  0.77791146328195	1.75030079238439\\
  0.77265422715956	1.9316355678989\\
  0.680672530371707	1.84753972529463\\
  0.869236005554505	1.83505378950395\\
  0.685027595263561	1.76149953067773\\
  0.868972261685458	1.93104947041213\\
  0.677681877340157	1.93623393525759\\
  0.871305270865915	1.74261054173183\\
  0.782888680301212	1.66363844564008\\
  0.690736373741109	1.67750262194269\\
  0.772419788164852	2.02760194393274\\
  0.587166510225909	1.85936061571538\\
  0.965524735206065	1.83449699689152\\
  0.592059748920951	1.77617924676285\\
  0.875150396192194	1.65306185947414\\
  0.676072448764615	2.02821734629384\\
  0.870551974180783	2.03128793975516\\
  0.967279971311982	1.93455994262396\\
  0.96581778394945	1.73847201110901\\
  0.583433597461463	1.94477865820488\\
  0.598109290478083	1.6946429896879\\
  0.69779417222443	1.59495810794155\\
  0.789412998561267	1.57882599712253\\
  0.580870180577277	2.03304563202047\\
  0.968116967628795	1.64579884496895\\
  0.498424408731736	1.79432787143425\\
  0.971135139518144	2.0393837929881\\
  0.605316192736598	1.61417651396426\\
  0.493383124100792	1.87485587158301\\
  0.773823977049585	2.12801593688636\\
  1.06400796844318	1.83783194549277\\
  0.880749765338864	1.56577736060242\\
  0.504430160613832	1.71506254608703\\
  0.675867314644245	2.12415441745334\\
  1.06207720872667	1.73794452337092\\
  0.489305425188258	1.95722170075303\\
  1.06824865346996	1.94227027903629\\
  0.87402162556633	2.13649730693992\\
  0.511405463287108	1.63649748251875\\
  0.706202224859364	1.5132904818415\\
  0.797479053970777	1.49527322619521\\
  0.61368655594453	1.53421638986133\\
  0.57949067036967	2.12479912468879\\
  0.972389329102438	1.5558229265639\\
  1.06239956234439	1.64189023271406\\
  0.486194664551219	2.04201759111512\\
  1.07486691614438	2.05201865809381\\
  0.977151741949434	2.14973383228876\\
  0.403544128491065	1.81594857820979\\
  0.888089623381426	1.48014937230238\\
  0.409124370629687	1.73877857517617\\
  0.398739526985389	1.8940127531806\\
  0.519359592815851	1.55807877844755\\
  1.16536216742177	1.84515676508447\\
  0.776908111614515	2.23361082089173\\
  0.677095979918387	2.22474393401756\\
  1.16073596557438	1.74110394836157\\
  0.415487674252681	1.66195069701072\\
  0.394706499280634	1.97353249640317\\
  0.484058483814397	2.12985732878335\\
  1.06492866439167	1.54898714820607\\
  0.715967648601952	1.4319352972039\\
  0.623231511379021	1.45420685988438\\
  1.17258209033932	1.95430348389887\\
  0.879436567754491	2.2474490064837\\
  0.978610850376516	1.46791627556477\\
  0.807088256982131	1.41240444971873\\
  0.579314841123639	2.22070689097395\\
  1.15862968224728	1.64139204985031\\
  0.422644824594292	1.5849180922286\\
  0.528306030742865	1.47925688608002\\
  0.391444340150606	2.05508278579068\\
  1.08394170330611	2.16788340661222\\
  0.897163935717124	1.39558834444886\\
};

\addplot [color=mycolor3,line width=1.0pt,mark size=0.75pt,only marks,mark=*,mark options={solid},forget plot]
  table[row sep=crcr]{%
  2.14790294580586	0\\
  2.0774383712634	0\\
  2.21949270670086	0\\
  2.11860667457809	0.100886032122766\\
  2.19306739841964	0.0996848817463469\\
  2.00756291682289	0\\
  2.04562185314843	0.102281092657422\\
  2.29275228016598	0\\
  2.26956237086364	0.0986766248201585\\
  2.17088574604349	0.197353249640317\\
  1.97356645270023	0.10387191856317\\
  2.09338251667329	0.199369763492694\\
  1.93774499802338	0\\
  2.3682389956838	0\\
  2.25080495586599	0.195722170075303\\
  2.34866604090364	0.0978610850376516\\
  2.01772064245533	0.201772064245533\\
  1.90190171067431	0.105661206148573\\
  2.33373438984585	0.194477865820488\\
  1.86745346811207	0\\
  1.94334076049101	0.204562185314843\\
  2.15294387082833	0.293583255112955\\
  2.44652712594129	0\\
  2.4309733227561	0.0972389329102441\\
  2.23649545693561	0.291716798730731\\
  2.07220912122333	0.296029874460475\\
  2.42029241907199	0.193623393525759\\
  1.83009249813319	0.107652499890188\\
  1.99369763492694	0.299054645239041\\
  2.32348072230911	0.290435090288638\\
  1.86969453413706	0.20774383712634\\
  1.79615357729256	0\\
  2.52821225566634	0\\
  1.91683461033256	0.302658096368299\\
  2.51710411583487	0.0968116967628796\\
  2.13925652402536	0.388955731640975\\
  2.41454445987362	0.289745335184835\\
  2.22666902554623	0.387246787051518\\
  2.05508278579068	0.391444340150606\\
  1.75760326410548	0.109850204006593\\
  1.79624050452574	0.211322412297146\\
  2.51112623826857	0.19316355678989\\
  2.31796268147868	0.38632711357978\\
  1.72330290456268	0\\
  1.97353249640317	0.394706499280634\\
  1.84105966783359	0.306843277972265\\
  2.51036431153577	0.28965742056182\\
  2.61391581259775	0\\
  2.41381183801516	0.386209894082425\\
  2.60770801666351	0.096581778394945\\
  1.8940127531806	0.398739526985389\\
  1.72243999824301	0.215304999780376\\
  1.68389397871178	0.112259598580785\\
  2.12985732878335	0.484058483814398\\
  2.22138090308373	0.482908891974725\\
  1.76582261557389	0.311615755689511\\
  2.04201759111512	0.486194664551219\\
  2.60691678505637	0.193104947041213\\
  1.64834723119281	0\\
  2.31725936449456	0.482762367603032\\
  1.81594857820979	0.403544128491065\\
  1.95722170075303	0.489305425188258\\
  2.51492792541115	0.386911988524793\\
  1.64775306009889	0.219700408013185\\
  2.61165592254235	0.290183991393595\\
  1.6084160442585	0.114886860304178\\
  2.41819992827995	0.483639985655991\\
  2.70428979505846	0\\
  1.69057929837717	0.316983618445719\\
  2.70346925857698	0.0965524735206063\\
  1.87485587158301	0.493383124100792\\
  1.57071632445187	0\\
  1.73877857517617	0.409124370629687\\
  2.12479912468879	0.57949067036967\\
  2.22070689097395	0.579314841123639\\
};

\addplot [color=OliveGreen,line width=1.0pt,mark size=0.75pt,only marks,mark=*,mark options={solid},forget plot]
  table[row sep=crcr]{%
  0.121499554870527	5.10298130456213\\
  0.242999109741054	4.9814817496916\\
  0	5.22448085943266\\
  0	4.94669980281412\\
  0.25121783975008	5.27557463475167\\
  0.376826759625119	5.14996571487663\\
  0.11777856673367	4.82892123608045\\
  0.12560891987504	5.40118355462671\\
  0.364498664611581	4.85998219482108\\
  0.502435679500159	5.02435679500159\\
  0.235557133467339	4.71114266934678\\
  0	4.69133605766379\\
  0	5.52679247450175\\
  0.390395634244265	5.46553887941971\\
  0.520527512325686	5.33540700133829\\
  0.260263756162843	5.59567075750113\\
  0.485998219482108	4.73848263995055\\
  0.114422830674727	4.57691322698906\\
  0.650659390407108	5.20527512325686\\
  0.628044599375199	4.89874787512655\\
  0.353335700201009	4.59336410261311\\
  0.130131878081422	5.72580263558255\\
  0.228845661349453	4.46249039631433\\
  0	4.45644691892319\\
  0.78079126848853	5.07514324517544\\
  0.607497774352635	4.61698308508002\\
  0.67547945192157	5.53893150575688\\
  0.540383561537256	5.67402739614119\\
  0.753653519250239	4.77313895525151\\
  0.471114266934678	4.47558553587944\\
  0.810575342305885	5.40383561537256\\
  0.11141117297308	4.34503574595011\\
  0.405287671152942	5.80912328652551\\
  0	5.85593451366397\\
  0.34326849202418	4.34806756563961\\
  0.910923146569951	4.94501136709402\\
  0.945671232690199	5.26873972498825\\
  0.270191780768628	5.94421917690982\\
  0.728997329223162	4.4954835302095\\
  0	4.24024544413602\\
  0.22282234594616	4.23362457297703\\
  0.588892833668348	4.35780696914577\\
  0.879262439125279	4.64753003537647\\
  0.457691322698906	4.23364473496488\\
  1.08076712307451	5.13364383460394\\
  1.04105502465137	4.8148794890126\\
  0.108724242157334	4.13152120197869\\
  0.135095890384314	6.07931506729413\\
  0.84318497255786	5.76176397914538\\
  0.702654143798217	5.90229480790502\\
  0.334233518919239	4.12221340000395\\
  0.983715801317503	5.62123315038573\\
  0.850496884093689	4.37398397533897\\
  0.562123315038573	6.04282563666466\\
  0.706671400402017	4.2400284024121\\
  1.00487135900032	4.52192111550143\\
  0	4.04108685910588\\
  1.12424663007715	5.48070232162609\\
  0.572114153373633	4.11922190429016\\
  0.217448484314668	4.02279695982135\\
  1.21586301345883	4.99854794421962\\
  0	6.21441095767845\\
  0.42159248627893	6.18335646542431\\
  1.17118690273279	4.68474761093118\\
  0.445644691892319	4.01080222703087\\
  1.26477745883679	5.34017149286645\\
  0.106344391029102	3.93474246807678\\
  0.971996438964216	4.25248442046844\\
  0.326172726472002	3.91407271766402\\
  0.824449967135687	4.12224983567843\\
  0.281061657519287	6.32388729418395\\
  1.13048027887536	4.39631219562639\\
  0.686536984048359	4.00479907361543\\
  0	3.85745604511321\\
  1.35095890384314	4.86345205383531\\
};

% \addplot [color=mycolor3,line width=1.0pt,mark size=0.75pt,only marks,mark=*,mark options={solid},forget plot]
%   table[row sep=crcr]{%
% 2.25785805020487	1.90135414754094\\
% 2.23399206072187	1.76367794267516\\
% 2.11641353121019	1.88125647218684\\
% 2.28472322723296	2.04422604541897\\
% 2.40497181813996	1.92397745451197\\
% 2.13902341598356	2.02018878176225\\
% 2.37669268442618	1.78251951331963\\
% 2.09656886554461	1.74714072128717\\
% 2.43650023405664	2.07102519894814\\
% 2.35157059023354	1.64609941316348\\
% 2.21304491363042	1.63066467320136\\
% 1.99883500169851	1.99883500169851\\
% 2.16447463632597	2.16447463632597\\
% };
\addplot [color=white,line width=0.5pt,mark size=1pt,only marks,mark=x,mark options={solid},forget plot]
  table[row sep=crcr]{%
0.121499554870527	5.10298130456213\\
2.14790294580586	0\\
0.774493574103036	1.83942223849471\\
3.1589884266337	2.06549243279896\\
};
\end{axis}
\end{tikzpicture}%

\caption{\emph{An example of nearest neighbor based stencils, used for approximating the differential operator $\mathcal{L}$ on a nonuniform node layout adapted for pricing of two-dimensional basket options with the underlying assets $s_1$ and $s_2$, and strike price $K$. The central node of each displayed stencil is denoted by a white cross mark. All stencils are of the size $n_j=n=75$.}}
\label{fig:gridsten}
\end{figure}

\par
For the time discretization, in all of our reported research, we use the second order backward differentiation method (BDF2). The BDF2 scheme requires two known previous states in order to compute the current one. To initiate the method, the Euler backward method (BDF1) is often used for the first time step. In order to avoid factoring two different matrices, we use BDF2 with BDF1 as described in \cite{larsson2008multi}, so that we get a single differentiation matrix with nonuniform time steps.

\par
We split the time interval $[0,T]$ into $M$ non-uniform steps of length $\tau^{k} = t^{M-k}-t^{M-k+1}$, $k = 1,\ldots,M$ and define the BDF2 weights as % shown in \eqref{eq:timeweights}.

\begin{equation}
\label{eq:timeweights}
\beta_0^k = \tau^k\frac{1+\omega_k}{1+2\omega_k},\quad
\beta_1^k = \frac{(1+\omega_k)^2}{1+2\omega_k},\quad
\beta_2^k = \frac{\omega_k^2}{1+2\omega_k},
\end{equation}
where $\omega_k=\tau^k/\tau^{k-1}$, $k=2,\ldots,M$. In \cite{larsson2008multi}, it is shown how the time steps can be chosen in such a way that $\beta_0^k\equiv \beta_0$. Therefore, the coefficient matrix is the same in all time steps, and only one matrix factorization is needed. Applying the BDF2 scheme to \eqref{eqdRBFFD} we obtain a fully discretized system of equation
%\begin{equation}
%\underbrace{\begin{bmatrix}
%E_{II}+\beta_0L_{II}\; & \beta_0 L_{IB} \\
%\beta_0 B_{IB} & E_{BB}+\beta_0 B_{BB}
%\end{bmatrix}}_{C}
%\begin{bmatrix}
%\ul{u}_I^{\,n}\\ \ul{u}_B^{\,n}
%\end{bmatrix}
%=
%\begin{bmatrix}
%\vec{f}_I^{\,n}\\ \vec{f}_B^{\,n}
%\end{bmatrix},
%\label{impl:system}
%\end{equation}
\begin{equation}
(\underbrace{\mathbf{E}-\beta_0 \mathbf{L}}_{\mathbf{C}})
\mathbf{u}^k
= \beta_1^k\mathbf{u}^{k-1} - \beta_2^k\mathbf{u}^{k-2}
\label{impl:system}
\end{equation}
where $\mathbf{E}$ is the identity matrix of the appropriate size. To solve this system, we employ the iterative GMRES method with an incomplete LU factorization as the preconditioner.

\par
RBF-FD may be seen as a generalization of classical FD methods --- where a polynomial interpolant is used instead of an RBF. Ever since its introduction, the RBF-FD methods have been successfully applied for solving convection-diffusion equations~\cite{chandhini2007local, stevens2009use}, incompressible Navier--Stokes~\cite{shu2003local, shan2008application, chinchapatnam2009compact}, and elliptic equations~\cite{tolstykh2003using, wright2006scattered}. To the best of our knowledge, RBF-FD methods were introduced to finance with a master thesis ~\cite{wang2013radial} at Uppsala University in 2013, and the results reported in this thesis represent the continuation of that work. In parallel with our research, classical versions of RBF-FD using infinitely smooth RBFs with constant shape parameters have been applied on equidistant Cartesian grids for pricing of different contracts~\cite{kadalbajoo2013application, kadalbajoo2015application, kumar2015numerical, kadalbajoo2017radial}. Although those articles noted the importance of RBF shape parameters for the RBF-FD approximation stability and accuracy, no special attention was paid to choosing them appropriately. Moreover, the RBF-FD examples in those articles were not exploiting the great RBF advantage of being mesh-free, as the method was applied to pricing problems using node layouts that correspond to standard equidistant FD grids. The first results of option pricing with the RBF-FD method using nonuniform node layouts and recommendations for choosing the shape parameter for GA RBFs were reported in \textbf{Paper \ref{paper1}}.
%
%%
\section{Scattering Nodes}
In general, pricing PDEs are defined on infinite real domains. In many cases, the domain may be limited from one side, e.g., because a stock as an underlying cannot be negative, but in most cases the domain remains open towards $+\infty$. Since we want to use a numerical discretization scheme, it is first required to truncate the domain and assign appropriate boundary conditions at the boundaries. When it comes to pricing multi-asset options under single-factor models, we truncate the far positive side of each dimension at $s_{\max}=aKD$, $a\in\mathbb{N}$. In practice, most of the time, using $a=4$ keeps the approximation in the area around the strike price $K$ safe from the artificial boundary effects. In that case, the close-field boundary is usually set at $s_{\min}=0$. The details about the domain truncation for the multi-factor models, as well as the specifics about the boundary conditions, can be found in the appended papers next to each problem used in the numerical experiments.
\par Once the computational domain boundaries are defined, we discretize the domain by scattering nodes across it. In order to study RBF-FD approximation and to be able to compare it with the standard FD methods, we started by using equidistant-Cartesian-grid-based node layouts. \textbf{Figure \ref{fig:gridreg}} shows equidistant-Cartesian-gird-based node layouts for arithmetic basket option pricing problems of up to three dimensions in space. Although not fully exploiting the mesh-free feature of the RBF-FD methods, the presented node layouts for $D\geq2$ are truncated diagonally via a hyperplane that is parallel to the hyperplane of discontinuity in the first derivative of the payoff function for call and put basket options --- as it is not necessary to have computations in the parts of the domain that are far away from the strike value.  
\par
For many contracts, it is common to have discontinuities in the first derivative of the payoff function, and for some instruments even in the payoff function itself. Those discontinuities pose an obstacle for accurate numerical approximations and can often limit the order of convergence of the numerical methods. Knowing the location of a discontinuity allows as to scatter the nodes such that their density is higher around the discontinuity, and therefore the accuracy improved in that area. One way to do that is to use a one-dimensional nonuniform node scattering scheme from ~\cite{foulon2010adi}.
\begin{figure}[H]
\centering
\rmfamily
\definecolor{mycolor1}{rgb}{0.00000,0.44700,0.74100}%
\definecolor{mycolor2}{rgb}{0.85000,0.32500,0.09800}%
\definecolor{mycolor3}{rgb}{0.92900,0.69400,0.12500}%
%
\begin{tikzpicture}[trim axis left, trim axis right,baseline]

\begin{axis}[%
hide y axis,
width=0.55\textwidth,
y=0.001cm,
xmin=0,
xmax=4,
xtick={0,1,4},
xticklabels={$0$,$K$,$4K$},
axis x line*=bottom,
xlabel={$s_1$},
title={1D}
]
\addplot [color=black,mark size=0.5pt,only marks,mark=*,mark options={solid},forget plot]
  table[row sep=crcr]{%
  0	0\\
  0.08	0\\
  0.16	0\\
  0.24	0\\
  0.32	0\\
  0.4	0\\
  0.48	0\\
  0.56	0\\
  0.64	0\\
  0.72	0\\
  0.8	0\\
  0.88	0\\
  0.96	0\\
  1.04	0\\
  1.12	0\\
  1.2	0\\
  1.28	0\\
  1.36	0\\
  1.44	0\\
  1.52	0\\
  1.6	0\\
  1.68	0\\
  1.76	0\\
  1.84	0\\
  1.92	0\\
  2	0\\
  2.08	0\\
  2.16	0\\
  2.24	0\\
  2.32	0\\
  2.4	0\\
  2.48	0\\
  2.56	0\\
  2.64	0\\
  2.72	0\\
  2.8	0\\
  2.88	0\\
  2.96	0\\
  3.04	0\\
  3.12	0\\
  3.2	0\\
  3.28	0\\
  3.36	0\\
  3.44	0\\
  3.52	0\\
  3.6	0\\
  3.68	0\\
  3.76	0\\
  3.84	0\\
  3.92	0\\
  4	0\\
};
\addplot [color=mycolor2,mark size=1pt,only marks,mark=square*,mark options={solid},forget plot]
  table[row sep=crcr]{%
4	0\\
};
\addplot [color=mycolor1,mark size=1pt,only marks,mark=triangle*,mark options={solid,scale=1.5},forget plot]
  table[row sep=crcr]{%
0	0\\
};
\end{axis}
\end{tikzpicture}%
\\
\vspace{11pt}
% This file was created by matlab2tikz.
%
%The latest updates can be retrieved from
%  http://www.mathworks.com/matlabcentral/fileexchange/22022-matlab2tikz-matlab2tikz
%where you can also make suggestions and rate matlab2tikz.
%
\rmfamily
\definecolor{mycolor1}{rgb}{0.00000,0.44700,0.74100}%
\definecolor{mycolor2}{rgb}{0.85000,0.32500,0.09800}%
\definecolor{mycolor3}{rgb}{0.92900,0.69400,0.12500}%
%
\begin{tikzpicture}[trim axis left, trim axis right,baseline]

\begin{axis}[%
  grid=major,
axis x line*=bottom,
axis y line*=left,
width=0.55\textwidth,
height=0.55\textwidth,
xmin=0,
xmax=4,
ymin=0,
ymax=4,
xtick={0,0.5,1,4},
xticklabels={$0$,$K$,$2K$,$8K$},
ytick={0,0.5,1,4},
yticklabels={$0$,$K$,$2K$,$8K$},
xlabel={$s_1$},
ylabel={$s_2$},
title={2D}
]
\addplot [color=black,mark size=0.5pt,only marks,mark=*,mark options={solid},forget plot]
  table[row sep=crcr]{%
0	0\\
0	0.0816326530612245\\
0.0816326530612245	0\\
0	0.163265306122449\\
0.0816326530612245	0.0816326530612245\\
0.163265306122449	0\\
0	0.244897959183673\\
0.0816326530612245	0.163265306122449\\
0.163265306122449	0.0816326530612245\\
0.244897959183673	0\\
0	0.326530612244898\\
0.0816326530612245	0.244897959183673\\
0.163265306122449	0.163265306122449\\
0.244897959183673	0.0816326530612245\\
0.326530612244898	0\\
0	0.408163265306122\\
0.0816326530612245	0.326530612244898\\
0.163265306122449	0.244897959183673\\
0.244897959183673	0.163265306122449\\
0.326530612244898	0.0816326530612245\\
0.408163265306122	0\\
0	0.489795918367347\\
0.0816326530612245	0.408163265306122\\
0.163265306122449	0.326530612244898\\
0.244897959183673	0.244897959183673\\
0.326530612244898	0.163265306122449\\
0.408163265306122	0.0816326530612245\\
0.489795918367347	0\\
0	0.571428571428571\\
0.0816326530612245	0.489795918367347\\
0.163265306122449	0.408163265306122\\
0.244897959183673	0.326530612244898\\
0.326530612244898	0.244897959183673\\
0.408163265306122	0.163265306122449\\
0.489795918367347	0.0816326530612245\\
0.571428571428571	0\\
0	0.653061224489796\\
0.0816326530612245	0.571428571428571\\
0.163265306122449	0.489795918367347\\
0.244897959183673	0.408163265306122\\
0.326530612244898	0.326530612244898\\
0.408163265306122	0.244897959183673\\
0.489795918367347	0.163265306122449\\
0.571428571428571	0.0816326530612245\\
0.653061224489796	0\\
0	0.73469387755102\\
0.0816326530612245	0.653061224489796\\
0.163265306122449	0.571428571428571\\
0.244897959183673	0.489795918367347\\
0.326530612244898	0.408163265306122\\
0.408163265306122	0.326530612244898\\
0.489795918367347	0.244897959183673\\
0.571428571428572	0.163265306122449\\
0.653061224489796	0.0816326530612245\\
0.73469387755102	0\\
0	0.816326530612245\\
0.0816326530612245	0.73469387755102\\
0.163265306122449	0.653061224489796\\
0.244897959183673	0.571428571428571\\
0.326530612244898	0.489795918367347\\
0.408163265306122	0.408163265306122\\
0.489795918367347	0.326530612244898\\
0.571428571428571	0.244897959183674\\
0.653061224489796	0.163265306122449\\
0.73469387755102	0.0816326530612245\\
0.816326530612245	0\\
0	0.897959183673469\\
0.0816326530612245	0.816326530612245\\
0.163265306122449	0.73469387755102\\
0.244897959183673	0.653061224489796\\
0.326530612244898	0.571428571428571\\
0.408163265306122	0.489795918367347\\
0.489795918367347	0.408163265306122\\
0.571428571428571	0.326530612244898\\
0.653061224489796	0.244897959183674\\
0.73469387755102	0.163265306122449\\
0.816326530612245	0.0816326530612246\\
0.897959183673469	0\\
0	0.979591836734694\\
0.0816326530612245	0.897959183673469\\
0.163265306122449	0.816326530612245\\
0.244897959183673	0.73469387755102\\
0.326530612244898	0.653061224489796\\
0.408163265306122	0.571428571428571\\
0.489795918367347	0.489795918367347\\
0.571428571428571	0.408163265306122\\
0.653061224489796	0.326530612244898\\
0.73469387755102	0.244897959183674\\
0.816326530612245	0.163265306122449\\
0.897959183673469	0.0816326530612245\\
0.979591836734694	0\\
0	1.06122448979592\\
0.0816326530612245	0.979591836734694\\
0.163265306122449	0.897959183673469\\
0.244897959183673	0.816326530612245\\
0.326530612244898	0.73469387755102\\
0.408163265306123	0.653061224489796\\
0.489795918367347	0.571428571428572\\
0.571428571428571	0.489795918367347\\
0.653061224489796	0.408163265306122\\
0.73469387755102	0.326530612244898\\
0.816326530612245	0.244897959183673\\
0.897959183673469	0.163265306122449\\
0.979591836734694	0.0816326530612246\\
1.06122448979592	0\\
0	1.14285714285714\\
0.0816326530612245	1.06122448979592\\
0.163265306122449	0.979591836734694\\
0.244897959183673	0.897959183673469\\
0.326530612244898	0.816326530612245\\
0.408163265306122	0.73469387755102\\
0.489795918367347	0.653061224489796\\
0.571428571428571	0.571428571428571\\
0.653061224489796	0.489795918367347\\
0.73469387755102	0.408163265306122\\
0.816326530612245	0.326530612244898\\
0.897959183673469	0.244897959183673\\
0.979591836734694	0.163265306122449\\
1.06122448979592	0.0816326530612246\\
1.14285714285714	0\\
0	1.22448979591837\\
0.0816326530612245	1.14285714285714\\
0.163265306122449	1.06122448979592\\
0.244897959183673	0.979591836734694\\
0.326530612244898	0.897959183673469\\
0.408163265306123	0.816326530612245\\
0.489795918367347	0.73469387755102\\
0.571428571428571	0.653061224489796\\
0.653061224489796	0.571428571428571\\
0.73469387755102	0.489795918367347\\
0.816326530612245	0.408163265306122\\
0.897959183673469	0.326530612244898\\
0.979591836734694	0.244897959183674\\
1.06122448979592	0.163265306122449\\
1.14285714285714	0.0816326530612246\\
1.22448979591837	0\\
0	1.30612244897959\\
0.0816326530612245	1.22448979591837\\
0.163265306122449	1.14285714285714\\
0.244897959183673	1.06122448979592\\
0.326530612244898	0.979591836734694\\
0.408163265306122	0.897959183673469\\
0.489795918367347	0.816326530612245\\
0.571428571428571	0.73469387755102\\
0.653061224489796	0.653061224489796\\
0.73469387755102	0.571428571428571\\
0.816326530612245	0.489795918367347\\
0.897959183673469	0.408163265306122\\
0.979591836734694	0.326530612244898\\
1.06122448979592	0.244897959183674\\
1.14285714285714	0.163265306122449\\
1.22448979591837	0.0816326530612246\\
1.30612244897959	0\\
0	1.38775510204082\\
0.0816326530612245	1.30612244897959\\
0.163265306122449	1.22448979591837\\
0.244897959183673	1.14285714285714\\
0.326530612244898	1.06122448979592\\
0.408163265306122	0.979591836734694\\
0.489795918367347	0.897959183673469\\
0.571428571428571	0.816326530612245\\
0.653061224489796	0.73469387755102\\
0.73469387755102	0.653061224489796\\
0.816326530612245	0.571428571428571\\
0.897959183673469	0.489795918367347\\
0.979591836734694	0.408163265306122\\
1.06122448979592	0.326530612244898\\
1.14285714285714	0.244897959183674\\
1.22448979591837	0.163265306122449\\
1.30612244897959	0.0816326530612246\\
1.38775510204082	0\\
0	1.46938775510204\\
0.0816326530612245	1.38775510204082\\
0.163265306122449	1.30612244897959\\
0.244897959183673	1.22448979591837\\
0.326530612244898	1.14285714285714\\
0.408163265306122	1.06122448979592\\
0.489795918367347	0.979591836734694\\
0.571428571428572	0.897959183673469\\
0.653061224489796	0.816326530612245\\
0.73469387755102	0.73469387755102\\
0.816326530612245	0.653061224489796\\
0.897959183673469	0.571428571428571\\
0.979591836734694	0.489795918367347\\
1.06122448979592	0.408163265306122\\
1.14285714285714	0.326530612244898\\
1.22448979591837	0.244897959183674\\
1.30612244897959	0.163265306122449\\
1.38775510204082	0.0816326530612246\\
1.46938775510204	0\\
0	1.55102040816327\\
0.0816326530612245	1.46938775510204\\
0.163265306122449	1.38775510204082\\
0.244897959183673	1.30612244897959\\
0.326530612244898	1.22448979591837\\
0.408163265306122	1.14285714285714\\
0.489795918367347	1.06122448979592\\
0.571428571428572	0.979591836734694\\
0.653061224489796	0.897959183673469\\
0.73469387755102	0.816326530612245\\
0.816326530612245	0.73469387755102\\
0.897959183673469	0.653061224489796\\
0.979591836734694	0.571428571428571\\
1.06122448979592	0.489795918367347\\
1.14285714285714	0.408163265306122\\
1.22448979591837	0.326530612244898\\
1.30612244897959	0.244897959183674\\
1.38775510204082	0.163265306122449\\
1.46938775510204	0.0816326530612246\\
1.55102040816327	0\\
0	1.63265306122449\\
0.0816326530612245	1.55102040816327\\
0.163265306122449	1.46938775510204\\
0.244897959183673	1.38775510204082\\
0.326530612244898	1.30612244897959\\
0.408163265306122	1.22448979591837\\
0.489795918367347	1.14285714285714\\
0.571428571428571	1.06122448979592\\
0.653061224489796	0.979591836734694\\
0.73469387755102	0.897959183673469\\
0.816326530612245	0.816326530612245\\
0.897959183673469	0.73469387755102\\
0.979591836734694	0.653061224489796\\
1.06122448979592	0.571428571428572\\
1.14285714285714	0.489795918367347\\
1.22448979591837	0.408163265306122\\
1.30612244897959	0.326530612244898\\
1.38775510204082	0.244897959183674\\
1.46938775510204	0.163265306122449\\
1.55102040816327	0.0816326530612244\\
1.63265306122449	0\\
0	1.71428571428571\\
0.0816326530612245	1.63265306122449\\
0.163265306122449	1.55102040816327\\
0.244897959183673	1.46938775510204\\
0.326530612244898	1.38775510204082\\
0.408163265306122	1.30612244897959\\
0.489795918367347	1.22448979591837\\
0.571428571428571	1.14285714285714\\
0.653061224489796	1.06122448979592\\
0.73469387755102	0.979591836734694\\
0.816326530612245	0.897959183673469\\
0.897959183673469	0.816326530612245\\
0.979591836734694	0.73469387755102\\
1.06122448979592	0.653061224489796\\
1.14285714285714	0.571428571428571\\
1.22448979591837	0.489795918367347\\
1.30612244897959	0.408163265306122\\
1.38775510204082	0.326530612244898\\
1.46938775510204	0.244897959183674\\
1.55102040816327	0.163265306122449\\
1.63265306122449	0.0816326530612244\\
1.71428571428571	0\\
0	1.79591836734694\\
0.0816326530612245	1.71428571428571\\
0.163265306122449	1.63265306122449\\
0.244897959183673	1.55102040816327\\
0.326530612244898	1.46938775510204\\
0.408163265306122	1.38775510204082\\
0.489795918367347	1.30612244897959\\
0.571428571428571	1.22448979591837\\
0.653061224489796	1.14285714285714\\
0.73469387755102	1.06122448979592\\
0.816326530612245	0.979591836734694\\
0.897959183673469	0.897959183673469\\
0.979591836734694	0.816326530612245\\
1.06122448979592	0.73469387755102\\
1.14285714285714	0.653061224489796\\
1.22448979591837	0.571428571428571\\
1.30612244897959	0.489795918367347\\
1.38775510204082	0.408163265306122\\
1.46938775510204	0.326530612244898\\
1.55102040816327	0.244897959183674\\
1.63265306122449	0.163265306122449\\
1.71428571428571	0.0816326530612244\\
1.79591836734694	0\\
0	1.87755102040816\\
0.0816326530612245	1.79591836734694\\
0.163265306122449	1.71428571428571\\
0.244897959183673	1.63265306122449\\
0.326530612244898	1.55102040816327\\
0.408163265306122	1.46938775510204\\
0.489795918367347	1.38775510204082\\
0.571428571428572	1.30612244897959\\
0.653061224489796	1.22448979591837\\
0.73469387755102	1.14285714285714\\
0.816326530612245	1.06122448979592\\
0.897959183673469	0.979591836734694\\
0.979591836734694	0.897959183673469\\
1.06122448979592	0.816326530612245\\
1.14285714285714	0.73469387755102\\
1.22448979591837	0.653061224489796\\
1.30612244897959	0.571428571428571\\
1.38775510204082	0.489795918367347\\
1.46938775510204	0.408163265306122\\
1.55102040816327	0.326530612244898\\
1.63265306122449	0.244897959183674\\
1.71428571428571	0.163265306122449\\
1.79591836734694	0.0816326530612246\\
1.87755102040816	0\\
0	1.95918367346939\\
0.0816326530612245	1.87755102040816\\
0.163265306122449	1.79591836734694\\
0.244897959183673	1.71428571428571\\
0.326530612244898	1.63265306122449\\
0.408163265306122	1.55102040816327\\
0.489795918367347	1.46938775510204\\
0.571428571428571	1.38775510204082\\
0.653061224489796	1.30612244897959\\
0.73469387755102	1.22448979591837\\
0.816326530612245	1.14285714285714\\
0.897959183673469	1.06122448979592\\
0.979591836734694	0.979591836734694\\
1.06122448979592	0.897959183673469\\
1.14285714285714	0.816326530612245\\
1.22448979591837	0.73469387755102\\
1.30612244897959	0.653061224489796\\
1.38775510204082	0.571428571428571\\
1.46938775510204	0.489795918367347\\
1.55102040816327	0.408163265306122\\
1.63265306122449	0.326530612244898\\
1.71428571428571	0.244897959183674\\
1.79591836734694	0.163265306122449\\
1.87755102040816	0.0816326530612244\\
1.95918367346939	0\\
0	2.04081632653061\\
0.0816326530612245	1.95918367346939\\
0.163265306122449	1.87755102040816\\
0.244897959183673	1.79591836734694\\
0.326530612244898	1.71428571428571\\
0.408163265306122	1.63265306122449\\
0.489795918367347	1.55102040816327\\
0.571428571428572	1.46938775510204\\
0.653061224489796	1.38775510204082\\
0.73469387755102	1.30612244897959\\
0.816326530612245	1.22448979591837\\
0.897959183673469	1.14285714285714\\
0.979591836734694	1.06122448979592\\
1.06122448979592	0.979591836734694\\
1.14285714285714	0.897959183673469\\
1.22448979591837	0.816326530612245\\
1.30612244897959	0.73469387755102\\
1.38775510204082	0.653061224489796\\
1.46938775510204	0.571428571428571\\
1.55102040816327	0.489795918367347\\
1.63265306122449	0.408163265306122\\
1.71428571428571	0.326530612244898\\
1.79591836734694	0.244897959183674\\
1.87755102040816	0.163265306122449\\
1.95918367346939	0.0816326530612244\\
2.04081632653061	0\\
0	2.12244897959184\\
0.0816326530612245	2.04081632653061\\
0.163265306122449	1.95918367346939\\
0.244897959183673	1.87755102040816\\
0.326530612244898	1.79591836734694\\
0.408163265306123	1.71428571428571\\
0.489795918367347	1.63265306122449\\
0.571428571428571	1.55102040816327\\
0.653061224489796	1.46938775510204\\
0.73469387755102	1.38775510204082\\
0.816326530612245	1.30612244897959\\
0.897959183673469	1.22448979591837\\
0.979591836734694	1.14285714285714\\
1.06122448979592	1.06122448979592\\
1.14285714285714	0.979591836734694\\
1.22448979591837	0.897959183673469\\
1.30612244897959	0.816326530612245\\
1.38775510204082	0.73469387755102\\
1.46938775510204	0.653061224489796\\
1.55102040816327	0.571428571428571\\
1.63265306122449	0.489795918367347\\
1.71428571428571	0.408163265306122\\
1.79591836734694	0.326530612244898\\
1.87755102040816	0.244897959183673\\
1.95918367346939	0.163265306122449\\
2.04081632653061	0.0816326530612246\\
2.12244897959184	0\\
0	2.20408163265306\\
0.0816326530612245	2.12244897959184\\
0.163265306122449	2.04081632653061\\
0.244897959183673	1.95918367346939\\
0.326530612244898	1.87755102040816\\
0.408163265306122	1.79591836734694\\
0.489795918367347	1.71428571428571\\
0.571428571428571	1.63265306122449\\
0.653061224489796	1.55102040816327\\
0.73469387755102	1.46938775510204\\
0.816326530612245	1.38775510204082\\
0.897959183673469	1.30612244897959\\
0.979591836734694	1.22448979591837\\
1.06122448979592	1.14285714285714\\
1.14285714285714	1.06122448979592\\
1.22448979591837	0.979591836734694\\
1.30612244897959	0.897959183673469\\
1.38775510204082	0.816326530612245\\
1.46938775510204	0.73469387755102\\
1.55102040816327	0.653061224489796\\
1.63265306122449	0.571428571428571\\
1.71428571428571	0.489795918367347\\
1.79591836734694	0.408163265306122\\
1.87755102040816	0.326530612244898\\
1.95918367346939	0.244897959183674\\
2.04081632653061	0.163265306122449\\
2.12244897959184	0.0816326530612246\\
2.20408163265306	0\\
0	2.28571428571429\\
0.0816326530612245	2.20408163265306\\
0.163265306122449	2.12244897959184\\
0.244897959183673	2.04081632653061\\
0.326530612244898	1.95918367346939\\
0.408163265306122	1.87755102040816\\
0.489795918367347	1.79591836734694\\
0.571428571428571	1.71428571428571\\
0.653061224489796	1.63265306122449\\
0.73469387755102	1.55102040816327\\
0.816326530612245	1.46938775510204\\
0.897959183673469	1.38775510204082\\
0.979591836734694	1.30612244897959\\
1.06122448979592	1.22448979591837\\
1.14285714285714	1.14285714285714\\
1.22448979591837	1.06122448979592\\
1.30612244897959	0.979591836734694\\
1.38775510204082	0.897959183673469\\
1.46938775510204	0.816326530612245\\
1.55102040816327	0.734693877551021\\
1.63265306122449	0.653061224489796\\
1.71428571428571	0.571428571428571\\
1.79591836734694	0.489795918367347\\
1.87755102040816	0.408163265306122\\
1.95918367346939	0.326530612244898\\
2.04081632653061	0.244897959183673\\
2.12244897959184	0.163265306122449\\
2.20408163265306	0.0816326530612246\\
2.28571428571429	0\\
0	2.36734693877551\\
0.0816326530612245	2.28571428571429\\
0.163265306122449	2.20408163265306\\
0.244897959183673	2.12244897959184\\
0.326530612244898	2.04081632653061\\
0.408163265306122	1.95918367346939\\
0.489795918367347	1.87755102040816\\
0.571428571428571	1.79591836734694\\
0.653061224489796	1.71428571428571\\
0.73469387755102	1.63265306122449\\
0.816326530612245	1.55102040816327\\
0.897959183673469	1.46938775510204\\
0.979591836734694	1.38775510204082\\
1.06122448979592	1.30612244897959\\
1.14285714285714	1.22448979591837\\
1.22448979591837	1.14285714285714\\
1.30612244897959	1.06122448979592\\
1.38775510204082	0.979591836734694\\
1.46938775510204	0.897959183673469\\
1.55102040816327	0.816326530612245\\
1.63265306122449	0.73469387755102\\
1.71428571428571	0.653061224489796\\
1.79591836734694	0.571428571428572\\
1.87755102040816	0.489795918367347\\
1.95918367346939	0.408163265306122\\
2.04081632653061	0.326530612244898\\
2.12244897959184	0.244897959183673\\
2.20408163265306	0.163265306122449\\
2.28571428571429	0.0816326530612246\\
2.36734693877551	0\\
0	2.44897959183673\\
0.0816326530612245	2.36734693877551\\
0.163265306122449	2.28571428571429\\
0.244897959183673	2.20408163265306\\
0.326530612244898	2.12244897959184\\
0.408163265306123	2.04081632653061\\
0.489795918367347	1.95918367346939\\
0.571428571428571	1.87755102040816\\
0.653061224489796	1.79591836734694\\
0.73469387755102	1.71428571428571\\
0.816326530612245	1.63265306122449\\
0.897959183673469	1.55102040816327\\
0.979591836734694	1.46938775510204\\
1.06122448979592	1.38775510204082\\
1.14285714285714	1.30612244897959\\
1.22448979591837	1.22448979591837\\
1.30612244897959	1.14285714285714\\
1.38775510204082	1.06122448979592\\
1.46938775510204	0.979591836734694\\
1.55102040816327	0.897959183673469\\
1.63265306122449	0.816326530612245\\
1.71428571428571	0.73469387755102\\
1.79591836734694	0.653061224489796\\
1.87755102040816	0.571428571428571\\
1.95918367346939	0.489795918367347\\
2.04081632653061	0.408163265306122\\
2.12244897959184	0.326530612244898\\
2.20408163265306	0.244897959183673\\
2.28571428571429	0.163265306122449\\
2.36734693877551	0.0816326530612246\\
2.44897959183673	0\\
0	2.53061224489796\\
0.0816326530612245	2.44897959183673\\
0.163265306122449	2.36734693877551\\
0.244897959183673	2.28571428571429\\
0.326530612244898	2.20408163265306\\
0.408163265306122	2.12244897959184\\
0.489795918367347	2.04081632653061\\
0.571428571428572	1.95918367346939\\
0.653061224489796	1.87755102040816\\
0.73469387755102	1.79591836734694\\
0.816326530612245	1.71428571428571\\
0.897959183673469	1.63265306122449\\
0.979591836734694	1.55102040816327\\
1.06122448979592	1.46938775510204\\
1.14285714285714	1.38775510204082\\
1.22448979591837	1.30612244897959\\
1.30612244897959	1.22448979591837\\
1.38775510204082	1.14285714285714\\
1.46938775510204	1.06122448979592\\
1.55102040816327	0.979591836734694\\
1.63265306122449	0.897959183673469\\
1.71428571428571	0.816326530612245\\
1.79591836734694	0.734693877551021\\
1.87755102040816	0.653061224489796\\
1.95918367346939	0.571428571428571\\
2.04081632653061	0.489795918367347\\
2.12244897959184	0.408163265306122\\
2.20408163265306	0.326530612244898\\
2.28571428571429	0.244897959183673\\
2.36734693877551	0.163265306122449\\
2.44897959183673	0.0816326530612246\\
2.53061224489796	0\\
0	2.61224489795918\\
0.0816326530612245	2.53061224489796\\
0.163265306122449	2.44897959183673\\
0.244897959183673	2.36734693877551\\
0.326530612244898	2.28571428571429\\
0.408163265306122	2.20408163265306\\
0.489795918367347	2.12244897959184\\
0.571428571428571	2.04081632653061\\
0.653061224489796	1.95918367346939\\
0.73469387755102	1.87755102040816\\
0.816326530612245	1.79591836734694\\
0.897959183673469	1.71428571428571\\
0.979591836734694	1.63265306122449\\
1.06122448979592	1.55102040816327\\
1.14285714285714	1.46938775510204\\
1.22448979591837	1.38775510204082\\
1.30612244897959	1.30612244897959\\
1.38775510204082	1.22448979591837\\
1.46938775510204	1.14285714285714\\
1.55102040816327	1.06122448979592\\
1.63265306122449	0.979591836734694\\
1.71428571428571	0.897959183673469\\
1.79591836734694	0.816326530612245\\
1.87755102040816	0.73469387755102\\
1.95918367346939	0.653061224489796\\
2.04081632653061	0.571428571428571\\
2.12244897959184	0.489795918367347\\
2.20408163265306	0.408163265306122\\
2.28571428571429	0.326530612244898\\
2.36734693877551	0.244897959183673\\
2.44897959183673	0.163265306122449\\
2.53061224489796	0.0816326530612246\\
2.61224489795918	0\\
0	2.69387755102041\\
0.0816326530612245	2.61224489795918\\
0.163265306122449	2.53061224489796\\
0.244897959183673	2.44897959183673\\
0.326530612244898	2.36734693877551\\
0.408163265306122	2.28571428571429\\
0.489795918367347	2.20408163265306\\
0.571428571428571	2.12244897959184\\
0.653061224489796	2.04081632653061\\
0.73469387755102	1.95918367346939\\
0.816326530612245	1.87755102040816\\
0.897959183673469	1.79591836734694\\
0.979591836734694	1.71428571428571\\
1.06122448979592	1.63265306122449\\
1.14285714285714	1.55102040816327\\
1.22448979591837	1.46938775510204\\
1.30612244897959	1.38775510204082\\
1.38775510204082	1.30612244897959\\
1.46938775510204	1.22448979591837\\
1.55102040816327	1.14285714285714\\
1.63265306122449	1.06122448979592\\
1.71428571428571	0.979591836734694\\
1.79591836734694	0.897959183673469\\
1.87755102040816	0.816326530612245\\
1.95918367346939	0.734693877551021\\
2.04081632653061	0.653061224489796\\
2.12244897959184	0.571428571428572\\
2.20408163265306	0.489795918367347\\
2.28571428571429	0.408163265306122\\
2.36734693877551	0.326530612244898\\
2.44897959183673	0.244897959183673\\
2.53061224489796	0.163265306122449\\
2.61224489795918	0.0816326530612246\\
2.69387755102041	0\\
0	2.77551020408163\\
0.0816326530612245	2.69387755102041\\
0.163265306122449	2.61224489795918\\
0.244897959183673	2.53061224489796\\
0.326530612244898	2.44897959183673\\
0.408163265306122	2.36734693877551\\
0.489795918367347	2.28571428571429\\
0.571428571428571	2.20408163265306\\
0.653061224489796	2.12244897959184\\
0.73469387755102	2.04081632653061\\
0.816326530612245	1.95918367346939\\
0.897959183673469	1.87755102040816\\
0.979591836734694	1.79591836734694\\
1.06122448979592	1.71428571428571\\
1.14285714285714	1.63265306122449\\
1.22448979591837	1.55102040816327\\
1.30612244897959	1.46938775510204\\
1.38775510204082	1.38775510204082\\
1.46938775510204	1.30612244897959\\
1.55102040816327	1.22448979591837\\
1.63265306122449	1.14285714285714\\
1.71428571428571	1.06122448979592\\
1.79591836734694	0.979591836734694\\
1.87755102040816	0.897959183673469\\
1.95918367346939	0.816326530612245\\
2.04081632653061	0.73469387755102\\
2.12244897959184	0.653061224489796\\
2.20408163265306	0.571428571428572\\
2.28571428571429	0.489795918367347\\
2.36734693877551	0.408163265306122\\
2.44897959183673	0.326530612244898\\
2.53061224489796	0.244897959183673\\
2.61224489795918	0.163265306122449\\
2.69387755102041	0.0816326530612246\\
2.77551020408163	0\\
0	2.85714285714286\\
0.0816326530612245	2.77551020408163\\
0.163265306122449	2.69387755102041\\
0.244897959183673	2.61224489795918\\
0.326530612244898	2.53061224489796\\
0.408163265306122	2.44897959183673\\
0.489795918367347	2.36734693877551\\
0.571428571428571	2.28571428571429\\
0.653061224489796	2.20408163265306\\
0.73469387755102	2.12244897959184\\
0.816326530612245	2.04081632653061\\
0.897959183673469	1.95918367346939\\
0.979591836734694	1.87755102040816\\
1.06122448979592	1.79591836734694\\
1.14285714285714	1.71428571428571\\
1.22448979591837	1.63265306122449\\
1.30612244897959	1.55102040816327\\
1.38775510204082	1.46938775510204\\
1.46938775510204	1.38775510204082\\
1.55102040816327	1.30612244897959\\
1.63265306122449	1.22448979591837\\
1.71428571428571	1.14285714285714\\
1.79591836734694	1.06122448979592\\
1.87755102040816	0.979591836734694\\
1.95918367346939	0.897959183673469\\
2.04081632653061	0.816326530612245\\
2.12244897959184	0.73469387755102\\
2.20408163265306	0.653061224489796\\
2.28571428571429	0.571428571428572\\
2.36734693877551	0.489795918367347\\
2.44897959183673	0.408163265306122\\
2.53061224489796	0.326530612244898\\
2.61224489795918	0.244897959183673\\
2.69387755102041	0.163265306122449\\
2.77551020408163	0.0816326530612246\\
2.85714285714286	0\\
0	2.93877551020408\\
0.0816326530612245	2.85714285714286\\
0.163265306122449	2.77551020408163\\
0.244897959183673	2.69387755102041\\
0.326530612244898	2.61224489795918\\
0.408163265306122	2.53061224489796\\
0.489795918367347	2.44897959183673\\
0.571428571428572	2.36734693877551\\
0.653061224489796	2.28571428571429\\
0.73469387755102	2.20408163265306\\
0.816326530612245	2.12244897959184\\
0.897959183673469	2.04081632653061\\
0.979591836734694	1.95918367346939\\
1.06122448979592	1.87755102040816\\
1.14285714285714	1.79591836734694\\
1.22448979591837	1.71428571428571\\
1.30612244897959	1.63265306122449\\
1.38775510204082	1.55102040816327\\
1.46938775510204	1.46938775510204\\
1.55102040816327	1.38775510204082\\
1.63265306122449	1.30612244897959\\
1.71428571428571	1.22448979591837\\
1.79591836734694	1.14285714285714\\
1.87755102040816	1.06122448979592\\
1.95918367346939	0.979591836734694\\
2.04081632653061	0.897959183673469\\
2.12244897959184	0.816326530612245\\
2.20408163265306	0.73469387755102\\
2.28571428571429	0.653061224489796\\
2.36734693877551	0.571428571428572\\
2.44897959183673	0.489795918367347\\
2.53061224489796	0.408163265306122\\
2.61224489795918	0.326530612244898\\
2.69387755102041	0.244897959183673\\
2.77551020408163	0.163265306122449\\
2.85714285714286	0.0816326530612246\\
2.93877551020408	0\\
0	3.02040816326531\\
0.0816326530612245	2.93877551020408\\
0.163265306122449	2.85714285714286\\
0.244897959183673	2.77551020408163\\
0.326530612244898	2.69387755102041\\
0.408163265306122	2.61224489795918\\
0.489795918367347	2.53061224489796\\
0.571428571428571	2.44897959183673\\
0.653061224489796	2.36734693877551\\
0.73469387755102	2.28571428571429\\
0.816326530612245	2.20408163265306\\
0.897959183673469	2.12244897959184\\
0.979591836734694	2.04081632653061\\
1.06122448979592	1.95918367346939\\
1.14285714285714	1.87755102040816\\
1.22448979591837	1.79591836734694\\
1.30612244897959	1.71428571428571\\
1.38775510204082	1.63265306122449\\
1.46938775510204	1.55102040816327\\
1.55102040816327	1.46938775510204\\
1.63265306122449	1.38775510204082\\
1.71428571428571	1.30612244897959\\
1.79591836734694	1.22448979591837\\
1.87755102040816	1.14285714285714\\
1.95918367346939	1.06122448979592\\
2.04081632653061	0.979591836734694\\
2.12244897959184	0.897959183673469\\
2.20408163265306	0.816326530612245\\
2.28571428571429	0.73469387755102\\
2.36734693877551	0.653061224489796\\
2.44897959183673	0.571428571428572\\
2.53061224489796	0.489795918367347\\
2.61224489795918	0.408163265306122\\
2.69387755102041	0.326530612244898\\
2.77551020408163	0.244897959183673\\
2.85714285714286	0.163265306122449\\
2.93877551020408	0.0816326530612246\\
3.02040816326531	0\\
0	3.10204081632653\\
0.0816326530612245	3.02040816326531\\
0.163265306122449	2.93877551020408\\
0.244897959183673	2.85714285714286\\
0.326530612244898	2.77551020408163\\
0.408163265306122	2.69387755102041\\
0.489795918367347	2.61224489795918\\
0.571428571428572	2.53061224489796\\
0.653061224489796	2.44897959183673\\
0.73469387755102	2.36734693877551\\
0.816326530612245	2.28571428571429\\
0.897959183673469	2.20408163265306\\
0.979591836734694	2.12244897959184\\
1.06122448979592	2.04081632653061\\
1.14285714285714	1.95918367346939\\
1.22448979591837	1.87755102040816\\
1.30612244897959	1.79591836734694\\
1.38775510204082	1.71428571428571\\
1.46938775510204	1.63265306122449\\
1.55102040816327	1.55102040816327\\
1.63265306122449	1.46938775510204\\
1.71428571428571	1.38775510204082\\
1.79591836734694	1.30612244897959\\
1.87755102040816	1.22448979591837\\
1.95918367346939	1.14285714285714\\
2.04081632653061	1.06122448979592\\
2.12244897959184	0.979591836734694\\
2.20408163265306	0.897959183673469\\
2.28571428571429	0.816326530612244\\
2.36734693877551	0.73469387755102\\
2.44897959183673	0.653061224489796\\
2.53061224489796	0.571428571428572\\
2.61224489795918	0.489795918367347\\
2.69387755102041	0.408163265306122\\
2.77551020408163	0.326530612244898\\
2.85714285714286	0.244897959183673\\
2.93877551020408	0.163265306122449\\
3.02040816326531	0.0816326530612246\\
3.10204081632653	0\\
0	3.18367346938776\\
0.0816326530612245	3.10204081632653\\
0.163265306122449	3.02040816326531\\
0.244897959183673	2.93877551020408\\
0.326530612244898	2.85714285714286\\
0.408163265306122	2.77551020408163\\
0.489795918367347	2.69387755102041\\
0.571428571428571	2.61224489795918\\
0.653061224489796	2.53061224489796\\
0.73469387755102	2.44897959183673\\
0.816326530612245	2.36734693877551\\
0.897959183673469	2.28571428571429\\
0.979591836734694	2.20408163265306\\
1.06122448979592	2.12244897959184\\
1.14285714285714	2.04081632653061\\
1.22448979591837	1.95918367346939\\
1.30612244897959	1.87755102040816\\
1.38775510204082	1.79591836734694\\
1.46938775510204	1.71428571428571\\
1.55102040816327	1.63265306122449\\
1.63265306122449	1.55102040816327\\
1.71428571428571	1.46938775510204\\
1.79591836734694	1.38775510204082\\
1.87755102040816	1.30612244897959\\
1.95918367346939	1.22448979591837\\
2.04081632653061	1.14285714285714\\
2.12244897959184	1.06122448979592\\
2.20408163265306	0.979591836734694\\
2.28571428571429	0.897959183673469\\
2.36734693877551	0.816326530612245\\
2.44897959183673	0.73469387755102\\
2.53061224489796	0.653061224489796\\
2.61224489795918	0.571428571428572\\
2.69387755102041	0.489795918367347\\
2.77551020408163	0.408163265306122\\
2.85714285714286	0.326530612244898\\
2.93877551020408	0.244897959183673\\
3.02040816326531	0.163265306122449\\
3.10204081632653	0.0816326530612241\\
3.18367346938776	0\\
0	3.26530612244898\\
0.0816326530612245	3.18367346938776\\
0.163265306122449	3.10204081632653\\
0.244897959183673	3.02040816326531\\
0.326530612244898	2.93877551020408\\
0.408163265306122	2.85714285714286\\
0.489795918367347	2.77551020408163\\
0.571428571428571	2.69387755102041\\
0.653061224489796	2.61224489795918\\
0.73469387755102	2.53061224489796\\
0.816326530612245	2.44897959183673\\
0.897959183673469	2.36734693877551\\
0.979591836734694	2.28571428571429\\
1.06122448979592	2.20408163265306\\
1.14285714285714	2.12244897959184\\
1.22448979591837	2.04081632653061\\
1.30612244897959	1.95918367346939\\
1.38775510204082	1.87755102040816\\
1.46938775510204	1.79591836734694\\
1.55102040816327	1.71428571428571\\
1.63265306122449	1.63265306122449\\
1.71428571428571	1.55102040816327\\
1.79591836734694	1.46938775510204\\
1.87755102040816	1.38775510204082\\
1.95918367346939	1.30612244897959\\
2.04081632653061	1.22448979591837\\
2.12244897959184	1.14285714285714\\
2.20408163265306	1.06122448979592\\
2.28571428571429	0.979591836734694\\
2.36734693877551	0.897959183673469\\
2.44897959183673	0.816326530612245\\
2.53061224489796	0.73469387755102\\
2.61224489795918	0.653061224489796\\
2.69387755102041	0.571428571428572\\
2.77551020408163	0.489795918367347\\
2.85714285714286	0.408163265306122\\
2.93877551020408	0.326530612244898\\
3.02040816326531	0.244897959183674\\
3.10204081632653	0.163265306122449\\
3.18367346938776	0.0816326530612246\\
3.26530612244898	0\\
0	3.3469387755102\\
0.0816326530612245	3.26530612244898\\
0.163265306122449	3.18367346938776\\
0.244897959183673	3.10204081632653\\
0.326530612244898	3.02040816326531\\
0.408163265306122	2.93877551020408\\
0.489795918367347	2.85714285714286\\
0.571428571428572	2.77551020408163\\
0.653061224489796	2.69387755102041\\
0.73469387755102	2.61224489795918\\
0.816326530612245	2.53061224489796\\
0.897959183673469	2.44897959183673\\
0.979591836734694	2.36734693877551\\
1.06122448979592	2.28571428571429\\
1.14285714285714	2.20408163265306\\
1.22448979591837	2.12244897959184\\
1.30612244897959	2.04081632653061\\
1.38775510204082	1.95918367346939\\
1.46938775510204	1.87755102040816\\
1.55102040816327	1.79591836734694\\
1.63265306122449	1.71428571428571\\
1.71428571428571	1.63265306122449\\
1.79591836734694	1.55102040816327\\
1.87755102040816	1.46938775510204\\
1.95918367346939	1.38775510204082\\
2.04081632653061	1.30612244897959\\
2.12244897959184	1.22448979591837\\
2.20408163265306	1.14285714285714\\
2.28571428571429	1.06122448979592\\
2.36734693877551	0.979591836734694\\
2.44897959183673	0.897959183673469\\
2.53061224489796	0.816326530612245\\
2.61224489795918	0.73469387755102\\
2.69387755102041	0.653061224489796\\
2.77551020408163	0.571428571428572\\
2.85714285714286	0.489795918367347\\
2.93877551020408	0.408163265306122\\
3.02040816326531	0.326530612244898\\
3.10204081632653	0.244897959183674\\
3.18367346938776	0.163265306122449\\
3.26530612244898	0.0816326530612246\\
3.3469387755102	0\\
0	3.42857142857143\\
0.0816326530612245	3.3469387755102\\
0.163265306122449	3.26530612244898\\
0.244897959183673	3.18367346938776\\
0.326530612244898	3.10204081632653\\
0.408163265306122	3.02040816326531\\
0.489795918367347	2.93877551020408\\
0.571428571428571	2.85714285714286\\
0.653061224489796	2.77551020408163\\
0.73469387755102	2.69387755102041\\
0.816326530612245	2.61224489795918\\
0.897959183673469	2.53061224489796\\
0.979591836734694	2.44897959183673\\
1.06122448979592	2.36734693877551\\
1.14285714285714	2.28571428571429\\
1.22448979591837	2.20408163265306\\
1.30612244897959	2.12244897959184\\
1.38775510204082	2.04081632653061\\
1.46938775510204	1.95918367346939\\
1.55102040816327	1.87755102040816\\
1.63265306122449	1.79591836734694\\
1.71428571428571	1.71428571428571\\
1.79591836734694	1.63265306122449\\
1.87755102040816	1.55102040816327\\
1.95918367346939	1.46938775510204\\
2.04081632653061	1.38775510204082\\
2.12244897959184	1.30612244897959\\
2.20408163265306	1.22448979591837\\
2.28571428571429	1.14285714285714\\
2.36734693877551	1.06122448979592\\
2.44897959183673	0.979591836734694\\
2.53061224489796	0.897959183673469\\
2.61224489795918	0.816326530612245\\
2.69387755102041	0.73469387755102\\
2.77551020408163	0.653061224489796\\
2.85714285714286	0.571428571428571\\
2.93877551020408	0.489795918367347\\
3.02040816326531	0.408163265306122\\
3.10204081632653	0.326530612244898\\
3.18367346938775	0.244897959183674\\
3.26530612244898	0.163265306122449\\
3.3469387755102	0.0816326530612246\\
3.42857142857143	0\\
0	3.51020408163265\\
0.0816326530612245	3.42857142857143\\
0.163265306122449	3.3469387755102\\
0.244897959183673	3.26530612244898\\
0.326530612244898	3.18367346938776\\
0.408163265306122	3.10204081632653\\
0.489795918367347	3.02040816326531\\
0.571428571428571	2.93877551020408\\
0.653061224489796	2.85714285714286\\
0.73469387755102	2.77551020408163\\
0.816326530612245	2.69387755102041\\
0.897959183673469	2.61224489795918\\
0.979591836734694	2.53061224489796\\
1.06122448979592	2.44897959183673\\
1.14285714285714	2.36734693877551\\
1.22448979591837	2.28571428571429\\
1.30612244897959	2.20408163265306\\
1.38775510204082	2.12244897959184\\
1.46938775510204	2.04081632653061\\
1.55102040816327	1.95918367346939\\
1.63265306122449	1.87755102040816\\
1.71428571428571	1.79591836734694\\
1.79591836734694	1.71428571428571\\
1.87755102040816	1.63265306122449\\
1.95918367346939	1.55102040816327\\
2.04081632653061	1.46938775510204\\
2.12244897959184	1.38775510204082\\
2.20408163265306	1.30612244897959\\
2.28571428571429	1.22448979591837\\
2.36734693877551	1.14285714285714\\
2.44897959183673	1.06122448979592\\
2.53061224489796	0.979591836734694\\
2.61224489795918	0.897959183673469\\
2.69387755102041	0.816326530612245\\
2.77551020408163	0.73469387755102\\
2.85714285714286	0.653061224489796\\
2.93877551020408	0.571428571428571\\
3.02040816326531	0.489795918367347\\
3.10204081632653	0.408163265306122\\
3.18367346938776	0.326530612244898\\
3.26530612244898	0.244897959183674\\
3.3469387755102	0.163265306122449\\
3.42857142857143	0.0816326530612246\\
3.51020408163265	0\\
0	3.59183673469388\\
0.0816326530612245	3.51020408163265\\
0.163265306122449	3.42857142857143\\
0.244897959183673	3.3469387755102\\
0.326530612244898	3.26530612244898\\
0.408163265306122	3.18367346938776\\
0.489795918367347	3.10204081632653\\
0.571428571428571	3.02040816326531\\
0.653061224489796	2.93877551020408\\
0.73469387755102	2.85714285714286\\
0.816326530612245	2.77551020408163\\
0.897959183673469	2.69387755102041\\
0.979591836734694	2.61224489795918\\
1.06122448979592	2.53061224489796\\
1.14285714285714	2.44897959183673\\
1.22448979591837	2.36734693877551\\
1.30612244897959	2.28571428571429\\
1.38775510204082	2.20408163265306\\
1.46938775510204	2.12244897959184\\
1.55102040816327	2.04081632653061\\
1.63265306122449	1.95918367346939\\
1.71428571428571	1.87755102040816\\
1.79591836734694	1.79591836734694\\
1.87755102040816	1.71428571428571\\
1.95918367346939	1.63265306122449\\
2.04081632653061	1.55102040816327\\
2.12244897959184	1.46938775510204\\
2.20408163265306	1.38775510204082\\
2.28571428571429	1.30612244897959\\
2.36734693877551	1.22448979591837\\
2.44897959183673	1.14285714285714\\
2.53061224489796	1.06122448979592\\
2.61224489795918	0.979591836734694\\
2.69387755102041	0.897959183673469\\
2.77551020408163	0.816326530612245\\
2.85714285714286	0.73469387755102\\
2.93877551020408	0.653061224489796\\
3.02040816326531	0.571428571428572\\
3.10204081632653	0.489795918367347\\
3.18367346938776	0.408163265306122\\
3.26530612244898	0.326530612244898\\
3.3469387755102	0.244897959183674\\
3.42857142857143	0.163265306122449\\
3.51020408163265	0.0816326530612241\\
3.59183673469388	0\\
0	3.6734693877551\\
0.0816326530612245	3.59183673469388\\
0.163265306122449	3.51020408163265\\
0.244897959183673	3.42857142857143\\
0.326530612244898	3.3469387755102\\
0.408163265306123	3.26530612244898\\
0.489795918367347	3.18367346938776\\
0.571428571428571	3.10204081632653\\
0.653061224489796	3.02040816326531\\
0.73469387755102	2.93877551020408\\
0.816326530612245	2.85714285714286\\
0.897959183673469	2.77551020408163\\
0.979591836734694	2.69387755102041\\
1.06122448979592	2.61224489795918\\
1.14285714285714	2.53061224489796\\
1.22448979591837	2.44897959183673\\
1.30612244897959	2.36734693877551\\
1.38775510204082	2.28571428571429\\
1.46938775510204	2.20408163265306\\
1.55102040816327	2.12244897959184\\
1.63265306122449	2.04081632653061\\
1.71428571428571	1.95918367346939\\
1.79591836734694	1.87755102040816\\
1.87755102040816	1.79591836734694\\
1.95918367346939	1.71428571428571\\
2.04081632653061	1.63265306122449\\
2.12244897959184	1.55102040816327\\
2.20408163265306	1.46938775510204\\
2.28571428571429	1.38775510204082\\
2.36734693877551	1.30612244897959\\
2.44897959183673	1.22448979591837\\
2.53061224489796	1.14285714285714\\
2.61224489795918	1.06122448979592\\
2.69387755102041	0.979591836734694\\
2.77551020408163	0.897959183673469\\
2.85714285714286	0.816326530612245\\
2.93877551020408	0.73469387755102\\
3.02040816326531	0.653061224489796\\
3.10204081632653	0.571428571428571\\
3.18367346938776	0.489795918367347\\
3.26530612244898	0.408163265306122\\
3.3469387755102	0.326530612244898\\
3.42857142857143	0.244897959183674\\
3.51020408163265	0.163265306122449\\
3.59183673469388	0.0816326530612246\\
3.6734693877551	0\\
0	3.75510204081633\\
0.0816326530612245	3.6734693877551\\
0.163265306122449	3.59183673469388\\
0.244897959183673	3.51020408163265\\
0.326530612244898	3.42857142857143\\
0.408163265306122	3.3469387755102\\
0.489795918367347	3.26530612244898\\
0.571428571428572	3.18367346938776\\
0.653061224489796	3.10204081632653\\
0.73469387755102	3.02040816326531\\
0.816326530612245	2.93877551020408\\
0.897959183673469	2.85714285714286\\
0.979591836734694	2.77551020408163\\
1.06122448979592	2.69387755102041\\
1.14285714285714	2.61224489795918\\
1.22448979591837	2.53061224489796\\
1.30612244897959	2.44897959183673\\
1.38775510204082	2.36734693877551\\
1.46938775510204	2.28571428571429\\
1.55102040816327	2.20408163265306\\
1.63265306122449	2.12244897959184\\
1.71428571428571	2.04081632653061\\
1.79591836734694	1.95918367346939\\
1.87755102040816	1.87755102040816\\
1.95918367346939	1.79591836734694\\
2.04081632653061	1.71428571428571\\
2.12244897959184	1.63265306122449\\
2.20408163265306	1.55102040816327\\
2.28571428571429	1.46938775510204\\
2.36734693877551	1.38775510204082\\
2.44897959183673	1.30612244897959\\
2.53061224489796	1.22448979591837\\
2.61224489795918	1.14285714285714\\
2.69387755102041	1.06122448979592\\
2.77551020408163	0.979591836734694\\
2.85714285714286	0.897959183673469\\
2.93877551020408	0.816326530612245\\
3.02040816326531	0.73469387755102\\
3.10204081632653	0.653061224489796\\
3.18367346938776	0.571428571428571\\
3.26530612244898	0.489795918367347\\
3.3469387755102	0.408163265306122\\
3.42857142857143	0.326530612244898\\
3.51020408163265	0.244897959183673\\
3.59183673469388	0.163265306122449\\
3.6734693877551	0.0816326530612246\\
3.75510204081633	0\\
0	3.83673469387755\\
0.0816326530612245	3.75510204081633\\
0.163265306122449	3.6734693877551\\
0.244897959183673	3.59183673469388\\
0.326530612244898	3.51020408163265\\
0.408163265306122	3.42857142857143\\
0.489795918367347	3.3469387755102\\
0.571428571428571	3.26530612244898\\
0.653061224489796	3.18367346938776\\
0.73469387755102	3.10204081632653\\
0.816326530612245	3.02040816326531\\
0.897959183673469	2.93877551020408\\
0.979591836734694	2.85714285714286\\
1.06122448979592	2.77551020408163\\
1.14285714285714	2.69387755102041\\
1.22448979591837	2.61224489795918\\
1.30612244897959	2.53061224489796\\
1.38775510204082	2.44897959183673\\
1.46938775510204	2.36734693877551\\
1.55102040816327	2.28571428571429\\
1.63265306122449	2.20408163265306\\
1.71428571428571	2.12244897959184\\
1.79591836734694	2.04081632653061\\
1.87755102040816	1.95918367346939\\
1.95918367346939	1.87755102040816\\
2.04081632653061	1.79591836734694\\
2.12244897959184	1.71428571428571\\
2.20408163265306	1.63265306122449\\
2.28571428571429	1.55102040816327\\
2.36734693877551	1.46938775510204\\
2.44897959183673	1.38775510204082\\
2.53061224489796	1.30612244897959\\
2.61224489795918	1.22448979591837\\
2.69387755102041	1.14285714285714\\
2.77551020408163	1.06122448979592\\
2.85714285714286	0.979591836734694\\
2.93877551020408	0.897959183673469\\
3.02040816326531	0.816326530612245\\
3.10204081632653	0.73469387755102\\
3.18367346938776	0.653061224489796\\
3.26530612244898	0.571428571428571\\
3.3469387755102	0.489795918367347\\
3.42857142857143	0.408163265306122\\
3.51020408163265	0.326530612244898\\
3.59183673469388	0.244897959183674\\
3.6734693877551	0.163265306122449\\
3.75510204081633	0.0816326530612241\\
3.83673469387755	0\\
0	3.91836734693878\\
0.0816326530612245	3.83673469387755\\
0.163265306122449	3.75510204081633\\
0.244897959183673	3.6734693877551\\
0.326530612244898	3.59183673469388\\
0.408163265306122	3.51020408163265\\
0.489795918367347	3.42857142857143\\
0.571428571428571	3.3469387755102\\
0.653061224489796	3.26530612244898\\
0.73469387755102	3.18367346938776\\
0.816326530612245	3.10204081632653\\
0.897959183673469	3.02040816326531\\
0.979591836734694	2.93877551020408\\
1.06122448979592	2.85714285714286\\
1.14285714285714	2.77551020408163\\
1.22448979591837	2.69387755102041\\
1.30612244897959	2.61224489795918\\
1.38775510204082	2.53061224489796\\
1.46938775510204	2.44897959183673\\
1.55102040816327	2.36734693877551\\
1.63265306122449	2.28571428571429\\
1.71428571428571	2.20408163265306\\
1.79591836734694	2.12244897959184\\
1.87755102040816	2.04081632653061\\
1.95918367346939	1.95918367346939\\
2.04081632653061	1.87755102040816\\
2.12244897959184	1.79591836734694\\
2.20408163265306	1.71428571428571\\
2.28571428571429	1.63265306122449\\
2.36734693877551	1.55102040816327\\
2.44897959183673	1.46938775510204\\
2.53061224489796	1.38775510204082\\
2.61224489795918	1.30612244897959\\
2.69387755102041	1.22448979591837\\
2.77551020408163	1.14285714285714\\
2.85714285714286	1.06122448979592\\
2.93877551020408	0.979591836734694\\
3.02040816326531	0.897959183673469\\
3.10204081632653	0.816326530612245\\
3.18367346938775	0.734693877551021\\
3.26530612244898	0.653061224489796\\
3.3469387755102	0.571428571428572\\
3.42857142857143	0.489795918367347\\
3.51020408163265	0.408163265306122\\
3.59183673469388	0.326530612244898\\
3.6734693877551	0.244897959183674\\
3.75510204081633	0.163265306122449\\
3.83673469387755	0.0816326530612246\\
3.91836734693878	0\\
0	4\\
0.0816326530612245	3.91836734693878\\
0.163265306122449	3.83673469387755\\
0.244897959183673	3.75510204081633\\
0.326530612244898	3.6734693877551\\
0.408163265306122	3.59183673469388\\
0.489795918367347	3.51020408163265\\
0.571428571428571	3.42857142857143\\
0.653061224489796	3.3469387755102\\
0.73469387755102	3.26530612244898\\
0.816326530612245	3.18367346938776\\
0.897959183673469	3.10204081632653\\
0.979591836734694	3.02040816326531\\
1.06122448979592	2.93877551020408\\
1.14285714285714	2.85714285714286\\
1.22448979591837	2.77551020408163\\
1.30612244897959	2.69387755102041\\
1.38775510204082	2.61224489795918\\
1.46938775510204	2.53061224489796\\
1.55102040816327	2.44897959183673\\
1.63265306122449	2.36734693877551\\
1.71428571428571	2.28571428571429\\
1.79591836734694	2.20408163265306\\
1.87755102040816	2.12244897959184\\
1.95918367346939	2.04081632653061\\
2.04081632653061	1.95918367346939\\
2.12244897959184	1.87755102040816\\
2.20408163265306	1.79591836734694\\
2.28571428571429	1.71428571428571\\
2.36734693877551	1.63265306122449\\
2.44897959183673	1.55102040816327\\
2.53061224489796	1.46938775510204\\
2.61224489795918	1.38775510204082\\
2.69387755102041	1.30612244897959\\
2.77551020408163	1.22448979591837\\
2.85714285714286	1.14285714285714\\
2.93877551020408	1.06122448979592\\
3.02040816326531	0.979591836734694\\
3.10204081632653	0.897959183673469\\
3.18367346938776	0.816326530612245\\
3.26530612244898	0.73469387755102\\
3.3469387755102	0.653061224489796\\
3.42857142857143	0.571428571428572\\
3.51020408163265	0.489795918367347\\
3.59183673469388	0.408163265306122\\
3.6734693877551	0.326530612244898\\
3.75510204081633	0.244897959183673\\
3.83673469387755	0.163265306122449\\
3.91836734693878	0.0816326530612246\\
4	0\\
};
\addplot [color=mycolor2,mark size=1pt,only marks,mark=square*,mark options={solid},forget plot]
  table[row sep=crcr]{%
0	4\\
0.0816326530612245	3.91836734693878\\
0.163265306122449	3.83673469387755\\
0.244897959183673	3.75510204081633\\
0.326530612244898	3.6734693877551\\
0.408163265306122	3.59183673469388\\
0.489795918367347	3.51020408163265\\
0.571428571428571	3.42857142857143\\
0.653061224489796	3.3469387755102\\
0.73469387755102	3.26530612244898\\
0.816326530612245	3.18367346938776\\
0.897959183673469	3.10204081632653\\
0.979591836734694	3.02040816326531\\
1.06122448979592	2.93877551020408\\
1.14285714285714	2.85714285714286\\
1.22448979591837	2.77551020408163\\
1.30612244897959	2.69387755102041\\
1.38775510204082	2.61224489795918\\
1.46938775510204	2.53061224489796\\
1.55102040816327	2.44897959183673\\
1.63265306122449	2.36734693877551\\
1.71428571428571	2.28571428571429\\
1.79591836734694	2.20408163265306\\
1.87755102040816	2.12244897959184\\
1.95918367346939	2.04081632653061\\
2.04081632653061	1.95918367346939\\
2.12244897959184	1.87755102040816\\
2.20408163265306	1.79591836734694\\
2.28571428571429	1.71428571428571\\
2.36734693877551	1.63265306122449\\
2.44897959183673	1.55102040816327\\
2.53061224489796	1.46938775510204\\
2.61224489795918	1.38775510204082\\
2.69387755102041	1.30612244897959\\
2.77551020408163	1.22448979591837\\
2.85714285714286	1.14285714285714\\
2.93877551020408	1.06122448979592\\
3.02040816326531	0.979591836734694\\
3.10204081632653	0.897959183673469\\
3.18367346938776	0.816326530612245\\
3.26530612244898	0.73469387755102\\
3.3469387755102	0.653061224489796\\
3.42857142857143	0.571428571428572\\
3.51020408163265	0.489795918367347\\
3.59183673469388	0.408163265306122\\
3.6734693877551	0.326530612244898\\
3.75510204081633	0.244897959183673\\
3.83673469387755	0.163265306122449\\
3.91836734693878	0.0816326530612246\\
4	0\\
};
\addplot [color=mycolor1,mark size=1pt,only marks,mark=triangle*,mark options={solid,scale=1.5},forget plot]
  table[row sep=crcr]{%
0	0\\
};
\end{axis}
\end{tikzpicture}%
\\
\vspace{11pt}
\rmfamily
\definecolor{mycolor1}{rgb}{0.00000,0.44700,0.74100}%
\definecolor{mycolor2}{rgb}{0.85000,0.32500,0.09800}%
\definecolor{mycolor3}{rgb}{0.92900,0.69400,0.12500}%
\begin{tikzpicture}[trim axis left, trim axis right,baseline]

\begin{axis}[%
width=0.55\textwidth,
height=0.55\textwidth,
scale only axis,
plot box ratio=1 1 1,
grid=both,
xlabel=$s_1$,
ylabel=$s_2$,
zlabel=$s_3$,
xtick={0,1,3,12},
xticklabels={$0$,$K$,$3K$,$12K$},
ytick={0,1,3,12},
yticklabels={$0$,$K$,$3K$,$12K$},
ztick={0,1,3,12},
zticklabels={$0$,$K$,$3K$,$12K$},
xmin=0,
xmax=12,
tick align=outside,
ymin=0,
ymax=12,
zmin=0,
zmax=12,
view={110}{20},
axis background/.style={fill=white},
axis x line*=bottom,
axis y line*=left,
axis z line*=left,
title={3D}
]
\addplot3 [color=black,mark size=0.5pt,only marks,mark=*,mark options={solid}]
 table[row sep=crcr] {%
0	0	0\\
0	0.5	0\\
0	1	0\\
0	1.5	0\\
0	2	0\\
0	2.5	0\\
0	3	0\\
0	3.5	0\\
0	4	0\\
0	4.5	0\\
0	5	0\\
0	5.5	0\\
0	6	0\\
0	6.5	0\\
0	7	0\\
0	7.5	0\\
0	8	0\\
0	8.5	0\\
0	9	0\\
0	9.5	0\\
0	10	0\\
0	10.5	0\\
0	11	0\\
0	11.5	0\\
0	12	0\\
0.5	0	0\\
0.5	0.5	0\\
0.5	1	0\\
0.5	1.5	0\\
0.5	2	0\\
0.5	2.5	0\\
0.5	3	0\\
0.5	3.5	0\\
0.5	4	0\\
0.5	4.5	0\\
0.5	5	0\\
0.5	5.5	0\\
0.5	6	0\\
0.5	6.5	0\\
0.5	7	0\\
0.5	7.5	0\\
0.5	8	0\\
0.5	8.5	0\\
0.5	9	0\\
0.5	9.5	0\\
0.5	10	0\\
0.5	10.5	0\\
0.5	11	0\\
0.5	11.5	0\\
1	0	0\\
1	0.5	0\\
1	1	0\\
1	1.5	0\\
1	2	0\\
1	2.5	0\\
1	3	0\\
1	3.5	0\\
1	4	0\\
1	4.5	0\\
1	5	0\\
1	5.5	0\\
1	6	0\\
1	6.5	0\\
1	7	0\\
1	7.5	0\\
1	8	0\\
1	8.5	0\\
1	9	0\\
1	9.5	0\\
1	10	0\\
1	10.5	0\\
1	11	0\\
1.5	0	0\\
1.5	0.5	0\\
1.5	1	0\\
1.5	1.5	0\\
1.5	2	0\\
1.5	2.5	0\\
1.5	3	0\\
1.5	3.5	0\\
1.5	4	0\\
1.5	4.5	0\\
1.5	5	0\\
1.5	5.5	0\\
1.5	6	0\\
1.5	6.5	0\\
1.5	7	0\\
1.5	7.5	0\\
1.5	8	0\\
1.5	8.5	0\\
1.5	9	0\\
1.5	9.5	0\\
1.5	10	0\\
1.5	10.5	0\\
2	0	0\\
2	0.5	0\\
2	1	0\\
2	1.5	0\\
2	2	0\\
2	2.5	0\\
2	3	0\\
2	3.5	0\\
2	4	0\\
2	4.5	0\\
2	5	0\\
2	5.5	0\\
2	6	0\\
2	6.5	0\\
2	7	0\\
2	7.5	0\\
2	8	0\\
2	8.5	0\\
2	9	0\\
2	9.5	0\\
2	10	0\\
2.5	0	0\\
2.5	0.5	0\\
2.5	1	0\\
2.5	1.5	0\\
2.5	2	0\\
2.5	2.5	0\\
2.5	3	0\\
2.5	3.5	0\\
2.5	4	0\\
2.5	4.5	0\\
2.5	5	0\\
2.5	5.5	0\\
2.5	6	0\\
2.5	6.5	0\\
2.5	7	0\\
2.5	7.5	0\\
2.5	8	0\\
2.5	8.5	0\\
2.5	9	0\\
2.5	9.5	0\\
3	0	0\\
3	0.5	0\\
3	1	0\\
3	1.5	0\\
3	2	0\\
3	2.5	0\\
3	3	0\\
3	3.5	0\\
3	4	0\\
3	4.5	0\\
3	5	0\\
3	5.5	0\\
3	6	0\\
3	6.5	0\\
3	7	0\\
3	7.5	0\\
3	8	0\\
3	8.5	0\\
3	9	0\\
3.5	0	0\\
3.5	0.5	0\\
3.5	1	0\\
3.5	1.5	0\\
3.5	2	0\\
3.5	2.5	0\\
3.5	3	0\\
3.5	3.5	0\\
3.5	4	0\\
3.5	4.5	0\\
3.5	5	0\\
3.5	5.5	0\\
3.5	6	0\\
3.5	6.5	0\\
3.5	7	0\\
3.5	7.5	0\\
3.5	8	0\\
3.5	8.5	0\\
4	0	0\\
4	0.5	0\\
4	1	0\\
4	1.5	0\\
4	2	0\\
4	2.5	0\\
4	3	0\\
4	3.5	0\\
4	4	0\\
4	4.5	0\\
4	5	0\\
4	5.5	0\\
4	6	0\\
4	6.5	0\\
4	7	0\\
4	7.5	0\\
4	8	0\\
4.5	0	0\\
4.5	0.5	0\\
4.5	1	0\\
4.5	1.5	0\\
4.5	2	0\\
4.5	2.5	0\\
4.5	3	0\\
4.5	3.5	0\\
4.5	4	0\\
4.5	4.5	0\\
4.5	5	0\\
4.5	5.5	0\\
4.5	6	0\\
4.5	6.5	0\\
4.5	7	0\\
4.5	7.5	0\\
5	0	0\\
5	0.5	0\\
5	1	0\\
5	1.5	0\\
5	2	0\\
5	2.5	0\\
5	3	0\\
5	3.5	0\\
5	4	0\\
5	4.5	0\\
5	5	0\\
5	5.5	0\\
5	6	0\\
5	6.5	0\\
5	7	0\\
5.5	0	0\\
5.5	0.5	0\\
5.5	1	0\\
5.5	1.5	0\\
5.5	2	0\\
5.5	2.5	0\\
5.5	3	0\\
5.5	3.5	0\\
5.5	4	0\\
5.5	4.5	0\\
5.5	5	0\\
5.5	5.5	0\\
5.5	6	0\\
5.5	6.5	0\\
6	0	0\\
6	0.5	0\\
6	1	0\\
6	1.5	0\\
6	2	0\\
6	2.5	0\\
6	3	0\\
6	3.5	0\\
6	4	0\\
6	4.5	0\\
6	5	0\\
6	5.5	0\\
6	6	0\\
6.5	0	0\\
6.5	0.5	0\\
6.5	1	0\\
6.5	1.5	0\\
6.5	2	0\\
6.5	2.5	0\\
6.5	3	0\\
6.5	3.5	0\\
6.5	4	0\\
6.5	4.5	0\\
6.5	5	0\\
6.5	5.5	0\\
7	0	0\\
7	0.5	0\\
7	1	0\\
7	1.5	0\\
7	2	0\\
7	2.5	0\\
7	3	0\\
7	3.5	0\\
7	4	0\\
7	4.5	0\\
7	5	0\\
7.5	0	0\\
7.5	0.5	0\\
7.5	1	0\\
7.5	1.5	0\\
7.5	2	0\\
7.5	2.5	0\\
7.5	3	0\\
7.5	3.5	0\\
7.5	4	0\\
7.5	4.5	0\\
8	0	0\\
8	0.5	0\\
8	1	0\\
8	1.5	0\\
8	2	0\\
8	2.5	0\\
8	3	0\\
8	3.5	0\\
8	4	0\\
8.5	0	0\\
8.5	0.5	0\\
8.5	1	0\\
8.5	1.5	0\\
8.5	2	0\\
8.5	2.5	0\\
8.5	3	0\\
8.5	3.5	0\\
9	0	0\\
9	0.5	0\\
9	1	0\\
9	1.5	0\\
9	2	0\\
9	2.5	0\\
9	3	0\\
9.5	0	0\\
9.5	0.5	0\\
9.5	1	0\\
9.5	1.5	0\\
9.5	2	0\\
9.5	2.5	0\\
10	0	0\\
10	0.5	0\\
10	1	0\\
10	1.5	0\\
10	2	0\\
10.5	0	0\\
10.5	0.5	0\\
10.5	1	0\\
10.5	1.5	0\\
11	0	0\\
11	0.5	0\\
11	1	0\\
11.5	0	0\\
11.5	0.5	0\\
12	0	0\\
0	0	0.5\\
0	0.5	0.5\\
0	1	0.5\\
0	1.5	0.5\\
0	2	0.5\\
0	2.5	0.5\\
0	3	0.5\\
0	3.5	0.5\\
0	4	0.5\\
0	4.5	0.5\\
0	5	0.5\\
0	5.5	0.5\\
0	6	0.5\\
0	6.5	0.5\\
0	7	0.5\\
0	7.5	0.5\\
0	8	0.5\\
0	8.5	0.5\\
0	9	0.5\\
0	9.5	0.5\\
0	10	0.5\\
0	10.5	0.5\\
0	11	0.5\\
0	11.5	0.5\\
0.5	0	0.5\\
0.5	0.5	0.5\\
0.5	1	0.5\\
0.5	1.5	0.5\\
0.5	2	0.5\\
0.5	2.5	0.5\\
0.5	3	0.5\\
0.5	3.5	0.5\\
0.5	4	0.5\\
0.5	4.5	0.5\\
0.5	5	0.5\\
0.5	5.5	0.5\\
0.5	6	0.5\\
0.5	6.5	0.5\\
0.5	7	0.5\\
0.5	7.5	0.5\\
0.5	8	0.5\\
0.5	8.5	0.5\\
0.5	9	0.5\\
0.5	9.5	0.5\\
0.5	10	0.5\\
0.5	10.5	0.5\\
0.5	11	0.5\\
1	0	0.5\\
1	0.5	0.5\\
1	1	0.5\\
1	1.5	0.5\\
1	2	0.5\\
1	2.5	0.5\\
1	3	0.5\\
1	3.5	0.5\\
1	4	0.5\\
1	4.5	0.5\\
1	5	0.5\\
1	5.5	0.5\\
1	6	0.5\\
1	6.5	0.5\\
1	7	0.5\\
1	7.5	0.5\\
1	8	0.5\\
1	8.5	0.5\\
1	9	0.5\\
1	9.5	0.5\\
1	10	0.5\\
1	10.5	0.5\\
1.5	0	0.5\\
1.5	0.5	0.5\\
1.5	1	0.5\\
1.5	1.5	0.5\\
1.5	2	0.5\\
1.5	2.5	0.5\\
1.5	3	0.5\\
1.5	3.5	0.5\\
1.5	4	0.5\\
1.5	4.5	0.5\\
1.5	5	0.5\\
1.5	5.5	0.5\\
1.5	6	0.5\\
1.5	6.5	0.5\\
1.5	7	0.5\\
1.5	7.5	0.5\\
1.5	8	0.5\\
1.5	8.5	0.5\\
1.5	9	0.5\\
1.5	9.5	0.5\\
1.5	10	0.5\\
2	0	0.5\\
2	0.5	0.5\\
2	1	0.5\\
2	1.5	0.5\\
2	2	0.5\\
2	2.5	0.5\\
2	3	0.5\\
2	3.5	0.5\\
2	4	0.5\\
2	4.5	0.5\\
2	5	0.5\\
2	5.5	0.5\\
2	6	0.5\\
2	6.5	0.5\\
2	7	0.5\\
2	7.5	0.5\\
2	8	0.5\\
2	8.5	0.5\\
2	9	0.5\\
2	9.5	0.5\\
2.5	0	0.5\\
2.5	0.5	0.5\\
2.5	1	0.5\\
2.5	1.5	0.5\\
2.5	2	0.5\\
2.5	2.5	0.5\\
2.5	3	0.5\\
2.5	3.5	0.5\\
2.5	4	0.5\\
2.5	4.5	0.5\\
2.5	5	0.5\\
2.5	5.5	0.5\\
2.5	6	0.5\\
2.5	6.5	0.5\\
2.5	7	0.5\\
2.5	7.5	0.5\\
2.5	8	0.5\\
2.5	8.5	0.5\\
2.5	9	0.5\\
3	0	0.5\\
3	0.5	0.5\\
3	1	0.5\\
3	1.5	0.5\\
3	2	0.5\\
3	2.5	0.5\\
3	3	0.5\\
3	3.5	0.5\\
3	4	0.5\\
3	4.5	0.5\\
3	5	0.5\\
3	5.5	0.5\\
3	6	0.5\\
3	6.5	0.5\\
3	7	0.5\\
3	7.5	0.5\\
3	8	0.5\\
3	8.5	0.5\\
3.5	0	0.5\\
3.5	0.5	0.5\\
3.5	1	0.5\\
3.5	1.5	0.5\\
3.5	2	0.5\\
3.5	2.5	0.5\\
3.5	3	0.5\\
3.5	3.5	0.5\\
3.5	4	0.5\\
3.5	4.5	0.5\\
3.5	5	0.5\\
3.5	5.5	0.5\\
3.5	6	0.5\\
3.5	6.5	0.5\\
3.5	7	0.5\\
3.5	7.5	0.5\\
3.5	8	0.5\\
4	0	0.5\\
4	0.5	0.5\\
4	1	0.5\\
4	1.5	0.5\\
4	2	0.5\\
4	2.5	0.5\\
4	3	0.5\\
4	3.5	0.5\\
4	4	0.5\\
4	4.5	0.5\\
4	5	0.5\\
4	5.5	0.5\\
4	6	0.5\\
4	6.5	0.5\\
4	7	0.5\\
4	7.5	0.5\\
4.5	0	0.5\\
4.5	0.5	0.5\\
4.5	1	0.5\\
4.5	1.5	0.5\\
4.5	2	0.5\\
4.5	2.5	0.5\\
4.5	3	0.5\\
4.5	3.5	0.5\\
4.5	4	0.5\\
4.5	4.5	0.5\\
4.5	5	0.5\\
4.5	5.5	0.5\\
4.5	6	0.5\\
4.5	6.5	0.5\\
4.5	7	0.5\\
5	0	0.5\\
5	0.5	0.5\\
5	1	0.5\\
5	1.5	0.5\\
5	2	0.5\\
5	2.5	0.5\\
5	3	0.5\\
5	3.5	0.5\\
5	4	0.5\\
5	4.5	0.5\\
5	5	0.5\\
5	5.5	0.5\\
5	6	0.5\\
5	6.5	0.5\\
5.5	0	0.5\\
5.5	0.5	0.5\\
5.5	1	0.5\\
5.5	1.5	0.5\\
5.5	2	0.5\\
5.5	2.5	0.5\\
5.5	3	0.5\\
5.5	3.5	0.5\\
5.5	4	0.5\\
5.5	4.5	0.5\\
5.5	5	0.5\\
5.5	5.5	0.5\\
5.5	6	0.5\\
6	0	0.5\\
6	0.5	0.5\\
6	1	0.5\\
6	1.5	0.5\\
6	2	0.5\\
6	2.5	0.5\\
6	3	0.5\\
6	3.5	0.5\\
6	4	0.5\\
6	4.5	0.5\\
6	5	0.5\\
6	5.5	0.5\\
6.5	0	0.5\\
6.5	0.5	0.5\\
6.5	1	0.5\\
6.5	1.5	0.5\\
6.5	2	0.5\\
6.5	2.5	0.5\\
6.5	3	0.5\\
6.5	3.5	0.5\\
6.5	4	0.5\\
6.5	4.5	0.5\\
6.5	5	0.5\\
7	0	0.5\\
7	0.5	0.5\\
7	1	0.5\\
7	1.5	0.5\\
7	2	0.5\\
7	2.5	0.5\\
7	3	0.5\\
7	3.5	0.5\\
7	4	0.5\\
7	4.5	0.5\\
7.5	0	0.5\\
7.5	0.5	0.5\\
7.5	1	0.5\\
7.5	1.5	0.5\\
7.5	2	0.5\\
7.5	2.5	0.5\\
7.5	3	0.5\\
7.5	3.5	0.5\\
7.5	4	0.5\\
8	0	0.5\\
8	0.5	0.5\\
8	1	0.5\\
8	1.5	0.5\\
8	2	0.5\\
8	2.5	0.5\\
8	3	0.5\\
8	3.5	0.5\\
8.5	0	0.5\\
8.5	0.5	0.5\\
8.5	1	0.5\\
8.5	1.5	0.5\\
8.5	2	0.5\\
8.5	2.5	0.5\\
8.5	3	0.5\\
9	0	0.5\\
9	0.5	0.5\\
9	1	0.5\\
9	1.5	0.5\\
9	2	0.5\\
9	2.5	0.5\\
9.5	0	0.5\\
9.5	0.5	0.5\\
9.5	1	0.5\\
9.5	1.5	0.5\\
9.5	2	0.5\\
10	0	0.5\\
10	0.5	0.5\\
10	1	0.5\\
10	1.5	0.5\\
10.5	0	0.5\\
10.5	0.5	0.5\\
10.5	1	0.5\\
11	0	0.5\\
11	0.5	0.5\\
11.5	0	0.5\\
0	0	1\\
0	0.5	1\\
0	1	1\\
0	1.5	1\\
0	2	1\\
0	2.5	1\\
0	3	1\\
0	3.5	1\\
0	4	1\\
0	4.5	1\\
0	5	1\\
0	5.5	1\\
0	6	1\\
0	6.5	1\\
0	7	1\\
0	7.5	1\\
0	8	1\\
0	8.5	1\\
0	9	1\\
0	9.5	1\\
0	10	1\\
0	10.5	1\\
0	11	1\\
0.5	0	1\\
0.5	0.5	1\\
0.5	1	1\\
0.5	1.5	1\\
0.5	2	1\\
0.5	2.5	1\\
0.5	3	1\\
0.5	3.5	1\\
0.5	4	1\\
0.5	4.5	1\\
0.5	5	1\\
0.5	5.5	1\\
0.5	6	1\\
0.5	6.5	1\\
0.5	7	1\\
0.5	7.5	1\\
0.5	8	1\\
0.5	8.5	1\\
0.5	9	1\\
0.5	9.5	1\\
0.5	10	1\\
0.5	10.5	1\\
1	0	1\\
1	0.5	1\\
1	1	1\\
1	1.5	1\\
1	2	1\\
1	2.5	1\\
1	3	1\\
1	3.5	1\\
1	4	1\\
1	4.5	1\\
1	5	1\\
1	5.5	1\\
1	6	1\\
1	6.5	1\\
1	7	1\\
1	7.5	1\\
1	8	1\\
1	8.5	1\\
1	9	1\\
1	9.5	1\\
1	10	1\\
1.5	0	1\\
1.5	0.5	1\\
1.5	1	1\\
1.5	1.5	1\\
1.5	2	1\\
1.5	2.5	1\\
1.5	3	1\\
1.5	3.5	1\\
1.5	4	1\\
1.5	4.5	1\\
1.5	5	1\\
1.5	5.5	1\\
1.5	6	1\\
1.5	6.5	1\\
1.5	7	1\\
1.5	7.5	1\\
1.5	8	1\\
1.5	8.5	1\\
1.5	9	1\\
1.5	9.5	1\\
2	0	1\\
2	0.5	1\\
2	1	1\\
2	1.5	1\\
2	2	1\\
2	2.5	1\\
2	3	1\\
2	3.5	1\\
2	4	1\\
2	4.5	1\\
2	5	1\\
2	5.5	1\\
2	6	1\\
2	6.5	1\\
2	7	1\\
2	7.5	1\\
2	8	1\\
2	8.5	1\\
2	9	1\\
2.5	0	1\\
2.5	0.5	1\\
2.5	1	1\\
2.5	1.5	1\\
2.5	2	1\\
2.5	2.5	1\\
2.5	3	1\\
2.5	3.5	1\\
2.5	4	1\\
2.5	4.5	1\\
2.5	5	1\\
2.5	5.5	1\\
2.5	6	1\\
2.5	6.5	1\\
2.5	7	1\\
2.5	7.5	1\\
2.5	8	1\\
2.5	8.5	1\\
3	0	1\\
3	0.5	1\\
3	1	1\\
3	1.5	1\\
3	2	1\\
3	2.5	1\\
3	3	1\\
3	3.5	1\\
3	4	1\\
3	4.5	1\\
3	5	1\\
3	5.5	1\\
3	6	1\\
3	6.5	1\\
3	7	1\\
3	7.5	1\\
3	8	1\\
3.5	0	1\\
3.5	0.5	1\\
3.5	1	1\\
3.5	1.5	1\\
3.5	2	1\\
3.5	2.5	1\\
3.5	3	1\\
3.5	3.5	1\\
3.5	4	1\\
3.5	4.5	1\\
3.5	5	1\\
3.5	5.5	1\\
3.5	6	1\\
3.5	6.5	1\\
3.5	7	1\\
3.5	7.5	1\\
4	0	1\\
4	0.5	1\\
4	1	1\\
4	1.5	1\\
4	2	1\\
4	2.5	1\\
4	3	1\\
4	3.5	1\\
4	4	1\\
4	4.5	1\\
4	5	1\\
4	5.5	1\\
4	6	1\\
4	6.5	1\\
4	7	1\\
4.5	0	1\\
4.5	0.5	1\\
4.5	1	1\\
4.5	1.5	1\\
4.5	2	1\\
4.5	2.5	1\\
4.5	3	1\\
4.5	3.5	1\\
4.5	4	1\\
4.5	4.5	1\\
4.5	5	1\\
4.5	5.5	1\\
4.5	6	1\\
4.5	6.5	1\\
5	0	1\\
5	0.5	1\\
5	1	1\\
5	1.5	1\\
5	2	1\\
5	2.5	1\\
5	3	1\\
5	3.5	1\\
5	4	1\\
5	4.5	1\\
5	5	1\\
5	5.5	1\\
5	6	1\\
5.5	0	1\\
5.5	0.5	1\\
5.5	1	1\\
5.5	1.5	1\\
5.5	2	1\\
5.5	2.5	1\\
5.5	3	1\\
5.5	3.5	1\\
5.5	4	1\\
5.5	4.5	1\\
5.5	5	1\\
5.5	5.5	1\\
6	0	1\\
6	0.5	1\\
6	1	1\\
6	1.5	1\\
6	2	1\\
6	2.5	1\\
6	3	1\\
6	3.5	1\\
6	4	1\\
6	4.5	1\\
6	5	1\\
6.5	0	1\\
6.5	0.5	1\\
6.5	1	1\\
6.5	1.5	1\\
6.5	2	1\\
6.5	2.5	1\\
6.5	3	1\\
6.5	3.5	1\\
6.5	4	1\\
6.5	4.5	1\\
7	0	1\\
7	0.5	1\\
7	1	1\\
7	1.5	1\\
7	2	1\\
7	2.5	1\\
7	3	1\\
7	3.5	1\\
7	4	1\\
7.5	0	1\\
7.5	0.5	1\\
7.5	1	1\\
7.5	1.5	1\\
7.5	2	1\\
7.5	2.5	1\\
7.5	3	1\\
7.5	3.5	1\\
8	0	1\\
8	0.5	1\\
8	1	1\\
8	1.5	1\\
8	2	1\\
8	2.5	1\\
8	3	1\\
8.5	0	1\\
8.5	0.5	1\\
8.5	1	1\\
8.5	1.5	1\\
8.5	2	1\\
8.5	2.5	1\\
9	0	1\\
9	0.5	1\\
9	1	1\\
9	1.5	1\\
9	2	1\\
9.5	0	1\\
9.5	0.5	1\\
9.5	1	1\\
9.5	1.5	1\\
10	0	1\\
10	0.5	1\\
10	1	1\\
10.5	0	1\\
10.5	0.5	1\\
11	0	1\\
0	0	1.5\\
0	0.5	1.5\\
0	1	1.5\\
0	1.5	1.5\\
0	2	1.5\\
0	2.5	1.5\\
0	3	1.5\\
0	3.5	1.5\\
0	4	1.5\\
0	4.5	1.5\\
0	5	1.5\\
0	5.5	1.5\\
0	6	1.5\\
0	6.5	1.5\\
0	7	1.5\\
0	7.5	1.5\\
0	8	1.5\\
0	8.5	1.5\\
0	9	1.5\\
0	9.5	1.5\\
0	10	1.5\\
0	10.5	1.5\\
0.5	0	1.5\\
0.5	0.5	1.5\\
0.5	1	1.5\\
0.5	1.5	1.5\\
0.5	2	1.5\\
0.5	2.5	1.5\\
0.5	3	1.5\\
0.5	3.5	1.5\\
0.5	4	1.5\\
0.5	4.5	1.5\\
0.5	5	1.5\\
0.5	5.5	1.5\\
0.5	6	1.5\\
0.5	6.5	1.5\\
0.5	7	1.5\\
0.5	7.5	1.5\\
0.5	8	1.5\\
0.5	8.5	1.5\\
0.5	9	1.5\\
0.5	9.5	1.5\\
0.5	10	1.5\\
1	0	1.5\\
1	0.5	1.5\\
1	1	1.5\\
1	1.5	1.5\\
1	2	1.5\\
1	2.5	1.5\\
1	3	1.5\\
1	3.5	1.5\\
1	4	1.5\\
1	4.5	1.5\\
1	5	1.5\\
1	5.5	1.5\\
1	6	1.5\\
1	6.5	1.5\\
1	7	1.5\\
1	7.5	1.5\\
1	8	1.5\\
1	8.5	1.5\\
1	9	1.5\\
1	9.5	1.5\\
1.5	0	1.5\\
1.5	0.5	1.5\\
1.5	1	1.5\\
1.5	1.5	1.5\\
1.5	2	1.5\\
1.5	2.5	1.5\\
1.5	3	1.5\\
1.5	3.5	1.5\\
1.5	4	1.5\\
1.5	4.5	1.5\\
1.5	5	1.5\\
1.5	5.5	1.5\\
1.5	6	1.5\\
1.5	6.5	1.5\\
1.5	7	1.5\\
1.5	7.5	1.5\\
1.5	8	1.5\\
1.5	8.5	1.5\\
1.5	9	1.5\\
2	0	1.5\\
2	0.5	1.5\\
2	1	1.5\\
2	1.5	1.5\\
2	2	1.5\\
2	2.5	1.5\\
2	3	1.5\\
2	3.5	1.5\\
2	4	1.5\\
2	4.5	1.5\\
2	5	1.5\\
2	5.5	1.5\\
2	6	1.5\\
2	6.5	1.5\\
2	7	1.5\\
2	7.5	1.5\\
2	8	1.5\\
2	8.5	1.5\\
2.5	0	1.5\\
2.5	0.5	1.5\\
2.5	1	1.5\\
2.5	1.5	1.5\\
2.5	2	1.5\\
2.5	2.5	1.5\\
2.5	3	1.5\\
2.5	3.5	1.5\\
2.5	4	1.5\\
2.5	4.5	1.5\\
2.5	5	1.5\\
2.5	5.5	1.5\\
2.5	6	1.5\\
2.5	6.5	1.5\\
2.5	7	1.5\\
2.5	7.5	1.5\\
2.5	8	1.5\\
3	0	1.5\\
3	0.5	1.5\\
3	1	1.5\\
3	1.5	1.5\\
3	2	1.5\\
3	2.5	1.5\\
3	3	1.5\\
3	3.5	1.5\\
3	4	1.5\\
3	4.5	1.5\\
3	5	1.5\\
3	5.5	1.5\\
3	6	1.5\\
3	6.5	1.5\\
3	7	1.5\\
3	7.5	1.5\\
3.5	0	1.5\\
3.5	0.5	1.5\\
3.5	1	1.5\\
3.5	1.5	1.5\\
3.5	2	1.5\\
3.5	2.5	1.5\\
3.5	3	1.5\\
3.5	3.5	1.5\\
3.5	4	1.5\\
3.5	4.5	1.5\\
3.5	5	1.5\\
3.5	5.5	1.5\\
3.5	6	1.5\\
3.5	6.5	1.5\\
3.5	7	1.5\\
4	0	1.5\\
4	0.5	1.5\\
4	1	1.5\\
4	1.5	1.5\\
4	2	1.5\\
4	2.5	1.5\\
4	3	1.5\\
4	3.5	1.5\\
4	4	1.5\\
4	4.5	1.5\\
4	5	1.5\\
4	5.5	1.5\\
4	6	1.5\\
4	6.5	1.5\\
4.5	0	1.5\\
4.5	0.5	1.5\\
4.5	1	1.5\\
4.5	1.5	1.5\\
4.5	2	1.5\\
4.5	2.5	1.5\\
4.5	3	1.5\\
4.5	3.5	1.5\\
4.5	4	1.5\\
4.5	4.5	1.5\\
4.5	5	1.5\\
4.5	5.5	1.5\\
4.5	6	1.5\\
5	0	1.5\\
5	0.5	1.5\\
5	1	1.5\\
5	1.5	1.5\\
5	2	1.5\\
5	2.5	1.5\\
5	3	1.5\\
5	3.5	1.5\\
5	4	1.5\\
5	4.5	1.5\\
5	5	1.5\\
5	5.5	1.5\\
5.5	0	1.5\\
5.5	0.5	1.5\\
5.5	1	1.5\\
5.5	1.5	1.5\\
5.5	2	1.5\\
5.5	2.5	1.5\\
5.5	3	1.5\\
5.5	3.5	1.5\\
5.5	4	1.5\\
5.5	4.5	1.5\\
5.5	5	1.5\\
6	0	1.5\\
6	0.5	1.5\\
6	1	1.5\\
6	1.5	1.5\\
6	2	1.5\\
6	2.5	1.5\\
6	3	1.5\\
6	3.5	1.5\\
6	4	1.5\\
6	4.5	1.5\\
6.5	0	1.5\\
6.5	0.5	1.5\\
6.5	1	1.5\\
6.5	1.5	1.5\\
6.5	2	1.5\\
6.5	2.5	1.5\\
6.5	3	1.5\\
6.5	3.5	1.5\\
6.5	4	1.5\\
7	0	1.5\\
7	0.5	1.5\\
7	1	1.5\\
7	1.5	1.5\\
7	2	1.5\\
7	2.5	1.5\\
7	3	1.5\\
7	3.5	1.5\\
7.5	0	1.5\\
7.5	0.5	1.5\\
7.5	1	1.5\\
7.5	1.5	1.5\\
7.5	2	1.5\\
7.5	2.5	1.5\\
7.5	3	1.5\\
8	0	1.5\\
8	0.5	1.5\\
8	1	1.5\\
8	1.5	1.5\\
8	2	1.5\\
8	2.5	1.5\\
8.5	0	1.5\\
8.5	0.5	1.5\\
8.5	1	1.5\\
8.5	1.5	1.5\\
8.5	2	1.5\\
9	0	1.5\\
9	0.5	1.5\\
9	1	1.5\\
9	1.5	1.5\\
9.5	0	1.5\\
9.5	0.5	1.5\\
9.5	1	1.5\\
10	0	1.5\\
10	0.5	1.5\\
10.5	0	1.5\\
0	0	2\\
0	0.5	2\\
0	1	2\\
0	1.5	2\\
0	2	2\\
0	2.5	2\\
0	3	2\\
0	3.5	2\\
0	4	2\\
0	4.5	2\\
0	5	2\\
0	5.5	2\\
0	6	2\\
0	6.5	2\\
0	7	2\\
0	7.5	2\\
0	8	2\\
0	8.5	2\\
0	9	2\\
0	9.5	2\\
0	10	2\\
0.5	0	2\\
0.5	0.5	2\\
0.5	1	2\\
0.5	1.5	2\\
0.5	2	2\\
0.5	2.5	2\\
0.5	3	2\\
0.5	3.5	2\\
0.5	4	2\\
0.5	4.5	2\\
0.5	5	2\\
0.5	5.5	2\\
0.5	6	2\\
0.5	6.5	2\\
0.5	7	2\\
0.5	7.5	2\\
0.5	8	2\\
0.5	8.5	2\\
0.5	9	2\\
0.5	9.5	2\\
1	0	2\\
1	0.5	2\\
1	1	2\\
1	1.5	2\\
1	2	2\\
1	2.5	2\\
1	3	2\\
1	3.5	2\\
1	4	2\\
1	4.5	2\\
1	5	2\\
1	5.5	2\\
1	6	2\\
1	6.5	2\\
1	7	2\\
1	7.5	2\\
1	8	2\\
1	8.5	2\\
1	9	2\\
1.5	0	2\\
1.5	0.5	2\\
1.5	1	2\\
1.5	1.5	2\\
1.5	2	2\\
1.5	2.5	2\\
1.5	3	2\\
1.5	3.5	2\\
1.5	4	2\\
1.5	4.5	2\\
1.5	5	2\\
1.5	5.5	2\\
1.5	6	2\\
1.5	6.5	2\\
1.5	7	2\\
1.5	7.5	2\\
1.5	8	2\\
1.5	8.5	2\\
2	0	2\\
2	0.5	2\\
2	1	2\\
2	1.5	2\\
2	2	2\\
2	2.5	2\\
2	3	2\\
2	3.5	2\\
2	4	2\\
2	4.5	2\\
2	5	2\\
2	5.5	2\\
2	6	2\\
2	6.5	2\\
2	7	2\\
2	7.5	2\\
2	8	2\\
2.5	0	2\\
2.5	0.5	2\\
2.5	1	2\\
2.5	1.5	2\\
2.5	2	2\\
2.5	2.5	2\\
2.5	3	2\\
2.5	3.5	2\\
2.5	4	2\\
2.5	4.5	2\\
2.5	5	2\\
2.5	5.5	2\\
2.5	6	2\\
2.5	6.5	2\\
2.5	7	2\\
2.5	7.5	2\\
3	0	2\\
3	0.5	2\\
3	1	2\\
3	1.5	2\\
3	2	2\\
3	2.5	2\\
3	3	2\\
3	3.5	2\\
3	4	2\\
3	4.5	2\\
3	5	2\\
3	5.5	2\\
3	6	2\\
3	6.5	2\\
3	7	2\\
3.5	0	2\\
3.5	0.5	2\\
3.5	1	2\\
3.5	1.5	2\\
3.5	2	2\\
3.5	2.5	2\\
3.5	3	2\\
3.5	3.5	2\\
3.5	4	2\\
3.5	4.5	2\\
3.5	5	2\\
3.5	5.5	2\\
3.5	6	2\\
3.5	6.5	2\\
4	0	2\\
4	0.5	2\\
4	1	2\\
4	1.5	2\\
4	2	2\\
4	2.5	2\\
4	3	2\\
4	3.5	2\\
4	4	2\\
4	4.5	2\\
4	5	2\\
4	5.5	2\\
4	6	2\\
4.5	0	2\\
4.5	0.5	2\\
4.5	1	2\\
4.5	1.5	2\\
4.5	2	2\\
4.5	2.5	2\\
4.5	3	2\\
4.5	3.5	2\\
4.5	4	2\\
4.5	4.5	2\\
4.5	5	2\\
4.5	5.5	2\\
5	0	2\\
5	0.5	2\\
5	1	2\\
5	1.5	2\\
5	2	2\\
5	2.5	2\\
5	3	2\\
5	3.5	2\\
5	4	2\\
5	4.5	2\\
5	5	2\\
5.5	0	2\\
5.5	0.5	2\\
5.5	1	2\\
5.5	1.5	2\\
5.5	2	2\\
5.5	2.5	2\\
5.5	3	2\\
5.5	3.5	2\\
5.5	4	2\\
5.5	4.5	2\\
6	0	2\\
6	0.5	2\\
6	1	2\\
6	1.5	2\\
6	2	2\\
6	2.5	2\\
6	3	2\\
6	3.5	2\\
6	4	2\\
6.5	0	2\\
6.5	0.5	2\\
6.5	1	2\\
6.5	1.5	2\\
6.5	2	2\\
6.5	2.5	2\\
6.5	3	2\\
6.5	3.5	2\\
7	0	2\\
7	0.5	2\\
7	1	2\\
7	1.5	2\\
7	2	2\\
7	2.5	2\\
7	3	2\\
7.5	0	2\\
7.5	0.5	2\\
7.5	1	2\\
7.5	1.5	2\\
7.5	2	2\\
7.5	2.5	2\\
8	0	2\\
8	0.5	2\\
8	1	2\\
8	1.5	2\\
8	2	2\\
8.5	0	2\\
8.5	0.5	2\\
8.5	1	2\\
8.5	1.5	2\\
9	0	2\\
9	0.5	2\\
9	1	2\\
9.5	0	2\\
9.5	0.5	2\\
10	0	2\\
0	0	2.5\\
0	0.5	2.5\\
0	1	2.5\\
0	1.5	2.5\\
0	2	2.5\\
0	2.5	2.5\\
0	3	2.5\\
0	3.5	2.5\\
0	4	2.5\\
0	4.5	2.5\\
0	5	2.5\\
0	5.5	2.5\\
0	6	2.5\\
0	6.5	2.5\\
0	7	2.5\\
0	7.5	2.5\\
0	8	2.5\\
0	8.5	2.5\\
0	9	2.5\\
0	9.5	2.5\\
0.5	0	2.5\\
0.5	0.5	2.5\\
0.5	1	2.5\\
0.5	1.5	2.5\\
0.5	2	2.5\\
0.5	2.5	2.5\\
0.5	3	2.5\\
0.5	3.5	2.5\\
0.5	4	2.5\\
0.5	4.5	2.5\\
0.5	5	2.5\\
0.5	5.5	2.5\\
0.5	6	2.5\\
0.5	6.5	2.5\\
0.5	7	2.5\\
0.5	7.5	2.5\\
0.5	8	2.5\\
0.5	8.5	2.5\\
0.5	9	2.5\\
1	0	2.5\\
1	0.5	2.5\\
1	1	2.5\\
1	1.5	2.5\\
1	2	2.5\\
1	2.5	2.5\\
1	3	2.5\\
1	3.5	2.5\\
1	4	2.5\\
1	4.5	2.5\\
1	5	2.5\\
1	5.5	2.5\\
1	6	2.5\\
1	6.5	2.5\\
1	7	2.5\\
1	7.5	2.5\\
1	8	2.5\\
1	8.5	2.5\\
1.5	0	2.5\\
1.5	0.5	2.5\\
1.5	1	2.5\\
1.5	1.5	2.5\\
1.5	2	2.5\\
1.5	2.5	2.5\\
1.5	3	2.5\\
1.5	3.5	2.5\\
1.5	4	2.5\\
1.5	4.5	2.5\\
1.5	5	2.5\\
1.5	5.5	2.5\\
1.5	6	2.5\\
1.5	6.5	2.5\\
1.5	7	2.5\\
1.5	7.5	2.5\\
1.5	8	2.5\\
2	0	2.5\\
2	0.5	2.5\\
2	1	2.5\\
2	1.5	2.5\\
2	2	2.5\\
2	2.5	2.5\\
2	3	2.5\\
2	3.5	2.5\\
2	4	2.5\\
2	4.5	2.5\\
2	5	2.5\\
2	5.5	2.5\\
2	6	2.5\\
2	6.5	2.5\\
2	7	2.5\\
2	7.5	2.5\\
2.5	0	2.5\\
2.5	0.5	2.5\\
2.5	1	2.5\\
2.5	1.5	2.5\\
2.5	2	2.5\\
2.5	2.5	2.5\\
2.5	3	2.5\\
2.5	3.5	2.5\\
2.5	4	2.5\\
2.5	4.5	2.5\\
2.5	5	2.5\\
2.5	5.5	2.5\\
2.5	6	2.5\\
2.5	6.5	2.5\\
2.5	7	2.5\\
3	0	2.5\\
3	0.5	2.5\\
3	1	2.5\\
3	1.5	2.5\\
3	2	2.5\\
3	2.5	2.5\\
3	3	2.5\\
3	3.5	2.5\\
3	4	2.5\\
3	4.5	2.5\\
3	5	2.5\\
3	5.5	2.5\\
3	6	2.5\\
3	6.5	2.5\\
3.5	0	2.5\\
3.5	0.5	2.5\\
3.5	1	2.5\\
3.5	1.5	2.5\\
3.5	2	2.5\\
3.5	2.5	2.5\\
3.5	3	2.5\\
3.5	3.5	2.5\\
3.5	4	2.5\\
3.5	4.5	2.5\\
3.5	5	2.5\\
3.5	5.5	2.5\\
3.5	6	2.5\\
4	0	2.5\\
4	0.5	2.5\\
4	1	2.5\\
4	1.5	2.5\\
4	2	2.5\\
4	2.5	2.5\\
4	3	2.5\\
4	3.5	2.5\\
4	4	2.5\\
4	4.5	2.5\\
4	5	2.5\\
4	5.5	2.5\\
4.5	0	2.5\\
4.5	0.5	2.5\\
4.5	1	2.5\\
4.5	1.5	2.5\\
4.5	2	2.5\\
4.5	2.5	2.5\\
4.5	3	2.5\\
4.5	3.5	2.5\\
4.5	4	2.5\\
4.5	4.5	2.5\\
4.5	5	2.5\\
5	0	2.5\\
5	0.5	2.5\\
5	1	2.5\\
5	1.5	2.5\\
5	2	2.5\\
5	2.5	2.5\\
5	3	2.5\\
5	3.5	2.5\\
5	4	2.5\\
5	4.5	2.5\\
5.5	0	2.5\\
5.5	0.5	2.5\\
5.5	1	2.5\\
5.5	1.5	2.5\\
5.5	2	2.5\\
5.5	2.5	2.5\\
5.5	3	2.5\\
5.5	3.5	2.5\\
5.5	4	2.5\\
6	0	2.5\\
6	0.5	2.5\\
6	1	2.5\\
6	1.5	2.5\\
6	2	2.5\\
6	2.5	2.5\\
6	3	2.5\\
6	3.5	2.5\\
6.5	0	2.5\\
6.5	0.5	2.5\\
6.5	1	2.5\\
6.5	1.5	2.5\\
6.5	2	2.5\\
6.5	2.5	2.5\\
6.5	3	2.5\\
7	0	2.5\\
7	0.5	2.5\\
7	1	2.5\\
7	1.5	2.5\\
7	2	2.5\\
7	2.5	2.5\\
7.5	0	2.5\\
7.5	0.5	2.5\\
7.5	1	2.5\\
7.5	1.5	2.5\\
7.5	2	2.5\\
8	0	2.5\\
8	0.5	2.5\\
8	1	2.5\\
8	1.5	2.5\\
8.5	0	2.5\\
8.5	0.5	2.5\\
8.5	1	2.5\\
9	0	2.5\\
9	0.5	2.5\\
9.5	0	2.5\\
0	0	3\\
0	0.5	3\\
0	1	3\\
0	1.5	3\\
0	2	3\\
0	2.5	3\\
0	3	3\\
0	3.5	3\\
0	4	3\\
0	4.5	3\\
0	5	3\\
0	5.5	3\\
0	6	3\\
0	6.5	3\\
0	7	3\\
0	7.5	3\\
0	8	3\\
0	8.5	3\\
0	9	3\\
0.5	0	3\\
0.5	0.5	3\\
0.5	1	3\\
0.5	1.5	3\\
0.5	2	3\\
0.5	2.5	3\\
0.5	3	3\\
0.5	3.5	3\\
0.5	4	3\\
0.5	4.5	3\\
0.5	5	3\\
0.5	5.5	3\\
0.5	6	3\\
0.5	6.5	3\\
0.5	7	3\\
0.5	7.5	3\\
0.5	8	3\\
0.5	8.5	3\\
1	0	3\\
1	0.5	3\\
1	1	3\\
1	1.5	3\\
1	2	3\\
1	2.5	3\\
1	3	3\\
1	3.5	3\\
1	4	3\\
1	4.5	3\\
1	5	3\\
1	5.5	3\\
1	6	3\\
1	6.5	3\\
1	7	3\\
1	7.5	3\\
1	8	3\\
1.5	0	3\\
1.5	0.5	3\\
1.5	1	3\\
1.5	1.5	3\\
1.5	2	3\\
1.5	2.5	3\\
1.5	3	3\\
1.5	3.5	3\\
1.5	4	3\\
1.5	4.5	3\\
1.5	5	3\\
1.5	5.5	3\\
1.5	6	3\\
1.5	6.5	3\\
1.5	7	3\\
1.5	7.5	3\\
2	0	3\\
2	0.5	3\\
2	1	3\\
2	1.5	3\\
2	2	3\\
2	2.5	3\\
2	3	3\\
2	3.5	3\\
2	4	3\\
2	4.5	3\\
2	5	3\\
2	5.5	3\\
2	6	3\\
2	6.5	3\\
2	7	3\\
2.5	0	3\\
2.5	0.5	3\\
2.5	1	3\\
2.5	1.5	3\\
2.5	2	3\\
2.5	2.5	3\\
2.5	3	3\\
2.5	3.5	3\\
2.5	4	3\\
2.5	4.5	3\\
2.5	5	3\\
2.5	5.5	3\\
2.5	6	3\\
2.5	6.5	3\\
3	0	3\\
3	0.5	3\\
3	1	3\\
3	1.5	3\\
3	2	3\\
3	2.5	3\\
3	3	3\\
3	3.5	3\\
3	4	3\\
3	4.5	3\\
3	5	3\\
3	5.5	3\\
3	6	3\\
3.5	0	3\\
3.5	0.5	3\\
3.5	1	3\\
3.5	1.5	3\\
3.5	2	3\\
3.5	2.5	3\\
3.5	3	3\\
3.5	3.5	3\\
3.5	4	3\\
3.5	4.5	3\\
3.5	5	3\\
3.5	5.5	3\\
4	0	3\\
4	0.5	3\\
4	1	3\\
4	1.5	3\\
4	2	3\\
4	2.5	3\\
4	3	3\\
4	3.5	3\\
4	4	3\\
4	4.5	3\\
4	5	3\\
4.5	0	3\\
4.5	0.5	3\\
4.5	1	3\\
4.5	1.5	3\\
4.5	2	3\\
4.5	2.5	3\\
4.5	3	3\\
4.5	3.5	3\\
4.5	4	3\\
4.5	4.5	3\\
5	0	3\\
5	0.5	3\\
5	1	3\\
5	1.5	3\\
5	2	3\\
5	2.5	3\\
5	3	3\\
5	3.5	3\\
5	4	3\\
5.5	0	3\\
5.5	0.5	3\\
5.5	1	3\\
5.5	1.5	3\\
5.5	2	3\\
5.5	2.5	3\\
5.5	3	3\\
5.5	3.5	3\\
6	0	3\\
6	0.5	3\\
6	1	3\\
6	1.5	3\\
6	2	3\\
6	2.5	3\\
6	3	3\\
6.5	0	3\\
6.5	0.5	3\\
6.5	1	3\\
6.5	1.5	3\\
6.5	2	3\\
6.5	2.5	3\\
7	0	3\\
7	0.5	3\\
7	1	3\\
7	1.5	3\\
7	2	3\\
7.5	0	3\\
7.5	0.5	3\\
7.5	1	3\\
7.5	1.5	3\\
8	0	3\\
8	0.5	3\\
8	1	3\\
8.5	0	3\\
8.5	0.5	3\\
9	0	3\\
0	0	3.5\\
0	0.5	3.5\\
0	1	3.5\\
0	1.5	3.5\\
0	2	3.5\\
0	2.5	3.5\\
0	3	3.5\\
0	3.5	3.5\\
0	4	3.5\\
0	4.5	3.5\\
0	5	3.5\\
0	5.5	3.5\\
0	6	3.5\\
0	6.5	3.5\\
0	7	3.5\\
0	7.5	3.5\\
0	8	3.5\\
0	8.5	3.5\\
0.5	0	3.5\\
0.5	0.5	3.5\\
0.5	1	3.5\\
0.5	1.5	3.5\\
0.5	2	3.5\\
0.5	2.5	3.5\\
0.5	3	3.5\\
0.5	3.5	3.5\\
0.5	4	3.5\\
0.5	4.5	3.5\\
0.5	5	3.5\\
0.5	5.5	3.5\\
0.5	6	3.5\\
0.5	6.5	3.5\\
0.5	7	3.5\\
0.5	7.5	3.5\\
0.5	8	3.5\\
1	0	3.5\\
1	0.5	3.5\\
1	1	3.5\\
1	1.5	3.5\\
1	2	3.5\\
1	2.5	3.5\\
1	3	3.5\\
1	3.5	3.5\\
1	4	3.5\\
1	4.5	3.5\\
1	5	3.5\\
1	5.5	3.5\\
1	6	3.5\\
1	6.5	3.5\\
1	7	3.5\\
1	7.5	3.5\\
1.5	0	3.5\\
1.5	0.5	3.5\\
1.5	1	3.5\\
1.5	1.5	3.5\\
1.5	2	3.5\\
1.5	2.5	3.5\\
1.5	3	3.5\\
1.5	3.5	3.5\\
1.5	4	3.5\\
1.5	4.5	3.5\\
1.5	5	3.5\\
1.5	5.5	3.5\\
1.5	6	3.5\\
1.5	6.5	3.5\\
1.5	7	3.5\\
2	0	3.5\\
2	0.5	3.5\\
2	1	3.5\\
2	1.5	3.5\\
2	2	3.5\\
2	2.5	3.5\\
2	3	3.5\\
2	3.5	3.5\\
2	4	3.5\\
2	4.5	3.5\\
2	5	3.5\\
2	5.5	3.5\\
2	6	3.5\\
2	6.5	3.5\\
2.5	0	3.5\\
2.5	0.5	3.5\\
2.5	1	3.5\\
2.5	1.5	3.5\\
2.5	2	3.5\\
2.5	2.5	3.5\\
2.5	3	3.5\\
2.5	3.5	3.5\\
2.5	4	3.5\\
2.5	4.5	3.5\\
2.5	5	3.5\\
2.5	5.5	3.5\\
2.5	6	3.5\\
3	0	3.5\\
3	0.5	3.5\\
3	1	3.5\\
3	1.5	3.5\\
3	2	3.5\\
3	2.5	3.5\\
3	3	3.5\\
3	3.5	3.5\\
3	4	3.5\\
3	4.5	3.5\\
3	5	3.5\\
3	5.5	3.5\\
3.5	0	3.5\\
3.5	0.5	3.5\\
3.5	1	3.5\\
3.5	1.5	3.5\\
3.5	2	3.5\\
3.5	2.5	3.5\\
3.5	3	3.5\\
3.5	3.5	3.5\\
3.5	4	3.5\\
3.5	4.5	3.5\\
3.5	5	3.5\\
4	0	3.5\\
4	0.5	3.5\\
4	1	3.5\\
4	1.5	3.5\\
4	2	3.5\\
4	2.5	3.5\\
4	3	3.5\\
4	3.5	3.5\\
4	4	3.5\\
4	4.5	3.5\\
4.5	0	3.5\\
4.5	0.5	3.5\\
4.5	1	3.5\\
4.5	1.5	3.5\\
4.5	2	3.5\\
4.5	2.5	3.5\\
4.5	3	3.5\\
4.5	3.5	3.5\\
4.5	4	3.5\\
5	0	3.5\\
5	0.5	3.5\\
5	1	3.5\\
5	1.5	3.5\\
5	2	3.5\\
5	2.5	3.5\\
5	3	3.5\\
5	3.5	3.5\\
5.5	0	3.5\\
5.5	0.5	3.5\\
5.5	1	3.5\\
5.5	1.5	3.5\\
5.5	2	3.5\\
5.5	2.5	3.5\\
5.5	3	3.5\\
6	0	3.5\\
6	0.5	3.5\\
6	1	3.5\\
6	1.5	3.5\\
6	2	3.5\\
6	2.5	3.5\\
6.5	0	3.5\\
6.5	0.5	3.5\\
6.5	1	3.5\\
6.5	1.5	3.5\\
6.5	2	3.5\\
7	0	3.5\\
7	0.5	3.5\\
7	1	3.5\\
7	1.5	3.5\\
7.5	0	3.5\\
7.5	0.5	3.5\\
7.5	1	3.5\\
8	0	3.5\\
8	0.5	3.5\\
8.5	0	3.5\\
0	0	4\\
0	0.5	4\\
0	1	4\\
0	1.5	4\\
0	2	4\\
0	2.5	4\\
0	3	4\\
0	3.5	4\\
0	4	4\\
0	4.5	4\\
0	5	4\\
0	5.5	4\\
0	6	4\\
0	6.5	4\\
0	7	4\\
0	7.5	4\\
0	8	4\\
0.5	0	4\\
0.5	0.5	4\\
0.5	1	4\\
0.5	1.5	4\\
0.5	2	4\\
0.5	2.5	4\\
0.5	3	4\\
0.5	3.5	4\\
0.5	4	4\\
0.5	4.5	4\\
0.5	5	4\\
0.5	5.5	4\\
0.5	6	4\\
0.5	6.5	4\\
0.5	7	4\\
0.5	7.5	4\\
1	0	4\\
1	0.5	4\\
1	1	4\\
1	1.5	4\\
1	2	4\\
1	2.5	4\\
1	3	4\\
1	3.5	4\\
1	4	4\\
1	4.5	4\\
1	5	4\\
1	5.5	4\\
1	6	4\\
1	6.5	4\\
1	7	4\\
1.5	0	4\\
1.5	0.5	4\\
1.5	1	4\\
1.5	1.5	4\\
1.5	2	4\\
1.5	2.5	4\\
1.5	3	4\\
1.5	3.5	4\\
1.5	4	4\\
1.5	4.5	4\\
1.5	5	4\\
1.5	5.5	4\\
1.5	6	4\\
1.5	6.5	4\\
2	0	4\\
2	0.5	4\\
2	1	4\\
2	1.5	4\\
2	2	4\\
2	2.5	4\\
2	3	4\\
2	3.5	4\\
2	4	4\\
2	4.5	4\\
2	5	4\\
2	5.5	4\\
2	6	4\\
2.5	0	4\\
2.5	0.5	4\\
2.5	1	4\\
2.5	1.5	4\\
2.5	2	4\\
2.5	2.5	4\\
2.5	3	4\\
2.5	3.5	4\\
2.5	4	4\\
2.5	4.5	4\\
2.5	5	4\\
2.5	5.5	4\\
3	0	4\\
3	0.5	4\\
3	1	4\\
3	1.5	4\\
3	2	4\\
3	2.5	4\\
3	3	4\\
3	3.5	4\\
3	4	4\\
3	4.5	4\\
3	5	4\\
3.5	0	4\\
3.5	0.5	4\\
3.5	1	4\\
3.5	1.5	4\\
3.5	2	4\\
3.5	2.5	4\\
3.5	3	4\\
3.5	3.5	4\\
3.5	4	4\\
3.5	4.5	4\\
4	0	4\\
4	0.5	4\\
4	1	4\\
4	1.5	4\\
4	2	4\\
4	2.5	4\\
4	3	4\\
4	3.5	4\\
4	4	4\\
4.5	0	4\\
4.5	0.5	4\\
4.5	1	4\\
4.5	1.5	4\\
4.5	2	4\\
4.5	2.5	4\\
4.5	3	4\\
4.5	3.5	4\\
5	0	4\\
5	0.5	4\\
5	1	4\\
5	1.5	4\\
5	2	4\\
5	2.5	4\\
5	3	4\\
5.5	0	4\\
5.5	0.5	4\\
5.5	1	4\\
5.5	1.5	4\\
5.5	2	4\\
5.5	2.5	4\\
6	0	4\\
6	0.5	4\\
6	1	4\\
6	1.5	4\\
6	2	4\\
6.5	0	4\\
6.5	0.5	4\\
6.5	1	4\\
6.5	1.5	4\\
7	0	4\\
7	0.5	4\\
7	1	4\\
7.5	0	4\\
7.5	0.5	4\\
8	0	4\\
0	0	4.5\\
0	0.5	4.5\\
0	1	4.5\\
0	1.5	4.5\\
0	2	4.5\\
0	2.5	4.5\\
0	3	4.5\\
0	3.5	4.5\\
0	4	4.5\\
0	4.5	4.5\\
0	5	4.5\\
0	5.5	4.5\\
0	6	4.5\\
0	6.5	4.5\\
0	7	4.5\\
0	7.5	4.5\\
0.5	0	4.5\\
0.5	0.5	4.5\\
0.5	1	4.5\\
0.5	1.5	4.5\\
0.5	2	4.5\\
0.5	2.5	4.5\\
0.5	3	4.5\\
0.5	3.5	4.5\\
0.5	4	4.5\\
0.5	4.5	4.5\\
0.5	5	4.5\\
0.5	5.5	4.5\\
0.5	6	4.5\\
0.5	6.5	4.5\\
0.5	7	4.5\\
1	0	4.5\\
1	0.5	4.5\\
1	1	4.5\\
1	1.5	4.5\\
1	2	4.5\\
1	2.5	4.5\\
1	3	4.5\\
1	3.5	4.5\\
1	4	4.5\\
1	4.5	4.5\\
1	5	4.5\\
1	5.5	4.5\\
1	6	4.5\\
1	6.5	4.5\\
1.5	0	4.5\\
1.5	0.5	4.5\\
1.5	1	4.5\\
1.5	1.5	4.5\\
1.5	2	4.5\\
1.5	2.5	4.5\\
1.5	3	4.5\\
1.5	3.5	4.5\\
1.5	4	4.5\\
1.5	4.5	4.5\\
1.5	5	4.5\\
1.5	5.5	4.5\\
1.5	6	4.5\\
2	0	4.5\\
2	0.5	4.5\\
2	1	4.5\\
2	1.5	4.5\\
2	2	4.5\\
2	2.5	4.5\\
2	3	4.5\\
2	3.5	4.5\\
2	4	4.5\\
2	4.5	4.5\\
2	5	4.5\\
2	5.5	4.5\\
2.5	0	4.5\\
2.5	0.5	4.5\\
2.5	1	4.5\\
2.5	1.5	4.5\\
2.5	2	4.5\\
2.5	2.5	4.5\\
2.5	3	4.5\\
2.5	3.5	4.5\\
2.5	4	4.5\\
2.5	4.5	4.5\\
2.5	5	4.5\\
3	0	4.5\\
3	0.5	4.5\\
3	1	4.5\\
3	1.5	4.5\\
3	2	4.5\\
3	2.5	4.5\\
3	3	4.5\\
3	3.5	4.5\\
3	4	4.5\\
3	4.5	4.5\\
3.5	0	4.5\\
3.5	0.5	4.5\\
3.5	1	4.5\\
3.5	1.5	4.5\\
3.5	2	4.5\\
3.5	2.5	4.5\\
3.5	3	4.5\\
3.5	3.5	4.5\\
3.5	4	4.5\\
4	0	4.5\\
4	0.5	4.5\\
4	1	4.5\\
4	1.5	4.5\\
4	2	4.5\\
4	2.5	4.5\\
4	3	4.5\\
4	3.5	4.5\\
4.5	0	4.5\\
4.5	0.5	4.5\\
4.5	1	4.5\\
4.5	1.5	4.5\\
4.5	2	4.5\\
4.5	2.5	4.5\\
4.5	3	4.5\\
5	0	4.5\\
5	0.5	4.5\\
5	1	4.5\\
5	1.5	4.5\\
5	2	4.5\\
5	2.5	4.5\\
5.5	0	4.5\\
5.5	0.5	4.5\\
5.5	1	4.5\\
5.5	1.5	4.5\\
5.5	2	4.5\\
6	0	4.5\\
6	0.5	4.5\\
6	1	4.5\\
6	1.5	4.5\\
6.5	0	4.5\\
6.5	0.5	4.5\\
6.5	1	4.5\\
7	0	4.5\\
7	0.5	4.5\\
7.5	0	4.5\\
0	0	5\\
0	0.5	5\\
0	1	5\\
0	1.5	5\\
0	2	5\\
0	2.5	5\\
0	3	5\\
0	3.5	5\\
0	4	5\\
0	4.5	5\\
0	5	5\\
0	5.5	5\\
0	6	5\\
0	6.5	5\\
0	7	5\\
0.5	0	5\\
0.5	0.5	5\\
0.5	1	5\\
0.5	1.5	5\\
0.5	2	5\\
0.5	2.5	5\\
0.5	3	5\\
0.5	3.5	5\\
0.5	4	5\\
0.5	4.5	5\\
0.5	5	5\\
0.5	5.5	5\\
0.5	6	5\\
0.5	6.5	5\\
1	0	5\\
1	0.5	5\\
1	1	5\\
1	1.5	5\\
1	2	5\\
1	2.5	5\\
1	3	5\\
1	3.5	5\\
1	4	5\\
1	4.5	5\\
1	5	5\\
1	5.5	5\\
1	6	5\\
1.5	0	5\\
1.5	0.5	5\\
1.5	1	5\\
1.5	1.5	5\\
1.5	2	5\\
1.5	2.5	5\\
1.5	3	5\\
1.5	3.5	5\\
1.5	4	5\\
1.5	4.5	5\\
1.5	5	5\\
1.5	5.5	5\\
2	0	5\\
2	0.5	5\\
2	1	5\\
2	1.5	5\\
2	2	5\\
2	2.5	5\\
2	3	5\\
2	3.5	5\\
2	4	5\\
2	4.5	5\\
2	5	5\\
2.5	0	5\\
2.5	0.5	5\\
2.5	1	5\\
2.5	1.5	5\\
2.5	2	5\\
2.5	2.5	5\\
2.5	3	5\\
2.5	3.5	5\\
2.5	4	5\\
2.5	4.5	5\\
3	0	5\\
3	0.5	5\\
3	1	5\\
3	1.5	5\\
3	2	5\\
3	2.5	5\\
3	3	5\\
3	3.5	5\\
3	4	5\\
3.5	0	5\\
3.5	0.5	5\\
3.5	1	5\\
3.5	1.5	5\\
3.5	2	5\\
3.5	2.5	5\\
3.5	3	5\\
3.5	3.5	5\\
4	0	5\\
4	0.5	5\\
4	1	5\\
4	1.5	5\\
4	2	5\\
4	2.5	5\\
4	3	5\\
4.5	0	5\\
4.5	0.5	5\\
4.5	1	5\\
4.5	1.5	5\\
4.5	2	5\\
4.5	2.5	5\\
5	0	5\\
5	0.5	5\\
5	1	5\\
5	1.5	5\\
5	2	5\\
5.5	0	5\\
5.5	0.5	5\\
5.5	1	5\\
5.5	1.5	5\\
6	0	5\\
6	0.5	5\\
6	1	5\\
6.5	0	5\\
6.5	0.5	5\\
7	0	5\\
0	0	5.5\\
0	0.5	5.5\\
0	1	5.5\\
0	1.5	5.5\\
0	2	5.5\\
0	2.5	5.5\\
0	3	5.5\\
0	3.5	5.5\\
0	4	5.5\\
0	4.5	5.5\\
0	5	5.5\\
0	5.5	5.5\\
0	6	5.5\\
0	6.5	5.5\\
0.5	0	5.5\\
0.5	0.5	5.5\\
0.5	1	5.5\\
0.5	1.5	5.5\\
0.5	2	5.5\\
0.5	2.5	5.5\\
0.5	3	5.5\\
0.5	3.5	5.5\\
0.5	4	5.5\\
0.5	4.5	5.5\\
0.5	5	5.5\\
0.5	5.5	5.5\\
0.5	6	5.5\\
1	0	5.5\\
1	0.5	5.5\\
1	1	5.5\\
1	1.5	5.5\\
1	2	5.5\\
1	2.5	5.5\\
1	3	5.5\\
1	3.5	5.5\\
1	4	5.5\\
1	4.5	5.5\\
1	5	5.5\\
1	5.5	5.5\\
1.5	0	5.5\\
1.5	0.5	5.5\\
1.5	1	5.5\\
1.5	1.5	5.5\\
1.5	2	5.5\\
1.5	2.5	5.5\\
1.5	3	5.5\\
1.5	3.5	5.5\\
1.5	4	5.5\\
1.5	4.5	5.5\\
1.5	5	5.5\\
2	0	5.5\\
2	0.5	5.5\\
2	1	5.5\\
2	1.5	5.5\\
2	2	5.5\\
2	2.5	5.5\\
2	3	5.5\\
2	3.5	5.5\\
2	4	5.5\\
2	4.5	5.5\\
2.5	0	5.5\\
2.5	0.5	5.5\\
2.5	1	5.5\\
2.5	1.5	5.5\\
2.5	2	5.5\\
2.5	2.5	5.5\\
2.5	3	5.5\\
2.5	3.5	5.5\\
2.5	4	5.5\\
3	0	5.5\\
3	0.5	5.5\\
3	1	5.5\\
3	1.5	5.5\\
3	2	5.5\\
3	2.5	5.5\\
3	3	5.5\\
3	3.5	5.5\\
3.5	0	5.5\\
3.5	0.5	5.5\\
3.5	1	5.5\\
3.5	1.5	5.5\\
3.5	2	5.5\\
3.5	2.5	5.5\\
3.5	3	5.5\\
4	0	5.5\\
4	0.5	5.5\\
4	1	5.5\\
4	1.5	5.5\\
4	2	5.5\\
4	2.5	5.5\\
4.5	0	5.5\\
4.5	0.5	5.5\\
4.5	1	5.5\\
4.5	1.5	5.5\\
4.5	2	5.5\\
5	0	5.5\\
5	0.5	5.5\\
5	1	5.5\\
5	1.5	5.5\\
5.5	0	5.5\\
5.5	0.5	5.5\\
5.5	1	5.5\\
6	0	5.5\\
6	0.5	5.5\\
6.5	0	5.5\\
0	0	6\\
0	0.5	6\\
0	1	6\\
0	1.5	6\\
0	2	6\\
0	2.5	6\\
0	3	6\\
0	3.5	6\\
0	4	6\\
0	4.5	6\\
0	5	6\\
0	5.5	6\\
0	6	6\\
0.5	0	6\\
0.5	0.5	6\\
0.5	1	6\\
0.5	1.5	6\\
0.5	2	6\\
0.5	2.5	6\\
0.5	3	6\\
0.5	3.5	6\\
0.5	4	6\\
0.5	4.5	6\\
0.5	5	6\\
0.5	5.5	6\\
1	0	6\\
1	0.5	6\\
1	1	6\\
1	1.5	6\\
1	2	6\\
1	2.5	6\\
1	3	6\\
1	3.5	6\\
1	4	6\\
1	4.5	6\\
1	5	6\\
1.5	0	6\\
1.5	0.5	6\\
1.5	1	6\\
1.5	1.5	6\\
1.5	2	6\\
1.5	2.5	6\\
1.5	3	6\\
1.5	3.5	6\\
1.5	4	6\\
1.5	4.5	6\\
2	0	6\\
2	0.5	6\\
2	1	6\\
2	1.5	6\\
2	2	6\\
2	2.5	6\\
2	3	6\\
2	3.5	6\\
2	4	6\\
2.5	0	6\\
2.5	0.5	6\\
2.5	1	6\\
2.5	1.5	6\\
2.5	2	6\\
2.5	2.5	6\\
2.5	3	6\\
2.5	3.5	6\\
3	0	6\\
3	0.5	6\\
3	1	6\\
3	1.5	6\\
3	2	6\\
3	2.5	6\\
3	3	6\\
3.5	0	6\\
3.5	0.5	6\\
3.5	1	6\\
3.5	1.5	6\\
3.5	2	6\\
3.5	2.5	6\\
4	0	6\\
4	0.5	6\\
4	1	6\\
4	1.5	6\\
4	2	6\\
4.5	0	6\\
4.5	0.5	6\\
4.5	1	6\\
4.5	1.5	6\\
5	0	6\\
5	0.5	6\\
5	1	6\\
5.5	0	6\\
5.5	0.5	6\\
6	0	6\\
0	0	6.5\\
0	0.5	6.5\\
0	1	6.5\\
0	1.5	6.5\\
0	2	6.5\\
0	2.5	6.5\\
0	3	6.5\\
0	3.5	6.5\\
0	4	6.5\\
0	4.5	6.5\\
0	5	6.5\\
0	5.5	6.5\\
0.5	0	6.5\\
0.5	0.5	6.5\\
0.5	1	6.5\\
0.5	1.5	6.5\\
0.5	2	6.5\\
0.5	2.5	6.5\\
0.5	3	6.5\\
0.5	3.5	6.5\\
0.5	4	6.5\\
0.5	4.5	6.5\\
0.5	5	6.5\\
1	0	6.5\\
1	0.5	6.5\\
1	1	6.5\\
1	1.5	6.5\\
1	2	6.5\\
1	2.5	6.5\\
1	3	6.5\\
1	3.5	6.5\\
1	4	6.5\\
1	4.5	6.5\\
1.5	0	6.5\\
1.5	0.5	6.5\\
1.5	1	6.5\\
1.5	1.5	6.5\\
1.5	2	6.5\\
1.5	2.5	6.5\\
1.5	3	6.5\\
1.5	3.5	6.5\\
1.5	4	6.5\\
2	0	6.5\\
2	0.5	6.5\\
2	1	6.5\\
2	1.5	6.5\\
2	2	6.5\\
2	2.5	6.5\\
2	3	6.5\\
2	3.5	6.5\\
2.5	0	6.5\\
2.5	0.5	6.5\\
2.5	1	6.5\\
2.5	1.5	6.5\\
2.5	2	6.5\\
2.5	2.5	6.5\\
2.5	3	6.5\\
3	0	6.5\\
3	0.5	6.5\\
3	1	6.5\\
3	1.5	6.5\\
3	2	6.5\\
3	2.5	6.5\\
3.5	0	6.5\\
3.5	0.5	6.5\\
3.5	1	6.5\\
3.5	1.5	6.5\\
3.5	2	6.5\\
4	0	6.5\\
4	0.5	6.5\\
4	1	6.5\\
4	1.5	6.5\\
4.5	0	6.5\\
4.5	0.5	6.5\\
4.5	1	6.5\\
5	0	6.5\\
5	0.5	6.5\\
5.5	0	6.5\\
0	0	7\\
0	0.5	7\\
0	1	7\\
0	1.5	7\\
0	2	7\\
0	2.5	7\\
0	3	7\\
0	3.5	7\\
0	4	7\\
0	4.5	7\\
0	5	7\\
0.5	0	7\\
0.5	0.5	7\\
0.5	1	7\\
0.5	1.5	7\\
0.5	2	7\\
0.5	2.5	7\\
0.5	3	7\\
0.5	3.5	7\\
0.5	4	7\\
0.5	4.5	7\\
1	0	7\\
1	0.5	7\\
1	1	7\\
1	1.5	7\\
1	2	7\\
1	2.5	7\\
1	3	7\\
1	3.5	7\\
1	4	7\\
1.5	0	7\\
1.5	0.5	7\\
1.5	1	7\\
1.5	1.5	7\\
1.5	2	7\\
1.5	2.5	7\\
1.5	3	7\\
1.5	3.5	7\\
2	0	7\\
2	0.5	7\\
2	1	7\\
2	1.5	7\\
2	2	7\\
2	2.5	7\\
2	3	7\\
2.5	0	7\\
2.5	0.5	7\\
2.5	1	7\\
2.5	1.5	7\\
2.5	2	7\\
2.5	2.5	7\\
3	0	7\\
3	0.5	7\\
3	1	7\\
3	1.5	7\\
3	2	7\\
3.5	0	7\\
3.5	0.5	7\\
3.5	1	7\\
3.5	1.5	7\\
4	0	7\\
4	0.5	7\\
4	1	7\\
4.5	0	7\\
4.5	0.5	7\\
5	0	7\\
0	0	7.5\\
0	0.5	7.5\\
0	1	7.5\\
0	1.5	7.5\\
0	2	7.5\\
0	2.5	7.5\\
0	3	7.5\\
0	3.5	7.5\\
0	4	7.5\\
0	4.5	7.5\\
0.5	0	7.5\\
0.5	0.5	7.5\\
0.5	1	7.5\\
0.5	1.5	7.5\\
0.5	2	7.5\\
0.5	2.5	7.5\\
0.5	3	7.5\\
0.5	3.5	7.5\\
0.5	4	7.5\\
1	0	7.5\\
1	0.5	7.5\\
1	1	7.5\\
1	1.5	7.5\\
1	2	7.5\\
1	2.5	7.5\\
1	3	7.5\\
1	3.5	7.5\\
1.5	0	7.5\\
1.5	0.5	7.5\\
1.5	1	7.5\\
1.5	1.5	7.5\\
1.5	2	7.5\\
1.5	2.5	7.5\\
1.5	3	7.5\\
2	0	7.5\\
2	0.5	7.5\\
2	1	7.5\\
2	1.5	7.5\\
2	2	7.5\\
2	2.5	7.5\\
2.5	0	7.5\\
2.5	0.5	7.5\\
2.5	1	7.5\\
2.5	1.5	7.5\\
2.5	2	7.5\\
3	0	7.5\\
3	0.5	7.5\\
3	1	7.5\\
3	1.5	7.5\\
3.5	0	7.5\\
3.5	0.5	7.5\\
3.5	1	7.5\\
4	0	7.5\\
4	0.5	7.5\\
4.5	0	7.5\\
0	0	8\\
0	0.5	8\\
0	1	8\\
0	1.5	8\\
0	2	8\\
0	2.5	8\\
0	3	8\\
0	3.5	8\\
0	4	8\\
0.5	0	8\\
0.5	0.5	8\\
0.5	1	8\\
0.5	1.5	8\\
0.5	2	8\\
0.5	2.5	8\\
0.5	3	8\\
0.5	3.5	8\\
1	0	8\\
1	0.5	8\\
1	1	8\\
1	1.5	8\\
1	2	8\\
1	2.5	8\\
1	3	8\\
1.5	0	8\\
1.5	0.5	8\\
1.5	1	8\\
1.5	1.5	8\\
1.5	2	8\\
1.5	2.5	8\\
2	0	8\\
2	0.5	8\\
2	1	8\\
2	1.5	8\\
2	2	8\\
2.5	0	8\\
2.5	0.5	8\\
2.5	1	8\\
2.5	1.5	8\\
3	0	8\\
3	0.5	8\\
3	1	8\\
3.5	0	8\\
3.5	0.5	8\\
4	0	8\\
0	0	8.5\\
0	0.5	8.5\\
0	1	8.5\\
0	1.5	8.5\\
0	2	8.5\\
0	2.5	8.5\\
0	3	8.5\\
0	3.5	8.5\\
0.5	0	8.5\\
0.5	0.5	8.5\\
0.5	1	8.5\\
0.5	1.5	8.5\\
0.5	2	8.5\\
0.5	2.5	8.5\\
0.5	3	8.5\\
1	0	8.5\\
1	0.5	8.5\\
1	1	8.5\\
1	1.5	8.5\\
1	2	8.5\\
1	2.5	8.5\\
1.5	0	8.5\\
1.5	0.5	8.5\\
1.5	1	8.5\\
1.5	1.5	8.5\\
1.5	2	8.5\\
2	0	8.5\\
2	0.5	8.5\\
2	1	8.5\\
2	1.5	8.5\\
2.5	0	8.5\\
2.5	0.5	8.5\\
2.5	1	8.5\\
3	0	8.5\\
3	0.5	8.5\\
3.5	0	8.5\\
0	0	9\\
0	0.5	9\\
0	1	9\\
0	1.5	9\\
0	2	9\\
0	2.5	9\\
0	3	9\\
0.5	0	9\\
0.5	0.5	9\\
0.5	1	9\\
0.5	1.5	9\\
0.5	2	9\\
0.5	2.5	9\\
1	0	9\\
1	0.5	9\\
1	1	9\\
1	1.5	9\\
1	2	9\\
1.5	0	9\\
1.5	0.5	9\\
1.5	1	9\\
1.5	1.5	9\\
2	0	9\\
2	0.5	9\\
2	1	9\\
2.5	0	9\\
2.5	0.5	9\\
3	0	9\\
0	0	9.5\\
0	0.5	9.5\\
0	1	9.5\\
0	1.5	9.5\\
0	2	9.5\\
0	2.5	9.5\\
0.5	0	9.5\\
0.5	0.5	9.5\\
0.5	1	9.5\\
0.5	1.5	9.5\\
0.5	2	9.5\\
1	0	9.5\\
1	0.5	9.5\\
1	1	9.5\\
1	1.5	9.5\\
1.5	0	9.5\\
1.5	0.5	9.5\\
1.5	1	9.5\\
2	0	9.5\\
2	0.5	9.5\\
2.5	0	9.5\\
0	0	10\\
0	0.5	10\\
0	1	10\\
0	1.5	10\\
0	2	10\\
0.5	0	10\\
0.5	0.5	10\\
0.5	1	10\\
0.5	1.5	10\\
1	0	10\\
1	0.5	10\\
1	1	10\\
1.5	0	10\\
1.5	0.5	10\\
2	0	10\\
0	0	10.5\\
0	0.5	10.5\\
0	1	10.5\\
0	1.5	10.5\\
0.5	0	10.5\\
0.5	0.5	10.5\\
0.5	1	10.5\\
1	0	10.5\\
1	0.5	10.5\\
1.5	0	10.5\\
0	0	11\\
0	0.5	11\\
0	1	11\\
0.5	0	11\\
0.5	0.5	11\\
1	0	11\\
0	0	11.5\\
0	0.5	11.5\\
0.5	0	11.5\\
0	0	12\\
};
 \addplot3 [color=mycolor2,mark size=1pt,only marks,mark=square*,mark options={solid}]
 table[row sep=crcr] {%
0	12	0\\
0.5	11.5	0\\
1	11	0\\
1.5	10.5	0\\
2	10	0\\
2.5	9.5	0\\
3	9	0\\
3.5	8.5	0\\
4	8	0\\
4.5	7.5	0\\
5	7	0\\
5.5	6.5	0\\
6	6	0\\
6.5	5.5	0\\
7	5	0\\
7.5	4.5	0\\
8	4	0\\
8.5	3.5	0\\
9	3	0\\
9.5	2.5	0\\
10	2	0\\
10.5	1.5	0\\
11	1	0\\
11.5	0.5	0\\
12	0	0\\
0	11.5	0.5\\
0.5	11	0.5\\
1	10.5	0.5\\
1.5	10	0.5\\
2	9.5	0.5\\
2.5	9	0.5\\
3	8.5	0.5\\
3.5	8	0.5\\
4	7.5	0.5\\
4.5	7	0.5\\
5	6.5	0.5\\
5.5	6	0.5\\
6	5.5	0.5\\
6.5	5	0.5\\
7	4.5	0.5\\
7.5	4	0.5\\
8	3.5	0.5\\
8.5	3	0.5\\
9	2.5	0.5\\
9.5	2	0.5\\
10	1.5	0.5\\
10.5	1	0.5\\
11	0.5	0.5\\
11.5	0	0.5\\
0	11	1\\
0.5	10.5	1\\
1	10	1\\
1.5	9.5	1\\
2	9	1\\
2.5	8.5	1\\
3	8	1\\
3.5	7.5	1\\
4	7	1\\
4.5	6.5	1\\
5	6	1\\
5.5	5.5	1\\
6	5	1\\
6.5	4.5	1\\
7	4	1\\
7.5	3.5	1\\
8	3	1\\
8.5	2.5	1\\
9	2	1\\
9.5	1.5	1\\
10	1	1\\
10.5	0.5	1\\
11	0	1\\
0	10.5	1.5\\
0.5	10	1.5\\
1	9.5	1.5\\
1.5	9	1.5\\
2	8.5	1.5\\
2.5	8	1.5\\
3	7.5	1.5\\
3.5	7	1.5\\
4	6.5	1.5\\
4.5	6	1.5\\
5	5.5	1.5\\
5.5	5	1.5\\
6	4.5	1.5\\
6.5	4	1.5\\
7	3.5	1.5\\
7.5	3	1.5\\
8	2.5	1.5\\
8.5	2	1.5\\
9	1.5	1.5\\
9.5	1	1.5\\
10	0.5	1.5\\
10.5	0	1.5\\
0	10	2\\
0.5	9.5	2\\
1	9	2\\
1.5	8.5	2\\
2	8	2\\
2.5	7.5	2\\
3	7	2\\
3.5	6.5	2\\
4	6	2\\
4.5	5.5	2\\
5	5	2\\
5.5	4.5	2\\
6	4	2\\
6.5	3.5	2\\
7	3	2\\
7.5	2.5	2\\
8	2	2\\
8.5	1.5	2\\
9	1	2\\
9.5	0.5	2\\
10	0	2\\
0	9.5	2.5\\
0.5	9	2.5\\
1	8.5	2.5\\
1.5	8	2.5\\
2	7.5	2.5\\
2.5	7	2.5\\
3	6.5	2.5\\
3.5	6	2.5\\
4	5.5	2.5\\
4.5	5	2.5\\
5	4.5	2.5\\
5.5	4	2.5\\
6	3.5	2.5\\
6.5	3	2.5\\
7	2.5	2.5\\
7.5	2	2.5\\
8	1.5	2.5\\
8.5	1	2.5\\
9	0.5	2.5\\
9.5	0	2.5\\
0	9	3\\
0.5	8.5	3\\
1	8	3\\
1.5	7.5	3\\
2	7	3\\
2.5	6.5	3\\
3	6	3\\
3.5	5.5	3\\
4	5	3\\
4.5	4.5	3\\
5	4	3\\
5.5	3.5	3\\
6	3	3\\
6.5	2.5	3\\
7	2	3\\
7.5	1.5	3\\
8	1	3\\
8.5	0.5	3\\
9	0	3\\
0	8.5	3.5\\
0.5	8	3.5\\
1	7.5	3.5\\
1.5	7	3.5\\
2	6.5	3.5\\
2.5	6	3.5\\
3	5.5	3.5\\
3.5	5	3.5\\
4	4.5	3.5\\
4.5	4	3.5\\
5	3.5	3.5\\
5.5	3	3.5\\
6	2.5	3.5\\
6.5	2	3.5\\
7	1.5	3.5\\
7.5	1	3.5\\
8	0.5	3.5\\
8.5	0	3.5\\
0	8	4\\
0.5	7.5	4\\
1	7	4\\
1.5	6.5	4\\
2	6	4\\
2.5	5.5	4\\
3	5	4\\
3.5	4.5	4\\
4	4	4\\
4.5	3.5	4\\
5	3	4\\
5.5	2.5	4\\
6	2	4\\
6.5	1.5	4\\
7	1	4\\
7.5	0.5	4\\
8	0	4\\
0	7.5	4.5\\
0.5	7	4.5\\
1	6.5	4.5\\
1.5	6	4.5\\
2	5.5	4.5\\
2.5	5	4.5\\
3	4.5	4.5\\
3.5	4	4.5\\
4	3.5	4.5\\
4.5	3	4.5\\
5	2.5	4.5\\
5.5	2	4.5\\
6	1.5	4.5\\
6.5	1	4.5\\
7	0.5	4.5\\
7.5	0	4.5\\
0	7	5\\
0.5	6.5	5\\
1	6	5\\
1.5	5.5	5\\
2	5	5\\
2.5	4.5	5\\
3	4	5\\
3.5	3.5	5\\
4	3	5\\
4.5	2.5	5\\
5	2	5\\
5.5	1.5	5\\
6	1	5\\
6.5	0.5	5\\
7	0	5\\
0	6.5	5.5\\
0.5	6	5.5\\
1	5.5	5.5\\
1.5	5	5.5\\
2	4.5	5.5\\
2.5	4	5.5\\
3	3.5	5.5\\
3.5	3	5.5\\
4	2.5	5.5\\
4.5	2	5.5\\
5	1.5	5.5\\
5.5	1	5.5\\
6	0.5	5.5\\
6.5	0	5.5\\
0	6	6\\
0.5	5.5	6\\
1	5	6\\
1.5	4.5	6\\
2	4	6\\
2.5	3.5	6\\
3	3	6\\
3.5	2.5	6\\
4	2	6\\
4.5	1.5	6\\
5	1	6\\
5.5	0.5	6\\
6	0	6\\
0	5.5	6.5\\
0.5	5	6.5\\
1	4.5	6.5\\
1.5	4	6.5\\
2	3.5	6.5\\
2.5	3	6.5\\
3	2.5	6.5\\
3.5	2	6.5\\
4	1.5	6.5\\
4.5	1	6.5\\
5	0.5	6.5\\
5.5	0	6.5\\
0	5	7\\
0.5	4.5	7\\
1	4	7\\
1.5	3.5	7\\
2	3	7\\
2.5	2.5	7\\
3	2	7\\
3.5	1.5	7\\
4	1	7\\
4.5	0.5	7\\
5	0	7\\
0	4.5	7.5\\
0.5	4	7.5\\
1	3.5	7.5\\
1.5	3	7.5\\
2	2.5	7.5\\
2.5	2	7.5\\
3	1.5	7.5\\
3.5	1	7.5\\
4	0.5	7.5\\
4.5	0	7.5\\
0	4	8\\
0.5	3.5	8\\
1	3	8\\
1.5	2.5	8\\
2	2	8\\
2.5	1.5	8\\
3	1	8\\
3.5	0.5	8\\
4	0	8\\
0	3.5	8.5\\
0.5	3	8.5\\
1	2.5	8.5\\
1.5	2	8.5\\
2	1.5	8.5\\
2.5	1	8.5\\
3	0.5	8.5\\
3.5	0	8.5\\
0	3	9\\
0.5	2.5	9\\
1	2	9\\
1.5	1.5	9\\
2	1	9\\
2.5	0.5	9\\
3	0	9\\
0	2.5	9.5\\
0.5	2	9.5\\
1	1.5	9.5\\
1.5	1	9.5\\
2	0.5	9.5\\
2.5	0	9.5\\
0	2	10\\
0.5	1.5	10\\
1	1	10\\
1.5	0.5	10\\
2	0	10\\
0	1.5	10.5\\
0.5	1	10.5\\
1	0.5	10.5\\
1.5	0	10.5\\
0	1	11\\
0.5	0.5	11\\
1	0	11\\
0	0.5	11.5\\
0.5	0	11.5\\
0	0	12\\
};
 \addplot3 [color=mycolor1,mark size=1pt,only marks,mark=triangle*,mark options={solid,scale=1.5}]
 table[row sep=crcr] {%
0	0	0\\
};
 \end{axis}
\end{tikzpicture}%
\\
\caption{\emph{Equidistant-Cartesian-grid-based node layouts for pricing options with a different number of underlying assets. The close-field boundary conditions are enforced in the blue triangle node, and the far-field boundary conditions are enforced in the red square nodes.}}
\label{fig:gridreg}
\end{figure}
\noindent That scheme in the case of arithmetic basket options can be expanded to higher dimensions by generating one-dimensional layouts along the axes of the domain, and connecting them (via equidistant node scattering) diagonally across the domain, as seen in \textbf{Figure \ref{fig:gridadap}}. In \textbf{Papers \ref{paper1}}, \textbf{\ref{paper2}}, \textbf{\ref{paper3}}, we show the advantages of this payoff-function-adapted node layout over the equidistant-Cartesian-grid-based node layouts.   

\begin{figure}[H]
\centering
\rmfamily
\definecolor{mycolor1}{rgb}{0.00000,0.44700,0.74100}%
\definecolor{mycolor2}{rgb}{0.85000,0.32500,0.09800}%
\definecolor{mycolor3}{rgb}{0.92900,0.69400,0.12500}%
%
\begin{tikzpicture}[trim axis left, trim axis right,baseline]

\begin{axis}[%
hide y axis,
width=0.55\textwidth,
y=0.001cm,
xmin=0,
xmax=4,
xtick={0,1,4},
xticklabels={$0$,$K$,$4K$},
axis x line*=bottom,
xlabel={$s_1$},
title={1D}
]
\addplot [color=black,mark size=0.5pt,only marks,mark=*,mark options={solid},forget plot]
  table[row sep=crcr]{%
  0.0902259783077568	0\\
0.173530760370657	0\\
0.250548095593546	0\\
0.321863901004856	0\\
0.388020718678599	0\\
0.449521843177093	0\\
0.506835150414338	0\\
0.560396657068511	0\\
0.610613837622161	0\\
0.657868724264807	0\\
0.702520813240772	0\\
0.744909799752577	0\\
0.785358162225933	0\\
0.824173615596403	0\\
0.861651452281338	0\\
0.898076788646281	0\\
0.933726734056037	0\\
0.968872499011691	0\\
1.00378145841144	0\\
1.0387191856317	0\\
1.07395147290293	0\\
1.10974635335043	0\\
1.14637614008299	0\\
1.1841194978419	0\\
1.22326356297064	0\\
1.26410612783317	0\\
1.30695790629887	0\\
1.35214489752923	0\\
1.40001086604879	0\\
1.45091995696797	0\\
1.50525946625312	0\\
1.56344278711909	0\\
1.62591255495917	0\\
1.69314401473779	0\\
1.76564863646384	0\\
1.84397800624955	0\\
1.92872802255661	0\\
2.02054342955294	0\\
2.12012272206801	0\\
2.2282234594616	0\\
2.34566802883189	0\\
2.47334990140706	0\\
2.61224042971633	0\\
2.76339623725088	0\\
2.92796725683199	0\\
3.10720547883922	0\\
3.30247447585162	0\\
3.51525977616151	0\\
3.7471801650786	0\\
% 0	0\\
% 0.0770340550727521	0\\
% 0.150565167875602	0\\
% 0.22087241180731	0\\
% 0.288222624683509	0\\
% 0.35287142146905	0\\
% 0.415064164416141	0\\
% 0.475036894290259	0\\
% 0.533017226218129	0\\
% 0.589225213557786	0\\
% 0.643874183069379	0\\
% 0.697171544556443	0\\
% 0.749319578050495	0\\
% 0.800516201526539	0\\
% 0.850955722063224	0\\
% 0.90082957329852	0\\
% 0.950327041979806	0\\
% 0.999635986365828	0\\
% 1.04894354920711	0\\
% 1.09843686801079	0\\
% 1.14830378528553	0\\
% 1.19873356146218	0\\
% 1.24991759319585	0\\
% 1.3020501397756	0\\
% 1.35532906039881	0\\
% 1.40995656510817	0\\
% 1.46613998224163	0\\
% 1.5240925453078	0\\
% 1.58403420227325	0\\
% 1.64619245033336	0\\
% 1.71080319933483	0\\
% 1.7781116671267	0\\
% 1.84837331023826	0\\
% 1.92185479341589	0\\
% 1.99883500169851	0\\
% 2.0796060988729	0\\
% 2.16447463632597	0\\
% 2.25376271650239	0\\
% 2.34780921538343	0\\
% 2.44697106862643	0\\
% 2.55162462624648	0\\
% 2.66216708098149	0\\
% 2.77901797576201	0\\
% 2.9026207960068	0\\
% 3.03344465278772	0\\
% 3.17198606325186	0\\
% 3.31877083505822	0\\
% 3.47435606198103	0\\
% 3.63933223825353	0\\
% 3.8143254996769	0\\
% 4	0\\
};
\addplot [color=mycolor2,mark size=1pt,only marks,mark=square*,mark options={solid},forget plot]
  table[row sep=crcr]{%
4	0\\
};
\addplot [color=mycolor1,mark size=1pt,only marks,mark=triangle*,mark options={solid,scale=1.5},forget plot]
  table[row sep=crcr]{%
0	0\\
};
\end{axis}
\end{tikzpicture}%
\\
\vspace{11pt}
% This file was created by matlab2tikz.
%
%The latest updates can be retrieved from
%  http://www.mathworks.com/matlabcentral/fileexchange/22022-matlab2tikz-matlab2tikz
%where you can also make suggestions and rate matlab2tikz.
\rmfamily
\definecolor{mycolor1}{rgb}{0.00000,0.44700,0.74100}%
\definecolor{mycolor2}{rgb}{0.85000,0.32500,0.09800}%
\definecolor{mycolor3}{rgb}{0.92900,0.69400,0.12500}%
\definecolor{mycolor4}{rgb}{0.49400,0.18400,0.55600}%
\definecolor{mycolor5}{rgb}{0.46600,0.67400,0.18800}%
\definecolor{mycolor6}{rgb}{0.30100,0.74500,0.93300}%
%
\begin{tikzpicture}[trim axis left, trim axis right, baseline]

  \begin{axis}[
  grid=major,
  %%tick label style = {font=\sansmath\sffamily},
  axis x line*=bottom,
  axis y line*=left,
  width=0.55\textwidth,
  height=0.55\textwidth,
  xmin=0,
  xmax=8,
  ymin=0,
  ymax=8,
  xlabel={$s_1$},
  ylabel={$s_2$},
  xtick={0,1,2,8},
  xticklabels={$0$,$K$,$2K$,$8K$},
  ytick={0,1,2,8},
  yticklabels={$0$,$K$,$2K$,$8K$},
  title={2D}
  ]
  \addplot [color=black,mark size=0.5pt,only marks,mark=*,mark options={solid},forget plot]
    table[row sep=crcr]{%
    0	0\\
    0	0.180451956615514\\
    0.180451956615514	0\\
    0	0.347061520741313\\
    0.173530760370657	0.173530760370657\\
    0.347061520741313	0\\
    0	0.501096191187092\\
    0.167032063729031	0.334064127458061\\
    0.334064127458061	0.167032063729031\\
    0.501096191187092	0\\
    0	0.643727802009712\\
    0.160931950502428	0.482795851507284\\
    0.321863901004856	0.321863901004856\\
    0.482795851507284	0.160931950502428\\
    0.643727802009712	0\\
    0	0.776041437357198\\
    0.15520828747144	0.620833149885759\\
    0.310416574942879	0.465624862414319\\
    0.465624862414319	0.310416574942879\\
    0.620833149885759	0.15520828747144\\
    0.776041437357198	0\\
    0	0.899043686354185\\
    0.149840614392364	0.749203071961821\\
    0.299681228784728	0.599362457569457\\
    0.449521843177093	0.449521843177093\\
    0.599362457569457	0.299681228784728\\
    0.749203071961821	0.149840614392364\\
    0.899043686354185	0\\
    0	1.01367030082868\\
    0.144810042975525	0.86886025785315\\
    0.28962008595105	0.724050214877625\\
    0.434430128926575	0.5792401719021\\
    0.5792401719021	0.434430128926575\\
    0.724050214877625	0.28962008595105\\
    0.86886025785315	0.144810042975525\\
    1.01367030082868	0\\
    0	1.12079331413702\\
    0.140099164267128	0.980694149869895\\
    0.280198328534256	0.840594985602767\\
    0.420297492801383	0.700495821335639\\
    0.560396657068511	0.560396657068511\\
    0.700495821335639	0.420297492801383\\
    0.840594985602767	0.280198328534256\\
    0.980694149869895	0.140099164267128\\
    1.12079331413702	0\\
    0	1.22122767524432\\
    0.135691963916036	1.08553571132829\\
    0.271383927832072	0.949843747412251\\
    0.407075891748107	0.814151783496215\\
    0.542767855664143	0.678459819580179\\
    0.678459819580179	0.542767855664143\\
    0.814151783496215	0.407075891748107\\
    0.949843747412251	0.271383927832072\\
    1.08553571132829	0.135691963916036\\
    1.22122767524432	0\\
    0	1.31573744852961\\
    0.131573744852961	1.18416370367665\\
    0.263147489705923	1.05258995882369\\
    0.394721234558884	0.921016213970729\\
    0.526294979411845	0.789442469117768\\
    0.657868724264807	0.657868724264807\\
    0.789442469117768	0.526294979411845\\
    0.921016213970729	0.394721234558884\\
    1.05258995882369	0.263147489705923\\
    1.18416370367665	0.131573744852961\\
    1.31573744852961	0\\
    0	1.40504162648154\\
    0.127731056952868	1.27731056952868\\
    0.255462113905735	1.14957951257581\\
    0.383193170858603	1.02184845562294\\
    0.510924227811471	0.894117398670074\\
    0.638655284764338	0.766386341717206\\
    0.766386341717206	0.638655284764338\\
    0.894117398670074	0.510924227811471\\
    1.02184845562294	0.383193170858603\\
    1.14957951257581	0.255462113905735\\
    1.27731056952868	0.127731056952868\\
    1.40504162648154	0\\
    0	1.48981959950515\\
    0.124151633292096	1.36566796621306\\
    0.248303266584192	1.24151633292096\\
    0.372454899876289	1.11736469962887\\
    0.496606533168385	0.99321306633677\\
    0.620758166460481	0.869061433044674\\
    0.744909799752577	0.744909799752577\\
    0.869061433044674	0.620758166460481\\
    0.99321306633677	0.496606533168385\\
    1.11736469962887	0.372454899876289\\
    1.24151633292096	0.248303266584192\\
    1.36566796621306	0.124151633292096\\
    1.48981959950515	0\\
    0	1.57071632445187\\
    0.120824332650144	1.44989199180172\\
    0.241648665300287	1.32906765915158\\
    0.362472997950431	1.20824332650144\\
    0.483297330600574	1.08741899385129\\
    0.604121663250718	0.966594661201149\\
    0.724945995900862	0.845770328551005\\
    0.845770328551005	0.724945995900862\\
    0.966594661201149	0.604121663250718\\
    1.08741899385129	0.483297330600574\\
    1.20824332650144	0.362472997950431\\
    1.32906765915158	0.241648665300287\\
    1.44989199180172	0.120824332650144\\
    1.57071632445187	0\\
    0	1.64834723119281\\
    0.117739087942343	1.53060814325046\\
    0.235478175884687	1.41286905530812\\
    0.35321726382703	1.29512996736578\\
    0.470956351769373	1.17739087942343\\
    0.588695439711716	1.05965179148109\\
    0.70643452765406	0.941912703538746\\
    0.824173615596403	0.824173615596403\\
    0.941912703538746	0.70643452765406\\
    1.05965179148109	0.588695439711716\\
    1.17739087942343	0.470956351769373\\
    1.29512996736578	0.35321726382703\\
    1.41286905530812	0.235478175884687\\
    1.53060814325046	0.117739087942343\\
    1.64834723119281	0\\
    0	1.72330290456268\\
    0.114886860304178	1.6084160442585\\
    0.229773720608357	1.49352918395432\\
    0.344660580912535	1.37864232365014\\
    0.459547441216714	1.26375546334596\\
    0.574434301520892	1.14886860304178\\
    0.689321161825071	1.03398174273761\\
    0.804208022129249	0.919094882433428\\
    0.919094882433428	0.804208022129249\\
    1.03398174273761	0.689321161825071\\
    1.14886860304178	0.574434301520892\\
    1.26375546334596	0.459547441216714\\
    1.37864232365014	0.344660580912535\\
    1.49352918395432	0.229773720608357\\
    1.6084160442585	0.114886860304178\\
    1.72330290456268	0\\
    0	1.79615357729256\\
    0.112259598580785	1.68389397871178\\
    0.22451919716157	1.57163438013099\\
    0.336778795742356	1.45937478155021\\
    0.449038394323141	1.34711518296942\\
    0.561297992903926	1.23485558438864\\
    0.673557591484711	1.12259598580785\\
    0.785817190065496	1.01033638722707\\
    0.898076788646281	0.898076788646281\\
    1.01033638722707	0.785817190065496\\
    1.12259598580785	0.673557591484711\\
    1.23485558438864	0.561297992903926\\
    1.34711518296942	0.449038394323141\\
    1.45937478155021	0.336778795742356\\
    1.57163438013099	0.22451919716157\\
    1.68389397871178	0.112259598580785\\
    1.79615357729256	0\\
    0	1.86745346811207\\
    0.109850204006593	1.75760326410548\\
    0.219700408013185	1.64775306009889\\
    0.329550612019778	1.5379028560923\\
    0.43940081602637	1.4280526520857\\
    0.549251020032963	1.31820244807911\\
    0.659101224039555	1.20835224407252\\
    0.768951428046148	1.09850204006593\\
    0.87880163205274	0.988651836059333\\
    0.988651836059333	0.87880163205274\\
    1.09850204006593	0.768951428046148\\
    1.20835224407252	0.659101224039555\\
    1.31820244807911	0.549251020032963\\
    1.4280526520857	0.43940081602637\\
    1.5379028560923	0.329550612019778\\
    1.64775306009889	0.219700408013185\\
    1.75760326410548	0.109850204006593\\
    1.86745346811207	0\\
    0	1.93774499802338\\
    0.107652499890188	1.83009249813319\\
    0.215304999780376	1.72243999824301\\
    0.322957499670564	1.61478749835282\\
    0.430609999560752	1.50713499846263\\
    0.538262499450939	1.39948249857244\\
    0.645914999341127	1.29182999868225\\
    0.753567499231315	1.18417749879207\\
    0.861219999121503	1.07652499890188\\
    0.968872499011691	0.968872499011691\\
    1.07652499890188	0.861219999121503\\
    1.18417749879207	0.753567499231315\\
    1.29182999868225	0.645914999341127\\
    1.39948249857244	0.53826249945094\\
    1.50713499846263	0.430609999560752\\
    1.61478749835282	0.322957499670564\\
    1.72243999824301	0.215304999780376\\
    1.83009249813319	0.107652499890188\\
    1.93774499802338	0\\
    0	2.00756291682289\\
    0.105661206148573	1.90190171067431\\
    0.211322412297146	1.79624050452574\\
    0.316983618445719	1.69057929837717\\
    0.422644824594292	1.5849180922286\\
    0.528306030742865	1.47925688608002\\
    0.633967236891438	1.37359567993145\\
    0.739628443040011	1.26793447378288\\
    0.845289649188584	1.1622732676343\\
    0.950950855337157	1.05661206148573\\
    1.05661206148573	0.950950855337157\\
    1.1622732676343	0.845289649188584\\
    1.26793447378288	0.739628443040011\\
    1.37359567993145	0.633967236891438\\
    1.47925688608002	0.528306030742865\\
    1.5849180922286	0.422644824594292\\
    1.69057929837717	0.316983618445719\\
    1.79624050452574	0.211322412297146\\
    1.90190171067431	0.105661206148573\\
    2.00756291682289	0\\
    0	2.0774383712634\\
    0.10387191856317	1.97356645270023\\
    0.20774383712634	1.86969453413706\\
    0.311615755689511	1.76582261557389\\
    0.415487674252681	1.66195069701072\\
    0.519359592815851	1.55807877844755\\
    0.623231511379021	1.45420685988438\\
    0.727103429942191	1.35033494132121\\
    0.830975348505362	1.24646302275804\\
    0.934847267068532	1.14259110419487\\
    1.0387191856317	1.0387191856317\\
    1.14259110419487	0.934847267068532\\
    1.24646302275804	0.830975348505362\\
    1.35033494132121	0.727103429942192\\
    1.45420685988438	0.623231511379021\\
    1.55807877844755	0.519359592815851\\
    1.66195069701072	0.415487674252681\\
    1.76582261557389	0.311615755689511\\
    1.86969453413706	0.20774383712634\\
    1.97356645270023	0.10387191856317\\
    2.0774383712634	0\\
    0	2.14790294580586\\
    0.102281092657422	2.04562185314843\\
    0.204562185314843	1.94334076049101\\
    0.306843277972265	1.84105966783359\\
    0.409124370629687	1.73877857517617\\
    0.511405463287108	1.63649748251875\\
    0.61368655594453	1.53421638986133\\
    0.715967648601952	1.4319352972039\\
    0.818248741259374	1.32965420454648\\
    0.920529833916795	1.22737311188906\\
    1.02281092657422	1.12509201923164\\
    1.12509201923164	1.02281092657422\\
    1.22737311188906	0.920529833916795\\
    1.32965420454648	0.818248741259374\\
    1.4319352972039	0.715967648601952\\
    1.53421638986133	0.61368655594453\\
    1.63649748251875	0.511405463287109\\
    1.73877857517617	0.409124370629687\\
    1.84105966783359	0.306843277972265\\
    1.94334076049101	0.204562185314843\\
    2.04562185314843	0.102281092657422\\
    2.14790294580586	0\\
    0	2.21949270670086\\
    0.100886032122766	2.11860667457809\\
    0.201772064245533	2.01772064245533\\
    0.302658096368299	1.91683461033256\\
    0.403544128491065	1.81594857820979\\
    0.504430160613832	1.71506254608703\\
    0.605316192736598	1.61417651396426\\
    0.706202224859364	1.5132904818415\\
    0.807088256982131	1.41240444971873\\
    0.907974289104897	1.31151841759596\\
    1.00886032122766	1.2106323854732\\
    1.10974635335043	1.10974635335043\\
    1.2106323854732	1.00886032122766\\
    1.31151841759596	0.907974289104897\\
    1.41240444971873	0.807088256982131\\
    1.51329048184149	0.706202224859364\\
    1.61417651396426	0.605316192736598\\
    1.71506254608703	0.504430160613832\\
    1.81594857820979	0.403544128491065\\
    1.91683461033256	0.302658096368299\\
    2.01772064245533	0.201772064245533\\
    2.11860667457809	0.100886032122766\\
    2.21949270670086	0\\
    0	2.29275228016598\\
    0.0996848817463471	2.19306739841964\\
    0.199369763492694	2.09338251667329\\
    0.299054645239041	1.99369763492694\\
    0.398739526985389	1.8940127531806\\
    0.498424408731736	1.79432787143425\\
    0.598109290478083	1.6946429896879\\
    0.69779417222443	1.59495810794155\\
    0.797479053970777	1.49527322619521\\
    0.897163935717124	1.39558834444886\\
    0.996848817463471	1.29590346270251\\
    1.09653369920982	1.19621858095617\\
    1.19621858095617	1.09653369920982\\
    1.29590346270251	0.996848817463472\\
    1.39558834444886	0.897163935717124\\
    1.49527322619521	0.797479053970777\\
    1.59495810794155	0.69779417222443\\
    1.6946429896879	0.598109290478083\\
    1.79432787143425	0.498424408731736\\
    1.8940127531806	0.398739526985389\\
    1.99369763492694	0.299054645239041\\
    2.09338251667329	0.199369763492694\\
    2.19306739841964	0.0996848817463469\\
    2.29275228016598	0\\
    0	2.3682389956838\\
    0.0986766248201584	2.26956237086364\\
    0.197353249640317	2.17088574604349\\
    0.296029874460475	2.07220912122333\\
    0.394706499280634	1.97353249640317\\
    0.493383124100792	1.87485587158301\\
    0.592059748920951	1.77617924676285\\
    0.690736373741109	1.67750262194269\\
    0.789412998561267	1.57882599712253\\
    0.888089623381426	1.48014937230238\\
    0.986766248201584	1.38147274748222\\
    1.08544287302174	1.28279612266206\\
    1.1841194978419	1.1841194978419\\
    1.28279612266206	1.08544287302174\\
    1.38147274748222	0.986766248201584\\
    1.48014937230238	0.888089623381426\\
    1.57882599712253	0.789412998561267\\
    1.67750262194269	0.690736373741109\\
    1.77617924676285	0.59205974892095\\
    1.87485587158301	0.493383124100792\\
    1.97353249640317	0.394706499280634\\
    2.07220912122333	0.296029874460475\\
    2.17088574604349	0.197353249640317\\
    2.26956237086364	0.0986766248201585\\
    2.3682389956838	0\\
    0	2.44652712594129\\
    0.0978610850376515	2.34866604090364\\
    0.195722170075303	2.25080495586599\\
    0.293583255112955	2.15294387082833\\
    0.391444340150606	2.05508278579068\\
    0.489305425188258	1.95722170075303\\
    0.587166510225909	1.85936061571538\\
    0.685027595263561	1.76149953067773\\
    0.782888680301212	1.66363844564008\\
    0.880749765338864	1.56577736060242\\
    0.978610850376516	1.46791627556477\\
    1.07647193541417	1.37005519052712\\
    1.17433302045182	1.27219410548947\\
    1.27219410548947	1.17433302045182\\
    1.37005519052712	1.07647193541417\\
    1.46791627556477	0.978610850376515\\
    1.56577736060242	0.880749765338864\\
    1.66363844564008	0.782888680301213\\
    1.76149953067773	0.685027595263561\\
    1.85936061571538	0.587166510225909\\
    1.95722170075303	0.489305425188258\\
    2.05508278579068	0.391444340150606\\
    2.15294387082833	0.293583255112955\\
    2.25080495586599	0.195722170075303\\
    2.34866604090364	0.0978610850376516\\
    2.44652712594129	0\\
    0	2.52821225566634\\
    0.0972389329102438	2.4309733227561\\
    0.194477865820488	2.33373438984585\\
    0.291716798730731	2.23649545693561\\
    0.388955731640975	2.13925652402536\\
    0.486194664551219	2.04201759111512\\
    0.583433597461463	1.94477865820488\\
    0.680672530371707	1.84753972529463\\
    0.77791146328195	1.75030079238439\\
    0.875150396192194	1.65306185947414\\
    0.972389329102438	1.5558229265639\\
    1.06962826201268	1.45858399365366\\
    1.16686719492293	1.36134506074341\\
    1.26410612783317	1.26410612783317\\
    1.36134506074341	1.16686719492293\\
    1.45858399365366	1.06962826201268\\
    1.5558229265639	0.972389329102438\\
    1.65306185947414	0.875150396192194\\
    1.75030079238439	0.77791146328195\\
    1.84753972529463	0.680672530371707\\
    1.94477865820488	0.583433597461463\\
    2.04201759111512	0.486194664551219\\
    2.13925652402536	0.388955731640975\\
    2.23649545693561	0.291716798730731\\
    2.33373438984585	0.194477865820488\\
    2.4309733227561	0.0972389329102441\\
    2.52821225566634	0\\
    0	2.61391581259775\\
    0.0968116967628795	2.51710411583487\\
    0.193623393525759	2.42029241907199\\
    0.290435090288639	2.32348072230911\\
    0.387246787051518	2.22666902554623\\
    0.484058483814397	2.12985732878335\\
    0.580870180577277	2.03304563202047\\
    0.677681877340157	1.93623393525759\\
    0.774493574103036	1.83942223849471\\
    0.871305270865915	1.74261054173183\\
    0.968116967628795	1.64579884496895\\
    1.06492866439167	1.54898714820607\\
    1.16174036115455	1.45217545144319\\
    1.25855205791743	1.35536375468031\\
    1.35536375468031	1.25855205791743\\
    1.45217545144319	1.16174036115455\\
    1.54898714820607	1.06492866439167\\
    1.64579884496895	0.968116967628795\\
    1.74261054173183	0.871305270865915\\
    1.83942223849471	0.774493574103036\\
    1.93623393525759	0.677681877340157\\
    2.03304563202047	0.580870180577277\\
    2.12985732878335	0.484058483814398\\
    2.22666902554623	0.387246787051518\\
    2.32348072230911	0.290435090288638\\
    2.42029241907199	0.193623393525759\\
    2.51710411583487	0.0968116967628796\\
    2.61391581259775	0\\
    0	2.70428979505846\\
    0.096581778394945	2.60770801666351\\
    0.19316355678989	2.51112623826857\\
    0.289745335184835	2.41454445987362\\
    0.38632711357978	2.31796268147868\\
    0.482908891974725	2.22138090308373\\
    0.57949067036967	2.12479912468879\\
    0.676072448764615	2.02821734629384\\
    0.77265422715956	1.9316355678989\\
    0.869236005554505	1.83505378950395\\
    0.96581778394945	1.73847201110901\\
    1.06239956234439	1.64189023271406\\
    1.15898134073934	1.54530845431912\\
    1.25556311913428	1.44872667592417\\
    1.35214489752923	1.35214489752923\\
    1.44872667592417	1.25556311913428\\
    1.54530845431912	1.15898134073934\\
    1.64189023271406	1.06239956234439\\
    1.73847201110901	0.96581778394945\\
    1.83505378950395	0.869236005554505\\
    1.9316355678989	0.77265422715956\\
    2.02821734629384	0.676072448764615\\
    2.12479912468879	0.57949067036967\\
    2.22138090308373	0.482908891974725\\
    2.31796268147868	0.38632711357978\\
    2.41454445987362	0.289745335184835\\
    2.51112623826857	0.19316355678989\\
    2.60770801666351	0.096581778394945\\
    2.70428979505846	0\\
    0	2.80002173209759\\
    0.0965524735206065	2.70346925857698\\
    0.193104947041213	2.60691678505637\\
    0.289657420561819	2.51036431153577\\
    0.386209894082426	2.41381183801516\\
    0.482762367603032	2.31725936449456\\
    0.579314841123639	2.22070689097395\\
    0.675867314644245	2.12415441745334\\
    0.772419788164852	2.02760194393274\\
    0.868972261685458	1.93104947041213\\
    0.965524735206065	1.83449699689152\\
    1.06207720872667	1.73794452337092\\
    1.15862968224728	1.64139204985031\\
    1.25518215576788	1.5448395763297\\
    1.35173462928849	1.4482871028091\\
    1.4482871028091	1.35173462928849\\
    1.5448395763297	1.25518215576788\\
    1.64139204985031	1.15862968224728\\
    1.73794452337092	1.06207720872667\\
    1.83449699689152	0.965524735206065\\
    1.93104947041213	0.868972261685458\\
    2.02760194393274	0.772419788164852\\
    2.12415441745334	0.675867314644246\\
    2.22070689097395	0.579314841123639\\
    2.31725936449456	0.482762367603032\\
    2.41381183801516	0.386209894082425\\
    2.51036431153577	0.28965742056182\\
    2.60691678505637	0.193104947041213\\
    2.70346925857698	0.0965524735206063\\
    2.80002173209759	0\\
    0	2.90183991393594\\
    0.0967279971311982	2.80511191680475\\
    0.193455994262396	2.70838391967355\\
    0.290183991393595	2.61165592254235\\
    0.386911988524793	2.51492792541115\\
    0.483639985655991	2.41819992827995\\
    0.580367982787189	2.32147193114876\\
    0.677095979918387	2.22474393401756\\
    0.773823977049585	2.12801593688636\\
    0.870551974180783	2.03128793975516\\
    0.967279971311982	1.93455994262396\\
    1.06400796844318	1.83783194549277\\
    1.16073596557438	1.74110394836157\\
    1.25746396270558	1.64437595123037\\
    1.35419195983677	1.54764795409917\\
    1.45091995696797	1.45091995696797\\
    1.54764795409917	1.35419195983677\\
    1.64437595123037	1.25746396270558\\
    1.74110394836157	1.16073596557438\\
    1.83783194549277	1.06400796844318\\
    1.93455994262396	0.967279971311982\\
    2.03128793975516	0.870551974180783\\
    2.12801593688636	0.773823977049585\\
    2.22474393401756	0.677095979918387\\
    2.32147193114876	0.580367982787189\\
    2.41819992827995	0.483639985655991\\
    2.51492792541115	0.386911988524793\\
    2.61165592254235	0.290183991393595\\
    2.70838391967355	0.193455994262396\\
    2.80511191680475	0.0967279971311981\\
    2.90183991393594	0\\
    0	3.01051893250625\\
    0.0971135139518144	2.91340541855443\\
    0.194227027903629	2.81629190460262\\
    0.291340541855443	2.7191783906508\\
    0.388454055807258	2.62206487669899\\
    0.485567569759072	2.52495136274717\\
    0.582681083710886	2.42783784879536\\
    0.679794597662701	2.33072433484355\\
    0.776908111614515	2.23361082089173\\
    0.87402162556633	2.13649730693992\\
    0.971135139518144	2.0393837929881\\
    1.06824865346996	1.94227027903629\\
    1.16536216742177	1.84515676508447\\
    1.26247568137359	1.74804325113266\\
    1.3595891953254	1.65092973718085\\
    1.45670270927722	1.55381622322903\\
    1.55381622322903	1.45670270927722\\
    1.65092973718084	1.3595891953254\\
    1.74804325113266	1.26247568137359\\
    1.84515676508447	1.16536216742177\\
    1.94227027903629	1.06824865346996\\
    2.0393837929881	0.971135139518144\\
    2.13649730693992	0.87402162556633\\
    2.23361082089173	0.776908111614516\\
    2.33072433484355	0.679794597662701\\
    2.42783784879536	0.582681083710886\\
    2.52495136274717	0.485567569759072\\
    2.62206487669899	0.388454055807258\\
    2.7191783906508	0.291340541855444\\
    2.81629190460262	0.194227027903629\\
    2.91340541855443	0.0971135139518142\\
    3.01051893250625	0\\
    0	3.12688557423819\\
    0.0977151741949434	3.02917040004325\\
    0.195430348389887	2.9314552258483\\
    0.29314552258483	2.83374005165336\\
    0.390860696779774	2.73602487745842\\
    0.488575870974717	2.63830970326347\\
    0.58629104516966	2.54059452906853\\
    0.684006219364604	2.44287935487359\\
    0.781721393559547	2.34516418067864\\
    0.879436567754491	2.2474490064837\\
    0.977151741949434	2.14973383228876\\
    1.07486691614438	2.05201865809381\\
    1.17258209033932	1.95430348389887\\
    1.27029726453426	1.85658830970392\\
    1.36801243872921	1.75887313550898\\
    1.46572761292415	1.66115796131404\\
    1.56344278711909	1.56344278711909\\
    1.66115796131404	1.46572761292415\\
    1.75887313550898	1.36801243872921\\
    1.85658830970392	1.27029726453426\\
    1.95430348389887	1.17258209033932\\
    2.05201865809381	1.07486691614438\\
    2.14973383228876	0.977151741949434\\
    2.2474490064837	0.879436567754491\\
    2.34516418067864	0.781721393559547\\
    2.44287935487359	0.684006219364604\\
    2.54059452906853	0.586291045169661\\
    2.63830970326347	0.488575870974717\\
    2.73602487745842	0.390860696779773\\
    2.83374005165336	0.29314552258483\\
    2.9314552258483	0.195430348389887\\
    3.02917040004325	0.0977151741949434\\
    3.12688557423819	0\\
    0	3.25182510991834\\
    0.0985401548460102	3.15328495507233\\
    0.19708030969202	3.05474480022632\\
    0.295620464538031	2.95620464538031\\
    0.394160619384041	2.8576644905343\\
    0.492700774230051	2.75912433568829\\
    0.591240929076061	2.66058418084228\\
    0.689781083922072	2.56204402599627\\
    0.788321238768082	2.46350387115026\\
    0.886861393614092	2.36496371630425\\
    0.985401548460102	2.26642356145824\\
    1.08394170330611	2.16788340661222\\
    1.18248185815212	2.06934325176621\\
    1.28102201299813	1.9708030969202\\
    1.37956216784414	1.87226294207419\\
    1.47810232269015	1.77372278722818\\
    1.57664247753616	1.67518263238217\\
    1.67518263238217	1.57664247753616\\
    1.77372278722818	1.47810232269015\\
    1.87226294207419	1.37956216784414\\
    1.9708030969202	1.28102201299813\\
    2.06934325176621	1.18248185815212\\
    2.16788340661223	1.08394170330611\\
    2.26642356145824	0.985401548460102\\
    2.36496371630425	0.886861393614092\\
    2.46350387115026	0.788321238768082\\
    2.56204402599627	0.689781083922072\\
    2.66058418084228	0.591240929076061\\
    2.75912433568829	0.492700774230051\\
    2.8576644905343	0.394160619384041\\
    2.95620464538031	0.295620464538031\\
    3.05474480022632	0.197080309692021\\
    3.15328495507233	0.0985401548460101\\
    3.25182510991834	0\\
    0	3.38628802947558\\
    0.0995967067492818	3.2866913227263\\
    0.199193413498564	3.18709461597702\\
    0.298790120247845	3.08749790922774\\
    0.398386826997127	2.98790120247845\\
    0.497983533746409	2.88830449572917\\
    0.597580240495691	2.78870778897989\\
    0.697176947244972	2.68911108223061\\
    0.796773653994254	2.58951437548133\\
    0.896370360743536	2.48991766873204\\
    0.995967067492818	2.39032096198276\\
    1.0955637742421	2.29072425523348\\
    1.19516048099138	2.1911275484842\\
    1.29475718774066	2.09153084173492\\
    1.39435389448994	1.99193413498564\\
    1.49395060123923	1.89233742823635\\
    1.59354730798851	1.79274072148707\\
    1.69314401473779	1.69314401473779\\
    1.79274072148707	1.59354730798851\\
    1.89233742823635	1.49395060123923\\
    1.99193413498564	1.39435389448994\\
    2.09153084173492	1.29475718774066\\
    2.1911275484842	1.19516048099138\\
    2.29072425523348	1.0955637742421\\
    2.39032096198276	0.995967067492818\\
    2.48991766873204	0.896370360743536\\
    2.58951437548133	0.796773653994254\\
    2.68911108223061	0.697176947244972\\
    2.78870778897989	0.597580240495691\\
    2.88830449572917	0.497983533746409\\
    2.98790120247845	0.398386826997127\\
    3.08749790922774	0.298790120247845\\
    3.18709461597702	0.199193413498564\\
    3.2866913227263	0.0995967067492818\\
    3.38628802947558	0\\
    0	3.53129727292769\\
    0.100894207797934	3.43040306512975\\
    0.201788415595868	3.32950885733182\\
    0.302682623393802	3.22861464953388\\
    0.403576831191736	3.12772044173595\\
    0.504471038989669	3.02682623393802\\
    0.605365246787603	2.92593202614008\\
    0.706259454585537	2.82503781834215\\
    0.807153662383471	2.72414361054421\\
    0.908047870181405	2.62324940274628\\
    1.00894207797934	2.52235519494835\\
    1.10983628577727	2.42146098715041\\
    1.21073049357521	2.32056677935248\\
    1.31162470137314	2.21967257155455\\
    1.41251890917107	2.11877836375661\\
    1.51341311696901	2.01788415595868\\
    1.61430732476694	1.91698994816074\\
    1.71520153256488	1.81609574036281\\
    1.81609574036281	1.71520153256488\\
    1.91698994816074	1.61430732476694\\
    2.01788415595868	1.51341311696901\\
    2.11877836375661	1.41251890917107\\
    2.21967257155455	1.31162470137314\\
    2.32056677935248	1.21073049357521\\
    2.42146098715041	1.10983628577727\\
    2.52235519494835	1.00894207797934\\
    2.62324940274628	0.908047870181405\\
    2.72414361054422	0.807153662383471\\
    2.82503781834215	0.706259454585537\\
    2.92593202614008	0.605365246787604\\
    3.02682623393802	0.504471038989669\\
    3.12772044173595	0.403576831191736\\
    3.22861464953388	0.302682623393802\\
    3.32950885733182	0.201788415595868\\
    3.43040306512975	0.100894207797934\\
    3.53129727292769	0\\
    0	3.68795601249909\\
    0.102443222569419	3.58551278992967\\
    0.204886445138839	3.48306956736026\\
    0.307329667708258	3.38062634479084\\
    0.409772890277677	3.27818312222142\\
    0.512216112847096	3.175739899652\\
    0.614659335416516	3.07329667708258\\
    0.717102557985935	2.97085345451316\\
    0.819545780555354	2.86841023194374\\
    0.921989003124773	2.76596700937432\\
    1.02443222569419	2.6635237868049\\
    1.12687544826361	2.56108056423548\\
    1.22931867083303	2.45863734166606\\
    1.33176189340245	2.35619411909664\\
    1.43420511597187	2.25375089652722\\
    1.53664833854129	2.1513076739578\\
    1.63909156111071	2.04886445138839\\
    1.74153478368013	1.94642122881897\\
    1.84397800624955	1.84397800624955\\
    1.94642122881897	1.74153478368013\\
    2.04886445138839	1.63909156111071\\
    2.1513076739578	1.53664833854129\\
    2.25375089652722	1.43420511597187\\
    2.35619411909664	1.33176189340245\\
    2.45863734166606	1.22931867083303\\
    2.56108056423548	1.12687544826361\\
    2.6635237868049	1.02443222569419\\
    2.76596700937432	0.921989003124774\\
    2.86841023194374	0.819545780555354\\
    2.97085345451316	0.717102557985935\\
    3.07329667708258	0.614659335416516\\
    3.175739899652	0.512216112847097\\
    3.27818312222142	0.409772890277677\\
    3.38062634479084	0.307329667708258\\
    3.48306956736026	0.204886445138839\\
    3.58551278992968	0.102443222569419\\
    3.68795601249909	0\\
    0	3.85745604511321\\
    0.104255568786844	3.75320047632637\\
    0.208511137573687	3.64894490753952\\
    0.312766706360531	3.54468933875268\\
    0.417022275147374	3.44043376996584\\
    0.521277843934218	3.33617820117899\\
    0.625533412721061	3.23192263239215\\
    0.729788981507905	3.12766706360531\\
    0.834044550294748	3.02341149481846\\
    0.938300119081592	2.91915592603162\\
    1.04255568786844	2.81490035724477\\
    1.14681125665528	2.71064478845793\\
    1.25106682544212	2.60638921967109\\
    1.35532239422897	2.50213365088424\\
    1.45957796301581	2.3978780820974\\
    1.56383353180265	2.29362251331056\\
    1.6680891005895	2.18936694452371\\
    1.77234466937634	2.08511137573687\\
    1.87660023816318	1.98085580695003\\
    1.98085580695003	1.87660023816318\\
    2.08511137573687	1.77234466937634\\
    2.18936694452371	1.6680891005895\\
    2.29362251331056	1.56383353180265\\
    2.3978780820974	1.45957796301581\\
    2.50213365088424	1.35532239422897\\
    2.60638921967109	1.25106682544212\\
    2.71064478845793	1.14681125665528\\
    2.81490035724477	1.04255568786844\\
    2.91915592603162	0.938300119081592\\
    3.02341149481846	0.834044550294748\\
    3.12766706360531	0.729788981507904\\
    3.23192263239215	0.625533412721061\\
    3.33617820117899	0.521277843934218\\
    3.44043376996584	0.417022275147374\\
    3.54468933875268	0.31276670636053\\
    3.64894490753952	0.208511137573687\\
    3.75320047632637	0.104255568786844\\
    3.85745604511321	0\\
    0	4.04108685910588\\
    0.106344391029102	3.93474246807678\\
    0.212688782058204	3.82839807704768\\
    0.319033173087307	3.72205368601858\\
    0.425377564116409	3.61570929498947\\
    0.531721955145511	3.50936490396037\\
    0.638066346174613	3.40302051293127\\
    0.744410737203715	3.29667612190217\\
    0.850755128232818	3.19033173087307\\
    0.95709951926192	3.08398733984396\\
    1.06344391029102	2.97764294881486\\
    1.16978830132012	2.87129855778576\\
    1.27613269234923	2.76495416675666\\
    1.38247708337833	2.65860977572755\\
    1.48882147440743	2.55226538469845\\
    1.59516586543653	2.44592099366935\\
    1.70151025646564	2.33957660264025\\
    1.80785464749474	2.23323221161115\\
    1.91419903852384	2.12688782058204\\
    2.02054342955294	2.02054342955294\\
    2.12688782058204	1.91419903852384\\
    2.23323221161115	1.80785464749474\\
    2.33957660264025	1.70151025646563\\
    2.44592099366935	1.59516586543653\\
    2.55226538469845	1.48882147440743\\
    2.65860977572755	1.38247708337833\\
    2.76495416675666	1.27613269234923\\
    2.87129855778576	1.16978830132012\\
    2.97764294881486	1.06344391029102\\
    3.08398733984396	0.95709951926192\\
    3.19033173087307	0.850755128232818\\
    3.29667612190217	0.744410737203716\\
    3.40302051293127	0.638066346174613\\
    3.50936490396037	0.531721955145511\\
    3.61570929498947	0.425377564116408\\
    3.72205368601858	0.319033173087307\\
    3.82839807704768	0.212688782058204\\
    3.93474246807678	0.106344391029102\\
    4.04108685910588	0\\
    0	4.24024544413602\\
    0.108724242157334	4.13152120197869\\
    0.217448484314668	4.02279695982135\\
    0.326172726472002	3.91407271766402\\
    0.434896968629336	3.80534847550669\\
    0.543621210786669	3.69662423334935\\
    0.652345452944003	3.58789999119202\\
    0.761069695101337	3.47917574903468\\
    0.869793937258671	3.37045150687735\\
    0.978518179416005	3.26172726472002\\
    1.08724242157334	3.15300302256268\\
    1.19596666373067	3.04427878040535\\
    1.30469090588801	2.93555453824802\\
    1.41341514804534	2.82683029609068\\
    1.52213939020267	2.71810605393335\\
    1.63086363236001	2.60938181177601\\
    1.73958787451734	2.50065756961868\\
    1.84831211667468	2.39193332746135\\
    1.95703635883201	2.28320908530401\\
    2.06576060098934	2.17448484314668\\
    2.17448484314668	2.06576060098934\\
    2.28320908530401	1.95703635883201\\
    2.39193332746135	1.84831211667468\\
    2.50065756961868	1.73958787451734\\
    2.60938181177601	1.63086363236001\\
    2.71810605393335	1.52213939020267\\
    2.82683029609068	1.41341514804534\\
    2.93555453824802	1.30469090588801\\
    3.04427878040535	1.19596666373067\\
    3.15300302256268	1.08724242157334\\
    3.26172726472002	0.978518179416005\\
    3.37045150687735	0.869793937258671\\
    3.47917574903468	0.761069695101337\\
    3.58789999119202	0.652345452944004\\
    3.69662423334935	0.543621210786669\\
    3.80534847550669	0.434896968629335\\
    3.91407271766402	0.326172726472002\\
    4.02279695982135	0.217448484314668\\
    4.13152120197869	0.108724242157334\\
    4.24024544413602	0\\
    0	4.45644691892319\\
    0.11141117297308	4.34503574595011\\
    0.22282234594616	4.23362457297703\\
    0.334233518919239	4.12221340000395\\
    0.445644691892319	4.01080222703087\\
    0.557055864865399	3.89939105405779\\
    0.668467037838479	3.78797988108471\\
    0.779878210811559	3.67656870811163\\
    0.891289383784638	3.56515753513855\\
    1.00270055675772	3.45374636216547\\
    1.1141117297308	3.34233518919239\\
    1.22552290270388	3.23092401621931\\
    1.33693407567696	3.11951284324623\\
    1.44834524865004	3.00810167027315\\
    1.55975642162312	2.89669049730007\\
    1.6711675945962	2.78527932432699\\
    1.78257876756928	2.67386815135391\\
    1.89398994054236	2.56245697838084\\
    2.00540111351544	2.45104580540776\\
    2.11681228648852	2.33963463243468\\
    2.2282234594616	2.2282234594616\\
    2.33963463243468	2.11681228648852\\
    2.45104580540776	2.00540111351544\\
    2.56245697838084	1.89398994054236\\
    2.67386815135391	1.78257876756928\\
    2.78527932432699	1.6711675945962\\
    2.89669049730007	1.55975642162312\\
    3.00810167027315	1.44834524865004\\
    3.11951284324623	1.33693407567696\\
    3.23092401621931	1.22552290270388\\
    3.34233518919239	1.1141117297308\\
    3.45374636216547	1.00270055675772\\
    3.56515753513855	0.891289383784638\\
    3.67656870811163	0.779878210811559\\
    3.78797988108471	0.668467037838479\\
    3.89939105405779	0.557055864865398\\
    4.01080222703087	0.445644691892319\\
    4.12221340000395	0.334233518919239\\
    4.23362457297703	0.22282234594616\\
    4.34503574595011	0.11141117297308\\
    4.45644691892319	0\\
    0	4.69133605766379\\
    0.114422830674727	4.57691322698906\\
    0.228845661349453	4.46249039631433\\
    0.34326849202418	4.34806756563961\\
    0.457691322698906	4.23364473496488\\
    0.572114153373633	4.11922190429016\\
    0.686536984048359	4.00479907361543\\
    0.800959814723086	3.8903762429407\\
    0.915382645397812	3.77595341226598\\
    1.02980547607254	3.66153058159125\\
    1.14422830674727	3.54710775091652\\
    1.25865113742199	3.4326849202418\\
    1.37307396809672	3.31826208956707\\
    1.48749679877145	3.20383925889234\\
    1.60191962944617	3.08941642821762\\
    1.7163424601209	2.97499359754289\\
    1.83076529079562	2.86057076686816\\
    1.94518812147035	2.74614793619344\\
    2.05961095214508	2.63172510551871\\
    2.1740337828198	2.51730227484398\\
    2.28845661349453	2.40287944416926\\
    2.40287944416926	2.28845661349453\\
    2.51730227484398	2.1740337828198\\
    2.63172510551871	2.05961095214508\\
    2.74614793619344	1.94518812147035\\
    2.86057076686816	1.83076529079562\\
    2.97499359754289	1.7163424601209\\
    3.08941642821762	1.60191962944617\\
    3.20383925889234	1.48749679877145\\
    3.31826208956707	1.37307396809672\\
    3.4326849202418	1.25865113742199\\
    3.54710775091652	1.14422830674727\\
    3.66153058159125	1.02980547607254\\
    3.77595341226598	0.915382645397812\\
    3.8903762429407	0.800959814723086\\
    4.00479907361543	0.686536984048359\\
    4.11922190429016	0.572114153373633\\
    4.23364473496488	0.457691322698906\\
    4.34806756563961	0.34326849202418\\
    4.46249039631433	0.228845661349453\\
    4.57691322698906	0.114422830674727\\
    4.69133605766379	0\\
    0	4.94669980281412\\
    0.11777856673367	4.82892123608045\\
    0.235557133467339	4.71114266934678\\
    0.353335700201009	4.59336410261311\\
    0.471114266934678	4.47558553587944\\
    0.588892833668348	4.35780696914577\\
    0.706671400402017	4.2400284024121\\
    0.824449967135687	4.12224983567843\\
    0.942228533869356	4.00447126894476\\
    1.06000710060303	3.88669270221109\\
    1.1777856673367	3.76891413547742\\
    1.29556423407036	3.65113556874376\\
    1.41334280080403	3.53335700201009\\
    1.5311213675377	3.41557843527642\\
    1.64889993427137	3.29779986854275\\
    1.76667850100504	3.18002130180908\\
    1.88445706773871	3.06224273507541\\
    2.00223563447238	2.94446416834174\\
    2.12001420120605	2.82668560160807\\
    2.23779276793972	2.7089070348744\\
    2.35557133467339	2.59112846814073\\
    2.47334990140706	2.47334990140706\\
    2.59112846814073	2.35557133467339\\
    2.7089070348744	2.23779276793972\\
    2.82668560160807	2.12001420120605\\
    2.94446416834174	2.00223563447238\\
    3.06224273507541	1.88445706773871\\
    3.18002130180908	1.76667850100504\\
    3.29779986854275	1.64889993427137\\
    3.41557843527642	1.5311213675377\\
    3.53335700201009	1.41334280080403\\
    3.65113556874376	1.29556423407036\\
    3.76891413547742	1.1777856673367\\
    3.88669270221109	1.06000710060303\\
    4.00447126894476	0.942228533869356\\
    4.12224983567843	0.824449967135687\\
    4.2400284024121	0.706671400402016\\
    4.35780696914577	0.588892833668348\\
    4.47558553587944	0.471114266934678\\
    4.59336410261311	0.353335700201009\\
    4.71114266934678	0.235557133467339\\
    4.82892123608045	0.117778566733669\\
    4.94669980281412	0\\
    0	5.22448085943266\\
    0.121499554870527	5.10298130456213\\
    0.242999109741054	4.9814817496916\\
    0.364498664611581	4.85998219482108\\
    0.485998219482108	4.73848263995055\\
    0.607497774352635	4.61698308508002\\
    0.728997329223162	4.4954835302095\\
    0.850496884093689	4.37398397533897\\
    0.971996438964216	4.25248442046844\\
    1.09349599383474	4.13098486559792\\
    1.21499554870527	4.00948531072739\\
    1.3364951035758	3.88798575585686\\
    1.45799465844632	3.76648620098633\\
    1.57949421331685	3.64498664611581\\
    1.70099376818738	3.52348709124528\\
    1.8224933230579	3.40198753637475\\
    1.94399287792843	3.28048798150423\\
    2.06549243279896	3.1589884266337\\
    2.18699198766948	3.03748887176317\\
    2.30849154254001	2.91598931689265\\
    2.42999109741054	2.79448976202212\\
    2.55149065228107	2.67299020715159\\
    2.67299020715159	2.55149065228107\\
    2.79448976202212	2.42999109741054\\
    2.91598931689265	2.30849154254001\\
    3.03748887176317	2.18699198766948\\
    3.1589884266337	2.06549243279896\\
    3.28048798150423	1.94399287792843\\
    3.40198753637475	1.8224933230579\\
    3.52348709124528	1.70099376818738\\
    3.64498664611581	1.57949421331685\\
    3.76648620098634	1.45799465844632\\
    3.88798575585686	1.3364951035758\\
    4.00948531072739	1.21499554870527\\
    4.13098486559792	1.09349599383474\\
    4.25248442046844	0.971996438964216\\
    4.37398397533897	0.850496884093689\\
    4.4954835302095	0.728997329223161\\
    4.61698308508002	0.607497774352635\\
    4.73848263995055	0.485998219482108\\
    4.85998219482108	0.36449866461158\\
    4.9814817496916	0.242999109741054\\
    5.10298130456213	0.121499554870527\\
    5.22448085943266	0\\
    0	5.52679247450175\\
    0.12560891987504	5.40118355462671\\
    0.25121783975008	5.27557463475167\\
    0.376826759625119	5.14996571487663\\
    0.502435679500159	5.02435679500159\\
    0.628044599375199	4.89874787512655\\
    0.753653519250239	4.77313895525151\\
    0.879262439125279	4.64753003537647\\
    1.00487135900032	4.52192111550143\\
    1.13048027887536	4.39631219562639\\
    1.2560891987504	4.27070327575135\\
    1.38169811862544	4.14509435587631\\
    1.50730703850048	4.01948543600127\\
    1.63291595837552	3.89387651612623\\
    1.75852487825056	3.76826759625119\\
    1.8841337981256	3.64265867637615\\
    2.00974271800064	3.51704975650111\\
    2.13535163787568	3.39144083662608\\
    2.26096055775072	3.26583191675104\\
    2.38656947762576	3.140222996876\\
    2.5121783975008	3.01461407700096\\
    2.63778731737584	2.88900515712592\\
    2.76339623725088	2.76339623725088\\
    2.88900515712592	2.63778731737584\\
    3.01461407700096	2.5121783975008\\
    3.140222996876	2.38656947762576\\
    3.26583191675104	2.26096055775072\\
    3.39144083662607	2.13535163787568\\
    3.51704975650111	2.00974271800064\\
    3.64265867637615	1.8841337981256\\
    3.76826759625119	1.75852487825056\\
    3.89387651612623	1.63291595837552\\
    4.01948543600127	1.50730703850048\\
    4.14509435587631	1.38169811862544\\
    4.27070327575135	1.2560891987504\\
    4.39631219562639	1.13048027887536\\
    4.52192111550143	1.00487135900032\\
    4.64753003537647	0.879262439125279\\
    4.77313895525151	0.753653519250239\\
    4.89874787512655	0.628044599375199\\
    5.02435679500159	0.502435679500159\\
    5.14996571487663	0.37682675962512\\
    5.27557463475167	0.25121783975008\\
    5.40118355462671	0.12560891987504\\
    5.52679247450175	0\\
    0	5.85593451366397\\
    0.130131878081422	5.72580263558255\\
    0.260263756162843	5.59567075750113\\
    0.390395634244265	5.46553887941971\\
    0.520527512325686	5.33540700133829\\
    0.650659390407108	5.20527512325686\\
    0.78079126848853	5.07514324517544\\
    0.910923146569951	4.94501136709402\\
    1.04105502465137	4.8148794890126\\
    1.17118690273279	4.68474761093118\\
    1.30131878081422	4.55461573284976\\
    1.43145065889564	4.42448385476833\\
    1.56158253697706	4.29435197668691\\
    1.69171441505848	4.16422009860549\\
    1.8218462931399	4.03408822052407\\
    1.95197817122132	3.90395634244265\\
    2.08211004930275	3.77382446436123\\
    2.21224192738417	3.6436925862798\\
    2.34237380546559	3.51356070819838\\
    2.47250568354701	3.38342883011696\\
    2.60263756162843	3.25329695203554\\
    2.73276943970985	3.12316507395412\\
    2.86290131779127	2.9930331958727\\
    2.9930331958727	2.86290131779128\\
    3.12316507395412	2.73276943970985\\
    3.25329695203554	2.60263756162843\\
    3.38342883011696	2.47250568354701\\
    3.51356070819838	2.34237380546559\\
    3.6436925862798	2.21224192738417\\
    3.77382446436123	2.08211004930275\\
    3.90395634244265	1.95197817122132\\
    4.03408822052407	1.8218462931399\\
    4.16422009860549	1.69171441505848\\
    4.29435197668691	1.56158253697706\\
    4.42448385476833	1.43145065889564\\
    4.55461573284976	1.30131878081422\\
    4.68474761093118	1.17118690273279\\
    4.8148794890126	1.04105502465137\\
    4.94501136709402	0.910923146569951\\
    5.07514324517544	0.780791268488529\\
    5.20527512325686	0.650659390407109\\
    5.33540700133829	0.520527512325686\\
    5.46553887941971	0.390395634244265\\
    5.59567075750113	0.260263756162844\\
    5.72580263558255	0.130131878081422\\
    5.85593451366397	0\\
    0	6.21441095767845\\
    0.135095890384314	6.07931506729413\\
    0.270191780768628	5.94421917690982\\
    0.405287671152942	5.80912328652551\\
    0.540383561537256	5.67402739614119\\
    0.67547945192157	5.53893150575688\\
    0.810575342305885	5.40383561537256\\
    0.945671232690199	5.26873972498825\\
    1.08076712307451	5.13364383460394\\
    1.21586301345883	4.99854794421962\\
    1.35095890384314	4.86345205383531\\
    1.48605479422745	4.72835616345099\\
    1.62115068461177	4.59326027306668\\
    1.75624657499608	4.45816438268236\\
    1.8913424653804	4.32306849229805\\
    2.02643835576471	4.18797260191374\\
    2.16153424614903	4.05287671152942\\
    2.29663013653334	3.91778082114511\\
    2.43172602691765	3.78268493076079\\
    2.56682191730197	3.64758904037648\\
    2.70191780768628	3.51249314999217\\
    2.8370136980706	3.37739725960785\\
    2.97210958845491	3.24230136922354\\
    3.10720547883922	3.10720547883922\\
    3.24230136922354	2.97210958845491\\
    3.37739725960785	2.8370136980706\\
    3.51249314999217	2.70191780768628\\
    3.64758904037648	2.56682191730197\\
    3.78268493076079	2.43172602691765\\
    3.91778082114511	2.29663013653334\\
    4.05287671152942	2.16153424614903\\
    4.18797260191374	2.02643835576471\\
    4.32306849229805	1.8913424653804\\
    4.45816438268236	1.75624657499608\\
    4.59326027306668	1.62115068461177\\
    4.72835616345099	1.48605479422745\\
    4.86345205383531	1.35095890384314\\
    4.99854794421962	1.21586301345883\\
    5.13364383460394	1.08076712307451\\
    5.26873972498825	0.945671232690199\\
    5.40383561537256	0.810575342305885\\
    5.53893150575688	0.675479451921571\\
    5.67402739614119	0.540383561537256\\
    5.80912328652551	0.405287671152943\\
    5.94421917690982	0.270191780768628\\
    6.07931506729413	0.135095890384314\\
    6.21441095767845	0\\
    0	6.60494895170324\\
    0.140530828759643	6.46441812294359\\
    0.281061657519287	6.32388729418395\\
    0.42159248627893	6.18335646542431\\
    0.562123315038573	6.04282563666466\\
    0.702654143798217	5.90229480790502\\
    0.84318497255786	5.76176397914538\\
    0.983715801317503	5.62123315038573\\
    1.12424663007715	5.48070232162609\\
    1.26477745883679	5.34017149286645\\
    1.40530828759643	5.1996406641068\\
    1.54583911635608	5.05910983534716\\
    1.68636994511572	4.91857900658752\\
    1.82690077387536	4.77804817782787\\
    1.96743160263501	4.63751734906823\\
    2.10796243139465	4.49698652030859\\
    2.24849326015429	4.35645569154894\\
    2.38902408891394	4.2159248627893\\
    2.52955491767358	4.07539403402966\\
    2.67008574643322	3.93486320527001\\
    2.81061657519287	3.79433237651037\\
    2.95114740395251	3.65380154775073\\
    3.09167823271215	3.51327071899108\\
    3.2322090614718	3.37273989023144\\
    3.37273989023144	3.2322090614718\\
    3.51327071899108	3.09167823271215\\
    3.65380154775073	2.95114740395251\\
    3.79433237651037	2.81061657519287\\
    3.93486320527001	2.67008574643322\\
    4.07539403402966	2.52955491767358\\
    4.2159248627893	2.38902408891394\\
    4.35645569154894	2.24849326015429\\
    4.49698652030859	2.10796243139465\\
    4.63751734906823	1.96743160263501\\
    4.77804817782787	1.82690077387536\\
    4.91857900658752	1.68636994511572\\
    5.05910983534716	1.54583911635608\\
    5.1996406641068	1.40530828759643\\
    5.34017149286645	1.26477745883679\\
    5.48070232162609	1.12424663007715\\
    5.62123315038573	0.983715801317503\\
    5.76176397914538	0.843184972557861\\
    5.90229480790502	0.702654143798217\\
    6.04282563666466	0.562123315038574\\
    6.18335646542431	0.42159248627893\\
    6.32388729418395	0.281061657519287\\
    6.46441812294359	0.140530828759643\\
    6.60494895170324	0\\
    0	7.03051955232302\\
    0.146469157340063	6.88405039498295\\
    0.292938314680126	6.73758123764289\\
    0.439407472020189	6.59111208030283\\
    0.585876629360251	6.44464292296276\\
    0.732345786700314	6.2981737656227\\
    0.878814944040377	6.15170460828264\\
    1.02528410138044	6.00523545094258\\
    1.1717532587205	5.85876629360251\\
    1.31822241606057	5.71229713626245\\
    1.46469157340063	5.56582797892239\\
    1.61116073074069	5.41935882158232\\
    1.75762988808075	5.27288966424226\\
    1.90409904542082	5.1264205069022\\
    2.05056820276088	4.97995134956214\\
    2.19703736010094	4.83348219222207\\
    2.34350651744101	4.68701303488201\\
    2.48997567478107	4.54054387754195\\
    2.63644483212113	4.39407472020189\\
    2.78291398946119	4.24760556286182\\
    2.92938314680126	4.10113640552176\\
    3.07585230414132	3.9546672481817\\
    3.22232146148138	3.80819809084163\\
    3.36879061882144	3.66172893350157\\
    3.51525977616151	3.51525977616151\\
    3.66172893350157	3.36879061882144\\
    3.80819809084163	3.22232146148138\\
    3.9546672481817	3.07585230414132\\
    4.10113640552176	2.92938314680126\\
    4.24760556286182	2.78291398946119\\
    4.39407472020188	2.63644483212113\\
    4.54054387754195	2.48997567478107\\
    4.68701303488201	2.34350651744101\\
    4.83348219222207	2.19703736010094\\
    4.97995134956214	2.05056820276088\\
    5.1264205069022	1.90409904542082\\
    5.27288966424226	1.75762988808075\\
    5.41935882158232	1.61116073074069\\
    5.56582797892239	1.46469157340063\\
    5.71229713626245	1.31822241606057\\
    5.85876629360251	1.1717532587205\\
    6.00523545094258	1.02528410138044\\
    6.15170460828264	0.878814944040377\\
    6.2981737656227	0.732345786700313\\
    6.44464292296276	0.585876629360252\\
    6.59111208030283	0.439407472020188\\
    6.73758123764289	0.292938314680126\\
    6.88405039498295	0.146469157340062\\
    7.03051955232302	0\\
    0	7.49436033015719\\
    0.152946129186881	7.34141420097031\\
    0.305892258373763	7.18846807178343\\
    0.458838387560644	7.03552194259655\\
    0.611784516747526	6.88257581340967\\
    0.764730645934407	6.72962968422279\\
    0.917676775121289	6.5766835550359\\
    1.07062290430817	6.42373742584902\\
    1.22356903349505	6.27079129666214\\
    1.37651516268193	6.11784516747526\\
    1.52946129186881	5.96489903828838\\
    1.6824074210557	5.8119529091015\\
    1.83535355024258	5.65900677991461\\
    1.98829967942946	5.50606065072773\\
    2.14124580861634	5.35311452154085\\
    2.29419193780322	5.20016839235397\\
    2.4471380669901	5.04722226316709\\
    2.60008419617699	4.89427613398021\\
    2.75303032536387	4.74133000479333\\
    2.90597645455075	4.58838387560644\\
    3.05892258373763	4.43543774641956\\
    3.21186871292451	4.28249161723268\\
    3.36481484211139	4.1295454880458\\
    3.51776097129827	3.97659935885892\\
    3.67070710048516	3.82365322967204\\
    3.82365322967204	3.67070710048516\\
    3.97659935885892	3.51776097129827\\
    4.1295454880458	3.36481484211139\\
    4.28249161723268	3.21186871292451\\
    4.43543774641956	3.05892258373763\\
    4.58838387560644	2.90597645455075\\
    4.74133000479333	2.75303032536387\\
    4.89427613398021	2.60008419617699\\
    5.04722226316709	2.4471380669901\\
    5.20016839235397	2.29419193780322\\
    5.35311452154085	2.14124580861634\\
    5.50606065072773	1.98829967942946\\
    5.65900677991461	1.83535355024258\\
    5.8119529091015	1.6824074210557\\
    5.96489903828838	1.52946129186881\\
    6.11784516747526	1.37651516268193\\
    6.27079129666214	1.22356903349505\\
    6.42373742584902	1.07062290430817\\
    6.5766835550359	0.917676775121288\\
    6.72962968422279	0.764730645934407\\
    6.88257581340967	0.611784516747527\\
    7.03552194259655	0.458838387560644\\
    7.18846807178343	0.305892258373762\\
    7.34141420097031	0.152946129186882\\
    7.49436033015719	0\\
    0	8\\
    0.16	7.84\\
    0.32	7.68\\
    0.48	7.52\\
    0.64	7.36\\
    0.8	7.2\\
    0.96	7.04\\
    1.12	6.88\\
    1.28	6.72\\
    1.44	6.56\\
    1.6	6.4\\
    1.76	6.24\\
    1.92	6.08\\
    2.08	5.92\\
    2.24	5.76\\
    2.4	5.6\\
    2.56	5.44\\
    2.72	5.28\\
    2.88	5.12\\
    3.04	4.96\\
    3.2	4.8\\
    3.36	4.64\\
    3.52	4.48\\
    3.68	4.32\\
    3.84	4.16\\
    4	4\\
    4.16	3.84\\
    4.32	3.68\\
    4.48	3.52\\
    4.64	3.36\\
    4.8	3.2\\
    4.96	3.04\\
    5.12	2.88\\
    5.28	2.72\\
    5.44	2.56\\
    5.6	2.4\\
    5.76	2.24\\
    5.92	2.08\\
    6.08	1.92\\
    6.24	1.76\\
    6.4	1.6\\
    6.56	1.44\\
    6.72	1.28\\
    6.88	1.12\\
    7.04	0.96\\
    7.2	0.8\\
    7.36	0.64\\
    7.52	0.48\\
    7.68	0.32\\
    7.84	0.16\\
    8	0\\
    };


    \addplot [color=mycolor2,mark size=1pt,only marks,mark=square*,mark options={solid},forget plot]
      table[row sep=crcr]{%
    0	8\\
    0.16	7.84\\
    0.32	7.68\\
    0.48	7.52\\
    0.64	7.36\\
    0.8	7.2\\
    0.96	7.04\\
    1.12	6.88\\
    1.28	6.72\\
    1.44	6.56\\
    1.6	6.4\\
    1.76	6.24\\
    1.92	6.08\\
    2.08	5.92\\
    2.24	5.76\\
    2.4	5.6\\
    2.56	5.44\\
    2.72	5.28\\
    2.88	5.12\\
    3.04	4.96\\
    3.2	4.8\\
    3.36	4.64\\
    3.52	4.48\\
    3.68	4.32\\
    3.84	4.16\\
    4	4\\
    4.16	3.84\\
    4.32	3.68\\
    4.48	3.52\\
    4.64	3.36\\
    4.8	3.2\\
    4.96	3.04\\
    5.12	2.88\\
    5.28	2.72\\
    5.44	2.56\\
    5.6	2.4\\
    5.76	2.24\\
    5.92	2.08\\
    6.08	1.92\\
    6.24	1.76\\
    6.4	1.6\\
    6.56	1.44\\
    6.72	1.28\\
    6.88	1.12\\
    7.04	0.96\\
    7.2	0.8\\
    7.36	0.64\\
    7.52	0.48\\
    7.68	0.32\\
    7.84	0.16\\
    8	0\\
    };






    \addplot [color=mycolor1,mark size=1pt,only marks,mark=triangle*,mark options={solid,scale=1.5},forget plot]
      table[row sep=crcr]{%
    0	0\\
    };
    \end{axis}
    \end{tikzpicture}%
\\
\caption{\emph{Payoff-function-adapted node layouts for pricing options with a different number of underlying assets. The close-field boundary conditions are enforced in the blue triangle node, and the far-field boundary conditions are enforced in the red square nodes.}}
\label{fig:gridadap}
\end{figure}

\par
By using nonuniform node layouts in our numerical experiments, we have discovered that the RBF-FD stencils become very sensitive to non-smooth variations in their density, making the approximation numerically unstable. %
%Namely, if a discretized computational domain contains non-smooth changes in density of the node layout, it is very likely that the stencils constructed across those areas will have very large condition numbers, and therefore make the entire approximation numerically unstable.% 
For a successful implementation of RBF-FD methods, we need to be able to quickly generate node layouts with smoothly varying density. Unfortunately, not many practical results on node scattering are readily available, but research on this topic has become quite active lately. In \textbf{Paper \ref{paper2}}, we used a two-dimensional node layout suggested by Bengt Fornberg and Natasha Flyer in~\cite{fornberg2015fast}. An example of layouts with smoothly varying density for pricing two-dimensional arithmetic basket options under the Black--Scholes--Merton model and one-dimensional options under the Heston model are shown in \textbf{Figure \ref{fig:gridsmooth}}.
\begin{figure}[H]
\centering
% This file was created by matlab2tikz.
%
%The latest updates can be retrieved from
%  http://www.mathworks.com/matlabcentral/fileexchange/22022-matlab2tikz-matlab2tikz
%where you can also make suggestions and rate matlab2tikz.
%
\rmfamily
\definecolor{mycolor1}{rgb}{0.00000,0.44700,0.74100}%
\definecolor{mycolor2}{rgb}{0.85000,0.32500,0.09800}%
\definecolor{mycolor3}{rgb}{0.92900,0.69400,0.12500}%
\definecolor{mycolor4}{rgb}{0.49400,0.18400,0.55600}%
\definecolor{mycolor5}{rgb}{0.46600,0.67400,0.18800}%
\definecolor{mycolor6}{rgb}{0.30100,0.74500,0.93300}%
%
\begin{tikzpicture}[trim axis left, trim axis right, baseline]

  \begin{axis}[
    axis x line*=bottom,
    axis y line*=left,
  grid=major,
  %%tick label style = {font=\sansmath\sffamily},
  width=0.55\textwidth,
  height=0.55\textwidth,
  % at={(0\textwidth,0\textwidth)},
  % scale only axis,
  unbounded coords=jump,
  xmin=0,
  xmax=1,
  ymin=0,
  ymax=1,
  xlabel={$s_1$},
  ylabel={$s_2$},
  xtick={0,0.125,0.25,1},
  xticklabels={$0$,$K$,$2K$,$8K$},
  ytick={0,0.125,0.25,1},
  yticklabels={$0$,$K$,$2K$,$8K$},
  % yticklabels={,,}, %hides y ticks
  axis background/.style={fill=white},
  %title style={font=\bfseries},
  title={BS},
  legend pos=north east,
  legend style={legend cell align=left,align=left,draw=white!15!black}
  ]
  \addplot [color=black,mark size=0.5pt,only marks,mark=*,mark options={solid},forget plot]
    table[row sep=crcr]{%
  0	0\\
  0	0.0168560086809764\\
  0	0.0330471190368719\\
  0	0.048650812908091\\
  0	0.0637363890314228\\
  0	0.0783662560329692\\
  0	0.0925970189898784\\
  0	0.106480400604591\\
  0	0.120064029606358\\
  0	0.133392122566045\\
  0	0.146506080391629\\
  0	0.15944501700692\\
  0	0.172246234845103\\
  0	0.184945659622013\\
  0	0.197578245254389\\
  0	0.210178358658248\\
  0	0.222780153435572\\
  0	0.2354179410931\\
  0	0.24812656841679\\
  0	0.260941809952184\\
  0	0.273900785238784\\
  0	0.287042411563899\\
  0	0.300407904615414\\
  0	0.314041341636995\\
  0	0.327990304684728\\
  0	0.342306625577502\\
  0	0.357047259440772\\
  0	0.372275320808173\\
  0	0.388061325692705\\
  0	0.404484695758335\\
  0	0.421635597992042\\
  0	0.439617216955617\\
  0	0.458548589528036\\
  0	0.478568178135693\\
  0	0.499838424045652\\
  0	0.522551617013513\\
  0	0.546937556641267\\
  0	0.573273688695389\\
  0	0.60189871669143\\
  0	0.633231183412857\\
  0	0.667795307146035\\
  0	0.706257655796897\\
  0	0.749480443909005\\
  0	0.798601108459064\\
  0	0.85515491338436\\
  0	0.921270983218435\\
  0	1\\
  0.00900881848262941	0.240796299278469\\
  0.00925817492314862	0.191439487849569\\
  0.00925951971489372	0.166342281032336\\
  0.00930118804710417	0.203963801042126\\
  0.00948555897980787	0.264020903578867\\
  0.00949110461214825	0.179066059304175\\
  0.00959503464881397	0.216124039429967\\
  0.00961133503832103	0.228295913963984\\
  0.0100003270375712	0.286216113831334\\
  0.0100575529070423	0.140091135424387\\
  0.0100894148906082	0.25246296996576\\
  0.0102094971298005	0.378968868441565\\
  0.010438370572792	0.396162963832828\\
  0.0104860991477496	0.298095642920755\\
  0.0104918548885745	0.310226753961541\\
  0.0107234317104764	0.323077349435029\\
  0.0107374587518244	0.126749380052069\\
  0.0107453575339383	0.153374521835635\\
  0.0109204963074406	0.113709663933974\\
  0.0109994447045224	0.364340626308042\\
  0.0110185324708215	0.274996475073206\\
  0.0112721977576401	0.3492137786339\\
  0.0113003750097487	0.336077912957908\\
  0.0113601619978049	0.415035896197807\\
  0.0117576288766664	0.0997398618395776\\
  0.012429002352062	0.071933864629426\\
  0.0124499643941443	0.0865755037720903\\
  0.0126507540528436	0.0558908567586885\\
  0.0131402706050223	0.0233405510342291\\
  0.0135046523748407	0.437364311491031\\
  0.0145258145684575	0.0396919221241934\\
  0.0149398283283748	0.453138903775822\\
  0.0150065519565519	0.469841337678413\\
  0.0156894860895265	0.558518217699749\\
  0.0165579473285747	0.487594292899464\\
  0.0170857349375904	0\\
  0.01721850431458	0.529279057572216\\
  0.0176764167015756	0.583422828371698\\
  0.0178689806925814	0.612290638592001\\
  0.0178736873657911	0.50660865706964\\
  0.0199378212661716	0.236982610448697\\
  0.0202329177802562	0.16337507615854\\
  0.0205941171795222	0.189284778275272\\
  0.0206899525696901	0.176664473229922\\
  0.0207101247803055	0.201892700646121\\
  0.0211032994642955	0.262486788518237\\
  0.0211735014219027	0.21398939599712\\
  0.021436999103108	0.288604187170994\\
  0.0214626064775267	0.146552067791735\\
  0.0218080167703743	0.133399259202445\\
  0.0219669541945881	0.249521192901511\\
  0.0220358462248277	0.225662204154548\\
  0.0220818606032292	0.6518411694883\\
  0.0229254523538656	0.385471823293745\\
  0.0229437128171029	0.314459612184643\\
  0.0230389640482897	0.10652721767785\\
  0.0231100959209974	0.300852263121341\\
  0.0234174335724815	0.275014017358379\\
  0.0237093359628513	0.32874689154593\\
  0.0237224643312359	0.11948258037016\\
  0.0237267886268849	0.403297251396797\\
  0.0241456252504949	0.0130598367190341\\
  0.0242508428800952	0.357054799885627\\
  0.024875167782581	0.422492037568244\\
  0.0248913091840864	0.06434883159481\\
  0.0251807832516935	0.0808018911281175\\
  0.0259599946740111	0.370946623678575\\
  0.0268595659701676	0.342583078306376\\
  0.0271255261446224	0.0480114254175661\\
  0.027394073015444	0.693987158337202\\
  0.0277001336718635	0.0945597758963637\\
  0.0291435892213895	0.155703592571164\\
  0.029544322457484	0.0292894423335948\\
  0.0312675432071444	0.43853622323929\\
  0.0313049229529976	0.16939432902906\\
  0.0313884041946527	0.241212893109198\\
  0.0314938164997418	0.194898349260203\\
  0.0318356016789167	0.181990885228867\\
  0.032077141477227	0.258341828507764\\
  0.0327977556539102	0.286287260553465\\
  0.032842814192788	0.206526668548757\\
  0.0333477256656908	0.140166314062162\\
  0.0334166538766316	0\\
  0.0334292223416152	0.127434882494935\\
  0.0335135493683332	0.218336421819968\\
  0.0339053942547124	0.456769186387818\\
  0.0342257069265338	0.229967293638017\\
  0.0343212075294112	0.547915456016617\\
  0.0352036368994835	0.304290833773354\\
  0.0352181217207035	0.271545292997809\\
  0.0355453176399343	0.109041706496277\\
  0.0359937610418666	0.475984145627238\\
  0.0361754149033487	0.318096099256497\\
  0.0363699622731222	0.39195343196873\\
  0.0364795469531695	0.0722331954938723\\
  0.0378872035186777	0.332138648675441\\
  0.0384445650547317	0.756916235109692\\
  0.0388163786891931	0.0579203646156627\\
  0.039070440523976	0.0868756541406597\\
  0.0392385818212567	0.355526957291103\\
  0.0393711313062515	0.410429987945008\\
  0.0395342076303574	0.518982959867481\\
  0.0397109048909783	0.150193539218175\\
  0.0397284381224575	0.249469028089266\\
  0.0400103284777692	0.625945565868613\\
  0.0400606770125598	0.0401793441572961\\
  0.0403015464635155	0.0144068891681519\\
  0.0407964935316079	0.162167393144159\\
  0.0408768038599187	0.372644583339669\\
  0.0417772431532171	0.495758369731443\\
  0.0418403251129568	0.188991812059099\\
  0.0423701330825781	0.582423124673202\\
  0.0424512589778456	0.119584248341719\\
  0.0430151825171274	0.175390314778022\\
  0.0434426870301158	0.294604473975269\\
  0.0435125614963196	0.098985099224743\\
  0.0442193692198327	0.281884129034834\\
  0.0444102169638542	0.212857850667812\\
  0.0445161223806215	0.263745765411374\\
  0.0446658267008354	0.200371101841068\\
  0.0451267166268769	0.13306095734188\\
  0.0454514802841285	0.823038530431254\\
  0.0454575629439987	0.425943217636761\\
  0.0457345669017316	0.238710523600533\\
  0.0460736672785754	0.226837105828886\\
  0.046840527309314	0.343248810497531\\
  0.0482407115412353	0.0276239340943564\\
  0.0484138002417421	0.0767648117569623\\
  0.0489367862773852	0.310297256945277\\
  0.0490859188590455	0\\
  0.0494568516145155	0.144594608470347\\
  0.0503738216965568	0.386306079550053\\
  0.0505741707932443	0.0495897840491627\\
  0.0506089238661849	0.324060757047142\\
  0.0508870404633127	0.110929276910137\\
  0.0510798718879095	0.252637250004809\\
  0.0518017684142318	0.443114290988546\\
  0.0519271125518913	0.166833006132893\\
  0.0519805630256755	0.184668300877497\\
  0.0526890646276437	0.0637721651957046\\
  0.0528352457684092	0.155879921338698\\
  0.0539906089087773	0.946009391091223\\
  0.0542672193857206	0.364108918984737\\
  0.0543518977649217	0.276372113723906\\
  0.0543814927375355	0.0888166890221937\\
  0.0545506948223102	0.124480662415745\\
  0.0545767461108705	0.667254300992331\\
  0.0545953384239307	0.21980195437342\\
  0.054673770297687	0.462266361395313\\
  0.0549048131344366	0.402793101778429\\
  0.0554205465756411	0.554250117367433\\
  0.0554669776691062	0.206770554668581\\
  0.0556868894675315	0.299584669266417\\
  0.0561408067966668	0.0127819979439371\\
  0.0569733181616903	0.265498164701477\\
  0.0571565800508	0.195525669903275\\
  0.0574608793797813	0.100907148703474\\
  0.0577892621547543	0.238343471708214\\
  0.0578870846646451	0.138092678078078\\
  0.0582508142537236	0.175617903150816\\
  0.0595790971128458	0.0382533778678014\\
  0.0596794964716772	0.287611017468941\\
  0.0598014989454856	0.71638686577309\\
  0.0600040148699381	0.334606467251465\\
  0.061160704320123	0.0763832424002399\\
  0.061363696562348	0.351087791564539\\
  0.061558816116135	0.60432640643049\\
  0.0620324374777032	0.250614676911872\\
  0.0620950332130746	0.480769151047304\\
  0.062830512656318	0.151427643735048\\
  0.0630310756101696	0.228330536034424\\
  0.063371929150927	0.115290524467036\\
  0.0636723911330932	0.379696078473495\\
  0.0641030635788024	0.186356904212377\\
  0.0641303348307065	0.316742984358376\\
  0.0641533647689336	0.0524333487670637\\
  0.0641737863679923	0.427438125308434\\
  0.0641752965375964	0\\
  0.0642117220192571	0.525821854627834\\
  0.0642904951567178	0.0253751874331379\\
  0.064572179205886	0.165193850072387\\
  0.0646317701325846	0.214547845237889\\
  0.0657323358794379	0.129241384921416\\
  0.0669480347437335	0.0646363458880119\\
  0.0676249787163396	0.0928753959160075\\
  0.0678320582516432	0.202583635864199\\
  0.0682447435729926	0.276541836155834\\
  0.0682656633365608	0.302450156500599\\
  0.0686224051184443	0.263182623045115\\
  0.0688915811815915	0.502486160432629\\
  0.0701412960768442	0.105430140102124\\
  0.0709332436165571	0.240929230837622\\
  0.0718691647693244	0.0128413034917748\\
  0.0719389511728236	0.411447724769888\\
  0.072147996619215	0.0819360839599163\\
  0.072201534402217	0.176987144434546\\
  0.0727434679797758	0.14520776706642\\
  0.0728850529386619	0.394641170696069\\
  0.0729065233426286	0.368892452188438\\
  0.0732816428726063	0.450946142817134\\
  0.0732859159948493	0.289258819657826\\
  0.0734209478560754	0.636816693740172\\
  0.0734496512369767	0.0427317423275149\\
  0.0738326321327489	0.15732715732387\\
  0.0742198557298193	0.571031099696906\\
  0.0747780849967038	0.343497437946084\\
  0.0748547095658502	0.120774576785514\\
  0.0750278354431656	0.328888587913835\\
  0.0753792916503385	0.221951885253015\\
  0.0758907533505151	0.134400233858532\\
  0.0759793365068411	0.193333759845336\\
  0.0765530150505077	0.252582831992251\\
  0.0776372897049623	0.0580874114899988\\
  0.0779628614863664	0.0281767761625147\\
  0.0787571194795262	0\\
  0.0793215337808379	0.0720886916257169\\
  0.0794202403985623	0.312070720788713\\
  0.0801020089534679	0.167735603773271\\
  0.0803810118046327	0.0975800342946224\\
  0.0807154796603099	0.269855042738418\\
  0.0807976914445368	0.210636772616302\\
  0.0810762797817974	0.358036702289321\\
  0.0812509065126756	0.233172702335148\\
  0.08137386681193	0.110934271261397\\
  0.083862547930108	0.433797902859291\\
  0.0839413748628589	0.0860392324839676\\
  0.0841721760891093	0.298242340438271\\
  0.0842628646657882	0.184272961390968\\
  0.0846455642625513	0.47234790852239\\
  0.0854351964756874	0.127319934846707\\
  0.0854594066606839	0.545798389002846\\
  0.0854852119340943	0.0116943680862863\\
  0.085664292168153	0.152911061884698\\
  0.0859087881887524	0.0404202979357609\\
  0.0861311454229007	0.141282856651121\\
  0.0864958052708371	0.386020311121124\\
  0.0879391930014155	0.281514550645192\\
  0.0884482489040005	0.0536735123289254\\
  0.0885488483175274	0.200862347996707\\
  0.0889496541839414	0.2590533839316\\
  0.0890006301541852	0.244120461248913\\
  0.0890611079186548	0.221841543114437\\
  0.0895387934706494	0.337174059687373\\
  0.0906390434086824	0.0675297177095762\\
  0.0906919691154257	0.0238312938433635\\
  0.090883257289854	0.415885057339708\\
  0.0909716856795408	0.321604218472435\\
  0.0909800850876433	0.16286666035835\\
  0.0911874490008201	0.594280632743258\\
  0.0915236354630774	0.103894207067232\\
  0.0917371641689817	0.117566165260885\\
  0.0918643450860414	0.78881773675699\\
  0.0925812736666972	0.17530181425851\\
  0.0928958830072647	0\\
  0.0933843632559465	0.36956425372555\\
  0.0938695152830276	0.0797688688742368\\
  0.0939882948884113	0.497474552531948\\
  0.0953509481305956	0.0929560166136769\\
  0.0956141929719157	0.232829357143347\\
  0.0958018443036361	0.132938411224355\\
  0.0958326064322138	0.684685549848775\\
  0.0960267661829841	0.190961053776972\\
  0.0972578919293506	0.212114980803477\\
  0.097631699767247	0.0344722442721564\\
  0.097657838741123	0.269476674428492\\
  0.0981588493472886	0.144996094587137\\
  0.0985728435147429	0.453773223015266\\
  0.0992659060490749	0.305164711493383\\
  0.0994415715060304	0.350024818627389\\
  0.0994918456345783	0.0117568550908587\\
  0.099711094752881	0.0473162326349238\\
  0.0998552508091946	0.290436473536734\\
  0.100039312465009	0.398496955025141\\
  0.100565386979918	0.251958559901436\\
  0.100579625795685	0.0597549075373523\\
  0.101368150446184	0.155702577088725\\
  0.101511594115451	0.109984407083777\\
  0.101597739035573	0.522134965541448\\
  0.102189609217624	0.168383430266367\\
  0.102416078564209	0.122932989524272\\
  0.103272545398672	0.201033469891578\\
  0.104132049771686	0.072963404457238\\
  0.104293920361385	0.182835241986319\\
  0.105462606730916	0.0860277091672744\\
  0.106146781384246	0.024124072774968\\
  0.106274074554107	0.431320456894874\\
  0.106325910233713	0.332383654396482\\
  0.106586935007253	0.893413064992747\\
  0.106649558027104	0\\
  0.106707827260001	0.240020230378028\\
  0.107050877053842	0.0992464484984746\\
  0.107136317011163	0.222532032793341\\
  0.107541082505791	0.136310568752973\\
  0.108323214976999	0.27859485310604\\
  0.108726451808637	0.639851734761633\\
  0.109269519207138	0.316668364840283\\
  0.109769207789435	0.0387845392116543\\
  0.110227819435987	0.737867408768143\\
  0.110371049304233	0.378149986756557\\
  0.110845425510624	0.26260582646469\\
  0.111427773642952	0.0539809145665605\\
  0.111632531221739	0.210472038505763\\
  0.11229156911371	0.555741379760637\\
  0.1125	0.1125\\
  0.112587929021323	0.173822008774052\\
  0.112703985493261	0.0103757237870677\\
  0.112803322227867	0.16023906121114\\
  0.112843024845023	0.14769956764712\\
  0.113584667087935	0.192521016854422\\
  0.113599704324815	0.124637680462361\\
  0.114198055626379	0.47702296662096\\
  0.114445129899186	0.0663979343721794\\
  0.114475838394043	0.360423366469536\\
  0.114836092591099	0.411988262712745\\
  0.11509836866972	0.29961125545626\\
  0.115979844217477	0.249730173011439\\
  0.116712042142389	0.0785770687693734\\
  0.116820271460337	0.09082854278254\\
  0.116837483981077	0.231248998037338\\
  0.118850612913673	0.0211931051343662\\
  0.119151216992279	0.343522002497532\\
  0.119535835563115	0.135862173421955\\
  0.119673640085893	0.6039654392987\\
  0.119988973742735	0.102895919833647\\
  0.120070681077754	0\\
  0.120622415469332	0.0328959882816909\\
  0.120881285517369	0.0459101795717798\\
  0.120960113891363	0.18241369506509\\
  0.121770067364542	0.446864398276417\\
  0.122150949773603	0.393010179708878\\
  0.122504798738847	0.286325816352583\\
  0.12251313133916	0.271632781091595\\
  0.122588525058487	0.217226711221221\\
  0.122744477159826	0.201704400576489\\
  0.123567483159468	0.327176790949985\\
  0.123705976929831	0.168124889017167\\
  0.123924344751551	0.0581992909407849\\
  0.124315430455966	0.155179430363147\\
  0.124620076450171	0.500423477216613\\
  0.125	0.125\\
  0.125248381743961	0.113027387132281\\
  0.126713408926038	0.0105075181058861\\
  0.127093929820166	0.258014001143504\\
  0.127275790641706	0.143676805215223\\
  0.12762317454018	0.311315590706517\\
  0.127766144327169	0.0702000682356538\\
  0.128416286055731	0.0942405516293054\\
  0.128629413891131	0.0824340088260004\\
  0.129166312260354	0.241385792083405\\
  0.129967116255182	0.190974913200442\\
  0.130379904070721	0.227857333626798\\
  0.131623539268623	0.0235383711699568\\
  0.131638275814059	0.0382701640935073\\
  0.132967873063049	0.36817448896776\\
  0.133054372062639	0.0510048552786628\\
  0.133126855697897	0.177001297215335\\
  0.133207268726946	0\\
  0.133743807281895	0.296574854625302\\
  0.135156826238467	0.211127475507282\\
  0.13552057916521	0.163377950899527\\
  0.135555968914886	0.349834379197645\\
  0.135708011442592	0.423358785275181\\
  0.135981940966707	0.526166780424023\\
  0.136350196950622	0.126173352015636\\
  0.13640241016933	0.101941265448546\\
  0.136836591954612	0.113937453081662\\
  0.1375	0.1375\\
  0.137530197329269	0.151040839922097\\
  0.137536063537301	0.27902471289556\\
  0.137870174393314	0.0617578313491299\\
  0.139206298744748	0.00961075343370248\\
  0.13938973032392	0.0757846219685762\\
  0.14006770463943	0.0892974254335386\\
  0.140123449974364	0.249972849873552\\
  0.140219586626374	0.199044415984429\\
  0.140520655386045	0.332686839221557\\
  0.140725364675739	0.264413216089015\\
  0.141910238256207	0.0319143666424839\\
  0.142494172708678	0.465125564389008\\
  0.14323967222132	0.0451291023352705\\
  0.143279452206197	0.185201262524964\\
  0.143378966403813	0.673458346372923\\
  0.143989715495595	0.0201853456335485\\
  0.144266555361574	0.234131057753679\\
  0.144693070053371	0.220436068874998\\
  0.144904531573441	0.38347360240187\\
  0.145027930072019	0.316059545270919\\
  0.145442728789404	0.584123514460935\\
  0.145941187734019	0.171567531705586\\
  0.146103592999309	0\\
  0.146589800056863	0.550900648260509\\
  0.147210042526839	0.404224381527162\\
  0.147457669569824	0.119559931108814\\
  0.147544977624959	0.0565622983378068\\
  0.147562965431345	0.106193646605196\\
  0.14837832062531	0.158402040084862\\
  0.148585416069984	0.289741137002769\\
  0.148854108222658	0.144994987767903\\
  0.148949667789485	0.132048991204188\\
  0.149807813958093	0.0688165460121813\\
  0.149824616421354	0.207953696135806\\
  0.15025645405231	0.0823023898886913\\
  0.150259015365684	0.44124278931849\\
  0.151127818129951	0.095199469961117\\
  0.15238347502824	0.0105495603914474\\
  0.152548194167081	0.0372925866972138\\
  0.153951161525083	0.358651056832432\\
  0.154261779503138	0.193749816079673\\
  0.154996443961475	0.257520642990462\\
  0.155100951759364	0.0241860506841379\\
  0.155127720265069	0.488368764185881\\
  0.156010719384588	0.627647368246682\\
  0.156153304516724	0.272618655847682\\
  0.156160528984067	0.049164760989215\\
  0.156551077973885	0.179539605463146\\
  0.156789630481134	0.242941326555341\\
  0.156833896985687	0.303670488992774\\
  0.157648159754314	0.112019561364504\\
  0.157984702917568	0.842015297082432\\
  0.158192133147264	0.33658307203242\\
  0.158573323204214	0.228767614391005\\
  0.158800846723999	0\\
  0.159041322009413	0.0609095509127495\\
  0.159225087981896	0.165901157926886\\
  0.159306729369912	0.124413250570689\\
  0.160112256414541	0.138633814140959\\
  0.160325671631555	0.152242104159936\\
  0.160426142132046	0.07524652191054\\
  0.161510960384936	0.088430480103416\\
  0.162683203433027	0.420317964640514\\
  0.162977750114137	0.101441802555949\\
  0.1630385890026	0.728957083427853\\
  0.163456802369523	0.0341510544509034\\
  0.163704118397325	0.215705371009881\\
  0.163794703722933	0.319304322169457\\
  0.164454062981827	0.00906588648851562\\
  0.165030480431753	0.375118957155642\\
  0.165726072693848	0.202143263780624\\
  0.166200117259099	0.0197734022754923\\
  0.166586746001344	0.0460069720138427\\
  0.166872234508822	0.284161973018615\\
  0.167412101397136	0.514876557673001\\
  0.168084769014926	0.187334908450176\\
  0.16915196857443	0.114837293952026\\
  0.169846119695711	0.130184838439622\\
  0.169958478793305	0.25282836216837\\
  0.170331943368381	0.173896742097976\\
  0.170446123195369	0.0684025685789302\\
  0.171337721839757	0\\
  0.171463627124616	0.14500851718476\\
  0.171467238637224	0.0562751726508597\\
  0.171757333643534	0.237963479168788\\
  0.171882342440603	0.159368818195416\\
  0.172036765682883	0.349614598513891\\
  0.172356012376723	0.0822017068578018\\
  0.172911129630703	0.094626901026152\\
  0.173147641567053	0.394927752556478\\
  0.173705651112842	0.0283044382662697\\
  0.174220202267602	0.299073288619938\\
  0.174509049029214	0.267597857244422\\
  0.17484993084862	0.448653253787832\\
  0.176410947124486	0.0397516258813804\\
  0.176540979758427	0.0091144610875862\\
  0.177495941565122	0.105587711344986\\
  0.178362346490958	0.211676378705913\\
  0.179035211433521	0.19776371977535\\
  0.179051568949357	0.580685233539005\\
  0.179660404483444	0.120564547282945\\
  0.179807273937216	0.331651754832809\\
  0.180314834243958	0.225855960005878\\
  0.180492115916947	0.0506636131442283\\
  0.18102826464078	0.0192891053044195\\
  0.181312632349298	0.133758134050378\\
  0.182104782549811	0.0748985496174461\\
  0.182614425123755	0.0627316004198869\\
  0.183537149972615	0.171230416062161\\
  0.183596846134931	0.184842533024011\\
  0.183717470415488	0.0889631175940742\\
  0.183750919288766	0\\
  0.184292316415296	0.146105996396125\\
  0.184416609632111	0.313429991615697\\
  0.184975777325687	0.365149354205653\\
  0.184990658632646	0.158122614220303\\
  0.185338617125444	0.541990798492148\\
  0.185452536377486	0.279546363182019\\
  0.185746187461993	0.0313045613987863\\
  0.186023539217708	0.246945913283475\\
  0.186831293944324	0.473169494640682\\
  0.187339093113604	0.413241860842513\\
  0.188644964319533	0.111884766921405\\
  0.188896979429054	0.0430397840857915\\
  0.189111482134144	0.00964695609921599\\
  0.189741767981445	0.0996178858556627\\
  0.191306573153503	0.0561472093728883\\
  0.192214229993883	0.126047064443525\\
  0.192775944125064	0.0219356763431217\\
  0.192788717624773	0.261211745527507\\
  0.193168458455192	0.294871983706999\\
  0.193244484563996	0.0812055857363153\\
  0.193506411892478	0.0689656509596469\\
  0.194484759564003	0.204562389376003\\
  0.194781699256657	0.234030839836363\\
  0.19517147515635	0.139225736934054\\
  0.196075605813091	0\\
  0.196396670816254	0.219005465540969\\
  0.197242913306369	0.164752473077997\\
  0.197475711033101	0.0350606403409925\\
  0.197755295649698	0.178289141098761\\
  0.197820235347305	0.343242178100559\\
  0.197998128170703	0.382714611955953\\
  0.198889064968178	0.621276104331893\\
  0.199251681930507	0.0491522129144171\\
  0.199499051678844	0.435500022088113\\
  0.19989764748807	0.151703092678819\\
  0.20029476079176	0.500213610201864\\
  0.200398232325659	0.680394109649356\\
  0.200644368153935	0.0910129820251057\\
  0.200866926945771	0.00863966130798653\\
  0.201041534106839	0.117183128855938\\
  0.201072736490853	0.191786213651196\\
  0.202823638265838	0.0609449947279225\\
  0.202869209999454	0.104716182322415\\
  0.2035541629791	0.0190788604157728\\
  0.203621124697039	0.246890429538062\\
  0.204253911320595	0.273895984976925\\
  0.204453999046721	0.0752715450378894\\
  0.204643771166218	0.309047431916273\\
  0.206055264205113	0.13282878285334\\
  0.20710704943905	0.0298777791739391\\
  0.208118815702127	0.0442009253861729\\
  0.208345830483695	0\\
  0.208364922531887	0.791635077468113\\
  0.209643917902687	0.206889635021032\\
  0.210214227834235	0.360438432464742\\
  0.210981656581447	0.170822728560015\\
  0.211511821788278	0.327348112458203\\
  0.211804860302541	0.229932211656593\\
  0.212676374890166	0.0967087945821528\\
  0.212826099008546	0.00887922671851848\\
  0.212845311469261	0.0558014437069216\\
  0.213033418437731	0.0841770459779234\\
  0.213114861626002	0.121876119439112\\
  0.213232068557152	0.144191372209748\\
  0.213818868861663	0.564059918141884\\
  0.213960131365274	0.0672701068446325\\
  0.214879449227579	0.157558855678955\\
  0.21501545559873	0.109151762196647\\
  0.215109080285985	0.290067837926915\\
  0.215338419089128	0.0196994056916435\\
  0.215353052888104	0.184189582234441\\
  0.215553447217869	0.260446518025961\\
  0.215944952233761	0.0370308989881557\\
  0.216064259129695	0.402974698868317\\
  0.218745366160695	0.45575257848762\\
  0.22024808364689	0.132780975935697\\
  0.220594911959795	0\\
  0.220980553764414	0.216922561955314\\
  0.221043238866754	0.526840471471785\\
  0.221441054330508	0.0493959636076484\\
  0.221541803024836	0.197394026135119\\
  0.221927034897234	0.244562022774565\\
  0.222814868067252	0.0282192317772392\\
  0.223079462772038	0.0757097371393774\\
  0.223854831627415	0.378993744954983\\
  0.224436506055933	0.00888865393948487\\
  0.224898589387712	0.0916352590171472\\
  0.22524866887545	0.061182529376839\\
  0.225841176042297	0.274684857409257\\
  0.226791733725482	0.305097589666409\\
  0.227020635640704	0.147974576903904\\
  0.227222612890954	0.166735732297022\\
  0.227522434467402	0.0400071592657904\\
  0.227965940024527	0.116775037732119\\
  0.229025238550161	0.103758606723921\\
  0.22903452565735	0.0188022411500831\\
  0.230192629889635	0.229906474667567\\
  0.230561553308234	0.344054015510026\\
  0.231123579049714	0.180630826352495\\
  0.231918546649284	0.424436972994086\\
  0.232855806165651	0\\
  0.233318055164616	0.053356025129995\\
  0.233789727445938	0.0823563660835328\\
  0.234550554694482	0.482561834189572\\
  0.234958129226228	0.12804283769535\\
  0.235016730073647	0.0296161965637447\\
  0.235703841234137	0.209142680040279\\
  0.235936275207112	0.0679420801646127\\
  0.23627867748995	0.00900320582483908\\
  0.237025417968833	0.322848701843456\\
  0.237768044555172	0.156817530207187\\
  0.238499545625968	0.193680960397387\\
  0.238672962590579	0.041865439204077\\
  0.239269129407992	0.141006614251986\\
  0.239466100518024	0.251130238032243\\
  0.239904076824799	0.0936596807176189\\
  0.241642291098151	0.0198452314660206\\
  0.24237534496295	0.111578441953258\\
  0.242765614113618	0.170404331162084\\
  0.242965189410159	0.286980703729786\\
  0.243985817353714	0.0580265518425069\\
  0.244650106290354	0.399204384433928\\
  0.244763950316582	0.593739505015733\\
  0.245161463192649	0\\
  0.246242553280861	0.0780505158409771\\
  0.246443643532349	0.221263236818749\\
  0.24651748258445	0.0330779893459596\\
  0.246664500594823	0.267877419643706\\
  0.247376441927196	0.0092448625709971\\
  0.247892425185918	0.359435426530461\\
  0.248061317360031	0.0481742051344303\\
  0.250512408424563	0.122357053703788\\
  0.250632162712842	0.646347501680113\\
  0.251043201002133	0.101203501059945\\
  0.251602963422059	0.237627053345695\\
  0.252119304655797	0.18220868611936\\
  0.252756871055982	0.021110231854706\\
  0.252912247465103	0.1502630191219\\
  0.253100162283581	0.0669731460087698\\
  0.253486593568913	0.304371376946194\\
  0.253911891187431	0.135894442292894\\
  0.254468590320535	0.0882809219842657\\
  0.254656885581892	0.204113188657504\\
  0.25733365919461	0.506315285196932\\
  0.25754518172658	0\\
  0.257617290118105	0.042811763372715\\
  0.257778943048855	0.0102522098029303\\
  0.257896530450881	0.742103469549119\\
  0.258154812635181	0.445734223434276\\
  0.258169670106323	0.0313427207337327\\
  0.258630952689111	0.163892759754841\\
  0.259039487509538	0.54829756539184\\
  0.259066660299217	0.0562138251943249\\
  0.259556842080576	0.11170536005934\\
  0.261394820889895	0.252916231386386\\
  0.26262503788876	0.0776679677337288\\
  0.263321916131712	0.328622306822155\\
  0.264118127276596	0.378971991825619\\
  0.264777701138104	0.219780658436752\\
  0.265789947309536	0.0221515851698116\\
  0.266356256753182	0.192065759741136\\
  0.266519176381447	0.128148961987806\\
  0.267212116850958	0.14273127272063\\
  0.267463819102822	0.271378128453133\\
  0.267496286544929	0.0652828627059737\\
  0.267800794985164	0.0102567648077974\\
  0.268253045221161	0.175942650056638\\
  0.268295537850593	0.0479815183747349\\
  0.268758048373291	0.0942054105866824\\
  0.269729591511184	0.0349292699879292\\
  0.270040969128116	0\\
  0.271446439085814	0.416743138795542\\
  0.273299871325977	0.23557027285821\\
  0.273567795225794	0.350570592714457\\
  0.274994320568572	0.474627764973546\\
  0.275370554144009	0.155376531901595\\
  0.275498565217169	0.106892184218256\\
  0.275634689221768	0.0581602425528393\\
  0.275930947049065	0.205862177212222\\
  0.276743136241158	0.0829031976668666\\
  0.276786429358036	0.289975895262562\\
  0.277534686709128	0.0207299952140417\\
  0.278350681676346	0.00894059382008069\\
  0.279335786448548	0.120701600086837\\
  0.28041234112296	0.0447583353470268\\
  0.281057562377602	0.312071645317832\\
  0.281587990857944	0.0322008639479807\\
  0.281619077253439	0.0700689388039436\\
  0.282319888295822	0.256249495212913\\
  0.282683915444403	0\\
  0.283445649448221	0.0950101362498267\\
  0.283458814556079	0.136456374815444\\
  0.283537201741431	0.168868865486874\\
  0.286879172181229	0.0559386952747774\\
  0.287874269142823	0.219108915350825\\
  0.288771286202394	0.184634229078829\\
  0.289181688246061	0.0108211316309399\\
  0.290467358602726	0.0237778318232694\\
  0.291361450195347	0.111237284816366\\
  0.29177437204839	0.0793081481285242\\
  0.29211346543417	0.149253822057201\\
  0.292382821926944	0.0385063192311908\\
  0.292539819819164	0.393271145677908\\
  0.29478768437035	0.125816441911771\\
  0.294788580757151	0.332070062211438\\
  0.294803787744417	0.272870086411112\\
  0.295510590106772	0\\
  0.29618315383901	0.0643057241082508\\
  0.296393425539464	0.200389451963832\\
  0.297446406423662	0.241900113544944\\
  0.297544570592085	0.364233396971847\\
  0.297722881548528	0.050288080749054\\
  0.298814237972259	0.0916360400751738\\
  0.299120120667708	0.441525057628801\\
  0.299322866647018	0.59469615198571\\
  0.299520097416642	0.530759837925102\\
  0.300955094799577	0.00810470742016123\\
  0.301474696388161	0.162519564061044\\
  0.301475942756802	0.0195504157874425\\
  0.301809971861863	0.294311013917505\\
  0.303302767511725	0.0319994625971644\\
  0.305264951470562	0.104854969011275\\
  0.305666770125481	0.488058027835769\\
  0.306062401715977	0.136513741208303\\
  0.306738718146423	0.693261281853577\\
  0.306745060341778	0.17885436997678\\
  0.306750950813941	0.0735212388736186\\
  0.308200169861405	0.0441419948781192\\
  0.308205259432484	0.0603862758041205\\
  0.308559470899221	0\\
  0.310934607984386	0.213042076609839\\
  0.311492614034036	0.118828426770231\\
  0.313071570934899	0.0101997925117371\\
  0.313426031511725	0.255292943229887\\
  0.313601088197478	0.0863404757633326\\
  0.313932194472218	0.0229328021670567\\
  0.31551525573897	0.149873672798745\\
  0.316089799132776	0.194383124333142\\
  0.316605403812866	0.232259386539749\\
  0.316616299086418	0.313761122892682\\
  0.317848214689882	0.0360491244407892\\
  0.318243036552947	0.411579881877124\\
  0.318469852965095	0.0527382663513333\\
  0.320127723239281	0.344965499158653\\
  0.320320698624769	0.0998467259787069\\
  0.321425691944686	0.068382621749974\\
  0.321787367328102	0.27633728895033\\
  0.321871416011411	0\\
  0.322351536876542	0.165771082462938\\
  0.324433663390015	0.130108500744727\\
  0.325205308393412	0.00972637298170561\\
  0.326806094489094	0.0222034971782142\\
  0.32739573504554	0.113893062947725\\
  0.328576651791492	0.0814588109111438\\
  0.329094755293083	0.453128229969943\\
  0.329915974271922	0.372991216122689\\
  0.330532001307904	0.049710361496784\\
  0.330582013038358	0.0349791690021128\\
  0.331580230973624	0.181602194386731\\
  0.333111025002812	0.0652264633566005\\
  0.333136756360996	0.217933813354914\\
  0.33463456156765	0.143636150173268\\
  0.335490191691098	0\\
  0.336334178915461	0.0948858887775361\\
  0.336937954026783	0.0116954554548954\\
  0.337874444835004	0.242580798271768\\
  0.339897753751335	0.292978106112773\\
  0.341198212957754	0.0255851846396916\\
  0.341358543396129	0.198561077439586\\
  0.34164908773835	0.324422666761308\\
  0.342197549936856	0.159944782094313\\
  0.342410067894825	0.125600488085691\\
  0.342545302468842	0.0422283303852833\\
  0.343950634080858	0.0755089361460712\\
  0.344179004615517	0.0582767105658912\\
  0.344857625858124	0.108693690556394\\
  0.345593007720893	0.551787307732059\\
  0.345622962755066	0.264722448342976\\
  0.349031702359293	0.492068881899213\\
  0.34913460578469	0.0110391354609574\\
  0.349463070288524	0\\
  0.353005434687387	0.175701534355183\\
  0.353392028160536	0.0883532019601807\\
  0.354113310843263	0.138767943416249\\
  0.354955437198207	0.0233301377214797\\
  0.355043016365879	0.644956983634121\\
  0.355066946865179	0.0368730026090075\\
  0.355217008716887	0.350900744733989\\
  0.355618005083854	0.226918218886558\\
  0.356438166314062	0.431079082036589\\
  0.356525078654949	0.0525725922244305\\
  0.357541460813636	0.0716434645770989\\
  0.358177291520807	0.397757790966787\\
  0.361248574834717	0.119360841465567\\
  0.362322803837587	0.0115327407499914\\
  0.362447305823265	0.15559197149381\\
  0.36340555245971	0.101783014701797\\
  0.363841516484453	0\\
  0.365190867682947	0.192202633796841\\
  0.367158911336095	0.247581176230065\\
  0.367195502995641	0.307733645548671\\
  0.367817406559302	0.279446857539744\\
  0.369103159664043	0.0621922246312301\\
  0.369162777035673	0.043907640959971\\
  0.369167556046384	0.0274465304572429\\
  0.372181892215279	0.0800374195803602\\
  0.374499254224419	0.131994673155347\\
  0.374690606191434	0.212253398208308\\
  0.375122575865706	0.0124035634413512\\
  0.375440729272547	0.170846881261865\\
  0.378681983413947	0\\
  0.381903039091165	0.0943835091202114\\
  0.38323717750696	0.0567086143355543\\
  0.383743212061126	0.0395813862164207\\
  0.384846454847182	0.148627888988586\\
  0.384860772418944	0.0250528058221261\\
  0.385071825283277	0.370962930031262\\
  0.386386216250156	0.0750198849384183\\
  0.386601991242849	0.112125831486413\\
  0.387383464801771	0.330615711881238\\
  0.389370285193944	0.0114535942085971\\
  0.390696035753974	0.186186152781603\\
  0.392692039960929	0.264250089369608\\
  0.393519860300165	0.454097372202695\\
  0.394046845512526	0\\
  0.394497897747703	0.23194649094277\\
  0.395769771233853	0.405939862759319\\
  0.395920738443266	0.129096153169604\\
  0.396086268540623	0.508356125941403\\
  0.398070153711434	0.0485250291952329\\
  0.400208355274484	0.0673372570543623\\
  0.400528527552675	0.297116060478013\\
  0.400978810506645	0.088464561874499\\
  0.401294550819698	0.0302810076687541\\
  0.401316504876554	0.20719949333439\\
  0.401376441415815	0.161642134178283\\
  0.402955194611025	0.597044805388975\\
  0.403098228757113	0.0136739078168239\\
  0.40962732440755	0.105528263907619\\
  0.410005501594225	0\\
  0.414319748363103	0.0426212113318056\\
  0.415603435826412	0.143876399255084\\
  0.416042868991326	0.0608488117905113\\
  0.416497964212114	0.0835068402148712\\
  0.41719451003192	0.247156764094423\\
  0.417277542379809	0.348925261751899\\
  0.417991277971996	0.0264562424234738\\
  0.41825386300134	0.0111127314261926\\
  0.419944805760271	0.178463373566883\\
  0.420827794469998	0.122565828161642\\
  0.425716214977532	0.27579124249657\\
  0.426635690437074	0\\
  0.428554263336693	0.313956937378966\\
  0.431080936678311	0.201029567642178\\
  0.431316192301438	0.0730575394996646\\
  0.431622355779326	0.0998295998832971\\
  0.431773099024616	0.0498732855866208\\
  0.435057653237025	0.0133936510988084\\
  0.435435979495485	0.0313512440343678\\
  0.436131252389607	0.159227147196859\\
  0.437583417464644	0.405676642995391\\
  0.439895319366861	0.227146211200574\\
  0.443937602565015	0.117414523962998\\
  0.444025072714139	0\\
  0.447309809578362	0.0895137960061859\\
  0.447999512952417	0.455022517997427\\
  0.448696949155777	0.0674619314877447\\
  0.450617026721293	0.549382973278707\\
  0.451413275949476	0.139391030366683\\
  0.452207992630623	0.0294745589736348\\
  0.452283331714393	0.366179539314127\\
  0.452472261442339	0.0122982166530624\\
  0.452663943043887	0.17818173962311\\
  0.452846130500676	0.0493353308760084\\
  0.456236998511992	0.252615771351791\\
  0.460023188192319	0.292829451553137\\
  0.461083112931254	0.20456716751765\\
  0.461314915986378	0.327406651347342\\
  0.462273148464093	0\\
  0.466310526831185	0.106855291885749\\
  0.467297224565562	0.0813836253108451\\
  0.468681244999491	0.0374256411759792\\
  0.469063716963377	0.0157097360885439\\
  0.469701720868882	0.156494964341492\\
  0.471154918738836	0.0588314329762219\\
  0.48111788772864	0.127487440178764\\
  0.481493599849653	0\\
  0.482931366848862	0.229446057177804\\
  0.482971400911071	0.180671764934358\\
  0.485974134271959	0.0374511535305221\\
  0.487413543840461	0.0979914659713311\\
  0.487613040171867	0.0154162648121844\\
  0.488657934778207	0.0680222119886482\\
  0.490599056334192	0.268964974401508\\
  0.495775461235339	0.404863590854522\\
  0.498167968157511	0.501832031842489\\
  0.499529770210672	0.351645523003936\\
  0.501817176706908	0\\
  0.502022017913958	0.304038414990729\\
  0.504982714569924	0.0383209846391509\\
  0.505746241263271	0.146664843715847\\
  0.505946905103852	0.20382036836885\\
  0.507085352275534	0.0167059709031959\\
  0.50786542444936	0.0849772085183946\\
  0.509974074478105	0.116994624857075\\
  0.510843362627296	0.060567207496122\\
  0.517287499099727	0.245577934718036\\
  0.51922072215641	0.173387364674998\\
  0.523395280227554	0\\
  0.52818555099672	0.0378266456135525\\
  0.528534363810815	0.0166151941336975\\
  0.531862119322869	0.131020383655737\\
  0.532818516823522	0.0971396310530869\\
  0.534156613806435	0.0620476513277391\\
  0.543183710325448	0.22149464506848\\
  0.545746787176797	0.454253212823203\\
  0.546383917692217	0.299217813861153\\
  0.546404452243655	0\\
  0.547757191109357	0.190504332936244\\
  0.551971959479232	0.350384629761217\\
  0.55313026784428	0.0168243127580385\\
  0.554206088147992	0.0818996184327364\\
  0.556182011010363	0.156134928973627\\
  0.556447616978065	0.0385179496400091\\
  0.56160186241617	0.257290848856086\\
  0.562339523743279	0.120459552466343\\
  0.569694566482592	0.0588436272203566\\
  0.571052050210631	0\\
  0.580480576334196	0.0172340879258031\\
  0.582665491335779	0.0967433056975811\\
  0.58404395560592	0.202757941233328\\
  0.591094932861439	0.168504703452162\\
  0.59217287804647	0.0380477184023183\\
  0.593493190508234	0.406506809491766\\
  0.597421387072606	0.0694876779286465\\
  0.597583490408627	0\\
  0.60210410656353	0.128993340850036\\
  0.606140064239905	0.293084306509347\\
  0.608135957842241	0.237532276566343\\
  0.618235088093282	0.0222801675471766\\
  0.623708546335519	0.0520898172181544\\
  0.624245045295224	0.0948142686001779\\
  0.626291588242436	0\\
  0.635666223534624	0.180855101728823\\
  0.64154948429987	0.35845051570013\\
  0.649435964278186	0.139380186984509\\
  0.652994624743048	0.0272660509673008\\
  0.656518631650878	0.0640097123168313\\
  0.657528736656325	0\\
  0.660235163451603	0.235319864329991\\
  0.669958415226678	0.101648520228823\\
  0.686536269085143	0.0350852595144426\\
  0.690062313042738	0.309937686957262\\
  0.691722975924769	0\\
  0.697409198547835	0.189204075594593\\
  0.710028078886515	0.0787470173038922\\
  0.711857681094088	0.139286862526321\\
  0.729399475430738	0\\
  0.729490547377566	0.0346277685029349\\
  0.739184523038948	0.260815476961052\\
  0.76537378029135	0.0606978595920142\\
  0.766367408522739	0.117396102541982\\
  0.771209660422376	0\\
  0.789077203056588	0.210922796943412\\
  0.817971324494741	0\\
  0.821066752262789	0.0551294095192394\\
  0.839911963558909	0.160088036441091\\
  0.870724829273362	0\\
  0.891873527979063	0.108126472020937\\
  0.930813359856943	0\\
  0.945162725894684	0.0548372741053156\\
  1	0\\
  };
  % %\addlegendentry{data1}

  \addplot [color=mycolor1,mark size=1pt,only marks,mark=triangle*,mark options={solid,scale=1.5},forget plot]
    table[row sep=crcr]{%
  0	0\\
  };
  %\addlegendentry{data2}

  \addplot [color=mycolor2,mark size=1pt,only marks,mark=square*,mark options={solid},forget plot]
    table[row sep=crcr]{%
  0	1\\
  0.0539906089087773	0.946009391091223\\
  0.106586935007253	0.893413064992747\\
  0.157984702917568	0.842015297082432\\
  0.208364922531887	0.791635077468113\\
  0.257896530450881	0.742103469549119\\
  0.306738718146423	0.693261281853577\\
  0.355043016365879	0.644956983634121\\
  0.402955194611025	0.597044805388975\\
  0.450617026721293	0.549382973278707\\
  0.498167968157511	0.501832031842489\\
  0.545746787176797	0.454253212823203\\
  0.593493190508234	0.406506809491766\\
  0.64154948429987	0.35845051570013\\
  0.690062313042738	0.309937686957262\\
  0.739184523038948	0.260815476961052\\
  0.789077203056588	0.210922796943412\\
  0.839911963558909	0.160088036441091\\
  0.891873527979063	0.108126472020937\\
  0.945162725894684	0.0548372741053156\\
  1	0\\
  };
  %\addlegendentry{data3}

  \addplot [color=mycolor3, draw=none, mark=pentagon*, mark options={solid, mycolor3,scale=0.75}]
    table[row sep=crcr]{%
  0.1125	0.1125\\
  0.125	0.125\\
  0.1375	0.1375\\
  };
  %\addlegendentry{data4}

  \end{axis}
  \end{tikzpicture}%
\\
\vspace{11pt}
% This file was created by matlab2tikz.
%
%The latest updates can be retrieved from
%  http://www.mathworks.com/matlabcentral/fileexchange/22022-matlab2tikz-matlab2tikz
%where you can also make suggestions and rate matlab2tikz.
%
\rmfamily
\definecolor{mycolor1}{rgb}{0.00000,0.44700,0.74100}%
\definecolor{mycolor2}{rgb}{0.85000,0.32500,0.09800}%
\definecolor{mycolor3}{rgb}{0.92900,0.69400,0.12500}%
\definecolor{mycolor4}{rgb}{0.49400,0.18400,0.55600}%
\definecolor{mycolor5}{rgb}{0.46600,0.67400,0.18800}%
\definecolor{mycolor6}{rgb}{0.30100,0.74500,0.93300}%
%
\begin{tikzpicture}[trim axis left, trim axis right, baseline]

  \begin{axis}[
    axis x line*=bottom,
    axis y line*=left,
  grid=major,
  %%tick label style = {font=\sansmath\sffamily},
  width=0.55\textwidth,
  height=0.55\textwidth,
  % at={(0\textwidth,0\textwidth)},
  % scale only axis,
  unbounded coords=jump,
  xmin=0,
  xmax=1,
  ymin=0,
  ymax=1,
  xtick={0,0.25,1},
  xticklabels={$0$,$K$,$4K$},
  xlabel={$s$},
  ylabel={$v$},
  ytick={0, 0.2, 0.4, 0.6, 0.8, 1},
  yticklabels={$0$,$0.1$,$0.2$,$0.3$,$0.4$,$0.5$},
  % yticklabels={,,}, %hides y ticks
  axis background/.style={fill=white},
  %title style={font=\bfseries},
  title={HST},
  legend pos=north east,
  legend style={legend cell align=left,align=left,draw=white!15!black}
  ]
  \addplot [color=black,mark size=0.5pt,only marks,mark=*,mark options={solid},forget plot]
    table[row sep=crcr]{%
    0	0.001\\
    0	0.0168187505296433\\
    0	0.03241608873095\\
    0	0.0479033014713116\\
    0	0.0633897208445366\\
    0	0.0789858291198479\\
    0	0.0948064895022532\\
    0	0.110974510993003\\
    0	0.127624798003611\\
    0	0.144909410108953\\
    0	0.163003979469953\\
    0	0.182116129362376\\
    0	0.20249685310644\\
    0	0.224456331105068\\
    0	0.248386536879842\\
    0	0.274794502077065\\
    0	0.304352857346471\\
    0	0.33797946579834\\
    0	0.376968359847148\\
    0	0.423216382959283\\
    0	0.479641324488477\\
    0	0.551019300611469\\
    0	0.645857993453341\\
    0	0.781315527547944\\
    0	1\\
    0.0064884388643925	0.00739747352950067\\
    0.00680678159453577	0.024507978313706\\
    0.00970774806638557	0.0721269422743722\\
    0.00978373808983118	0.0864795505123035\\
    0.0101331330151388	0.0591151527518648\\
    0.0102938631911058	0.0364221909288827\\
    0.0109103880404063	0.100914377223029\\
    0.011709692065377	0.0474663837712988\\
    0.0118388206177656	0.114239997571135\\
    0.0136338389994147	0.140839833175224\\
    0.0137616478994845	0.127283499584583\\
    0.0150715377692652	0.155783827068037\\
    0.0154417124494077	0.172311309372683\\
    0.015470934107664	0.015504920301584\\
    0.0159538068350802	0.001\\
    0.0169116707901369	0.191469237822446\\
    0.0197242037809676	0.212233219100464\\
    0.0211391480842565	0.0268763703034514\\
    0.0220174387756894	0.233856891297887\\
    0.0222942535656742	0.0799759373049013\\
    0.0226497597593516	0.0657151868599118\\
    0.0234918206342378	0.264416547497128\\
    0.0239285499213649	0.00713169064986919\\
    0.0239568260382449	0.0947252210422286\\
    0.0243918891913886	0.039009204081929\\
    0.0245592817777545	0.0523642525918943\\
    0.0260992248211872	0.110407470107227\\
    0.0276879353924553	0.290108098353286\\
    0.0290276871039797	0.127143592465511\\
    0.0303458856387703	0.144232334414224\\
    0.0313261098822697	0.0182323421781905\\
    0.0317143047568545	0.001\\
    0.0335700347413698	0.160879977355428\\
    0.0339078439483111	0.350717086227886\\
    0.034108196040092	0.179525178310804\\
    0.0355359526496423	0.0714527717666451\\
    0.0360569547775396	0.0858836118333147\\
    0.0366330918587693	0.0298865883450941\\
    0.0369264965682882	0.0571976386809246\\
    0.0374303823858631	0.0428727998154523\\
    0.038339569544601	0.10127309735515\\
    0.0385814565343237	0.201670794289202\\
    0.0396127677626535	0.00861856355235861\\
    0.0407850359210592	0.116806907670375\\
    0.0440122743411237	0.134955590858022\\
    0.0442384071569869	0.247859800227807\\
    0.04652642510428	0.0196178042077513\\
    0.0467675155291464	0.313922451413856\\
    0.0472964549285347	0.001\\
    0.0483945185803461	0.221022257236043\\
    0.0486011150162608	0.397105063809593\\
    0.0488899542868703	0.0487078793228525\\
    0.0489683237397493	0.0638405381080175\\
    0.0490239424684588	0.077659730331456\\
    0.0494979676893947	0.0922435956880214\\
    0.0501061745608403	0.150222059089618\\
    0.0501817794469659	0.0326589071880599\\
    0.0510972964270354	0.447151761877607\\
    0.0518371376044242	0.168510542880969\\
    0.0536979490113692	0.187414463802323\\
    0.0541979996977226	0.108845220533584\\
    0.05461935271567	0.00818927270090638\\
    0.0551365409436224	0.124599275580257\\
    0.0555779836438537	0.270328556325451\\
    0.0602590583693506	0.0416270786989785\\
    0.0604486029771647	0.018788170283382\\
    0.0606746918949532	0.0563252331994035\\
    0.0626360303997046	0.0831077233806319\\
    0.0627146054325225	0.001\\
    0.0631659603182663	0.0976216469117928\\
    0.0631823918580469	0.0695121146051192\\
    0.0640988316831066	0.138821559619121\\
    0.0651286778981493	0.0298276605022845\\
    0.0656071659058072	0.204871220547805\\
    0.067013509817995	0.158995652366362\\
    0.0690873482971124	0.119690995710811\\
    0.0694866634346422	0.238395865995615\\
    0.0702095260823471	0.00841484500828786\\
    0.0716957500491606	0.0491267810932558\\
    0.0717845982096277	0.180155104238293\\
    0.0722825830421164	0.537628427705455\\
    0.0743521586506391	0.365484983879869\\
    0.0746997143011316	0.10598532176894\\
    0.0753975128929699	0.0199492261768311\\
    0.0756031233287984	0.291002151048066\\
    0.0762435147872152	0.0362477753323776\\
    0.0762470667398013	0.0620501919819033\\
    0.0766032049261186	0.0904371371118907\\
    0.0766357384270237	0.0762069185522187\\
    0.0779825491879939	0.001\\
    0.0797398627919963	0.147180495929654\\
    0.0804132075596921	0.130674180944729\\
    0.0844495325027431	0.219685574298383\\
    0.0845539466138033	0.0474904309798297\\
    0.0846829877543603	0.167010811007913\\
    0.0847553639570597	0.00881177955054861\\
    0.0847669685371318	0.259498492894729\\
    0.0856454306397393	0.0275572500762942\\
    0.0857978545377945	0.335974419622807\\
    0.0860653168999378	0.195930075480171\\
    0.0865406631254788	0.116051491020459\\
    0.0890398010300804	0.100343966568305\\
    0.0900891595337439	0.0720142786188911\\
    0.0902441261811509	0.0592836878062344\\
    0.0903807788377716	0.0858507385762115\\
    0.0928917451045627	0.0176435969022043\\
    0.0931135777173156	0.001\\
    0.0933180438557983	0.0387506792281224\\
    0.0959018835224824	0.153720688974181\\
    0.0961943752823302	0.137101536605627\\
    0.0983303007815648	0.181182377685578\\
    0.100403896997336	0.122678975588224\\
    0.100573428448484	0.0080495310151485\\
    0.100575121879803	0.0276147994051304\\
    0.100705183427835	0.0505912188589106\\
    0.101177885434084	0.480094556850489\\
    0.101873871764155	0.107743792795047\\
    0.103184555568205	0.078217720406282\\
    0.103254492158696	0.0928225710191791\\
    0.103590160203888	0.240934613685405\\
    0.104350764071459	0.0642028255217804\\
    0.104938553654324	0.310526701926597\\
    0.105286656048637	0.211407848252395\\
    0.107884050254108	0.0385501012297784\\
    0.108120531235085	0.001\\
    0.108770383319095	0.0184619110651763\\
    0.108987172470203	0.166287934881212\\
    0.108990372763607	0.275620792123958\\
    0.112436464569812	0.147952744740134\\
    0.114071437789764	0.132241368738264\\
    0.11431141005075	0.0497051280953351\\
    0.115358928701967	0.19314897110247\\
    0.115527537594171	0.00809157578669045\\
    0.115722678814337	0.116884693596586\\
    0.116163671797348	0.101660953515\\
    0.116353218676032	0.0739882350180339\\
    0.116495804489349	0.0291062221303747\\
    0.117058195432595	0.0873379835769461\\
    0.117067175699897	0.368774817594257\\
    0.118150219475208	0.614904992513273\\
    0.118215140069579	0.42582695820831\\
    0.118244903530318	0.0612944402047469\\
    0.123015845485706	0.001\\
    0.123439761032498	0.0182491335983855\\
    0.123773484849673	0.0402240886054868\\
    0.124658698093335	0.227397870337203\\
    0.124770173963616	0.177308846696926\\
    0.127020056848871	0.158645036878702\\
    0.129299258211921	0.256329863015195\\
    0.129456295922043	0.109584263323828\\
    0.12950775651625	0.141700030828524\\
    0.129586639662249	0.0678868920113719\\
    0.130016925481044	0.0949126135828521\\
    0.130032565962171	0.126038223340523\\
    0.130037129447164	0.052263451154143\\
    0.130427828621486	0.00803145633688348\\
    0.130733444200685	0.0810319235227477\\
    0.130899047160809	0.0293070955034769\\
    0.133050206757243	0.207734510977623\\
    0.136170700740509	0.328408107631867\\
    0.13722874195876	0.295592798817034\\
    0.137792107483705	0.0183600092814691\\
    0.137811595714003	0.001\\
    0.137933678091549	0.0407411151248369\\
    0.140311532213105	0.0608087834548083\\
    0.141687227709052	0.189728598550581\\
    0.142577540206527	0.117053717391294\\
    0.142921831819569	0.170134694310544\\
    0.143124745313716	0.101641691281467\\
    0.143863628592233	0.0741934080634566\\
    0.144048128282096	0.0873675283510808\\
    0.144221626785302	0.151958674224928\\
    0.144938727221565	0.0290319312761141\\
    0.145243415359369	0.00825728064392664\\
    0.145670202270311	0.135165140261125\\
    0.146607264076521	0.0502743509073257\\
    0.150464110486803	0.2423579476121\\
    0.152059915708714	0.220350541968291\\
    0.152384350429229	0.0180837948575786\\
    0.152519538117586	0.001\\
    0.153004790472079	0.0390528041669598\\
    0.153471009658844	0.0642682007660024\\
    0.155433449583192	0.123446454399224\\
    0.156299119294678	0.107379112157738\\
    0.156843617008088	0.0926623285332885\\
    0.157387257559584	0.274484971968933\\
    0.157646870319547	0.51202888815572\\
    0.157829151990561	0.0780823976286556\\
    0.158726847422825	0.394370255256928\\
    0.159734682694864	0.050864265520726\\
    0.159891409248813	0.0280873610539236\\
    0.160001647568312	0.162893886240139\\
    0.160023560003306	0.00831602448456733\\
    0.160436293223079	0.181731180061903\\
    0.160683746849451	0.145725776809233\\
    0.161749948518962	0.201684862803643\\
    0.166002270376325	0.0627408840187338\\
    0.166049849612798	0.357694901577838\\
    0.166848627592225	0.0386531279678506\\
    0.167151149098869	0.001\\
    0.167376505778541	0.132122528065795\\
    0.16743659146199	0.0180847731969486\\
    0.169323789463766	0.116240017622102\\
    0.169968683148354	0.101614676408726\\
    0.170269996965078	0.0878500483818373\\
    0.171293197794281	0.0746143726390767\\
    0.171350414531082	0.31266060628724\\
    0.173116255280395	0.0495802862011351\\
    0.174541359720708	0.00814829810204816\\
    0.174899043970341	0.0283070467799791\\
    0.175145367782377	0.229504249758451\\
    0.176407198725011	0.254410898921544\\
    0.17693578020535	0.155655675357679\\
    0.177629392202737	0.172424904546561\\
    0.179109862446292	0.0603905439548355\\
    0.180331805504906	0.140543999013788\\
    0.181396453095918	0.0391435508483251\\
    0.181520500235384	0.189787963563404\\
    0.181717662608121	0.001\\
    0.182271503862056	0.0179776981447449\\
    0.182442642304367	0.125010859848175\\
    0.182982249244653	0.210008864152766\\
    0.183708465824832	0.110551721254295\\
    0.184095709739835	0.0839978444749643\\
    0.184111315744275	0.0969286455137644\\
    0.185049187554155	0.071648670409787\\
    0.185381595152327	0.45902029665802\\
    0.18779424954708	0.0497362165129555\\
    0.188158892040919	0.287644827710182\\
    0.189178577907377	0.0285765919719121\\
    0.189261361978521	0.00816752770093855\\
    0.193278462516428	0.0614726862004519\\
    0.194672939274927	0.165087481074516\\
    0.194730075420582	0.14898276348209\\
    0.196114072635665	0.132995912733535\\
    0.196230105846325	0.001\\
    0.196311481050687	1\\
    0.196365508727076	0.0392632410003692\\
    0.196564230098896	0.0182809723720203\\
    0.197019071134829	0.117860699868689\\
    0.197423770968249	0.103230664188582\\
    0.197586198207224	0.0890466632522527\\
    0.197925002270852	0.0758859341227576\\
    0.199625183737045	0.243255625924186\\
    0.200905436174134	0.180494972065092\\
    0.201722596080541	0.0509343153701486\\
    0.201890105112478	0.200012754420232\\
    0.202207879545274	0.336192882287552\\
    0.203332106891402	0.00840719102887896\\
    0.203913771594024	0.0291217137949807\\
    0.204378123787726	0.222252601360616\\
    0.205520544436443	0.0647721785566879\\
    0.208762762240941	0.265609547065046\\
    0.209937649131453	0.0956113232346181\\
    0.210141647503765	0.142203044760471\\
    0.210448715369931	0.110714749069678\\
    0.210485587623871	0.0805528307343318\\
    0.210528056385356	0.126023186542624\\
    0.210699333577401	0.001\\
    0.210804647439585	0.0183058950935434\\
    0.210928198461878	0.0408264645325261\\
    0.211021486816784	0.374045106592292\\
    0.211657354153287	0.158615105738377\\
    0.213503182455789	0.418016366645332\\
    0.214342725921544	0.0534169546137112\\
    0.217077704282315	0.0303784864186859\\
    0.217574341024072	0.0669135596721855\\
    0.218043907724998	0.00818513444656839\\
    0.219556315819582	0.189617053816603\\
    0.21968324805356	0.291919919822367\\
    0.22016034635929	0.559754101046079\\
    0.220502970945619	0.171579931291222\\
    0.222632665851331	0.0907235357653776\\
    0.223491578489664	0.210180438658915\\
    0.223667796101188	0.104663321056313\\
    0.223753403585007	0.0189660295077552\\
    0.224048018618116	0.119562489264574\\
    0.224082478482275	0.677961049891567\\
    0.224126571315128	0.0781325326259119\\
    0.224665674559853	0.135600087587575\\
    0.225	0.045\\
    0.225136061283242	0.001\\
    0.226632693946853	0.151717722525455\\
    0.226784435383133	0.232946629189977\\
    0.227181980649187	0.0566013937858832\\
    0.229023394303911	0.0320851652760777\\
    0.232094921035476	0.0674639842448084\\
    0.232505064834103	0.00861399506874579\\
    0.234040410999747	0.254213579640862\\
    0.235640933995183	0.0968621570007109\\
    0.235841591961987	0.0202687231192634\\
    0.236159266619417	0.317524355700203\\
    0.236197560821206	0.0834268317289701\\
    0.237000476839117	0.111954644205162\\
    0.237172573048015	0.0436663873334561\\
    0.237574216477029	0.17974883005974\\
    0.237626113865681	0.12769224971047\\
    0.238690201277414	0.16237924517225\\
    0.239550897381701	0.001\\
    0.240337657460699	0.0559909465919549\\
    0.240346604279357	0.198705033352459\\
    0.241033166097022	0.144297882792181\\
    0.242091716649461	0.0320709362739964\\
    0.243018341425317	0.0715912993790034\\
    0.246549597660956	0.00835048025566478\\
    0.246896571990541	0.219563215680297\\
    0.247420020876892	0.0197382734350412\\
    0.248130087634947	0.102290897862744\\
    0.248806350494022	0.0869787439285797\\
    0.25	0.045\\
    0.25001378526047	0.488354792345846\\
    0.250104514290268	0.351207603314274\\
    0.250292444938999	0.27379312468785\\
    0.250318563984403	0.118846617260944\\
    0.251914452126818	0.0593974199754841\\
    0.252158597209744	0.134145328559022\\
    0.253954374716921	0.001\\
    0.254260586263306	0.0317771078757369\\
    0.254513666227732	0.0734135336792492\\
    0.254591831553003	0.154863186028477\\
    0.255002671725372	0.172595242515092\\
    0.256886914095393	0.239248322876539\\
    0.257789703477896	0.400455030593013\\
    0.258072264634108	0.190924703712824\\
    0.258963508149562	0.0201063613409238\\
    0.260519983718356	0.094600211231399\\
    0.260696707151192	0.109156401534328\\
    0.260979683741393	0.00836347635638345\\
    0.26243923732035	0.0413809760838013\\
    0.263458790904144	0.0531118537312775\\
    0.26354160026144	0.0818661906909726\\
    0.264753041165174	0.0660275530944222\\
    0.264938522381822	0.143447786272658\\
    0.265225718788208	0.126490220357194\\
    0.265430833692441	0.208885930427398\\
    0.26700128671526	0.297031323617031\\
    0.268356981523035	0.001\\
    0.268393306924823	0.0305508135529913\\
    0.270960902038974	0.0186355933619558\\
    0.271232908710147	0.164565339187751\\
    0.273213204276012	0.100330877098166\\
    0.273491043741265	0.115208661683393\\
    0.273925935210467	0.182280497798449\\
    0.275	0.045\\
    0.275353265459497	0.0734071837414897\\
    0.275409828974673	0.25590616634309\\
    0.275438092152458	0.00804095455902531\\
    0.275651982050626	0.0593649889469226\\
    0.275743761534364	0.0870316214024877\\
    0.278137013486969	0.225672763630716\\
    0.278920582624947	0.135175560769257\\
    0.279358833344379	0.151322379334146\\
    0.280130935426679	0.0333256263541926\\
    0.280318393232232	0.326214824954045\\
    0.282502567529061	0.43915599789581\\
    0.282607963732114	0.0195228041799749\\
    0.282769192056126	0.001\\
    0.28400023777943	0.199161920551628\\
    0.28517764008908	0.0511787214260768\\
    0.28564628997652	0.122371361396085\\
    0.286069098328496	0.106929719067932\\
    0.287264452303951	0.0664965646371083\\
    0.287625865662886	0.0931051352574604\\
    0.287982077990798	0.0791276047022465\\
    0.289478600889293	0.00768735594755733\\
    0.290207799579103	0.176827326731648\\
    0.291775548713749	0.0409568743373362\\
    0.291936702438729	0.0286644940538921\\
    0.292857635155943	0.370832282695979\\
    0.293808990497188	0.16005189286133\\
    0.294563289593437	0.276439743571403\\
    0.294652122797685	0.143618767813599\\
    0.295290621549026	0.0168612789105814\\
    0.295804925445872	0.0576179734025057\\
    0.297201497085329	0.001\\
    0.297438026524197	0.240839095988445\\
    0.298251489967949	0.129413629250463\\
    0.298845281605769	0.11425496579549\\
    0.298930158730312	0.214806552204115\\
    0.299655944528239	0.0996223897037727\\
    0.300003498082582	0.085091798313333\\
    0.300538132091821	0.535771975217838\\
    0.300842917839327	0.0709322397180865\\
    0.30316565278567	0.192243654125493\\
    0.303310650215316	0.0471703825926678\\
    0.304290509969049	0.0347060293549298\\
    0.304707070109686	0.00870233851985411\\
    0.305071265526481	0.0230936619389077\\
    0.308050813388351	0.172303020654047\\
    0.308267823228705	0.0587153466165621\\
    0.309611944383267	0.303929599369619\\
    0.309964067618992	0.154016450617594\\
    0.311432225707164	0.137991502775444\\
    0.311664434431979	0.001\\
    0.312043966447633	0.0931653229094011\\
    0.312241160321617	0.107989539409734\\
    0.31250400058344	0.079273997413037\\
    0.312550053132936	0.122572540340536\\
    0.314945186900788	0.0414725652721817\\
    0.315051987418328	0.0178618367390136\\
    0.316604281818855	0.0672942634359776\\
    0.317612174544755	0.341168652474456\\
    0.318365656395938	0.257173488540965\\
    0.318992434414199	0.00750538164834238\\
    0.319118434360294	0.228514309743057\\
    0.319134974520899	0.0294562986745562\\
    0.319893863070323	0.205306827178878\\
    0.320089329854604	0.0525498298709329\\
    0.322419278083981	0.184733198394596\\
    0.32440853766632	0.16436277978172\\
    0.32472098885538	0.100691138594355\\
    0.324906944557393	0.0868494523918953\\
    0.325628010429624	0.146251507635722\\
    0.325793919589636	0.115624318177265\\
    0.326168619745806	0.001\\
    0.326804032084025	0.130038889641267\\
    0.327383109636417	0.0189625389876851\\
    0.327590744625594	0.0395793348768739\\
    0.327901550949872	0.0742718856491184\\
    0.328210329402006	0.412154227747489\\
    0.329359738665649	0.0615943202318008\\
    0.333401429764275	0.00820966842421473\\
    0.33434830451207	0.0504912501284994\\
    0.334641377344773	0.283301713900835\\
    0.334670346974961	0.0289340066763609\\
    0.336422551813876	0.465115935903039\\
    0.337440083425952	0.0926897870852156\\
    0.33764718518843	0.107515257544614\\
    0.339196089715289	0.174757063064118\\
    0.339682150978087	0.195222778544675\\
    0.339720637845074	0.155043170389456\\
    0.339906565873115	0.0785856011737714\\
    0.340060739878912	0.217257357945156\\
    0.340590125522577	0.137276611356042\\
    0.340724777709183	0.001\\
    0.341130456116388	0.0178064251852264\\
    0.341161442734928	0.242248662532365\\
    0.341626637033215	0.120928166203064\\
    0.34162777729034	0.0397640133266278\\
    0.341923758282981	0.0649514344673563\\
    0.3443551632794	0.366451633559335\\
    0.347315278088595	0.0518012585439225\\
    0.347904878610056	0.31519071482467\\
    0.348072313084122	0.00770046801858966\\
    0.348472738147022	0.028625726702792\\
    0.349997612431152	0.101859566515144\\
    0.351107692978296	0.0876209445898644\\
    0.353449431437411	0.0744139078943838\\
    0.354507387844481	0.16453219991981\\
    0.354508759196384	0.146025948603268\\
    0.354662675928941	0.128416595167946\\
    0.355136119300029	0.0402353435954266\\
    0.355159097179876	0.0179159572190017\\
    0.355343773864587	0.001\\
    0.355470155075058	0.0620621425509151\\
    0.356292175183509	0.264763558805065\\
    0.356842372164267	0.183850721992155\\
    0.357175733456855	0.113232565707756\\
    0.358908545592139	0.205506407595106\\
    0.361859069045558	0.0514190732563695\\
    0.362135229931582	0.0290331631622741\\
    0.362587254657044	0.0959017640957742\\
    0.362671227294119	0.00792325465024724\\
    0.363214410688775	0.228722629721561\\
    0.364817517231599	0.0811256047299867\\
    0.365159100118261	0.580400331353613\\
    0.367749047586952	0.0665930211399179\\
    0.368325855343764	0.136373801851039\\
    0.369033794479231	0.0404839730780907\\
    0.369387693355804	0.15450156250202\\
    0.369410660916594	0.0183430464520283\\
    0.369910825992096	0.12033533673417\\
    0.370036647266591	0.001\\
    0.371842179961903	0.171675421855954\\
    0.372485413530762	0.105458068522397\\
    0.374392758441242	0.0535825994687677\\
    0.375046725609378	0.291358380754125\\
    0.375201876337064	0.396968636303304\\
    0.375487827645964	0.192771673583698\\
    0.375844271431295	0.0904821224646993\\
    0.376369632186214	0.0295029253392065\\
    0.377316146700946	0.0081881803812591\\
    0.377999399674748	0.251156415936804\\
    0.378139190934027	0.0769042176550531\\
    0.380734996711036	0.33175533848757\\
    0.381493351753961	0.690401276601207\\
    0.381776601521213	0.212746218110201\\
    0.382124482947695	0.064757822496535\\
    0.382373671103179	0.0415037556098709\\
    0.382598694013614	0.144882700122136\\
    0.383139776877874	0.128113461421863\\
    0.383853457711331	0.0187907877865494\\
    0.384814644168108	0.001\\
    0.385009736526885	0.112296772057116\\
    0.386657287467154	0.506897053720334\\
    0.387536691103443	0.0970272773765595\\
    0.388585790015825	0.0533633990651943\\
    0.388812671972852	0.179170340122411\\
    0.389108930616809	0.160061275773853\\
    0.390515024605565	0.0295789966919457\\
    0.390561388467506	0.0820374445579658\\
    0.392021664645115	0.00842246158835211\\
    0.392215206833937	1\\
    0.393135751484355	0.232983554594028\\
    0.394856340474475	0.0680078804747998\\
    0.396267757484687	0.0415959042044747\\
    0.39639710544597	0.362008202426827\\
    0.396624207283736	0.13853198196617\\
    0.397441024831343	0.435936460129789\\
    0.397823363140879	0.121573816427104\\
    0.39854035747006	0.0188546262872564\\
    0.398852463747967	0.195895158907828\\
    0.399418201936789	0.105893073065465\\
    0.399689252961319	0.001\\
    0.399745408817935	0.275746612833334\\
    0.401678117451611	0.054409621185411\\
    0.401915198670075	0.0915532986742349\\
    0.404243868127797	0.170321136776726\\
    0.40471186134728	0.0301277406576586\\
    0.405284091295476	0.0783447738488303\\
    0.406199615874371	0.151651900885346\\
    0.406893997828129	0.00843528693145611\\
    0.407586470644995	0.216697885376942\\
    0.409092007337109	0.0657260959620564\\
    0.410303634222474	0.0421835369768435\\
    0.410774831636644	0.131538807735733\\
    0.411399280570296	0.252929224787549\\
    0.412031796520582	0.114300337216938\\
    0.413281722241845	0.0191169011674138\\
    0.414493812170724	0.0987998790434007\\
    0.414672240606917	0.001\\
    0.416201658120483	0.0543867212774134\\
    0.416344478134583	0.311263795478654\\
    0.417002305157575	0.18348335056516\\
    0.41774538566074	0.0841677730327407\\
    0.419272196908353	0.0305040930740017\\
    0.420754395471447	0.162628594890826\\
    0.421547007753263	0.0696648737821068\\
    0.421656309124476	0.143429214319961\\
    0.421694267066379	0.00854925226541367\\
    0.422644030300786	0.203503311198141\\
    0.424521307996839	0.0422309631045381\\
    0.42461289012627	0.124229896373859\\
    0.426508684766033	0.234202639681128\\
    0.426841807820178	0.108244111224106\\
    0.42811100506787	0.019304592765056\\
    0.429309048154372	0.054819851142857\\
    0.429595397913531	0.0936914231741822\\
    0.42977569080122	0.001\\
    0.431141240205664	0.384192545714526\\
    0.432454244632342	0.0798821229034576\\
    0.433782985629469	0.0307397289107618\\
    0.43457820447324	0.285446415152738\\
    0.435336059989509	0.0666494219349188\\
    0.43613832356646	0.153205142284892\\
    0.436347798444684	0.172895872996493\\
    0.436554524377815	0.134619137573348\\
    0.43688535809439	0.00856098128063707\\
    0.438288911793402	0.480996339413227\\
    0.438351419842743	0.191575055638977\\
    0.438750388503101	0.043152601265646\\
    0.439507748182986	0.117341833808228\\
    0.440092036450071	0.25690853573822\\
    0.441717998183015	0.217858957529082\\
    0.442280331315103	0.101442603539097\\
    0.443026326030981	0.01932189453195\\
    0.444311734173558	0.0559238117812133\\
    0.444940125764912	0.0861654432309105\\
    0.445012044149578	0.001\\
    0.448127163946826	0.0309266923506232\\
    0.448132603098034	0.0710579280653737\\
    0.451134618973021	0.144058607641603\\
    0.45227378629602	0.00868088992063182\\
    0.452331621503966	0.127014010321912\\
    0.452540155693944	0.162954774139565\\
    0.453196842596362	0.0428288281278808\\
    0.454389211748091	0.347356116287308\\
    0.454787813049261	0.110776428043584\\
    0.456653911818182	0.203303908156757\\
    0.45682842877152	0.181637170973758\\
    0.457217857642648	0.0956689344802978\\
    0.457618030642048	0.238109628483245\\
    0.457804774871675	0.0551211273730149\\
    0.458122336352145	0.0197396642519224\\
    0.459524102350135	0.081329029053808\\
    0.460394140636699	0.001\\
    0.461389958119087	0.306228319747404\\
    0.462218958232659	0.0675157614872215\\
    0.463042561522291	0.0317568099092603\\
    0.464365598552383	0.436668690388126\\
    0.466810279244685	0.273411906837533\\
    0.466976279877397	0.13582034287366\\
    0.467608552973819	0.153207733189556\\
    0.46790674690787	0.00935621727117052\\
    0.468147818318078	0.119190296176192\\
    0.468739662654109	0.0440997833916189\\
    0.470623046409241	0.103479618846429\\
    0.472577196942976	0.577360175242979\\
    0.47267805801334	0.0888115520284029\\
    0.472778173529103	0.0570742563012938\\
    0.472891119195892	0.219781497283559\\
    0.473141620433346	0.170791415496169\\
    0.473218301320222	0.0210260605544042\\
    0.474682190035871	0.0738031363783997\\
    0.474868308176258	0.19346594758268\\
    0.475935264710339	0.001\\
    0.47818320274701	0.0336257983273477\\
    0.48074274385334	0.396609345076093\\
    0.482264533579364	0.249504874613469\\
    0.482694228812445	0.128456943989536\\
    0.48319694045386	0.00902038830944576\\
    0.483739273616548	0.145770344267822\\
    0.484129751867013	0.112169540700046\\
    0.484519187887573	0.0475776702167352\\
    0.484960082705203	0.0636248864863159\\
    0.486823282134734	0.0819210873585509\\
    0.487124727741645	0.0971665203546008\\
    0.487783888088289	0.0208443087225101\\
    0.490116652602002	0.161383335957553\\
    0.491225441880677	0.182030180134055\\
    0.491649193324781	0.001\\
    0.492035955875967	0.0337111334477875\\
    0.493024558736947	0.206938993677413\\
    0.496663125298167	0.32120127513063\\
    0.497391818521632	0.230129375727945\\
    0.49740297871496	0.27920617335223\\
    0.497507483971343	0.0724023845400864\\
    0.498050132996353	0.0579949298453706\\
    0.498644458596141	0.121312995540816\\
    0.498925382882079	0.00913334162361565\\
    0.499013735129075	0.138410288359047\\
    0.499973976417388	0.0449479757450603\\
    0.501145863829794	0.0898246027775322\\
    0.501147092125371	0.105928108465384\\
    0.501445133195605	0.503349284171018\\
    0.502135196624297	0.0213550757879046\\
    0.507550247325154	0.001\\
    0.507864669332552	0.171074341706198\\
    0.508078134085594	0.0336728749088528\\
    0.508307793016704	0.151955454617427\\
    0.510077144066697	0.458127468233445\\
    0.510157609258629	0.193160753869977\\
    0.510586333132756	0.078691852640799\\
    0.512304806013776	0.0629120629144312\\
    0.514200587454839	0.131735961372681\\
    0.514282733565626	0.0486870541933319\\
    0.514758422549515	0.250710301158757\\
    0.515030240842672	0.00943483404835209\\
    0.515291170235572	0.0988350606941517\\
    0.516095650265313	0.115258489337938\\
    0.516433170830241	0.216266530636613\\
    0.517682189754604	0.0215973099366651\\
    0.522293174431827	0.0346011562530397\\
    0.523653346593524	0.001\\
    0.52411999552612	0.0873995330711536\\
    0.524917481640612	0.0725761250045635\\
    0.525299839953375	0.160732797041949\\
    0.526831618521803	0.405900287594123\\
    0.527167675761795	0.181197680659415\\
    0.527490836147039	0.142731862214095\\
    0.528076283274232	0.289714723563819\\
    0.529466118566358	0.0465621836722996\\
    0.529776479229208	0.0599286747420875\\
    0.5306374624258	0.107091905584094\\
    0.531125327214231	0.124993904372554\\
    0.531258678634355	0.00939174312339408\\
    0.532481002184696	0.0221243979416852\\
    0.534705426622902	0.20243298971116\\
    0.535819477622774	0.330470460948604\\
    0.537036839755423	0.229173179282276\\
    0.538269200010183	0.0936999145127919\\
    0.538512511914005	0.0350330519620374\\
    0.539826253180299	0.0779892414145271\\
    0.539974069423668	0.001\\
    0.542059251744378	0.260631212647424\\
    0.544482515886458	0.169400390969318\\
    0.544859335500773	0.0639863542612551\\
    0.545071550675116	0.151769980133956\\
    0.545170133182567	0.0495894255164972\\
    0.545738930525182	0.116635012432012\\
    0.546361903577118	0.13448281584634\\
    0.547909048152786	0.00966138153016363\\
    0.54808865798493	0.0224470553229716\\
    0.55137044958778	0.187741398141306\\
    0.552375935558903	0.680947789568957\\
    0.552649619546509	0.10315953771424\\
    0.553278809764743	0.0359366283264923\\
    0.55457543722044	0.0881611114883416\\
    0.556528716644246	0.001\\
    0.558186180487993	0.0745135213532031\\
    0.558533382564167	0.208743545465111\\
    0.561267625887682	0.237144392880668\\
    0.561556701125311	0.0608664374302506\\
    0.56166272337186	0.125885777524088\\
    0.562268097728234	0.0475320626969628\\
    0.563231593612606	0.0229814000577606\\
    0.563255328192326	0.145898093171424\\
    0.564816204153194	0.00984156018472769\\
    0.565285112551079	0.164196813701597\\
    0.5656993827562	0.294811956678547\\
    0.5681049887415	0.110953670447476\\
    0.570041809378433	0.385143132368313\\
    0.57022657419303	0.0357040299134773\\
    0.570504796833151	0.0952604699079415\\
    0.571510183250578	0.426696679826125\\
    0.572245825093168	0.182682960525825\\
    0.573334381070904	0.001\\
    0.573519550806969	0.0801708873597765\\
    0.575020604544924	0.261967849120561\\
    0.576787268327378	0.0653824997625772\\
    0.577472688700624	0.135545312616676\\
    0.579520229890646	0.0231962887680584\\
    0.57953350900009	0.0513002824343801\\
    0.580000895582323	0.48837105119145\\
    0.580737156234512	0.220660832663279\\
    0.581883008325663	0.00986287713076488\\
    0.58426411057967	0.119907111954405\\
    0.584638179809228	0.153190427308814\\
    0.585309499648091	0.0380876958530632\\
    0.586339053054679	0.334066610620202\\
    0.586378248769242	0.19759536992964\\
    0.586653250735857	0.103771606300052\\
    0.589057973760151	0.0877347849133222\\
    0.589892493136417	1\\
    0.590409022937788	0.001\\
    0.591277424729553	0.071663999460233\\
    0.591413437784168	0.171057399856278\\
    0.593346115736458	0.57304949233119\\
    0.594582298791453	0.0246380577703866\\
    0.595635576487675	0.0558450990088521\\
    0.596868328105258	0.134652222050042\\
    0.598738829777321	0.0105240473066063\\
    0.600115617739716	0.240436991330201\\
    0.600359530163263	0.0400284019504659\\
    0.602802178282786	0.116315022862943\\
    0.603773091734709	0.10006386293313\\
    0.605761872462862	0.150599568300284\\
    0.605902930983378	0.0835990725796115\\
    0.606535021798871	0.213467524094106\\
    0.607056921644731	0.271210673545187\\
    0.607058552181599	0.18646342442635\\
    0.607771552039507	0.001\\
    0.608084306955161	0.0677333436878221\\
    0.609358319869383	0.0241361661652801\\
    0.612107257067796	0.0523508467950445\\
    0.612285600123026	0.367670317773093\\
    0.612791822308345	0.303040678933114\\
    0.615508512210431	0.128853074011563\\
    0.61602632487083	0.166151154207834\\
    0.616261849261829	0.0107017662269159\\
    0.617152777927979	0.0375413133803849\\
    0.620281218089127	0.096819735260088\\
    0.622757486159214	0.0799816426218843\\
    0.624083989643251	0.425838338821928\\
    0.624698101465179	0.112443402120939\\
    0.625441917407474	0.001\\
    0.625632956829962	0.144220850229928\\
    0.625645525665396	0.0643579009072677\\
    0.626427085960204	0.0232136105388252\\
    0.627927077682532	0.199035489883988\\
    0.630306973221674	0.0497329952050895\\
    0.631865040095916	0.225575826754716\\
    0.634011573974565	0.177143490613389\\
    0.63458358736445	0.0111668932878748\\
    0.634742874362826	0.25018189393905\\
    0.635091047859878	0.0354902092996598\\
    0.63696325473517	0.0913176748296006\\
    0.636991368841291	0.125429829305426\\
    0.640527543556626	0.157044590358236\\
    0.640640784972775	0.0740580429886253\\
    0.643441205451796	0.001\\
    0.644757361357534	0.106239700091549\\
    0.645005607721963	0.0577107905505074\\
    0.64651524903079	0.278741365830036\\
    0.646939323609288	0.0248998759528513\\
    0.649484825489116	0.138734895356783\\
    0.65073892011604	0.0414673054004473\\
    0.65100060233539	0.479625843516922\\
    0.65261686889849	0.0109993063061999\\
    0.653627282685768	0.351446724764618\\
    0.654416571595649	0.212739631590434\\
    0.654562568720785	0.18974187934933\\
    0.654788438057329	0.0873549834846506\\
    0.657123801122853	0.118483468822658\\
    0.659158249992405	0.0709180183738969\\
    0.65974131335425	0.309375401320824\\
    0.660601639713169	0.168862762556433\\
    0.661791747623867	0.001\\
    0.66189247625934	0.394652051410662\\
    0.662731700486733	0.0550786745811837\\
    0.662931065314425	0.0258420475317141\\
    0.665603486017054	0.099616084084948\\
    0.666551023906498	0.150190424387342\\
    0.669534026930142	0.0406979195201425\\
    0.670700592099279	0.231373558782614\\
    0.671164723675407	0.0118522814955743\\
    0.671598312738568	0.130325764162695\\
    0.674389301451355	0.0794629048405335\\
    0.675558707268686	0.257740849136353\\
    0.678660965208649	0.111669079232818\\
    0.679021956202579	0.0618471221625905\\
    0.680293708356908	0.18220099197221\\
    0.680517238798469	0.001\\
    0.681039491003708	0.0271660439265588\\
    0.683292951763985	0.204687093125544\\
    0.685442494031154	0.0929957131766709\\
    0.686097870980094	0.0447132116107853\\
    0.687520773687057	0.143275173668664\\
    0.689259805590052	0.0120396662250124\\
    0.690904753626255	0.162650332521257\\
    0.692028171275076	0.283400630185962\\
    0.694045575731611	0.123446310961041\\
    0.694296682170105	0.538911009784965\\
    0.694739294536909	0.0760600962045704\\
    0.697893941197542	0.0270785496039798\\
    0.69822187580297	0.0591112600716307\\
    0.69964286774087	0.001\\
    0.699898341864744	0.105420653164527\\
    0.700579237967784	0.434690684810214\\
    0.702666451883588	0.221903905361462\\
    0.70417594310642	0.335405416830239\\
    0.705164772104615	0.0422352481090219\\
    0.70646353075229	0.372622967098716\\
    0.706772160018244	0.179356953663548\\
    0.708012168786225	0.0891127363759603\\
    0.708385252318448	0.137287607902574\\
    0.708678354706986	0.0124240183296247\\
    0.710950838066143	0.246902331674408\\
    0.716023825821699	0.156248767710767\\
    0.716316988711698	0.0730466285565209\\
    0.717350803033471	0.0271271804138151\\
    0.717395225706434	0.198680579977907\\
    0.717425971384585	0.116836737286828\\
    0.717745146830509	0.0559295015320223\\
    0.719195461218324	0.001\\
    0.724610187928261	0.0979219137180479\\
    0.726850871547114	0.304589898062035\\
    0.727028302467984	0.0413206318468787\\
    0.727974631005001	0.702562861471453\\
    0.728393259575509	0.0130453090083606\\
    0.729837564864187	0.133336664555642\\
    0.732466010573931	0.172396093030168\\
    0.73298174958944	0.080001637379056\\
    0.733779310517865	0.268663916421228\\
    0.735106098614293	0.218062425926206\\
    0.736580453483612	0.061155527203243\\
    0.739203643541358	0.001\\
    0.739413034571886	0.112981612465338\\
    0.739923394649405	0.0274823115149249\\
    0.74264582851643	0.149708508990648\\
    0.746344809578568	0.0446134048173343\\
    0.747256483575423	0.191502846380512\\
    0.747547863110444	0.0945173452429884\\
    0.749510087893579	0.0128281181339464\\
    0.75317036176248	0.403488347082097\\
    0.753688677422961	0.0763340450969881\\
    0.753989393032915	0.127872993144205\\
    0.754842184751246	0.237351817051572\\
    0.756744455245836	0.356228493615302\\
    0.758691298889803	0.0591073824649935\\
    0.759115731338368	0.167243389525796\\
    0.759698013584323	0.001\\
    0.75983836801909	0.0299295009049929\\
    0.762747111081438	0.456695314270791\\
    0.762924489961732	0.292271510909532\\
    0.763722427287055	0.106963128693441\\
    0.769203161881151	0.145133180817729\\
    0.769753510479861	0.204966052971033\\
    0.769995025332956	0.0143304872359121\\
    0.771076444054773	0.0451791612434581\\
    0.771960152129543	0.0865154129897995\\
    0.777647641134139	0.260385243955924\\
    0.778558518253232	0.0671858233987608\\
    0.778900956360038	0.124154064291286\\
    0.780711341644039	0.001\\
    0.781256198364804	0.177966922244911\\
    0.784194209945853	0.0305359521466302\\
    0.787273858202767	0.103669076351328\\
    0.790456978537209	0.322266836606267\\
    0.791093732393512	0.0494498983981801\\
    0.791245102883616	0.0139654634782081\\
    0.791643044144658	1\\
    0.791675620239712	0.15429084211139\\
    0.791785867643239	0.224723795284473\\
    0.794400741946656	0.515966866990312\\
    0.796513186500372	0.0851662877384971\\
    0.79946406923831	0.600170501062848\\
    0.80079916851087	0.132626182581681\\
    0.802273199141531	0.0663900002558779\\
    0.802278788858646	0.001\\
    0.803366060460264	0.194004866011225\\
    0.805166797063659	0.380256155548918\\
    0.805921138459091	0.0332270827906005\\
    0.807416865753574	0.278450563173025\\
    0.808182620358926	0.111641055038369\\
    0.812983015466282	0.0155825720691139\\
    0.814394679827347	0.167780234079369\\
    0.816327247600686	0.0505273096041902\\
    0.818445826831299	0.0928032504270515\\
    0.818494952900292	0.246048685812074\\
    0.823726416778737	0.143517221791129\\
    0.824438152336681	0.001\\
    0.82455235528046	0.0714918503059655\\
    0.828804347920499	0.21146328368352\\
    0.830004179976677	0.120081618859034\\
    0.830519287830223	0.0340602172429881\\
    0.835226263438917	0.0151990657054844\\
    0.836524388365855	0.341978362159376\\
    0.83919023230877	0.183117649293557\\
    0.839221855686734	0.0537956193362352\\
    0.839771043851705	0.422728003796822\\
    0.841194972488585	0.0896657881224913\\
    0.844161130262712	0.303759150542408\\
    0.84723013965176	0.001\\
    0.847665580275625	0.157680875074453\\
    0.849237013532673	0.262039477282972\\
    0.851192325997286	0.109175142254367\\
    0.85282972422781	0.0714585141493565\\
    0.854318204611077	0.035371011351175\\
    0.854384415232404	0.133670259229993\\
    0.858017263977409	0.0160846343985244\\
    0.859886810339956	0.229401737227339\\
    0.86628116884656	0.0521460562105219\\
    0.867927393332684	0.197815513984804\\
    0.870327315597891	0.0943324064512647\\
    0.870698676956959	0.001\\
    0.876022113248208	0.168875907447211\\
    0.877382067404945	0.118106962641109\\
    0.877494160399592	0.0732535436720993\\
    0.881055105305614	0.0164575051466143\\
    0.881317602721726	0.366083833187549\\
    0.882028102829594	0.283153194173283\\
    0.882536231488304	0.143421373546712\\
    0.88371691485531	0.0363311701057444\\
    0.893404404202167	0.0564377760923533\\
    0.894556939797737	0.243526098946783\\
    0.894891255684485	0.001\\
    0.895693856830698	0.565543808536417\\
    0.897873055505335	0.460089206624845\\
    0.900009841366048	0.0848304150751377\\
    0.900472013963194	0.212132976809285\\
    0.900878879672293	0.107619864276068\\
    0.902003755561598	0.317309233166617\\
    0.906491575349975	0.0173386320186445\\
    0.906649556989047	0.184956975308291\\
    0.908947025638522	0.131575320043384\\
    0.910502659359962	0.0385263503675858\\
    0.913921487318868	0.156428527363267\\
    0.916196411694014	0.0653963305466984\\
    0.919859323644561	0.001\\
    0.922733416455395	0.266899287694769\\
    0.928287418460604	0.0871188437359387\\
    0.928779390126373	0.113287298571226\\
    0.931972425960832	0.020854447263783\\
    0.932274979386165	0.403714349952014\\
    0.932294234473633	0.0487553790215476\\
    0.934834444815541	0.200954100688394\\
    0.943118710469934	0.168179596266034\\
    0.944339498987783	0.136116788084165\\
    0.944416546008027	0.347231838414697\\
    0.945658727352316	0.001\\
    0.949914809134159	0.0722529577967304\\
    0.952618801254546	0.297650992139386\\
    0.953471481886421	0.226765874704757\\
    0.953509875508481	0.103218221186642\\
    0.954671033299111	0.044519814402682\\
    0.957530782963133	0.0199232359874518\\
    0.963557693388711	0.257691741209224\\
    0.969694448268183	0.188133554506462\\
    0.972105488002052	0.153023232757829\\
    0.972350213629979	0.001\\
    0.974189761086614	0.122987464030628\\
    0.975915846014618	0.065971470600724\\
    0.976796367712302	0.0926729205271357\\
    0.978673724077369	0.0403220218544877\\
    0.985096858805558	0.0153329383818049\\
    1	0.001\\
    1	0.0285097307387055\\
    1	0.0556512513770067\\
    1	0.0827770713009296\\
    1	0.110232139268335\\
    1	0.138371595309512\\
    1	0.167579080132279\\
    1	0.198287618398392\\
    1	0.231005498900014\\
    1	0.266350467868962\\
    1	0.30509718271678\\
    1	0.348245748743942\\
    1	0.397124273467859\\
    1	0.453547708894511\\
    1	0.520072942352945\\
    1	0.600425156484955\\
    1	0.70024373575407\\
    1	0.828458882366566\\
    1	1\\
    };
  % %\addlegendentry{data1}

  \addplot [color=mycolor1,mark size=1pt,only marks,mark=triangle*,mark options={solid,scale=1.5},forget plot]
    table[row sep=crcr]{%
    0	0.001\\
    0	0.0168187505296433\\
    0	0.03241608873095\\
    0	0.0479033014713116\\
    0	0.0633897208445366\\
    0	0.0789858291198479\\
    0	0.0948064895022532\\
    0	0.110974510993003\\
    0	0.127624798003611\\
    0	0.144909410108953\\
    0	0.163003979469953\\
    0	0.182116129362376\\
    0	0.20249685310644\\
    0	0.224456331105068\\
    0	0.248386536879842\\
    0	0.274794502077065\\
    0	0.304352857346471\\
    0	0.33797946579834\\
    0	0.376968359847148\\
    0	0.423216382959283\\
    0	0.479641324488477\\
    0	0.551019300611469\\
    0	0.645857993453341\\
    0	0.781315527547944\\
    0	1\\
  };
  %\addlegendentry{data2}

  \addplot [color=mycolor2,mark size=1pt,only marks,mark=square*,mark options={solid},forget plot]
    table[row sep=crcr]{%
    1	0.001\\
    1	0.0285097307387055\\
    1	0.0556512513770067\\
    1	0.0827770713009296\\
    1	0.110232139268335\\
    1	0.138371595309512\\
    1	0.167579080132279\\
    1	0.198287618398392\\
    1	0.231005498900014\\
    1	0.266350467868962\\
    1	0.30509718271678\\
    1	0.348245748743942\\
    1	0.397124273467859\\
    1	0.453547708894511\\
    1	0.520072942352945\\
    1	0.600425156484955\\
    1	0.70024373575407\\
    1	0.828458882366566\\
    1	1\\
  1	0\\
  };
  %\addlegendentry{data3}

  \addplot [color=mycolor3, draw=none, mark=pentagon*, mark options={solid, mycolor3,scale=0.75}]
    table[row sep=crcr]{%
    0.225	0.045\\
    0.25	0.045\\
    0.275	0.045\\
  };
  %\addlegendentry{data4}

  \end{axis}
  \end{tikzpicture}%
\\
\caption{\emph{Smoothly-varying-density node layouts for pricing two-dimensional arithmetic basket options under the Black--Scholes--Merton model and one-dimensional options under the Heston model. The close-field boundary conditions are enforced in the blue triangle node, and the far-field boundary conditions are enforced in the red square nodes. The yellow pentagons show the locations of interest.}}
\label{fig:gridsmooth}
\end{figure}

\par
The results in \textbf{Paper \ref{paper2}} show on several examples the advantages of such node layouts over the previously presented ones. The condition numbers of the differentiation matrices in all considered problems are significantly lower, and the accuracy is tremendously improved, as the density can be increased around the areas of interest more easily. 
\par
Unfortunately, quick node placing schemes of this sort for higher dimensions are still unavailable. One of the recent works~\cite{vlasiuk2017fast}, introduces a novel way to quickly generate three-dimensional smoothly varying node layouts, but other than that the resources about this problem are very scarce.
%
%%





%
\section{Choosing Shape Parameters}
The properties of the RBF-FD methods presented so far are quite attractive, but when it comes to implementing the methods in practice, with infinitely smooth RBFs we need to deal with the selection of the shape parameter $\varepsilon$. As it can be seen in the plot to the left of Figure \ref{fig:RBF}, the shape parameter tunes the support of the RBF. The larger the shape parameter is, the smaller the support --- and vice versa. Moreover, the greater the support of an RBF, the approximation becomes more accurate. Nevertheless, if an RBF becomes too flat, the system of equations \eqref{eqRBFFDmat} becomes nearly singular, and the computations of the differentiation weights become ill-conditioned. There have been several approaches to stabilize the RBF-FD stencils and make them independent of the choice of the shape parameter~\cite{fornberg2013stable}, but all those treatments came with significant increases in computational costs.  

\par
Therefore, in \textbf{Paper \ref{paper1}}, we suggested choosing the shape parameter such that it has the smallest value before the problem becomes ill-conditioned, in order to maintain high accuracy of the RBF-FD approximation. In that article, we showed by spatial error analysis and verified by numerical experiments, that an efficient RBF-FD method can be constructed if the shape parameter for approximating the Black--Scholes--Merton differential operator is chosen as 
\begin{equation}
\varepsilon=\varepsilon^*(h)\equiv\frac{\alpha}{h},
\label{eq:shape}
\end{equation}
where $\alpha$ is a real positive constant, and $h$ is the characteristic distance between the nodes. The constant $\alpha$ is obtained by linear regression between two points in the $h$-$\varepsilon$ plane with the minimal error for the given $h$, obtained experimentally. The suggestion has been verified on one-dimensional and two-dimensional European and American option pricing problems, on equidistant Cartesian and problem-adapted nonuniform layouts. The result is illustrated in \textbf{Figure \ref{fig:contour1}}.
\begin{figure}[H]
\centering
\rmfamily
\begin{tikzpicture}[trim axis left, trim axis right,baseline]
    \begin{axis}[
        axis on top,
        %grid=major,
        % tick label style = {font=\sansmath\sffamily},
        width=0.9*0.8032\textwidth,
        height=0.9*0.4519\textwidth,
        scale only axis,
        enlargelimits=false,
        xmode=log,
        xmin=1e-04,
        xmax=1e-01,
        ymode=log,
        ymin=1e-03,
        ymax=1e03,
        yminorticks=true,
        xminorticks=true,
        xlabel={$h$},
        % xticklabels={,,}, %hides y ticks
        ylabel={$\varepsilon$},
        ytick distance=10^1,
        % title={RBF-FD approximation: GA},
        ]

      \addplot graphics[xmin=1e-04,ymin=1e-03,xmax=1e-01,ymax=1e03] {figures/n3_p0new.png};

      \addplot[
        black, semithick,
        domain=1e-04:1e-01,
        samples=100,
        ]
        {0.0015/x} [every node/.style={yshift=8pt},sloped]
              node[pos=0.75] {$\varepsilon^*(h)$}
              ;
    \end{axis}
  \end{tikzpicture}

\rmfamily
\begin{tikzpicture}[baseline, trim axis left, trim axis right]
\pgfplotsset{xtick style={draw=none}, every axis y label/.append style={yshift=7.75cm}}
    \begin{axis}[
        axis on top,
        %grid=major,
        % tick label style = {font=\sansmath\sffamily},
        width=3*0.0115\textwidth,
        height=0.9*0.4519\textwidth,
        scale only axis,
        enlargelimits=false,
        xmin=0,
        xmax=1,
        ymode=log,
        ymin=1e-09,
        ymax=1e-02,
        % ytick distance=10^1,
        ytick={1e-09,1e-08,1e-07,1e-06,1e-05,1e-04,1e-03},
        extra y ticks={1e-02},
        extra y tick labels={$\geq10^{-2}$},
        %yminorticks=true,
        title={$\Delta u_{\text{max}}$},
        ylabel near ticks, yticklabel pos=right,
        xticklabels={,,},
        ]

      \addplot graphics[xmin=0,ymin=1e-09,xmax=1,ymax=1e-02]{figures/cbarnew.png};
    \end{axis}
  \end{tikzpicture}

\caption{\emph{The maximum absolute error $\Delta u_{\max}$ measured in the subdomain $\hat\Omega=[\frac{1}{3}K,\frac{5}{3}K]$ around the strike price $K$, as a function of $h$ and $\varepsilon$, for a one-dimensional European call option priced on an equidistant Cartesian grid with RBF-FD stencil size $n=3$. The black line shows the appropriate choice for the shape parameter. The RBF-FD approximation is performed using GA basis functions.}}
\label{fig:contour1}
\end{figure}
If we take a closer look at \textbf{Figure \ref{fig:contour1}}, we can see that as $h$ decreases, the high error areas are joining together from top and bottom --- leaving no space for high accuracy if $h$ is sufficiently small. The truncation error analysis results presented in \textbf{Paper \ref{paper1}} explain the presented behavior, and we consider the solutions for this issue in the following section. 
%
%%





%





%
\section{Role of Polynomials}
Many infinitely smooth RBFs have been successfully used for approximating differential operators of PDEs by RBF-FD. Nevertheless, the linear systems of equations that needed to be solved in order to obtain the weights $\mathbf{w}_j$ have been often ill-conditioned, especially as $h$ was becoming smaller. Several works~\cite{davydov2011adaptive, fornberg2011stabilization, flyer2012guide, larsson2013stable, fornberg2013stable, flyer2016enhancing}, addressed this problem by adding low-order polynomials together with RBFs into the presented interpolation. The linear system that we need to solve to obtain the differentiation weights for each node in our problem then becomes
\begin{equation}
{\footnotesize{
\label{eq:D2}
\left[\begin{array}{cc}
\mathbf{A}_j & \mathbf{P}_j^T \\
\mathbf{P}_j & \mathbf{0} \\
\end{array}\right]
\left[\begin{array}{c}
{\mathbf{w}}_j\\
{\mathbf{v}}_j\\
\end{array}\right]=
\left[\begin{array}{c}
\mathcal{L}\phi(\|\underline{x}_{j}-\underline{x}_j^{1}\|)\\
\vdots \\
\mathcal{L}\phi(\|\underline{x}_{j}-\underline{x}_j^{n_j}\|)\\
\mathcal{L}p_1(\underline{x}_j)\\
\vdots\\
\mathcal{L}p_{m_j}(\underline{x}_j)
\end{array}\right],
}}
\end{equation}
where $\mathbf{A}_j$ is the RBF matrix and $\mathbf{w}_j$ is the array of differentiation weights; $\mathbf{P}_j$ is the matrix of size $m_j \times n_j$ that contains all monomials up to degree $l$ (corresponding to $m_j$ monomial terms) that are evaluated in each node $\underline{x}_j^i$ of the stencil $\mathbf{x}_j$ and $\mathbf{0}$ is a zero square matrix of size $m_j \times m_j$; $\mathbf{v}_j$ is the array of dummy weights that are discarded, and $\{p_1, p_2, \ldots, p_{m_j}\}$ is the array of monomial functions indexed by their position relative to the total number of monomial terms $m_j$, such that it contains all the combinations of monomial terms up to degree $l$.

\par
We used monomials of degree $l=0$ augmented to GA basis functions in \textbf{Paper \ref{paper1}}, and the result is shown in \textbf{Figure} \ref{fig:contour2}. The figure no longer shows joining high error fields, which ensures well-conditioned convergence of the method as $h$ is decreased.
\par
Even though the problem of choosing the shape parameter for GA based RBF-FD schemes is thoroughly examined for option pricing problems in \textbf{Paper \ref{paper1}}, it remains unsolved for general applications. Nevertheless, recent developments~\cite{flyer2016on, bayona2017role}, have demonstrated that the RBF-FD approximation can be greatly improved by using high order polynomials together with PHSs as piecewise smooth RBFs in the interpolation, shown in the plot to the right of \textbf{Figure \ref{fig:RBF}}. With that approach, it seems as if the polynomial degree takes the role of controlling the rate of convergence. This allows us to use piecewise smooth PHSs as RBFs without a shape parameter, since the approximation accuracy is no longer controlled by the smoothness of the RBFs. Still, the RBFs do contribute to reduction of approximation errors, and therefore are necessary in order to have both stable and accurate approximation.
\begin{figure}[H]
\centering
\rmfamily
\begin{tikzpicture}[trim axis left, trim axis right,baseline]
    \begin{axis}[
        axis on top,
        %grid=major,
        % tick label style = {font=\sansmath\sffamily},
        width=0.9*0.8032\textwidth,
        height=0.9*0.4519\textwidth,
        scale only axis,
        enlargelimits=false,
        xmode=log,
        xmin=1e-04,
        xmax=1e-01,
        ymode=log,
        ymin=1e-03,
        ymax=1e03,
        yminorticks=true,
        xminorticks=true,
        % yticklabels={,,}, %hides y ticks
        xlabel={$h$},
        ylabel={$\varepsilon$},
        ytick distance=10^1,
        % title={RBF-FD approximation: GA with polynomials}
        ]

      \addplot graphics[xmin=1e-04,ymin=1e-03,xmax=1e-01,ymax=1e03] {figures/n3_p1new.png};

      \addplot[
        black,
        domain=1e-04:1e-01,
        samples=100,
        ]
        {0.0015/x} [every node/.style={yshift=7pt},sloped]
              node[pos=0.75] {$\varepsilon^*(h)$}
              ;
    \end{axis}
  \end{tikzpicture}

\rmfamily
\begin{tikzpicture}[baseline, trim axis left, trim axis right]
\pgfplotsset{xtick style={draw=none}, every axis y label/.append style={yshift=7.75cm}}
    \begin{axis}[
        axis on top,
        %grid=major,
        % tick label style = {font=\sansmath\sffamily},
        width=3*0.0115\textwidth,
        height=0.9*0.4519\textwidth,
        scale only axis,
        enlargelimits=false,
        xmin=0,
        xmax=1,
        ymode=log,
        ymin=1e-09,
        ymax=1e-02,
        % ytick distance=10^1,
        ytick={1e-09,1e-08,1e-07,1e-06,1e-05,1e-04,1e-03},
        extra y ticks={1e-02},
        extra y tick labels={$\geq10^{-2}$},
        %yminorticks=true,
        title={$\Delta u_{\text{max}}$},
        ylabel near ticks, yticklabel pos=right,
        xticklabels={,,},
        ]

      \addplot graphics[xmin=0,ymin=1e-09,xmax=1,ymax=1e-02]{figures/cbarnew.png};
    \end{axis}
  \end{tikzpicture}

\caption{\emph{The maximum absolute error $\Delta u_{\max}$ measured in the subdomain $\hat\Omega=[\frac{1}{3}K,\frac{5}{3}K]$ around the strike price $K$, as a function of $h$ and $\varepsilon$, for a one-dimensional European call option priced on an equidistant Cartesian grid with RBF-FD stencil size $n=3$. The black line shows the appropriate choice for the shape parameter. The RBF-FD approximation is performed using GA augmented with monomials of degree $l=0$.}}
\label{fig:contour2}
\end{figure}

\par
In \textbf{Paper \ref{paper2}}, we successfully apply the PHS based RBF-FD method to pricing two-dimensional European call and American put arithmetic basket options under the Black--Scholes--Merton model, and a one-dimensional European call option under the Heston model. The PHS based method, free of any hassle of picking the shape parameters, outperforms the standard FD method despite the computational overhead from the differentiation weights.
%
%%





%
\section{Smoothing of Payoff Functions}

\par
For many option pricing problems, the payoff has a discontinuity in the function itself or its derivatives. Such discontinuities limit the order of convergence obtained in numerical simulations. For that reason, before applying the numerical method, we suggest smoothing the payoff function according to~\cite{kreiss1970smoothing}. This approach has been successfully used for option pricing problems in, e.g., ~\cite{pooley2003convergence, during2015high}. The smoothing of the payoff function enables the numerical method to converge with the expected order of the discretization used. 

\begin{figure}[H]
\centering
% This file was created by matlab2tikz.
%
%The latest updates can be retrieved from
%  http://www.mathworks.com/matlabcentral/fileexchange/22022-matlab2tikz-matlab2tikz
%where you can also make suggestions and rate matlab2tikz.
%
\rmfamily
\definecolor{mycolor1}{rgb}{0.00000,0.44700,0.74100}%
\definecolor{mycolor2}{rgb}{0.85000,0.32500,0.09800}%
\definecolor{mycolor3}{rgb}{0.92900,0.69400,0.12500}%
\definecolor{mycolor4}{rgb}{0.49400,0.18400,0.55600}%
\definecolor{mycolor5}{rgb}{0.46600,0.67400,0.18800}%
\definecolor{mycolor6}{rgb}{0.30100,0.74500,0.93300}%
%
\begin{tikzpicture}[trim axis left, trim axis right, baseline]

  \begin{axis}[
  grid=major,
  %tick label style = {font=\sansmath\sffamily},
  width=0.45\textwidth,
  height=0.75\textwidth,
  at={(0\textwidth,0\textwidth)},
  scale only axis,
  unbounded coords=jump,
  xmode=log,
  xmin=1e-02,
  xmax=2e-01,
  xlabel={$h$},
  ymode=log,
  ymin=1e-06,
  ymax=1e-03,
  yminorticks=true,
  ytick distance=10^1,
  xminorticks=true,
  ylabel={$\Delta u_{\text{max}}$},
  axis background/.style={fill=white},
  %title style={font=\bfseries},
  title={Convergence},
  % legend pos=outer north,
  legend columns=3,
  transpose legend,
  legend style={legend cell align=left,align=left, at={(0,1.2)},anchor=west}
  ]

  \addplot [color=black, dashed]
    table[row sep=crcr]{%
  0.001	1e-07\\
  0.00125892541179417	1.58489319246111e-07\\
  0.00158489319246111	2.51188643150958e-07\\
  0.00199526231496888	3.98107170553497e-07\\
  0.00251188643150958	6.30957344480193e-07\\
  0.00316227766016838	1e-06\\
  0.00398107170553497	1.58489319246111e-06\\
  0.00501187233627272	2.51188643150958e-06\\
  0.00630957344480193	3.98107170553497e-06\\
  0.00794328234724281	6.30957344480193e-06\\
  0.01	1e-05\\
  0.0125892541179417	1.58489319246111e-05\\
  0.0158489319246111	2.51188643150958e-05\\
  0.0199526231496888	3.98107170553497e-05\\
  0.0251188643150958	6.30957344480194e-05\\
  0.0316227766016838	0.0001\\
  0.0398107170553497	0.000158489319246111\\
  0.0501187233627272	0.000251188643150958\\
  0.0630957344480193	0.000398107170553497\\
  0.0794328234724281	0.000630957344480193\\
  0.1	0.001\\
  0.125892541179417	0.00158489319246111\\
  0.158489319246111	0.00251188643150958\\
  0.199526231496888	0.00398107170553497\\
  0.251188643150958	0.00630957344480193\\
  0.316227766016838	0.01\\
  0.398107170553497	0.0158489319246111\\
  0.501187233627272	0.0251188643150958\\
  0.630957344480193	0.0398107170553497\\
  0.794328234724281	0.0630957344480193\\
  1	0.1\\
  };
  \addlegendentry{$\mathcal{O}(\Delta x^2)$}

  \addplot [color=black, dashdotted]
    table[row sep=crcr]{%
  0.001	1e-12\\
  0.00125892541179417	2.51188643150958e-12\\
  0.00158489319246111	6.30957344480194e-12\\
  0.00199526231496888	1.58489319246111e-11\\
  0.00251188643150958	3.98107170553497e-11\\
  0.00316227766016838	1e-10\\
  0.00398107170553497	2.51188643150958e-10\\
  0.00501187233627272	6.30957344480194e-10\\
  0.00630957344480193	1.58489319246111e-09\\
  0.00794328234724281	3.98107170553497e-09\\
  0.01	1e-08\\
  0.0125892541179417	2.51188643150958e-08\\
  0.0158489319246111	6.30957344480194e-08\\
  0.0199526231496888	1.58489319246111e-07\\
  0.0251188643150958	3.98107170553498e-07\\
  0.0316227766016838	1e-06\\
  0.0398107170553497	2.51188643150958e-06\\
  0.0501187233627272	6.30957344480193e-06\\
  0.0630957344480193	1.58489319246111e-05\\
  0.0794328234724281	3.98107170553497e-05\\
  0.1	0.0001\\
  0.125892541179417	0.000251188643150958\\
  0.158489319246111	0.000630957344480193\\
  0.199526231496888	0.00158489319246111\\
  0.251188643150958	0.00398107170553497\\
  0.316227766016838	0.01\\
  0.398107170553497	0.0251188643150958\\
  0.501187233627272	0.0630957344480193\\
  0.630957344480193	0.158489319246111\\
  0.794328234724281	0.398107170553497\\
  1	1\\
  };
  \addlegendentry{$\mathcal{O}(\Delta x^4)$}

  \addplot [color=black, thick, mark=diamond*, mark options={scale = 1.3, solid, black}]
    table[row sep=crcr]{%
    0.0100250626566416	7.5886574535161e-06\\
    % 0.0105540897097625	8.57895159101559e-06\\
    0.011142061281337	9.72228147643741e-06\\
    % 0.0117994100294985	1.10702521652658e-05\\
    0.0125391849529781	1.26596809826227e-05\\
    % 0.0133779264214047	1.46040730994315e-05\\
    0.014336917562724	1.69486755397803e-05\\
    % 0.0154440154440154	1.98552890761392e-05\\
    0.0167364016736402	2.35866771239822e-05\\
    % 0.0182648401826484	2.83557808533014e-05\\
    0.0201005025125628	3.46796687033725e-05\\
    % 0.0223463687150838	4.31932136169522e-05\\
    0.0251572327044025	5.51237913711251e-05\\
    % 0.0287769784172662	7.2569263399329e-05\\
    0.0336134453781513	9.94080377643199e-05\\
    % 0.0404040404040404	0.000143644202452091\\
    0.0506329113924051	0.000224138718419901\\
    % 0.0677966101694915	0.000390740290111056\\
    0.102564102564103	0.000821697635051559\\
  };
  \addlegendentry{$\text{FD}$}

  \addplot [color=mycolor1, thick, dashdotted, mark=square*, mark options={scale = 0.9,solid, mycolor1}]
    table[row sep=crcr]{%
    0.0100250626566416	5.67489189767789e-06\\
  % 0.0105540897097625	5.55102991257883e-06\\
  0.011142061281337	5.4245897675026e-06\\
  % 0.0117994100294985	5.30633081058385e-06\\
  0.0125391849529781	5.76537132542312e-06\\
  % 0.0133779264214047	7.74333548296188e-06\\
  0.014336917562724	1.01673887758155e-05\\
  % 0.0154440154440154	1.31270985772661e-05\\
  0.0167364016736402	1.68818812815268e-05\\
  % 0.0182648401826484	2.16678920812105e-05\\
  0.0201005025125628	2.78000505424987e-05\\
  % 0.0223463687150838	3.60763660614653e-05\\
  0.0251572327044025	4.74942057093787e-05\\
  % 0.0287769784172662	6.36334257922318e-05\\
  0.0336134453781513	8.70798482893939e-05\\
  % 0.0404040404040404	0.000122420681927398\\
  0.0506329113924051	0.000193301090642285\\
  % 0.0677966101694915	0.000392330278500077\\
  0.102564102564103	0.000954288395106928\\
  };
  \addlegendentry{$\text{RBF-FD-GS}_{\texttt{uniform}}$}

  \addplot [color=mycolor2, thick, mark=square*, mark options={scale = 0.9, solid, mycolor2}]
    table[row sep=crcr]{%
    0.01	4.14165843865555e-05\\
  % 0.0105263157894737	1.24037094471896e-05\\
  0.0111111111111111	1.22566457229217e-05\\
  % 0.0117647058823529	1.3068540435663e-05\\
  0.0125	1.40298458809716e-05\\
  % 0.0133333333333333	1.52056691880287e-05\\
  0.0142857142857143	1.664986889751e-05\\
  % 0.0153846153846154	1.83902580481697e-05\\
  0.0166666666666667	2.04396388880333e-05\\
  % 0.0181818181818182	2.30278475184913e-05\\
  0.02	2.64971538777616e-05\\
  % 0.0222222222222222	3.12860705828644e-05\\
  0.025	3.80636671498055e-05\\
  % 0.0285714285714286	4.81614980887175e-05\\
  0.0333333333333333	6.40359806401113e-05\\
  % 0.04	9.12091426900526e-05\\
  0.05	0.000153151719320656\\
  % 0.0666666666666667	0.000329175837351009\\
  0.1	0.000730421951017157\\
  };
  \addlegendentry{$\text{RBF-FD-GS}_{\texttt{nonuniform}}$}

  \addplot [color=mycolor3, thick, dashdotted, mark=triangle*, mark options={scale = 1.3,solid, mycolor3}]
    table[row sep=crcr]{%
    0.014124491030929	8.36410001815724e-06\\
  % 0.0148669096806806	9.1350846760721e-06\\
  0.0156917051052617	1.0028502564801e-05\\
  % 0.0166133930936617	1.10815043109155e-05\\
  0.0176501127404552	1.23307428878117e-05\\
  % 0.0188248322134005	1.38316136247436e-05\\
  0.0201670703622372	1.56518944708431e-05\\
  % 0.0217154113994819	1.79376384123464e-05\\
  0.0235212743230775	2.10393559688313e-05\\
  % 0.0256547337551516	2.49649788940307e-05\\
  0.0282138246343439	3.00189176842408e-05\\
  % 0.0313400329814846	3.6692234042502e-05\\
  0.0352453688425121	4.57511141431395e-05\\
  % 0.0402625627785849	5.851743112463e-05\\
  0.046945252681302	7.78916616573505e-05\\
  % 0.0562878035784234	0.000109920965315556\\
  0.0702728368926307	0.000169616596950958\\
  % 0.0935049164424369	0.000310529558508282\\
  0.139686059153916	0.000743167977149496\\
  };
  \addlegendentry{$\text{RBF-FD-PHS}_{\texttt{uniform}}$}

  \addplot [color=mycolor4, thick, mark=triangle*, mark options={scale = 1.3,solid, mycolor4}]
    table[row sep=crcr]{%
    0.0140893116711202	5.02720013236327e-06\\
  % 0.0148279399945695	5.65527662367907e-06\\
  0.0156482978157298	6.36362730365575e-06\\
  % 0.016564744693194	7.0885477311787e-06\\
  0.0175952133454409	7.93251233059677e-06\\
  % 0.0187623947869598	9.01480234062266e-06\\
  0.0200954286780432	1.02810154664346e-05\\
  % 0.0216323693691111	1.1851168449626e-05\\
  0.0234238773236666	1.39793874041304e-05\\
  % 0.0255389104378709	1.6629536537395e-05\\
  0.028073804789235	2.01883525170823e-05\\
  % 0.0311673593094027	2.5275475873842e-05\\
  0.0350271291661823	3.19452679388398e-05\\
  % 0.0399780181333785	4.16182867751698e-05\\
  0.046558865430144	5.78232547853892e-05\\
  % 0.0557332304127295	8.47688371809069e-05\\
  0.0694105608393805	0.000134546330187221\\
  % 0.0919844099636548	0.00024414135768297\\
  0.136319635318199	0.000687264699283568\\
  };
  \addlegendentry{$\text{RBF-FD-PHS}_{\texttt{nonuniform}}$}

  \addplot [color=mycolor5, thick, dashdotted, mark=*, mark options={solid, mycolor5}]
    table[row sep=crcr]{%
  %   0.014124491030929	1.56216283858429e-06\\
  % 0.0148669096806806	1.60011950256348e-06\\
  % 0.0156917051052617	1.66133358668564e-06\\
  % 0.0166133930936617	1.6449609870088e-06\\
  % 0.0176501127404552	1.69119461798347e-06\\
  % 0.0188248322134005	1.8310173203911e-06\\
  0.0201670703622372	8.86676792243582e-07\\
  0.0217154113994819	1.08094053489867e-06\\
  0.0235212743230775	1.59605544365503e-06\\
  0.0256547337551516	2.36856397064627e-06\\
  0.0282138246343439	3.22326044284102e-06\\
  0.0313400329814846	4.57864276742076e-06\\
  0.0352453688425121	7.23881122826828e-06\\
  0.0402625627785849	1.22034416352584e-05\\
  0.046945252681302	2.37328354763949e-05\\
  0.0562878035784234	4.89880039680132e-05\\
  0.0702728368926307	0.000117282713859648\\
  0.0935049164424369	0.000346118628974414\\
  % 0.139686059153916	0.00126121370849494\\
  };
  \addlegendentry{$\text{RBF-FD-PHS}^{\texttt{smoothed}}_{\texttt{uniform}}$}

  \addplot [color=mycolor6, thick, mark=*, mark options={solid, mycolor6}]
    table[row sep=crcr]{%
    % 0.0140893116711202	1.40203105304789e-06\\
    % 0.0148279399945695	1.40618983384723e-06\\
    % 0.0156482978157298	1.41872953168874e-06\\
    % 0.016564744693194	1.39141329373507e-06\\
    % 0.0175952133454409	1.36198724156136e-06\\
    % 0.0187623947869598	1.35229464069946e-06\\
    % 0.0200954286780432	1.33335907658125e-06\\
    % 0.0216323693691111	1.3002449169284e-06\\
    % 0.0234238773236666	1.27935297984075e-06\\
    % 0.0255389104378709	1.21859483323073e-06\\
    % 0.028073804789235	1.15139006272302e-06\\
    % 0.0311673593094027	1.04388269646372e-06\\
    0.0350271291661823	6.99972089132939e-07\\
    0.0399780181333785	1.27354519761577e-06\\
    0.046558865430144	2.22068321180727e-06\\
    0.0557332304127295	4.43330995397034e-06\\
    0.0694105608393805	5.66874455900854e-06\\
    0.0919844099636548	1.64849648476599e-05\\
    0.136319635318199	0.000117373463206207\\
  };
  \addlegendentry{$\text{RBF-FD-PHS}^{\texttt{smoothed}}_{\texttt{nonuniform}}$}
\end{axis}
\end{tikzpicture}%
\hspace{11pt}
% This file was created by matlab2tikz.
%
%The latest updates can be retrieved from
%  http://www.mathworks.com/matlabcentral/fileexchange/22022-matlab2tikz-matlab2tikz
%where you can also make suggestions and rate matlab2tikz.
%
\rmfamily
\definecolor{mycolor1}{rgb}{0.00000,0.44700,0.74100}%
\definecolor{mycolor2}{rgb}{0.85000,0.32500,0.09800}%
\definecolor{mycolor3}{rgb}{0.92900,0.69400,0.12500}%
\definecolor{mycolor4}{rgb}{0.49400,0.18400,0.55600}%
\definecolor{mycolor5}{rgb}{0.46600,0.67400,0.18800}%
\definecolor{mycolor6}{rgb}{0.30100,0.74500,0.93300}%
%
\begin{tikzpicture}[trim axis left, trim axis right, baseline]

  \begin{axis}[
  grid=major,
  %tick label style = {font=\sansmath\sffamily},
  width=0.4383\textwidth,
  height=0.75\textwidth,
  at={(0\textwidth,0\textwidth)},
  scale only axis,
  unbounded coords=jump,
  xmode=log,
  xmin=1e-01,
  xmax=1000,
  xlabel={$t_\texttt{CPU}$},
  ymode=log,
  ymin=1e-06,
  ymax=1e-03,
  yticklabels={,,}, %hides y ticks
  yminorticks=true,
  ytick distance=10^1,
  xminorticks=true,
  xmajorgrids,
  % xminorgrids,
  ymajorgrids,
  % yminorgrids,
  %ylabel={$\Delta u$},
  axis background/.style={fill=white},
  %title style={font=\bfseries},
  title={{\color{white}g} Performance {\color{white}g}},
  legend pos=north east,
  legend style={legend cell align=left,align=left,draw=white!15!black}
  ]
  \addplot [color=black, semithick, mark=diamond*, mark options={scale = 1.3, solid, black}]
    table[row sep=crcr]{%
    0.06582786	0.000821697635051559\\
    % 0.098248712	0.000390740290111056\\
    0.287278344	0.000224138718419901\\
    % 0.653154694	0.000143644202452091\\
    1.363033105	9.94080377643199e-05\\
    % 2.475437018	7.2569263399329e-05\\
    4.280939024	5.51237913711251e-05\\
    % 7.33640343	4.31932136169522e-05\\
    12.175330572	3.46796687033725e-05\\
    % 16.521671284	2.83557808533014e-05\\
    25.616239575	2.35866771239822e-05\\
    % 32.093849714	1.98552890761392e-05\\
    45.866742939	1.69486755397803e-05\\
    % 62.758118074	1.46040730994315e-05\\
    89.876309784	1.26596809826227e-05\\
    % 105.144092832	1.10702521652658e-05\\
    133.115589004	9.72228147643741e-06\\
    % 165.803878311	8.57895159101559e-06\\
    206.569398646	7.5886574535161e-06\\
  };
  \addlegendentry{fd2}

  \addplot [color=mycolor1, semithick, dashdotted, mark=square*, mark options={scale = 0.9,solid, mycolor1}]
    table[row sep=crcr]{%
    0.294333181	0.000954288395106928\\
    % 0.670892027	0.000392330278500077\\
    1.264234185	0.000193301090642285\\
    % 2.320541893	0.000122420681927398\\
    4.002661134	8.70798482893939e-05\\
    % 4.994922184	6.36334257922318e-05\\
    7.21062104	4.74942057093787e-05\\
    % 18.76967603	3.60763660614653e-05\\
    12.789741194	2.78000505424987e-05\\
    % 32.427884469	2.16678920812105e-05\\
    23.987790138	1.68818812815268e-05\\
    % 57.771913045	1.31270985772661e-05\\
    74.876368296	1.01673887758155e-05\\
    % 95.476722184	7.74333548296188e-06\\
    118.737986566	5.76537132542312e-06\\
    % 76.838053609	5.30633081058385e-06\\
    160.362616102	5.4245897675026e-06\\
    % 110.767437067	5.55102991257883e-06\\
    220.215181191	5.67489189767789e-06\\
  };
  \addlegendentry{gs reg}

  \addplot [color=mycolor2, semithick, mark=square*, mark options={scale = 0.9, solid, mycolor2}]
    table[row sep=crcr]{%
    0.331154439	0.000730421951017157\\
    % 0.784126736	0.000329175837351009\\
    1.321707978	0.000153151719320656\\
    % 2.407973012	9.12091426900526e-05\\
    4.492199157	6.40359806401113e-05\\
    % 5.900866537	4.81614980887175e-05\\
    9.359380277	3.80636671498055e-05\\
    % 22.562775191	3.12860705828644e-05\\
    16.399435412	2.64971538777616e-05\\
    % 40.433655359	2.30278475184913e-05\\
    30.67468922	2.04396388880333e-05\\
    % 77.191195997	1.83902580481697e-05\\
    75.199079076	1.664986889751e-05\\
    % 138.453168475	1.52056691880287e-05\\
    117.664701023	1.40298458809716e-05\\
    % 117.269431288	1.3068540435663e-05\\
    140.823239139	1.22566457229217e-05\\
    % 175.124017293	1.24037094471896e-05\\
    219.44733059	4.14165843865555e-05\\
  };
  \addlegendentry{gs adap}

  \addplot [color=mycolor3, semithick, dashdotted, mark=triangle*, mark options={scale = 1.3,solid, mycolor3}]
    table[row sep=crcr]{%
    0.837605647	0.000743167977149496\\
  % 2.025372492	0.000310529558508282\\
  3.639769825	0.000169616596950958\\
  % 7.256226552	0.000109920965315556\\
  12.290603411	7.78916616573505e-05\\
  % 18.44024452	5.851743112463e-05\\
  22.358980677	4.57511141431395e-05\\
  % 52.540076485	3.6692234042502e-05\\
  33.609578206	3.00189176842408e-05\\
  % 80.7768459	2.49649788940307e-05\\
  64.471801581	2.10393559688313e-05\\
  % 132.364579792	1.79376384123464e-05\\
  96.176644434	1.56518944708431e-05\\
  % 215.934641416	1.38316136247436e-05\\
  145.624530492	1.23307428878117e-05\\
  % 165.50100046	1.10815043109155e-05\\
  187.614598032	1.0028502564801e-05\\
  % 219.971299033	9.1350846760721e-06\\
  309.252198757	8.36410001815724e-06\\
  };
  \addlegendentry{phs reg}

  \addplot [color=mycolor4, semithick, mark=triangle*, mark options={scale = 1.3,solid, mycolor4}]
    table[row sep=crcr]{%
    1.053425471	0.000687264699283568\\
    % 2.34448544	0.00024414135768297\\
    4.408053073	0.000134546330187221\\
    % 6.595062342	8.47688371809069e-05\\
    12.539242522	5.78232547853892e-05\\
    % 18.967676855	4.16182867751698e-05\\
    24.935753374	3.19452679388398e-05\\
    % 57.789232721	2.5275475873842e-05\\
    43.468228217	2.01883525170823e-05\\
    % 92.989813402	1.6629536537395e-05\\
    71.169584803	1.39793874041304e-05\\
    % 156.944794247	1.1851168449626e-05\\
    157.613770699	1.02810154664346e-05\\
    % 267.749507248	9.01480234062266e-06\\
    181.066750603	7.93251233059677e-06\\
    % 215.198176558	7.0885477311787e-06\\
    300.717458059	6.36362730365575e-06\\
    % 312.620094495	5.65527662367907e-06\\
    372.861778238	5.02720013236327e-06\\
  };
  \addlegendentry{phs adap}

  \addplot [color=mycolor5, semithick, dashdotted, mark=*, mark options={solid, mycolor5}]
    table[row sep=crcr]{%
    % 204.797699017	1.56216283858429e-06\\
    % 137.21974258	1.60011950256348e-06\\
    % 108.867301159	1.6449609870088e-06\\
    % 122.361616773	1.66133358668564e-06\\
    % 108.698068286	1.69119461798347e-06\\
    % 85.366205378	1.8310173203911e-06\\
    100.308480392	8.86676792243582e-07\\
    80.801123362	1.08094053489867e-06\\
    60.677620782	1.59605544365503e-06\\
    45.861328437	2.36856397064627e-06\\
    43.590460923	3.22326044284102e-06\\
    27.272527995	4.57864276742076e-06\\
    22.378082203	7.23881122826828e-06\\
    15.556166938	1.22034416352584e-05\\
    13.325419311	2.37328354763949e-05\\
    7.335637489	4.89880039680132e-05\\
    4.719910087	0.000117282713859648\\
    2.619557301	0.000346118628974414\\
    % 1.731366747	0.00126121370849494\\
  };
  \addlegendentry{phs reg smoothed}

  \addplot [color=mycolor6, semithick, mark=*, mark options={solid, mycolor6}]
    table[row sep=crcr]{%
    1.444674318	0.000117373463206207\\
    2.478047665	1.64849648476599e-05\\
    4.160072468	5.66874455900854e-06\\
    7.925728684	4.43330995397034e-06\\
    10.689505454	2.22068321180727e-06\\
    16.261363834	1.27354519761577e-06\\
    21.123572908	6.99972089132939e-07\\
    % 24.955379512	1.04388269646372e-06\\
    % 42.464244356	1.15139006272302e-06\\
    % 45.719717467	1.21859483323073e-06\\
    % 64.871214182	1.27935297984075e-06\\
    % 84.073613238	1.35229464069946e-06\\
    % 109.512552142	1.39141329373507e-06\\
    % 115.041296553	1.36198724156136e-06\\
    % 122.191878887	1.41872953168874e-06\\
    % 137.262390987	1.40618983384723e-06\\
    % 156.076144824	1.40203105304789e-06\\
    % 170.189141117	1.33335907658125e-06\\
    % 173.101519462	1.3002449169284e-06\\
  };
  \addlegendentry{phs adap smoothed}
  \legend{};
\end{axis}
\end{tikzpicture}%

\caption{\emph{Absolute error $\Delta u_{\max}$ as a function of $h$ and CPU-time in seconds for a two-dimensional European call basket option.}}
\label{fig:smoothing}
\end{figure}

\par
In \textbf{Paper \ref{paper3}}, we apply the smoothing on the equidistant-Cartesian-grid-based node layouts as the theory in~\cite{kreiss1970smoothing} suggests, to price two-dimensional European basket options. Moreover, we find a way to use the smoothing on the payoff-function-adapted nonuniform node layouts to enable RBF-FD spatial order of convergence of four. The results of the numerical experiments show that replacing the payoff function with the smoothed one gives convergence of the appropriate order, at an acceptable increase in the computational cost. \textbf{Figure \ref{fig:smoothing}} shows the performance of the PHS based RBF-FD methods with smoothing against other RBF-FD and FD methods on uniform and nonuniform node layouts. 

\par
Nevertheless, the smoothing on the presented payoff-function-adapted nonuniform node layouts relies on the parallel placement of the nodes with the hyperplane of discontinuity. Smoothing of the initial data on the node layouts with smoothly varying density still remains an open question. 


%
\chapter{Outlook and Further Development}
\label{ch:outlook}

\par 
In this thesis, we present recent developments of RBF-FD methods and their potential to efficiently approximate solutions of PDEs for pricing of financial derivatives. We show how to apply RBF-FD to a number of option pricing problems in a way that utilizes the best features of the method such as mesh-free node placement, high order approximation, and sparsity of the differentiation matrices. By performing numerical experiments, we confirm the robustness of the method and compare its performance with other currently used and newly developed methods in financial practice. Moreover, we identify some challenges and limitations for which we suggest appropriate treatments, such as dealing with RBF shape parameters and smoothing of initial data, but there are still some problems left open.  

\par
As far as future research is concerned, we note that the smoothly-varying-density node placing algorithm can be efficiently employed only in two-dimensional domains. On the other hand, the payoff-function-adapted node layout is not as easily customizable for different cases as the smoothly-varying-density one. Some recent work has been done to come up with more robust and efficient ways of constructing customizable node layouts with smoothly varying density in higher dimensions \cite{vlasiuk2017fast}, but the results are yet to be seen in practice. Research on efficient generation of high-dimensional node layouts is expected to give a significant improvement in performance of the higher-dimensional RBF-FD methods and improve the competitiveness of these methods in different financial applications. Moreover, the problem of smoothing of payoff functions on smoothly-varying-density node layouts still remains open, as the available theory covers only equidistant Cartesian grids.

\par
Therefore, future development of RBF-FD is expected to result in a solid mesh-free high order method for multi-dimensional PDEs, that can be used together with dimension reduction techniques to efficiently solve problems of high dimensionality that we often encounter in finance.

%In this paper we study the benefits of using PHSs and node layouts with smoothly varying density for developing robust and efficient RBF-FD methods for option pricing. We present the improved RBF-FD scheme and successfully apply it to two types of multidimensional PDEs in finance: two-dimensional Eu- ropean call and American put basket options under the Black–Scholes–Merton model, and a European call option under the Heston model. We show numer- ically that the performance of the method is equally high when it comes to pricing American options compared to the European ones. By studying con- vergence, computational performance, and conditioning of the discrete systems, we demonstrate the desirable properties of the introduced approaches.
%The implemented RBF-FD methods significantly overperformed the stan- dard FD method in the numerical experiments, despite the computational over- head from the differentiation weights. As the computation of the differentiation weights is parallelizable, the performance dominance should be even higher when machines with higher number of cores are used.
%Using PHSs as RBFs, augmented with polynomials, in the RBF-FD approx- imations, show to be hassle free as a result of absence of the shape parameter. The PHSs take control of stabilizing the stencils as the degree of the augmented polynomials in the approximation dictates the formal order of the method.
%Although the used smoothly-varying-density node placing algorithm works only in two-dimensional domains, some recent work has been done to come up with more robust and efficient ways to construct adaptable smooth node layouts in higher dimensions [38]. Research on efficient generation of high-dimensional node layouts is expected to give a significant improvement in performance of the higher-dimensional RBF-FD methods and improve the competitiveness of these methods in different financial applications.

\backmatter
    % References
    % No restriction is set to the reference styles
    % Save your references in References.bib
%    \nocite{*} % Remove this for your own citations
    \bibliographystyle{teza}
    \bibliography{References}





%
\chapter{Contributions}





%
\section*{Paper I}
The author of the thesis contributed to this paper by developing all the codes and performing all the numerical experiments reported in the paper. The manuscript was written in close collaboration with the coauthor. 




%
\section*{Paper II}
The author of the thesis is the sole author of the paper.



%
\section*{Paper III}
The author of the thesis contributed to this paper by developing all the codes and performing all the numerical experiments reported in the paper. The manuscript was written in close collaboration with the coauthor. 




%
\section*{Paper IV}
The author of the thesis contributed to this paper by developing the RBF-FD codes. The manuscript was written and the numerical experiments were performed in close collaboration with the coauthor. 



%
\newpage
\section*{Paper V}
The author of the thesis contributed to this paper by developing the RBF-FD codes and writing the code necessary for performing numerical experiments on a supercomputer for all methods presented in the paper. Moreover, the author performed all the numerical experiments and produced all the figures for the manuscript. The manuscript was written in collaboration with the coauthors. 





%
\section*{Paper VI}
The author of the thesis contributed to this paper by developing the RBF-FD codes and writing the code necessary for performing numerical experiments on a supercomputer for all methods presented in the paper. Moreover, the author performed all the numerical experiments and produced all the figures for the manuscript. The manuscript was written in collaboration with the coauthors. The author was a member of the project management group.\\\\


\par
{\color{white}\texttt{
Slobodan Milovanović\\
slobodandjmilovanovic@gmail.com\\
https://millovanovic.me\\
@millovanovic\\
BTC: 3JoyXxvAp2jmuYQEBr7qZF47GwiwoHX1zh, bitcoin\\
ETH: 0xc01a2724Bb8231E9a81C8E3B4bA7ecD8cB88e501, ethereum\\}
Boston Consulting Group\\
\emph{It doesn't look like anything to me.}
2018.
}








\addtocontents{toc}{\vspace{\normalbaselineskip}}






\chapter{{\sffamily{\emph{Acknowledgments}}}}
{\noteunic
\par
As this doctoral thesis is based on academic research that was conducted over the period of five years, the ingredients for its creation stretch far beyond the simple working hours I spent performing my duties. In fact, like for many other international students, completing this doctoral work involved moving to a lovely town in a foreign country and befriending many wonderful people, while doing my best not to lose the friends that I already had. Therefore, I would like to express immense gratitude to everyone who made my time in Sweden so beautiful and supported me during this feat at Uppsala University --- regardless of where they were, where they are now, or where they are going to be.

\par
I start by thanking my supervisor Lina von Sydow for the tireless leadership and unlimited availability. From the first day, we had clear and direct communication which made our cooperation so comfortable, efficient, and joyful. Lina has been there for me in good times as much as in difficult times, both professionally and privately. I have learned so much about life from her. I am also grateful to my second supervisor Elisabeth Larsson for always having the right answers to my RBF questions. Two of them have been the best bosses one could ever wish for. I will never forget our numerous trips, especially the one when we went to Gotland together.

\par
Moreover, a distinguished share of gratitude also belongs to my closest coauthors Victor Shcherbakov and Josef Höök. I hope that Victor now understands why it is better to be the second one to graduate, as the latest reports highlight how RBF-FD outperforms RBF-PU. I am thankful to Josef for many things he taught me about Monte Carlo methods, and for sparking my interest in machine learning. As it turned out, those conversations in the hallways happened to lead me to my future career. 

\par
A workplace without colleagues does not count as a workplace. All of this hard work started with Simon Sticko in a tiny office ITC 2419. Working with Simon felt like working with a co-founder of a startup company. We were both passionate about our job, so we worked our way to the biggest, brightest office at the corner of the top floor of our building --- a.k.a., the Offie. Of course, since we were not working at our unicorn company, the big office came at a price of new officemates. Although that led to a lot more complicated attention sharing, I am genuinely grateful to Per Wahlund, Emil Kieri, Hanna Holmgren, Kristiina Ausmees, and Robin Eriksson, for all the flavors of friendship that we have developed in our Offie. Although we might never find out who broke the coffee mug, I will forever be in debt of gratitude to Simon for teaching me how to snowboard. I would also like to express special appreciation to Hanna and her partner Kristoffer Wahlberg for their kindness. They made me feel more welcome in Sweden than anyone else. Besides the officemates, I am especially thankful to Timofey Mukha for being such a great friend during this whole time. Moreover, I would not have found peace without acknowledging the mountaineering crew: Daniel Elfverson, Fredrik Hellman, Lina Meinecke, Karl Ljungkvist, and Adrien Coulier, who climbed Triglav, Kebnekaise, and some other high places with me. The outdoors credits would not be fair without remembering great climbing days I had with Andreas \& Ida Löscher. Finally, I am sincerely grateful to all of my colleagues for the great discussions and laughs we had during \emph{fika}. 

\par
Moving to Uppsala from the outside usually means moving around Uppsala many times more, due to its infamous accommodation situation. The bright side of that phenomenon is the flatmates that one learns to love because that is easier than sleeping outside in the cold winter. It was not so hard to love my flatmates, as all of them happened to be epic in their own ways. They were all friendly and helpful, but I would like to particularly thank Anton Axelsson for making me environmentally aware on an entirely new level, Daniel Elfverson for making me an outdoor freak, and Luka Šupraha whose seemingly infinite pool of knowledge in history helped us recreate the good old Yugoslavian atmosphere that everyone born in Yugoslavia is still nostalgic about.

\par
I would also like to salute my dear friends, scattered around the world, who have not let the distance tear us apart. Ivan Milić, whose brotherly love I always feel no matter where we are, and whose eclectic taste in music filled up with joy so many breaks from work. Nikola Božić, a true friend with unconditional support, who has been my career guide from the early days. Milica Kolarević, who has sincerely cared about me, has been there for me whenever I needed advice, and has never let me fail in anything. Đor\strike{d}e Pevčević, who has always been looking out for my best interests, and has selflessly shared his professional experience. Lara Vujović, who has regularly been checking whether I am doing my best. Ksenija Ilić, who has kept my hometown Valjevo feel like home for me.
\par
Academic life far away from home has its dark sides, too. We get to see some dear people so rarely, and some of them we will never get to see again. Unfortunately, during this doctoral journey, I lost one of my closest friends --- Filip Tanasković. We studied together at Valjevska Gimnazija in our hometown and graduated from the University of Belgrade as the best students at the Faculty of Mechanical Engineering. Then, he continued his education in Switzerland, while I chose Sweden. Filip was a true peer to me in every sense. It was exceptionally joyful to synchronize our life views whenever we would meet. Hereby, I would like to express lasting gratitude for all the great times we had together. I miss him so much.

\par
I cannot thank my family enough --- especially my parents Đor\strike{d}e \& Snežana Milovanović, for not making too many troubles while I was away. Moreover, the golden star acknowledgment with a watermelon on top goes to my favorite sister An\strike{d}elka Milovanović, who visited me the most and helped me survive some of the toughest weeks of this period. I also want to give my gratitude to the rest of the family who has been supportive of what I have been doing so far away from home. 

\par
All this time would be much lonelier for me if I did not have the warmth and kindness of my partner Katarina Veselinović by my side. We went through heaven and hell together, and I am so proud that we are managing together also at the end of this chapter.

\par
I would also like to extend my acknowledgments to Bogdan Đukić, Antonija Burčul, and Cecilia \& Johan Tilli for helping me explore the ideas of entrepreneurial life, which greatly influenced my career decisions. 

\par
Lastly, I am thankful to Petnica Science Center for always standing behind me and lending its resources to support my ideas. I would also like to acknowledge the support from Anna Mara Lundin Foundation, C. F. Liljewalchs Scholarship Foundation, SIAM Student Travel Fund, and Dositeja Fund for Young Talents of the Republic of Serbia --- for helping me fund numerous conferences and workshops.
}

















\newpage
\chapter*{{{\sffamily\emph{Summary}}}}
{\noteunic

\par
The purpose of this thesis is to present state of the art in radial basis function generated finite difference (RBF-FD) methods for pricing of financial derivatives. This doctoral work provides a detailed overview of RBF-FD properties and challenges that arise when the RBF-FD methods are used in financial applications. Moreover, with this dissertation, we aim to motivate further development of RBF-FD for solving multi-dimensional partial differential equations (PDEs) in finance.

\par
Across the financial markets of the world, financial derivatives such as futures, options, and others, are traded in substantial volumes. The value of all assets that underly outstanding derivatives transactions is several times larger than the gross world product (GWP). Financial derivatives are the most commonly used instruments when it comes to hedging risks, speculation based investing, and performing arbitrage. Therefore, knowing the prices of those financial instruments is of utmost importance at any given time. In order to make that possible in practice, it is often required to employ a set of skills incorporating knowledge in financial theory, engineering methods, mathematical tools, and programming practice --- which altogether constitute the field known as \emph{financial engineering}. 

\par
Many of the theoretical pricing models for financial derivatives can be represented using PDEs. In many cases, those equations are time-dependent, of high spatial dimension, and with challenging boundary conditions --- which most often makes them analytically unsolvable. In those cases, we need to utilize numerical approximation as a mean of estimating their solution. The fields of \emph{numerical analysis} and \emph{scientific computing} are concerned with obtaining approximate solutions while maintaining reasonable bounds on errors. Unfortunately, there is no universal numerical method which can be used to solve all problems of this type efficiently. In fact, there are tremendously many numerical methods for solving different types of differential equations, and all those methods are featured with their own limitations in performance, stability, and accuracy --- mostly dependent on details of the problems they aim to solve. Therefore, carefully selecting and developing numerical methods for particular applications has been the only way to build efficient PDE solvers in ongoing practice. 

\par
In this thesis, we present RBF-FD as a recent numerical method with potential to efficiently approximate solutions of PDEs in finance. Over the past years, besides the purely academic development and research of its numerical properties, the method has been mainly applied for simulations of atmospheric phenomena. As its name suggests, the RBF-FD method is of a finite difference type, from the radial basis function family. As a finite difference method, RBF-FD approximates differential equations by linear systems of algebraic equations, known as difference equations. Radial basis functions (RBFs) are used as interpolants that enable local approximations of differential operators that are necessary for constructing the difference equations. Constructed like that, the method is featured with a sparse matrix of the linear system of difference equations, and it is relatively simple to implement --- like the standard finite difference methods. Moreover, the method is mesh-free, meaning that it does not require a structured discretization of the computational domain which makes it equally easy to use in spaces of different dimensions, and it is of a customizable order of accuracy --- which are the features it inherits from the global radial basis function approximations. It is those properties that led us to recognize RBF-FD as a method with high potential for efficiently solving some analytically unfeasible and computationally challenging pricing problems in finance.

\par
Nevertheless, being a young method, RBF-FD is still under intense development, and we face many challenges when moving from simple theoretical cases toward more complex real-world applications. The core of this thesis deals with finding solutions for overcoming obstacles when financial derivatives are priced using RBF-FD to solve PDEs with multiple spatial dimensions. Thus, it represents a contribution to making the RBF-FD methods more reliable and efficient for use in financial applications. 

\par
The results in this thesis demonstrate how to successfully apply RBF-FD to different problems in finance by studying the effects of RBF shape parameters for Gaussian RBF-FD approximations, improving the approximation of differential operators in multiple dimensions by using polyharmonic splines augmented with polynomials, constructing suitable node layouts, and smoothing of the initial data in order to enable high order convergence of the method. Finally, we compare the RBF-FD method with other available methods on a plethora of pricing problems to give an objective image of the method's performance.
}

 
 
 
 
 
 
 
 
 
\newpage
\begin{swedish}
\chapter{\emph{Sammanfattning}}
{\noteunic

\par
Syftet med denna avhandling är att presentera forskningsfronten för finita differenser genererade via radiella basfunktioner (RBF-FD) för prissättning av finansiella derivat.
Baserat på de sex bifogade artiklarna presenterar denna avhandling en detaljerad överblick av egenskaper hos RBF-FD metoder samt de svårigheter som uppstår då dessa används inom finansiella tillämpningar.
Vidare ämnar denna avhandling motivera fortsatt utveckling av RBF-FD metoder för lösning av partiella differentialekvationer (PDEer) för finansiella tillämpningar.

\par
Inom världens finansiella markader handlas finansiella derivat, såsom terminer, optioner med mera, i stora volymer.
Värdet av alla underliggande tillgångar hos utställda finansiella derivat är flera gånger större än bruttovärldsprodukten.
Finansiella derivat är det instrument som används mest inom risk hedging, spekulationsbaserad investering, samt utnyttjande av möjligheter till arbitrage.
Det är därför synnerligen viktigt att veta priset på dessa finansiella derivat vid varje given tid.
För att möjliggöra detta behövs en uppsättning färdigheter såsom kunskaper inom finansiell teori, ingenjörsmässiga metoder, matematiska verktyg, och programmeringserfarenhet. Dessa bildar tillsammans området \emph{finansiell ingenjörskonst}.

\par
Många av de teoretiska modeller som finns för att prissätta finansiella derivat kan beskrivas med PDEer.
I många fall är dessa ekvationer tidsberoende, högdimensionella och med utmanande randvillkor -- vilket oftast gör dem analytiskt olösbara.
När så är fallet behöver numeriska metoder användas för att approximera lösningen.
Områdena numerisk analys och beräkningsvetenskap handlar om att beräkna approximativa lösningar och samtidigt garantera att felet i lösningen hålls inom rimliga gränser.
Dessvärre finns ingen universell numerisk metod som kan användas för att lösa alla problem effektivt.
Det finns i själva verket en stor mängd numeriska metoder för att lösa olika typer av differentialekvationer vilka alla har sina egna begränsningar med avseende på effektivitet, stabilitet och noggrannhet --- oftast beroende på egenskaper hos problemet som ska lösas.
Att noggrant välja och utveckla numeriska metoder för specifika problem har därför i praktiken varit det enda sättet att konstruera effektiva PDE-lösare.

\par
I denna avhandling presenteras RBF-FD som en nyutvecklad metod med potential att effektivt beräkna approximativa lösningar till PDEer inom finans.
Under de senaste åren, förutom rent akademisk utveckling och forskning kring metodens numeriska egenskaper, har den huvudsakligen används för simuleringar av atmosfäriska fenomen.
Som dess namn antyder, är RBF-FD en finit differensmetod, baserad på klassen av från radiella basfunktioner.
Som vanliga finita differenser, approximerar RBF-FD differential ekvationer som linjära system av algebraiska ekvationer, också känt som differensekvationer.
Radiella basfunktioner (RBFer) används som interpolanter för att möjliggöra lokala approximationer av differentialoperatorer som behövs för att konstruera differensekvationerna.
Genom att konstruera metoden på detta vis får metoden ett glest linjärt ekvationssystemet 
och är liksom vanliga finita differensmetoder relativt enkel att implementera.
Vidare är metoden nät-fri, vilket innebär att den inte kräver en strukturerad diskretisering av beräkningsområdet. Detta gör den lika enkel att använda oberoende av antal rumsliga dimensioner och den kan erhålla den noggrannhetsordning man önskar. De senare egenskaperna ärver metoden från globala approximationer med RBFer. Sammantaget gör detta att vi kan identifiera RBF-FD som en metod med hög potential att effektivt lösa analytiskt olösbara och beräkningsmässigt utmanande prissättningsproblem inom finans.

\par
Icke desto mindre är RBF-FD en ung metod som fortfarande befinner sig under intensiv utveckling
och vi möter många utmaningar när vi går från enkla teoretiska fall mot mer komplexa problem från riktiga tillämpningar.
Kärnan i denna avhandling är att hitta sätt att överbrygga de hinder som uppstår då finansiella derivat prissätts genom lösning av PDEer med hjälp av RBF-FD i flera rumsliga dimensioner.
På så sätt representerar denna avhandling ett bidrag till att göra RBF-FD metoder mer tillförlitliga och effektiva för användning inom finansiella tillämpningar.

\par
Resultaten i denna avhandling visar hur man framgångsrikt tillämpar RBF-FD på olika problem inom finans genom att studera effekten av formparameter för Gaussiska RBF-FD approximationer, visa hur man förbättrar approximationen av flerdimensionella differentialoperatorer genom att använda polyharmoniska splines förstärkta med polynom, hur man distribuerar nodpunkter på ett lämpligt sätt, samt hur man glättar initialdata för att uppnå hög konvergensordning. Slutligen jämför vi RBF-FD mot andra tillgängliga metoder på en stor mängd prissättningsproblem. Detta bidrar till att skapa en objektiv bild av metodens effektivitet.
}
\end{swedish}














\newpage
\begin{serbian}
\chapter{\emph{Преглед}}
{\noteunic
\par
\noindent Циљ ове докторске дисертације јесте да извести о тренутном стању развоја метода коначних разлика изведених из радијалних базних функција (КР-РБФ) за одређивање цене финансијских деривата. Овај докторски рад представља детаљан преглед својстава КР-РБФ и изазова који настају када се КР-РБФ методе примене у финансијској пракси. Такође, резултатима ове докторске тезе желимо да мотивишемо даљи развој КР-РБФ као поуздане методе за решавање високодимензионих парцијалних диференцијалних једначина (ПДЈ) које проистичу из финансијских модела.

\par
Финансијским дериватима као што су форвард уговори, опције и други, широм светских финансијских тржишта тргује се у огромним количинама. Укупна вредност имовине која у овом тренутку покрива активне финансијске деривате је неколико пута већа од вредности бруто светског производа. Финансијски деривати су најкоришћенији инструменти за управљање ризиком, инвестирање засновано на шпекулисању, као и коришћење арбитражних прилика. Стога, изузетно је важно знати њихове вредности у сваком тренутку. Како би то било могуће у пракси, неретко је потребно применити скуп вештина заснованих на познавању финансијске теорије, математичких алата, метода природних наука и техника програмирања --- које заједно чине арсенал знања струке познате као \emph{финансијско инжењерство}.  

\par
Многи теоријски модели за одређивање цена финансијских деривата могу се представити помоћу ПДЈ. У великом броју случаја, те једначине су временски зависне, високодимензионе, са проблематичним граничним условима --- што их најчешће чини аналитички нерешивима. У тим ситуацијама, принуђени смо да користимо нумеричко апроксимирање као средство за одређивање њихових приближних решења. Научно-техничке дисциплине као што су \emph{нумеричка анализа} и \emph{рачунарство}, баве се развојем метода за приближно решавање аналитички нерешивих проблема и изучавањем својстава тих метода. Нажалост, не постоји једна универзална нумеричка метода коју бисмо могли да користимо за ефикасно решавање свих проблема овог типа. Штавише, постоји огроман број нумеричких метода за приближно решавање различитих врста диференцијалних једначина, а свака од њих долази са себи својственим скупом ограничења у прецизности, ефикасности и стабилности --- која најчешће зависе од особина самог проблема који покушавамо да решимо. Стога, једини начин за ефикасно апроксимирање решења ПДЈ јесте пажљиво бирање, развијање и прилагођавање нумеричких метода за сваки проблем, понаособ.

\par
У овој дисертацији представљамо КР-РБФ као младу нумеричку методу која има изванредан потенцијал за ефикасно апроксимирање решења ПДЈ у финансијама. Претходних година, поред чистог академског развоја и истраживања везаних за њена нумеричка својства, ова метода је успешно коришћена у пракси за симулирање атмосферских феномена. Као што њено само име наговештава, КР-РБФ је врста методе коначних разлика (КР) која потиче из породице метода радијалних базних функција (РБФ). %Као КР метода, КР-РБФ апроксимира диференцијалне једначине линеарним системима алгебарских једначина. РБФ овде имају улогу интерполаната који омогућују локално апроксимирање диференцијалних оператора, потребно за постављање претходно поменутог система алгебарских једначина.% 
Тако конструисану, ову методу карактерише ретка матрица система и једноставно имплементирање --- слично као код класичних КР метода. Такође, ова метода не захтева мрежу за дискретизовање домена, већ дискретизовање можемо да обављамо слободним распростирањем тачака. То чини дискретизовање вишедимензионих домена врло једноставним, а поред тога метода је прилагодљивог реда прецизности --- што су особине које наслеђује од глобалних РБФ метода. Представљена својства служе као основ за препознавање КР-РБФ као прикладне методе за решавање одређених аналитички нерешивих и рачунарски изазовних проблема у финансијама.
\par
Ипак, то што је КР-РБФ метода у развоју носи са собом мноштво изазова са којима се суочавамо када са идеалних академских примера пређемо на проблеме стварног света. Срж ове дисертације јесте управо проналажење решења за препреке на које наилазимо када се цене финансијских деривата одређују решавањем вишедимензионих ПДЈ помоћу КР-РБФ. 

\par 
Проучавањем утицаја параметара облика Гаусових РБФ на КР-РБФ апроксимирање, коришћењем полихармонијских сплајнова са полиномима за апроксимирање вишедимензионих диференцијалних оператора, конструисањем прикладних распореда тачака током дискретизовања домена и глачањем почетних услова како би метод могао да конвергира високим редом --- резултати ове дисертације показују како успешно можемо применити КР-РБФ методу на разне проблеме у финансијама. Такође, ова докторска теза садржи резултате поређења КР-РБФ са другим методама на широком спектру финансијских проблема како би се створила објективнa сликa о перформансама представљене методе .

}
\end{serbian}
%%%%%%%%%%%%%%%%%%%%%%%%%%%%%%%%%%%%%%%%%%%%%%%%%%%%%%%%%%%%%%%%%%%%%%%%%%%%%%%%%%%%%%%
%%%%%%%%%%%%%%%%%%%%%%%%%%%%%%%%%%%%%%%%%%%%%%%%%%%%%%%%%%%%%%%%%%%%%%%%%%%%%%%%%%%%%%%
%%%%%%%%%%%%%%%%%%%%%%%%%%%%%%%%%%%%%%%%%%%%%%%%%%%%%%%%%%%%%%%%%%%%%%%%%%%%%%%%%%%%%%%  

\end{document}
